\label{theory}
\section*{Measuring Strategic Interest}


Essential to evaluating the relative importance countries may accord strategic motives when dispensing aid is to how one measures strategic motives.  However, as argued in the literature review, previous papers have been greatly inconsistent in how they have measured strategic interest, which in turn produces incoherence as to what exactly is being measured. It is not simply a matter of using different data to measure the same concept, for example choosing between using Polity data or Freedom House data to measure political instituions, but of using different data to measure different aspects of a concept. That is, while UN voting scores and arms transfers may be acceptable measures of strategic interest, surely nobody is arguing that they are conceptually equivalent in the same way as Polity and Freedom House are.\\

A large reason for this inconsistency is that while a dyad's strategic bilateral relationship is quite multifaceted, to date, there has not been a readily available measure of strategic interest which captures its various aspects the same way that scholars have done for political institutions in the form of Polity and Freedom house. The most relevant research to date has been concerned with how to measure foreign policy similarity, starting with \citep{demesquita:1975} Kendall $\tau_b$ measure and then \citet{signorino:1999} \textit{S Scores}, with new work continually being done \citep{gartzke:2006, hage:2011, }). However, foreign policy similiarity arguably only captures the political dimension of strategic interest, equally relevant is active military cooperation between two countries.\\

While military cooperation certainly has political dimensions, we would argue that it should be considered a separate aspect of strategic interest rather than an subset of political strategic interests. That is, military security is set apart by its capacity to affect a country's security in a manner that is more immediate, concrete and unilateral than other security concerns across countries more generally, for example as compared to access to natural resources, humanitarian sanctions or environmental policy. While military cooperation can certainly be mediated in the political arena, is qualitatively different - that is it is one thing to jointly condem the various atrocities of the North Korean government, it is quite another thing to take joint military action against it.

\subsection{A new measure of strategic interest}

Our measure of strategic interest attempts to introduce greater coherency to the literature by providing a more rigorous measure of these two aspects of strategic interest, political and military. We do so by first measuring the latent space of different dyadic variables that measure various aspects of the strategic relationshp between countries and then conducting a principal components analysis (PCA) to combine the resulting measures. As such, our political strategic interest is the first principal component that results from the PCA of the latent space for dyadic alliances, UN voting and membership in an intergovernmental organizations (IGOs). Meanwhile our military strategic interest measure is the first principal component of the PCA that results from the latent space for dyadic arms transfers, militarized interstate disputes (MIDs), and wars. Note that MIDs and wars are of course, the opposite of military cooperation; for these latent measures we reverse the scale to account for this.  We explain how we constructed these measures of strategic interest in greater detail below while we detail the data sources we relied on in the following section.

The main advantage of using calculating the latent space of different dyadic variables as opposed to using alternative specifications such as the \textit{S Score} algorithm is that we are consequently able to account for third order dependencies within the data. To review, first order dependency refers the propensity for some actors to send or recieve more ties than others, second-order dependency refers to reciprocity of exchange between actors while a third-order dependency refers to interaction among three or more actors. Dyadic data are rife with these types of dependencies, and aside from first-order dependency, they pose serious challenges the basic assumption of independence between observations. 

In particular, third order dependency includes the concepts of (i) transitivity, (ii) balance and (iii) clusterability. Formally, a triad $ijk$ is said to be transitive if for whenver $y_{ij} = 1$ and $y_{jk} = 1$, we also observe that $y_{ik}$. This follows the logic of `` a friend of a friend is a friend''. Meanwhile,  a triad $ijk$ is said to be balanced if $y_{ij} \times y_{jk} \times y_{ki} >0$. Conceptually, if the relatoionship between $i$ and $j$ is `positive', then both will relate to another unit $k$ identically, either both positive or both negative. Finally a triad $ijk$ is said to be cluserable if it is balanced or all the relations are all negative. It is a relaxation of the concept of balance and seeks to capture groups where the measurements are positive within groups and negative between groups. For a more detailed explanation of these concepts, see \citet{hoff:2002,hoff:2004,hoff:2005}. \\

In other words, third order dependencies suggest that ``knowing something about the relationship between $i$ and $j$ as well as between $i$ and $k$ may reveal something about the relationship between $i$ and $k$, even when we do not directly observe it'' \citep{hoff:2004}. Such a dependency is especially important to capture with regards to strategic relationships as dyadic relationships between two particular countries cannot help but be understood in the context of their relationship with other countries. \\

Following \citep{hoff:2005}, we run a null generalized bilinear mixed effects model (gbme) \footnote{Code for running the gbme can be found from Hoff's website at \url{http://www.stat.washington.edu/hoff/Code/hoff_2005_jasa/} } to estimate the latent space for each component of our strategic interest variables. Below we show a visualization for each component for the year 2000.


After estimating the latent spaces for these components, we then combine them in a principal components analysis to reduce the dimensionality of our measure while retaining as much variance as possible. That is, for example, alliances, UN voting and joint membership in IGOs all capture certain aspects of political strategic interest, and instead of choosing only one of them as our measure of strategic interest as other papers have done, we combine them in order to increase our explanatory power.\\

It is possible of course, to do a PCA on all different aspects of strategic interest --- alliances, UN voting, joint IGO membership, arms transfers, mids and wars --- together, and then run analyses on the largest components of the PCA. As discussed before, while we argue that political and military strategic interest are qualititatively different, we do acknowledge that both can inform each other. While we considered employing this approach, we decided to make the tradeoff for better interpretability of our measure over increased precision of our strategic interest measure. The interpretation of different components of a PCA measure is generally not straightforward. For example, we could end up with a first principal component that is explained by alliances \%50 of the time, IGOs \%40 of the time, arms transfers \% 5 of the time and the rest of the components a combined \%5 of the time and a second component that is explained by  MIDS \%60 of the time, alliances 30\% of the time, and the rest of the components a combined \%10 of the time. While we may be able to say that strategic interest matters, it would be more difficult to say in what way. In separating out the variables before hand for theoretical reasons, we increase the interpretability of any of our subsequent results while sacrificing a bit of explanatory power. At the same time, whatever results we do find should represent a harder test for the importance of political or military strategic interest because of this sacrifice.\\

We also conduct a series of post-estimatation validation tests for our resulting strategic variables  








