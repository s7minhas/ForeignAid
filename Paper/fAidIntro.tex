\section*{Introduction}
 

\indent Foreign aid describes the transfer of resources from one government to another. Although the very term suggests a humanitarian motive, scholars and experts have long debated whether it would be more accurate to ascribe foreign aid a strategic motive instead. With some exceptions \citep{bermeo:2008}, most scholars have found that donors prioritize strategic considerations when dispensing aid \citep{alesina:2000, berthelemy:2006}. \\
\indent This seeming consensus belies the fact that how scholars conceputalize and measure strategic considerations is quite inconsistent from paper to paper. Scholars have variously used bilateral trade intensity \citep{berthelemy:2006, berthelemy:2004}, colonial legacies \citep{berthelemy:2004},  UN voting \citep{alesina:2000}, political orientation of the recipient country \citep{easterly:2008} to measure strategic interest. Other scholars take a different approach and investigate how much aid allocations can be ascribed to humanitarian reasons. These are broadly split along economic need \citep{collier:2002,nunnenkamp:2006,thiele:2007} and the quality of political governance \citep{neumayer:2005,dollar:2006}, the implication being that countries that fail to give aid along these criterion are acting in their strategic interest.
\indent These measures are at best imperfect and at worst, uninterpretable. As \citept{bermo:2008} states,
\begin{quote} `Perhaps the most puzzling conclusion of the existing literature is that a focus on trade partners, former colonies, and allies is somehow evidence against a development focus of aid. Instead, one could interpret this as evidence that donors give aid to the countries in which they most wish to pursue development. In this sense, donor interests and recipient needs are not mutually exclusive categories.'
\end{quote} Meanwhile some have argued that donors who give to poor countries may not do it out of a humanitarian impulse but because it is cheaper to buy interest in poorer countries \citep{demesquita:2007,stone:2006}. Conversely, a donor country may give to a poorly governed, undemocratic country for humanitarian reasons as well, North Korea being a prominent example.
\indent  The lack of coherence in evaluating strategic interest extends to model specification. Papers which have empirically evaluated the dominance of strategic over humanitarian motives with some exceptions \citep{berthelemy:2006}, have done so by specifying models which pool all donors together or by running models for each donor country separately. We find this empirical choice puzzling - if foreign aid is indeed given for strategic reasons then surely a donor country should account for the foreign aid given by other countries when making their own allocations. The same should be equally true if foreign aid is given for humanitarian reasons - if a very needy country is already recieving an abundant amount of foreign aid from other countries, a particular donor country may decide to dispense aid to a less needy but overlooked recipient country. Pooled models do not address this issue as they do not distinguish between donor countries while donor by donor regressions cannot address this issue because by construction they do not account for the allocations of other donor countries. \\

In this paper, we seek to more definitevely evaluate the prominence of a strategic or humanitarian motive for foreign aid. We do so by creating a new measure of strategic interest by measuring the latent strategic distance between countries \citep{hoff:2002}. Such a measure improves upon existing measures of strategic interest in that the latent measure is fundamentally a network measure which accounts for third-party relationships. In our model specification, we also use a hierarchical random effects model with panel data\footnote{We can go into greater detail and talk about how it is also a zero-inflated model later in the model specification section, what do you think Shahryar?} to account for the possibility that foreign aid given by one donor is not given without consideration of allocations by other donors. In doing so, we are able to model both the variation that is common among donor countries as well as that which is specific to a particular donor country, combining the best of what a pooled regression or donor by donor regressions can offer. 
\indent When we do so, we find that.... [INCLUDE FINDINGS HERE] Moreover, onor by donor regressions have found that there is wide variations of motivations in allocating foreign aid among donors, we find that when we consider the donor countries together that.... [INCLUDE FINDINGS HERE]. 
In what follows, we first give a brief overview of the literature before blah blah blah. 



\label{intro}