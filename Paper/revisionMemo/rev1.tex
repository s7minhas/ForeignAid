\section*{Reviewer 1}

\subsection*{Major Comments}

\begin{enumerate}
	\item I think that this paper's primary finding—that donors increase aid to opponents but not to allies in the aftermath of natural disaster—is an important one. Both the consideration of the (vast) extant literature on the strategic motivations for aid allocation and the novel theorizing by the authors are done carefully and well. I also appreciate that the finding is supported by a wealth of anecdotal examples.
	\begin{itemize}
		\item \textcolor{blue}{ \emph{
		Insert great response.
		}}
	\end{itemize}	
	\item I really like the whole theory section of the paper, especially the careful reading of the strategic environment in aid and the naïve vs. more strategic reading of why donors may distinguish between allies and opponents in humanitarian aid-giving. I think this moves the literature forward in a productive way.
	\begin{itemize}
		\item \textcolor{blue}{ \emph{
		Insert great response.
		}}
	\end{itemize}
	\item Theorizing civil society aid: I think that the authors have a choice to make with regards to the discussion of civil society aid. On the one hand, I think that they could group civil society aid with development aid and essentially test humanitarian vs. all other kinds of aid. This would still allow them to both theorize and test the main relationship of the paper—that between strategic alliances and humanitarian aid. On the other hand, if they want to maintain the three-part distinction in types of aid, then there is more theoretical and empirical work to be done in teasing out civil society aid. I note some of the empirical issues in 4b-c. On theorizing:
	\begin{itemize}
		\item The authors state that civil society aid is "to empower grass-roots advocacy and improve governance and government accountability." This may be true, but there can also be a more strategic logic, especially in strategic opponents. For example, when the U.S. allocates civil society aid to Russia, much of it goes to pro-democracy groups or support for freedom of the press, which does support civil society but is also anti-regime. In this case, civil society aid may be about laying the groundwork for long-term change within strategic allies. It is worth looking into some specific civil society projects that happen after natural disasters to support the logic in the paper that natural disasters serve as an entrance into domestic politics within strategic opponents.
		\begin{itemize}
			\item \textcolor{blue}{ \emph{
			Insert great response.
			}}
		\end{itemize}		
		\item In general, the authors need to be more clear about how civil society aid is defined and how it is strategically different from development aid.
		\begin{itemize}
			\item \textcolor{blue}{ \emph{
			Insert great response.
			}}
		\end{itemize}		
	\end{itemize}
	\item Types of aid: The empirical tests rely on distinguishing between humanitarian, civil society, and development aid. I have several outstanding questions about these distinctions.
	\begin{itemize}
		\item Strategic labeling of humanitarian aid: I would like the authors to consider the possibility that there is a strategic logic to how donors label aid, which may vary between strategic allies and opponents. For example, as illustrated nicely in the authors' Iran example, in order for the U.S. to allocate any aid to Iran, it was necessary to create new aid levers outside of the normal aid bureaucracies and allocation processes. This was true in this specific case because of economic sanctions, but it may also be true in order to generate public appetite for aid going to strategic opponents. With a strategic ally, there are already preexisting development aid channels and it may be more possible bureaucratically to send resources through those channels (without needing a distinct aid category of humanitarian aid) to provide post-disaster support or to enable allied recipients to reallocate, say, budget support in the health sector to disaster relief (and allies tend to receive more fungible forms of aid in the first place). The humanitarian aid classification may thus be more necessary with strategic opponents compared to strategic allies. One way to check this would be to see whether some of the types of aid within development aid increase for strategic allies in the aftermath of natural disaster—for example, do donors allocate more to food aid (but maybe less to other sectors, so the net effect is zero?). This would still indicate humanitarian support for allies, just through different bureaucratic channels. 
		\begin{itemize}
			\item \textcolor{blue}{ \emph{
			Insert great response.
			}}
		\end{itemize}		
		\item Why is "women" categorized as civil society aid? Aid aimed at women's empowerment could as easily be categorized as development aid. It would be useful to understand at a project level what types of projects fall into this category to understand whether "women" is capturing, say, women's political organizations or women's economic empowerment. If it is simply a heterogenous category that doesn't fall neatly into any of the authors' categories, then they could re-run the models classifying "women" as civil society vs. development and see if it makes a difference. It is also worth noting that women are disproportionately affected by natural disasters, and there could be a humanitarian logic behind increasing women's programs in the years after a natural disaster.	
		\begin{itemize}
			\item \textcolor{blue}{ \emph{
			Insert great response.
			}}
		\end{itemize}		
		\item There seem to be some missing aid categories. Where do things like governance aid, budget aid, and technical assistance fall? I am particularly wondering whether these categories (especially budget aid) are falling into civil society aid through the "Government and Civil Society" tag, as these are decisively not support for civil society. How does civil society aid relate to Dietrich's notion of bypass aid? Is it always non-governmental?			
		\begin{itemize}
			\item \textcolor{blue}{ \emph{
			These aid categories were deliberately excluded from our analysis precisely because it would be difficult to categorize them along the distinctions that we make when we define humanitarian, civil society and development aid. These excluded aid categories take up around 30\% of the sum of total aid over the time period under consideration in this paper. 
			}}
		\end{itemize}				
	\end{itemize}
	\item  Empirics: The models rely on testing the interaction term between strategic proximity and the number of natural disasters.
	\begin{itemize}
		\item It would be good to see in an Appendix the factor analysis used to calculate the strategic proximity variable as well as the summary statistics on this variable.
		\begin{itemize}
			\item \textcolor{blue}{ \emph{
			Insert great response.
			}}
		\end{itemize}						
		\item I would like to see the models re-run using a dummy variable for whether a natural disaster occurred at all rather than the number of natural disasters. I don't see how the number of natural disasters affects the strategic calculus of whether to respond with humanitarian aid, especially since the number of disasters has little to do with their scale. Using a dummy variable would ease interpretation of the interaction terms and their constituent terms. It also seems more consistent with the authors' argument: they argue that natural disasters are a "shock" which prompts donors to respond based on varying levels of strategic alliance.
		\begin{itemize}
			\item \textcolor{blue}{ \emph{
			Insert great response.
			}}
		\end{itemize}						
		\item  If the authors do think that the scale of the natural disaster matters for the response, then using the number of deaths seems more appropriate that the number of disasters, since one large disaster could be far more damaging than five small ones.
		\begin{itemize}
			\item \textcolor{blue}{ \emph{
			Insert great response.
			}}
		\end{itemize}						
	\end{itemize}
\end{enumerate}

\subsection*{Minor Comments}

\begin{enumerate}
	\item The legend on Figure 1 did not come out clearly, and the different categories of aid cannot be easily distinguished.
	\begin{itemize}
		\item \textcolor{blue}{ \emph{
		Insert great response.
		}}
	\end{itemize}	
	\item I think H3 is phrased the opposite of what the authors intended.
	\begin{itemize}
		\item \textcolor{blue}{ \emph{
		H3 was indeed phrased the opposite of what we intended and we have since fixed this unfortunate oversight. 
		}}
	\end{itemize}	
\end{enumerate}