\section*{Reviewer 1}

\subsection*{Major Comments}

\begin{enumerate}
	\item I think that this paper's primary finding—that donors increase aid to opponents but not to allies in the aftermath of natural disaster—is an important one. Both the consideration of the (vast) extant literature on the strategic motivations for aid allocation and the novel theorizing by the authors are done carefully and well. I also appreciate that the finding is supported by a wealth of anecdotal examples.
	\begin{itemize}
		\item \textcolor{blue}{ \emph{
		Thank you for this comment!
		}}
	\end{itemize}	
	\item I really like the whole theory section of the paper, especially the careful reading of the strategic environment in aid and the naive vs. more strategic reading of why donors may distinguish between allies and opponents in humanitarian aid-giving. I think this moves the literature forward in a productive way.
	\begin{itemize}
		\item \textcolor{blue}{ \emph{
		Thank you for this comment!
		}}
	\end{itemize}
	\item Theorizing civil society aid: I think that the authors have a choice to make with regards to the discussion of civil society aid. On the one hand, I think that they could group civil society aid with development aid and essentially test humanitarian vs. all other kinds of aid. This would still allow them to both theorize and test the main relationship of the paper—that between strategic alliances and humanitarian aid. On the other hand, if they want to maintain the three-part distinction in types of aid, then there is more theoretical and empirical work to be done in teasing out civil society aid. I note some of the empirical issues in 4b-c. On theorizing:
	\begin{itemize}
		\item The authors state that civil society aid is ``to empower grass-roots advocacy and improve governance and government accountability.'' This may be true, but there can also be a more strategic logic, especially in strategic opponents. For example, when the U.S. allocates civil society aid to Russia, much of it goes to pro-democracy groups or support for freedom of the press, which does support civil society but is also anti-regime. In this case, civil society aid may be about laying the groundwork for long-term change within strategic allies. It is worth looking into some specific civil society projects that happen after natural disasters to support the logic in the paper that natural disasters serve as an entrance into domestic politics within strategic opponents. In general, the authors need to be more clear about how civil society aid is defined and how it is strategically different from development aid.
		\begin{itemize}
			\item \textcolor{blue}{ \emph{
			Thanks for pointing out where our theory needed further clarification. We absolutely agree with the reviewer that civil society aid can in fact be pursued for more strategic purposes; this was in fact the point that we were trying to make in our paper. We have hopefully since clarified our language to better reflect this. We have also tried to make a clearer distinction between civil society and development aid. Following the reviewer's suggestion, we have also identified some suggestive anecdotal evidence that this may in fact be occurring. We hope that the distinction that we make between civil society and development aid (in brief, supporting civil society has a much higher ability to affect domestic politics, which makes its strategic value much higher than development aid) makes it clearer why testing these two aid categories separately might be fruitful.  
			}}
		\end{itemize}		
	
	\end{itemize}
	\item Types of aid: The empirical tests rely on distinguishing between humanitarian, civil society, and development aid. I have several outstanding questions about these distinctions.
	\begin{itemize}
		\item Strategic labeling of humanitarian aid: I would like the authors to consider the possibility that there is a strategic logic to how donors label aid, which may vary between strategic allies and opponents. For example, as illustrated nicely in the authors' Iran example, in order for the U.S. to allocate any aid to Iran, it was necessary to create new aid levers outside of the normal aid bureaucracies and allocation processes. This was true in this specific case because of economic sanctions, but it may also be true in order to generate public appetite for aid going to strategic opponents. With a strategic ally, there are already preexisting development aid channels and it may be more possible bureaucratically to send resources through those channels (without needing a distinct aid category of humanitarian aid) to provide post-disaster support or to enable allied recipients to reallocate, say, budget support in the health sector to disaster relief (and allies tend to receive more fungible forms of aid in the first place). The humanitarian aid classification may thus be more necessary with strategic opponents compared to strategic allies. One way to check this would be to see whether some of the types of aid within development aid increase for strategic allies in the aftermath of natural disaster—for example, do donors allocate more to food aid (but maybe less to other sectors, so the net effect is zero?). This would still indicate humanitarian support for allies, just through different bureaucratic channels. 
		\begin{itemize}
	    \item \textcolor{blue}{ \emph{
		We have taken up your suggestion and explored the possibility of strategic labelling as illustrated through Figure ~\ref{fig:aidtype}. Here we break down how much development aid was given to countries experiencing that experienced either 0 disasters or 1-3 disasters across different types of development aid. The comparison between 0 or 1-3 disasters was used to maximized comparability, as around 40 percent of the country-years in the dataset had 0 disasters, while 43 percent experienced 1-3 disasters. This figure suggests that i) countries that are strategic allies (located at low levels of strategic distance) are more likely to get more aid related to economic infrastructure and services while ii) countries that experience disasters are much more likely to get debt relief when the they are strategic opponents (that is at high levels of strategic distance), compared to countries that experience 0 natural disasters. However, while Figure ~\ref{fig:aidtype} does seem to be consistent the reviewer's hunch that it may be easier to distribute different types of aid depending on whether one is a strategic opponent or strategic ally, there does not seem to be much in the way of strategic labelling going on. That is,  the additional aid for economic infrastructure services to strategic allies and the additional aid for debt relief appear to be given in on top of existing levels of aid; neither appear to be offsetting other types of development aid. 
		}}
					
	\begin{figure}[h!]
	\centering
	\includegraphics[width = .9\textwidth]{graphics/developmentAidValueByType.pdf}
	\caption{Value of Aid Commitments categorized by type of development aid and by the number of disasters}
			
			\label{fig:aidtype}
				\end{figure}
				\FloatBarrier
				
			\end{itemize}	
		\item Why is ``women'' categorized as civil society aid? Aid aimed at women's empowerment could as easily be categorized as development aid. It would be useful to understand at a project level what types of projects fall into this category to understand whether ``women'' is capturing, say, women's political organizations or women's economic empowerment. If it is simply a heterogenous category that doesn't fall neatly into any of the authors' categories, then they could re-run the models classifying ``women'' as civil society vs. development and see if it makes a difference. It is also worth noting that women are disproportionately affected by natural disasters, and there could be a humanitarian logic behind increasing women's programs in the years after a natural disaster.	
		\begin{itemize}
			\item \textcolor{blue}{ \emph{
			Reviewer 1's comments cuts to the tension between donor motivations and aid outcomes that we seek to distinguish between in this paper. Indeed, civil society aid as a whole has often been understood as not only promoting civil society for it's intrinsic sake but as an instrumental mechanism to bring about development \citep{van:2012,pearce:2002}. For that matter, humanitarian aid can also be understood as responses to acute crises that are necessary for laying the foundation for longer term developmental outcomes. Given the messy link between motivations for aid and aid outcomes that plague the literature more generally however, we choose in this paper to focus only on donor motivations. To that end, we code aid given to support women as civil society aid however because they are targeted toward promoting women's rights and gender equality, which are commonly accepted to form an important facet of civil society \citep{esplen:2016}. Nevertheless, we do acknowledge Reviewer 1's argument that it is possible that aid given to support women could have more direct developmental outcomes than what we had originally imagined when we coded this category under civil society aid. However, the overall substantive impact of this decision should be negligible as aid coded as being given to women takes up 0.1\% of the total amount of aid considered in this dataset and 3.6\% of the aid coded as `civil society aid' in this dataset. As such, recoding this variable is unlikely to affect our findings.  
			}}
		\end{itemize}		
		\item There seem to be some missing aid categories. Where do things like governance aid, budget aid, and technical assistance fall? I am particularly wondering whether these categories (especially budget aid) are falling into civil society aid through the ``Government and Civil Society'' tag, as these are decisively not support for civil society. How does civil society aid relate to Dietrich's notion of bypass aid? Is it always non-governmental?			
		\begin{itemize}
			\item \textcolor{blue}{ \emph{
		These aid categories were deliberately excluded from our analysis precisely because it would be difficult to categorize them along the distinctions that we make when we define humanitarian, civil society and development aid. These excluded aid categories take up around 30\% of the sum of total aid over the time period under consideration in this paper. 
			}}
		\end{itemize}				
	\end{itemize}
	\item  Empirics: The models rely on testing the interaction term between strategic proximity and the number of natural disasters.
	\begin{itemize}
		\item It would be good to see in an Appendix the factor analysis used to calculate the strategic proximity variable as well as the summary statistics on this variable.
		\begin{itemize}
			\item \textcolor{blue}{ \emph{
			Sure! Please see the Appendix in the revised paper. 
			}}
		\end{itemize}						
		\item I would like to see the models re-run using a dummy variable for whether a natural disaster occurred at all rather than the number of natural disasters. I don't see how the number of natural disasters affects the strategic calculus of whether to respond with humanitarian aid, especially since the number of disasters has little to do with their scale. Using a dummy variable would ease interpretation of the interaction terms and their constituent terms. It also seems more consistent with the authors' argument: they argue that natural disasters are a ``shock'' which prompts donors to respond based on varying levels of strategic alliance.
		\begin{itemize}
			\item \textcolor{blue}{ \emph{
			We have rerun the analysis using a dummy variable for whether a natural disaster occurred instead of a count of natural disasters. We show the substantive results of this analysis below in Figure~\ref{fig:binaryDisasterSimulation}. The findings from this analysis reflect those that we observe when we use the count variable. We have included these revised results in the appendix of our paper. 
			\item We choose to include these results in the appendix rather than as part of the main analysis in our paper because the relative lack of variation in the binary variable makes it impossible to present how aid is distributed differently to strategic allies vs. strategic opponents over increasing intensities of natural disasters. 
			}}
			
            \begin{figure}[h!]
            	\centering
            	\includegraphics[width=.9\textwidth]{graphics/simComboPlot_bin_disaster.pdf}
            	\caption{Simulated substantive effect plots for development aid for varying lags of variables of interest and whether or not a recipient country experienced a natural disaster across the range of the strategic distance measure.}
            	\label{fig:binaryDisasterSimulation}
            \end{figure}			
            \FloatBarrier
			
		\end{itemize}						
		\item  If the authors do think that the scale of the natural disaster matters for the response, then using the number of deaths seems more appropriate that the number of disasters, since one large disaster could be far more damaging than five small ones.
		\begin{itemize}
		\item \textcolor{blue}{ \emph{ We have rerun the analysis using the number killed from a natural disaster instead of a count of the number of natural disasters. We show the substantive results of this analysis below. }}
        \begin{figure}[h!]
        	\centering
        	\includegraphics[width=1\textwidth]{graphics/simComboPlot_no_killed.pdf}
        	\caption{Simulated substantive effect plots for development aid for varying lags of variables of interest and different levels of natural disaster severity (specifically, the log of the number killed) across the range of the strategic distance measure.}
        	\label{fig:devIntCoef}
        \end{figure}	
        \FloatBarrier		
		\item \textcolor{blue}{ \emph{ The substantive trends with respect to humanitarian aid and development aid are notably similar to results that rely on a count of the number of natural disasters. There is a difference, however, with respect to the finding for the civil society aid dependent variable. In our analysis with the count of the number of natural disasters we saw that at higher counts of natural disasters the slope between the amount of civil society aid given and strategic distance became positive. Here we see a less pronounced change in the slope between strategic distance when there are a higher number of deaths. This is perhaps explained by the fact that this measure has a missingness rate of 10.8\%. }}
		\item \textcolor{blue}{ \emph{ With regards to other potential measures, the EM-DAT database provides the data on number people injured, homeless, or affected and the dollar amount of the disaster. However such data has a high degree of missingness and, by their own admission, frequently imprecise or under-reported. For instance there is 79\% missingness for the number of injured, 36\% missingness for the total number of homeless and 33\% for the total damages. The number of affected has comparatively less missingness, with 9.6\%, however the EM-DAT Gudelines note that, ``The indicator affected is often reported and is widely used by different actors to convey the extent, impact, or severity of a disaster in non-spatial terms.  The ambiguity in the definitions and the different criteria and methods of estimation produce vastly different numbers, which are rarely comparable.'' Generally all the indicators have varying degrees of imprecision. For instance, the guidelines further state, ``Any related word like 'hospitalized' is considered as injured. If there is no precise number is given, such as 'hundreds of injured', 200 injured will be entered (although it is probably underestimated).'' Given these problems with these other potential measures, we decided to focus on the number of disasters as our measure of disaster intensity.}}
		\end{itemize}						
	\end{itemize}
\end{enumerate}

\subsection*{Minor Comments}

\begin{enumerate}
	\item The legend on Figure 1 did not come out clearly, and the different categories of aid cannot be easily distinguished.
	\begin{itemize}
		\item \textcolor{blue}{ \emph{
		Thanks for pointing this out! We have fixed the legend so that it is more legible.
		}}
	\end{itemize}	
	\item I think H3 is phrased the opposite of what the authors intended.
	\begin{itemize}
		\item \textcolor{blue}{ \emph{
		H3 was indeed phrased the opposite of what we intended and we have since fixed this unfortunate oversight. 
		}}
	\end{itemize}	
\end{enumerate}