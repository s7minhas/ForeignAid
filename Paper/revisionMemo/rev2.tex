\section*{Reviewer 2}

\subsection*{Major Comments}

\begin{enumerate}
	\item The manuscript asks whether donors respond differently to natural disasters in strategic allies and adversaries. In particular, the authors put forth hypotheses across three different types of aid - humanitarian, civil society and development - and posit that an interaction will exist between "strategic distance" and disasters. The idea is sound and worth testing, particularly with regard to civil society aid, but the execution is not as good as it could be and the paper feels dated in both literature review and time covered. I would strongly suggest reframing the paper to focus more on the civil society findings (as well as the humanitarian findings), updating the analysis and literature review, and paying additional attention to some empirical difficulties. I'll say some more on each of these.
	\begin{itemize}
		\item \textcolor{blue}{ \emph{
		Insert great response.
		}}
	\end{itemize}	
	\item I thought the most interesting finding is that donors may be using a humanitarian disaster to "sneak" civil society aid to groups in countries not aligned with themselves - if this holds up it is a really neat finding. It also suggests that recipients are right to be worried about donors having multiple purposes when responding to humanitarian crises. I would highlight the importance of this more, as it would be the newest finding in the paper. The North Korean example regarding "changing hearts and minds" could play into this - it is a clear example of influencing the opinion of the people, rather than the government, toward the donor.
	\begin{itemize}
		\item \textcolor{blue}{ \emph{
		Insert great response.
		}}
	\end{itemize}
	\item Breaking H1 into three parts (the middle one clearly a straw man) is not helpful. Also on framing I recommend removing the multiple anecdotes of disasters in wealthy states in the front of the paper - these have nothing to do with interactions between aid donors and recipients, they are confusing in the context of the questions asked in the paper, and removing them would streamline the argument.
	\begin{itemize}
		\item \textcolor{blue}{ \emph{
		Insert great response.
		}}
	\end{itemize}
	\item I believe H3 is stated exactly the opposite of what the authors intended to say. Also, it does not necessarily seem consistent with H1A/C- if states experience a lot of infrastructure destruction then much of the disaster aid may be designated to rebuild infrastructure. In this way it seems like a longer-term disaster response, and then the same logic that plays out in H1 might play out here too.
	\begin{itemize}
		\item \textcolor{blue}{ \emph{
		Insert great response.
		}}
	\end{itemize}
	\item  The literature review is dated. In particular, it misses some key contributions from recent years regarding donor intent and foreign aid. Most notable are Bueno de Mesquita and Smith (2009, 2015); Fleck and Kilby (2010); Clist (2011); Bermeo (2017, 2018). Multiple of these studies note the importance of considering changes in donor intent and behavior across three periods (Cold War, 1990s, post-2001), which the authors should certainly test in their empirics. Carter and Stone (2015) have written the definitive piece on UN voting and aid, which should certainly be referenced.
	\begin{itemize}
		\item \textcolor{blue}{ \emph{
		Insert great response.
		}}
	\end{itemize}						
	\item A key contribution the authors could be making is on the measure of strategic difference, using network analysis and combining information from three variables - UN voting, alliances, and membership in IGOs. It is difficult to assess the suitability of this with the information given. We don't know which IGOs were included in developing this measure. There is updated data for UN voting (Bailey, Strezhnev, and Voeten, 2017).
	\begin{itemize}
		\item \textcolor{blue}{ \emph{
		Insert great response.
		}}
	\end{itemize}	
	\item On the empirical setup, it would be nice to see results on total aid in addition to the results by category. Do disasters shift total levels of aid and does this interact with strategic distance? This would allow us to see if increases in some categories are happening at the expense of other categories. It would also provide a nice backdrop to show how the new measure of strategic distance performs in models similar to those that have used other measures (which the authors find sub-optimal). Does the measure make a difference?
	\begin{itemize}
		\item \textcolor{blue}{ \emph{
		Insert great response.
		}}
	\end{itemize}	
	\item Have the authors considered threshold effects? It is possible that there are discontinuities - perhaps for recipients that are either really close or really far from the donor in terms of strategic distance are treated differently but the many countries in the middle see no impact for small changes in strategic distance, regardless of disasters.		
	\begin{itemize}
		\item \textcolor{blue}{ \emph{
		Insert great response.
		}}
	\end{itemize}	
	\item The period of analysis is cause for concern. First, it is not really a good idea to start in 1975 using disaggregated aid data. Countries were not required to report the purposes of aid for earlier years and before 1995 there was significant lack of reporting by category and it was not uniform across donors. Some donors almost never marked the sector of aid (just reported the total amount by recipient) and others did report. This is particularly problematic in AidData (which the authors use), since this source only includes aid that is reported at the project level. So large sums of aid are excluded from AidData in earlier years because donors did not code its purpose. This makes it impossible to create meaningful categories for humanitarian, civil society, and development aid, since donors did not distinguish across types.
	\begin{itemize}
		\item \textcolor{blue}{ \emph{
		Insert great response.
		}}
	\end{itemize}		
	\item It is also problematic to end the analysis in 2006. Why exclude ten years of more recent data? It can be particularly problematic to do so since multiple studies have shown that patterns in aid giving vary across the Cold War, 1990s, and post-2001 period. This analysis is swamped by Cold War years and may not hold for the more recent periods.
	\begin{itemize}
		\item \textcolor{blue}{ \emph{
		Insert great response.
		}}
	\end{itemize}	
	\item The count of natural disasters seems like the wrong measure of disaster intensity. A measure of number of people affected or dollar value of damages would better measure need in the wake of a disaster.
	\begin{itemize}
		\item \textcolor{blue}{ \emph{
		Insert great response.
		}}
	\end{itemize}	
	\item Donor and year fixed effects would be more in-line with the theory and existing literature, rather than donor and recipient random effects. The theory would imply that within a donor in a given year, the donor awards aid differently across recipients. Although there could be within-recipient differences over time for individual donors as well, the need to account for time invariant donor characteristics (while still allowing dyad characteristics to vary over time) suggests that donor fixed effects are worth considering.	
	\begin{itemize}
		\item \textcolor{blue}{ \emph{
		Insert great response.
		}}
	\end{itemize}	
	\item Multiple studies have shown that donors vary the composition of aid based on recipient characteristics. The positive relationship between humanitarian aid and strategic difference even in the absence of a disaster, coupled with the negative relationship between development aid and strategic difference, suggests that donor may simply be giving different types of aid to allies and adversaries. Perhaps they are more worried about going through the government in adversaries and so use aid that is more easily channeled through NGOs (similar to the Dietrich logic on corruption/governance and aid channels). This would suggest that, even absent disasters, donors are giving to both allies and adversaries but doing so differently. The same applies when a disaster strikes: allies and allies may both get more aid (hard to tell from the way it is presented), but for allies it is development aid and for adversaries it is humanitarian aid. The authors should address these patterns and possible implications for their theory.	
	\begin{itemize}
		\item \textcolor{blue}{ \emph{
		Insert great response.
		}}
	\end{itemize}		
\end{enumerate}