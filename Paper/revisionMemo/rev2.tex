\section*{Reviewer 2}

\subsection*{Major Comments}

\begin{enumerate}
	\item The manuscript asks whether donors respond differently to natural disasters in strategic allies and adversaries. In particular, the authors put forth hypotheses across three different types of aid - humanitarian, civil society and development - and posit that an interaction will exist between ``strategic distance'' and disasters. The idea is sound and worth testing, particularly with regard to civil society aid, but the execution is not as good as it could be and the paper feels dated in both literature review and time covered. I would strongly suggest reframing the paper to focus more on the civil society findings (as well as the humanitarian findings), updating the analysis and literature review, and paying additional attention to some empirical difficulties. I'll say some more on each of these.
	\begin{itemize}
		\item \textcolor{blue}{ \emph{
		We thank the reviewer for taking the time to provide her/his comments!
		}}
	\end{itemize}	
	\item I thought the most interesting finding is that donors may be using a humanitarian disaster to ``sneak'' civil society aid to groups in countries not aligned with themselves - if this holds up it is a really neat finding. It also suggests that recipients are right to be worried about donors having multiple purposes when responding to humanitarian crises. I would highlight the importance of this more, as it would be the newest finding in the paper. The North Korean example regarding ``changing hearts and minds'' could play into this - it is a clear example of influencing the opinion of the people, rather than the government, toward the donor.
	\begin{itemize}
		\item \textcolor{blue}{ \emph{
		We largely agree with the reviewer but we also think that the findings in terms of the development and humanitarian aid findings are also quite interesting. Though the development aid findings largely mirror expectations from the literature we still see it as an important validation of our strategic interest measure. Just as important the fact that the humanitarian aid finding has essentially the opposite effect of development aid is also quite interesting and makes an important contribution to the literature. To make our perspective on these findings more clear we have revised the paper to reflect what we see as our contribution here.
		}}
	\end{itemize}
	\item Breaking H1 into three parts (the middle one clearly a straw man) is not helpful. Also on framing I recommend removing the multiple anecdotes of disasters in wealthy states in the front of the paper - these have nothing to do with interactions between aid donors and recipients, they are confusing in the context of the questions asked in the paper, and removing them would streamline the argument.
	\begin{itemize}
		\item \textcolor{blue}{ \emph{
		We would respectfully push back on this comment: We think that making the distinction between H1A and H1C is especially helpful for reasoning through why a finding that shows that donors give even more to strategic opponents is an indication strategic, rather than a humanitarian motivation.  We also think that the middle hypothesis (H1B) is quite valuable given that this is precisely the finding that the previous literature would expect for us to find. Overall, we think that having all three hypotheses explicitly laid out also helps the reader think through our theoretical reasoning as well as gives a better foundation for how to interpret the subsequent empirical findings that we find. That being said, the point on the type of anecdotal evidence that we use is well-taken and we have adapted the anecdotes that we use in our paper to better align with the rest of the paper. 
		}}
	\end{itemize}
	\item I believe H3 is stated exactly the opposite of what the authors intended to say. Also, it does not necessarily seem consistent with H1A/C- if states experience a lot of infrastructure destruction then much of the disaster aid may be designated to rebuild infrastructure. In this way it seems like a longer-term disaster response, and then the same logic that plays out in H1 might play out here too.
	\begin{itemize}
		\item \textcolor{blue}{ \emph{
			H3 was indeed phrased the opposite of what we intended and we have since fixed this unfortunate oversight. 
		}}
	\end{itemize}
	\item  The literature review is dated. In particular, it misses some key contributions from recent years regarding donor intent and foreign aid. Most notable are Bueno de Mesquita and Smith (2009, 2015); Fleck and Kilby (2010); Clist (2011); Bermeo (2017, 2018). Multiple of these studies note the importance of considering changes in donor intent and behavior across three periods (Cold War, 1990s, post-2001), which the authors should certainly test in their empirics. Carter and Stone (2015) have written the definitive piece on UN voting and aid, which should certainly be referenced.
	\begin{itemize}
		\item \textcolor{blue}{ \emph{
		The development of this paper was quite long and we appreciate Reviewer 2 pointing out these later works that we overlooked and have incorporated them into the manuscript. Inspired by this comment, we have also incorporated additional relevant readings from: \citet{andrabi:2017}, \citet{bryant_etal:2018},  \citep{carnegie:2017}, \citet{dreher_etal:2011}, \citep{dreher:fuchs:2015}, \citet{dreher:2018}, \citet{eisensee:2007}, \citet{fuchs:Vadlamannati:2013},  \citet{harmer:2005}, \citet{milner:2013}, \citet{neumayer:2003}, \citet{olsen:2003}, \citet{qian:2015}, \citet{stromberg:2007} which hopefully has made the literature review even more relevent and timely. 
		}}
	\end{itemize}						
	\item A key contribution the authors could be making is on the measure of strategic difference, using network analysis and combining information from three variables - UN voting, alliances, and membership in IGOs. It is difficult to assess the suitability of this with the information given. We don't know which IGOs were included in developing this measure. There is updated data for UN voting \citep{bailey:etal:2017}.
	\begin{itemize}
		\item \textcolor{blue}{ \emph{
		Thank you for this suggestion! We have added a bit more descriptive information on the IGOs included in the COW dataset to help give readers a better idea of the underlying data our analysis is based on. Given the number of IGOs in the data-set, we would run into space constraints if we tried to document each and every one of them. However, we have added a note in the text inviting readers to go to the COW website if they are interested in learning more. With regards to the UN data, note that the constraint that we face is that IGO data only goes up until 2005. 
		\item As the underlying data is updated, however, our approach can be easily used to generate updated measures of strategic interest. If we find that there is significant interest among scholars for our measure, we will then also be happy to regularly update the strategic interest variable and make it available through a website and/or the Harvard dataverse.
		}}
	\end{itemize}	
	\item On the empirical setup, it would be nice to see results on total aid in addition to the results by category. Do disasters shift total levels of aid and does this interact with strategic distance?  
	\begin{itemize}
		\item \textcolor{blue}{ \emph{
		Here we present results for a model on total aid as well. Here we find strong evidence that countries are more likely to give aid to those that are strategically proximate to them and we also find robust support for an interactive relationship between strategic distance and the number of disasters. We certainly find the results for total aid interesting, but our hypotheses are dependent on differentiating between different types of aid  and as such we choose to focus on that in the paper.
		}}
		
        \begin{figure}[h!]
        	\centering
        	\includegraphics[width=1\textwidth]{graphics/totAid_Coef.pdf}
        	\caption{Parameter estimates for models using our three original aid dependent variable and total aid.}
        	\label{fig:devIntCoef}
        \end{figure}
        \FloatBarrier		
		
	\end{itemize}	
	\item Have the authors considered threshold effects? It is possible that there are discontinuities - perhaps for recipients that are either really close or really far from the donor in terms of strategic distance are treated differently but the many countries in the middle see no impact for small changes in strategic distance, regardless of disasters.		
	\begin{itemize}
		\item \textcolor{blue}{ \emph{
		We thank the reviewer for this comment but we have not found evidence for non-linear effects between strategic distance and aid. We tested this by incorporating a squared version of the strategic distance into our specification along with the original term, and found no support for a non-linear relationship. We are happy to provide additional details if requested.
		}}
	\end{itemize}	
	\item The period of analysis is cause for concern. First, it is not really a good idea to start in 1975 using disaggregated aid data. Countries were not required to report the purposes of aid for earlier years and before 1995 there was significant lack of reporting by category and it was not uniform across donors. Some donors almost never marked the sector of aid (just reported the total amount by recipient) and others did report. This is particularly problematic in AidData (which the authors use), since this source only includes aid that is reported at the project level. So large sums of aid are excluded from AidData in earlier years because donors did not code its purpose. This makes it impossible to create meaningful categories for humanitarian, civil society, and development aid, since donors did not distinguish across types.
	\begin{itemize}
		\item \textcolor{blue}{ \emph{
		Note that for the Aid Data version 3.0, which is what we use in the paper, the AidData team themselves code aid projects according to different sectors. We have confirmed this both in terms of the documentation given for the version 3.0 data as well as in terms of the actual data. We double checked this by seeing if the sum of the disaggregated aid categories by purpose code equals the aggregated aid cateogries across purpose codes and found this to be so. As such, there should be no concerns about missing data in this regard. 
		}}
	\end{itemize}		
	\item It is also problematic to end the analysis in 2006. Why exclude ten years of more recent data? It can be particularly problematic to do so since multiple studies have shown that patterns in aid giving vary across the Cold War, 1990s, and post-2001 period. This analysis is swamped by Cold War years and may not hold for the more recent periods.
	\begin{itemize}
		\item \textcolor{blue}{ \emph{
		We agree with Reviewer 2 and ideally we would also have liked to extend the analysis past 2005. However, we face the constraint that the IGO data is simply not available past 2005 which restricts our ability to construct our strategic interest variable and consequently, also restricts our ability to model the relationship between strategic interest and aid. 
		\item To show the potential relevance of our findings for more recent periods we have run our models using only data from the post Cold War period. The results are presented below and mirror the findings presented in the paper. We have included these results in the appendix.
        \begin{figure}[h!]
        	\centering
        	\includegraphics[width=1\textwidth]{graphics/simComboPlot_post_coldwar.pdf}
        	\caption{Simulated substantive effect plots for development aid for varying lags of variables of interest and different levels of natural disaster severity across the range of the strategic distance measure for the post Cold War period.}
        	\label{fig:devIntCoef}
        \end{figure}	
        \FloatBarrier	
		\item Additionally, we also run our models using only data from 2002-2005 (post-2001 period in our sample). The results are presented below and mirror the findings presented in the paper. We have included these results in the appendix as well.
        \begin{figure}[h!]
        	\centering
        	\includegraphics[width=1\textwidth]{graphics/simComboPlot_post_2001.pdf}
        	\caption{Simulated substantive effect plots for development aid for varying lags of variables of interest and different levels of natural disaster severity across the range of the strategic distance measure for 2001-2005.}
        	\label{fig:devIntCoef}
        \end{figure}	
        \FloatBarrier	        
		}}
	\end{itemize}	
	\item The count of natural disasters seems like the wrong measure of disaster intensity. A measure of number of people affected or dollar value of damages would better measure need in the wake of a disaster.
	\begin{itemize}
		\item \textcolor{blue}{ \emph{ We have rerun the analysis using the number killed from a natural disaster instead of a count of the number of natural disasters. We show the substantive results of this analysis below. }}
        \begin{figure}[h!]
        	\centering
        	\includegraphics[width=1\textwidth]{graphics/simComboPlot_no_killed.pdf}
        	\caption{Simulated substantive effect plots for development aid for varying lags of variables of interest and different levels of natural disaster severity (specifically, the log of the number killed) across the range of the strategic distance measure.}
        	\label{fig:devIntCoef}
        \end{figure}	
        \FloatBarrier		
		\item \textcolor{blue}{ \emph{ The substantive trends with respect to humanitarian aid and development aid are notably similar to results that rely on a count of the number of natural disasters. There is a difference, however, with respect to the finding for the civil society aid dependent variable. In our analysis with the count of the number of natural disasters we saw that at higher counts of natural disasters the slope between the amount of civil society aid given and strategic distance became positive. Here we see a less pronounced change in the slope between strategic distance when there are a higher number of deaths. This is perhaps explained by the fact that this measure has a missingness rate of 10.8\%. }}
		\item \textcolor{blue}{ \emph{ With regards to other potential measures, the EM-DAT database provides the data on number people injured, homless, or affected and the dollar amount of the disaster. However such data has a high degree of missingness and, by their own admission, frequently imprecise or under-reported. For instance, there is 79\% missingness for the number of injured, 36\% missingness for the total number of homeless and 33\% for the total damages. The number of affected has comparatively less missingness, with 9.6\%, however the EM-DAT Gudelines note that, ``The indicator affected is often reported and is widely used by different actors to convey the extent, impact, or severity of a disaster in non-spatial terms.  The ambiguity in the definitions and the different criteria and methods of estimation produce vastly different numbers, which are rarely comparable.'' Generally all the indicators have varying degrees of imprecision. For instance, the guidelines further state, ``Any related word like 'hospitalized' is considered as injured. If there is no precise number is given, such as 'hundreds of injured', 200 injured will be entered (although it is probably underestimated).'' Given these problems with these other potential measures, we decided to focus on the number of disasters as our measure of disaster intensity.}}
	\end{itemize}	
	\item Donor and year fixed effects would be more in-line with the theory and existing literature, rather than donor and recipient random effects. The theory would imply that within a donor in a given year, the donor awards aid differently across recipients. Although there could be within-recipient differences over time for individual donors as well, the need to account for time invariant donor characteristics (while still allowing dyad characteristics to vary over time) suggests that donor fixed effects are worth considering.	
	\begin{itemize}
		\item \textcolor{blue}{ \emph{
		We have rerun the analysis using a fixed effects specification and show the results below. The results remain broadly the same.
		\item Additionally, when running a Hausman specification test for our models we fail to reject the null hypothesis at both the 90 and 95\% confidence intervals, providing at least some initial evidence that we are justified in our choice \citep{greene:2008}.
		}}

        \begin{figure}[h!]
        	\centering
        	\includegraphics[width=1\textwidth]{graphics/intCoef_fe_re_compare.pdf}
        	\caption{Comparison between parameter estimates using fixed and random effects.}
        	\label{fig:devIntCoef}
        \end{figure}
        \FloatBarrier
        		
	\end{itemize}	
	\item Multiple studies have shown that donors vary the composition of aid based on recipient characteristics. The positive relationship between humanitarian aid and strategic difference even in the absence of a disaster, coupled with the negative relationship between development aid and strategic difference, suggests that donor may simply be giving different types of aid to allies and adversaries. Perhaps they are more worried about going through the government in adversaries and so use aid that is more easily channeled through NGOs (similar to the Dietrich logic on corruption/governance and aid channels). This would suggest that, even absent disasters, donors are giving to both allies and adversaries but doing so differently. The same applies when a disaster strikes: allies and allies may both get more aid (hard to tell from the way it is presented), but for allies it is development aid and for adversaries it is humanitarian aid. The authors should address these patterns and possible implications for their theory.	
	\begin{itemize}
		\item \textcolor{blue}{ \emph{
		This is indeed a legitimate concern. Following a suggestion made by Reviewer 1, we seek to explore the extent to which this is an issue by looking at whether donors strategically shift the type, that is the label, of aid they dispense within the overall category of development aid \textbf{while keeping the overall level of aid the same}, which would be a cause for concern. This is illustrated in Figure ~\ref{fig:aidtype}. Here we break down how much development aid was given to countries experiencing that experienced either 0 disasters or 1-3 disasters across different types of development aid. The comparison between 0 or 1-3 disasters was used to maximized comparability, as around 40 percent of the country-years in the dataset had 0 disasters, while 43 percent experienced 1-3 disasters. This figure suggests that i) countries that are strategic allies (located at low levels of strategic distance) are more likely to get more aid related to economic infrastructure and services while ii) countries that experience disasters are much more likely to get debt relief when the they are strategic opponents (that is at high levels of strategic distance), compared to countries that experience 0 natural disasters. However, while Figure ~\ref{fig:aidtype} does seem to be consistent the reviewer's hunch that it may be easier to distribute different types of aid depending on whether one is a strategic opponent or strategic ally, there does not seem to be much in the way of strategic labelling going on. That is,  the additional aid for economic infrastructure services to strategic allies and the additional aid for debt relief appear to be given in on top of existing levels of aid; neither appear to be offsetting other types of development aid. 
		}}
					
	\begin{figure}[h!]
	\centering
	\includegraphics[width = .9\textwidth]{graphics/developmentAidValueByType.pdf}
	\caption{Value of Aid Commitments categorized by type of development aid and by the number of disasters}
			
			\label{fig:aidtype}
				\end{figure}
				\FloatBarrier
	
	\end{itemize}		
\end{enumerate}