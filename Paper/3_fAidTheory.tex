
\section*{How Natural Disasters Change the Social Context of Foreign Aid Allocations}
 
We seek to show that social context may matter more than previously thought in foreign aid allocations.  We argue that natural disasters can encourage countries to increase aid to their strategic opponents because such disasters can drastically shift the social context of the relationship.  In so doing, we draw on a rich literature from behavioral economics which shows that at least on the individual level, social context can have a substantial influence on behavior \citep{kahneman:2003,do:2011}. 

To draw an illustrative example, in one experiment, \citet{ariely:2008} finds that lawyers generally refused offers to work for \$30 an hour to help the elderly poor but generally agreed to work pro bono for the same group of people. He argues that the reason for this discrepancy is that in the former situation, the lawyers were generally primed to think about the transaction using market norms, for which \$30 would be much too little. The latter situation however, emphasizes the social aspect of the transaction, and the lawyers agreed to the terms as an act of charity.  If we conceptualize countries as being composed of individuals who may be similarly sensitive to the same dynamics, we might also expect countries to also act differently based on social context. As yet, however, this issue remains unexplored when the relevant actor is a country, not an individual.%[NOTE:  did a quick check and this seems true but let me know if you can think of something]. 

In the context of natural disasters, countries may be motivated to give more aid to their strategic opponents than they otherwise would because the social norms around natural disasters prompt them to act according to a more humanitarian frame of mind. That is, the loss of human life and destruction of infrastructure, which natural disasters provoke can serve to, at least temporarily, emphasize the human aspect of the bilateral relationship as opposed to the  political, economic, and military aspects that generally typifies foreign relations between two countries. That is not to say that natural disasters can always bridge this divide. For example, India and Pakistan have had an uneasy history in accepting aid from each other following natural disasters.\footnote{Ravishankar, Siddharth. `Cooperation between India and Pakistan after Natural Disasters.' \textit{Stimson Center}. 9 January 2015. Accessed September 2017: \url{https://www.stimson.org/content/cooperation-between-india-and-pakistan-after-natural-disasters}} In general, we contend only that natural disasters make it more \textit{likely} that a strategic adversary will contribute aid because they reframe the context of bilateral relations toward empathy and altruism.
%In India's case, this may be as much a function of its historic reluctance to accept \textit{any} international aid following a disaster as it is a function of its uneasy history with Pakistan
% add cautionary remark about the fact that we dont think the relationship goes the other way.

%That is, newspaper coverage of natural disasters tends to emphasize  
%Something from behavorial economics here about how different social contexts can prompt different responses, at least on the individual level? ]
% something about newspaper coverage?

However, it is also possible that countries are motivated to give more aid to their strategic opponents because the temporary suspension in the normal dynamics of the relationship represents a unique opportunity to permanently shift the nature of the relationship. That is, donor countries may recognize all too well that the natural disaster offers an opportunity to shift the terms of their relationship with the affected country and may \textit{strategically} decide to increase their aid in order to improve future relations. Disaster-afflicted countries appear to sensitive to this possibility. In 1999 for example, Venezuela experienced catastrophic flash floods and debris flows in Vargas State, which left as much as 10\% of the Vargas population dead \citep{wieczorek:2001}. U.S. troops helped in the relief efforts by running helicopter rescue missions and working to provide clean water. However, consistent with his antagonism toward U.S. hegemony in the region, President Hugo Chavez declined U.S. assistance in rebuilding a critical highway, saying that while, ``he would accept American equipment if Venezuelan soldiers operated it...he did not want U.S. troops in his country.''\footnote{Brand, Richard. `Chavez assailed on handling of Venezuelan flood disaster.' \textit{The Miami Herald}. 5 August 2001. Accessed September 2017: \url{http://www.latinamericanstudies.org/venezuela/venezuela-disaster.htm}.} Meanwhile, Iran categorically refused any aid from Israel following the 2003 Bam earthquake, though the Israeli government still encouraged its citizens to donate privately.\footnote{Popper, Nathaniel. ``Israelis Help Iran Victims Despite Rebuff.'' \textit{The Forward}. 2 January 2004. Accessed September 2017: \url{http://forward.com/news/6059/israelis-help-iran-victims-despite-rebuff/}} Indeed, even the U.S. first turned down Russian aid for Hurricane Katrina before ultimately accepting it.\footnote{`U.S. accepts Russian Katrina aid.' \textit{UPI}. 2 September 2005. Accessed September 2017. \url{https://www.upi.com/US-accepts-Russian-Katrina-aid/39221125680989/}.}  As such, while natural disasters may not be able to bridge the divide between two countries when the strategic relationship is contentious enough, generally the non-affected party seems to at least make the initial gesture of goodwill.\footnote{Note an exception: following Iran's 2012 earthquake, Israeli did not offer aid stating: ``We offered Iran assistance after earthquakes in the past, but they refused. So this time, we didn't even bother to ask if they're interested. Their refusal was pretty impolite, but we’re not making a big fuss about it'' Note despite its contentious relationship with Iran, Israel does explicitly practice `disaster diplomacy' with other countries including Japan, Haiti, Azerbaijan, Jordan and Turkey (Ahren, Raphael. ``Rebuffed in the past, Israel does not offer Iran earthquake aid.'' \textit{The Times of Israel}. 14 August 2012. Accessed September 2017: \url{https://www.timesofisrael.com/having-been-rejected-by-iran-once-before-israel-does-not-renew-offer-for-earthquake-aid/}).}$^{,}$ \footnote{Regardless of the actual motivation of donors when they give aid to their strategic opponents, there is anecdotal evidence to suggest that aid given under such circumstances can be effective at improving perceptions of the bilateral relationship. For example, in the wake of US and South Korean aid for the North Korean famine, one refugee summarized his reaction to the US Institute for Peace this way: ``We were taught all these years that the South Koreans and Americans were our enemies. Now we see they are trying to feed us. We are wondering who our real enemies are” \citet{natsios:1999}.}

% sm: i'm unclear on this last point
According to the above theoretical reasoning then, donors should be more likely to give aid to strategic opponents that experience natural disasters, either because of humanitarian or strategic purposes. In this paper, we first seek to establish first whether natural disasters can prompt strategic opponents to give more aid, which no paper has previously systematically done. We reserve whether they do so for strategic or altruistic reasons for future research. 

%Natural disasters can encourage countries to increase aid to their strategic opponents because 

%\section*{Determinants of Foreign Aid}
%\label{lit}

%Foreign aid describes the transfer of resources from one government to another. Although the word `aid' suggests that donor governments are motivated to dispense foreign aid to further a developmental or humanitarian objective, researchers have long debated whether strategic considerations might better characterize donor government motivations.
%lumsdaine:1993

% In addition, their results suggest that political or strategic considerations do not play a role in influencing aid decisions.

\section*{Measuring Strategic Relationships}

One reason for evaluating the motivations for aid and not aid outcomes is that aid given for strategic reasons may still further development objectives, albeit incidentally, while aid given for humanitarian reasons may also bring unexpected strategic benefits \citep{maizels:1984}. However, evaluating the motivations for aid is not a straightforward process -- any given aid project may work toward providing assistance to a recipient country as well as strategic benefits to a donor country. %The question to be answered then is what relative consideration is given to donor interest or humanitarian need when making aid allocation decisions.  %At the root of these dichotomies is the suspicion that despite rhetoric to the contrary, foreign aid has been dispersed to address donor interest to a much greater extent than it has been for recipient needs.  \\

Of critical importance to investigating whether strategic considerations (and by extension, the interaction between strategic considerations and humanitarian need) affect foreign aid considerations then is constructing a valid and reliable measure of strategic interest. In our review of the literature of the drivers of foreign aid however, we find that \citet{alesina:2000}'s remark that  ``unfortunately the measurement of what a `strategic interest' is varies from study to study and is occasionally tautological,'' still holds true.  Indeed, strategic interest has alternately been operationalized as: trade intensity \citep{berthelemy:2004,bermeo:2008,hoeffler:2011}, UN voting scores \citep{alesina:2000, alesina:2002,hoeffler:2011,dreher:2012}, arms transfers \citep{maizels:1984}, colonial legacy \citep{alesina:2000, bermeo:2008, berthelemy:2004,berthelemy:2006}, alliances \citep{bermeo:2008,schraeder.etal:1998}, regional dummies \citep{bermeo:2008,berthelemy:2006, maizels:1984}, bilateral dummies \citep{alesina:2000, berthelemy:2004, berthelemy:2006}\footnote{A US-Egypt or US-Israel dummy seems to be the most common instance of a bilateral dummy} or some combination of the above. Meanwhile other papers take a negative approach and argue that any shortfall between what would theoretically be expected from poverty-efficient aid allocation and actual aid allocation \citep{collier:2002,nunnenkamp:2006,thiele:2007}, or similarly between a theoretical allocation based on good governance and actual aid allocation\citep{dollar:2006,neumayer:2005}, is evidence of strategic interest at play. 

%However, as argued in the literature review, previous papers have been inconsistent in how they have measured strategic interest, which in turn produces incoherence as to what exactly is being measured.

Such inconsistency in the operationalization of strategic interest is not simply a matter of using different variables to measure the same concept but a matter of using different variables to measure different \textit{aspects} of the underlying concept. However, while a dyad's strategic bilateral relationship is quite multifaceted, to date, there has not been a readily available measure of strategic relationships which captures its various aspects the same way that scholars have done for other complex concepts.\footnote{For example, Polity and Freedom House have provided measures or political institutions while the World Bank's World Governance Indicators (WGI) project provides measures for six dimensions of governance} To address this problem, we create a new measure of strategic interest that is able to capture different aspects of strategic interest into one variable, which we discuss more fully in the next section.  

% surely nobody is arguing that they are conceptually equivalent in the same way as Polity and Freedom House.  How one \textit{measures} strategic relationships  then, is essential to evaluating the relative importance countries may accord strategic motives when dispensing aid. It is not simply a matter of using different data to measure the same concept but of using different data to measure different aspects of a concept. 

%The most relevant research to date has been concerned with how to measure foreign policy similarity, starting with \citet{demesquita:1975}'s Kendall $\tau_b$ measure followed by \citet{signorino:1999}'s S Scores, with new work continually being done in this arena \citep{gartzke:2006, hage:2011,dorazio:2012}). However, foreign policy similarity arguably only captures the political dimension of strategic relationships, equally relevant is active military cooperation between two countries.

% While each of these different measures captures different aspects of strategic interest, missing from each of these measures  However, missing from both the individual and panel level models is the extent to which the ODA recipients receive from other donor countries affects foreign aid allocations.

% `In addition to the heterogeneity bias that may arise in these cross-sectional studies,' there is also a problem in that they implicitly assume that when a donor makes its
%However, missing from both the individual and panel level models  

%\indent\indent  While military cooperation certainly has political dimensions, we would argue that it should be considered a separate aspect of a strategic relationship rather than an subset of political strategic relationship. That is, military security is set apart by its capacity to affect a country's security in a manner that is more immediate, concrete and unilateral than other security concerns across countries more generally, as compared to, for example, access to natural resources, humanitarian sanctions or environmental policy. Though military cooperation can certainly be mediated in the political arena, it is qualitatively different -- that is it is one thing to jointly condemn the various atrocities of the North Korean government, it is quite another thing to take joint military action against it.

\subsection*{A new measure of strategic relationships}

% The sociomatrices described in the last section summarize the relations among all actors in any given country-year. As mentioned above, one way to think about them is that they provide information on the interactions between the N(N − 1) dyads of domestic actors. A simpler way to think about them is that they describe a single network. Analyzing the sociomatrices using network techniques allows me to reduce the information provided by the event data into a low-dimensional space.

%Meanwhile our military strategic relationship measure is the first principal component of the PCA that results from the latent distance between dyadic arms transfers, militarized interstate disputes (MIDs), and wars. Note that MIDs and wars are of course, the opposite of military cooperation; for these latent measures we reverse the scale to account for this. 

Our measure of strategic relationships introduces greater coherency to the literature by providing a more rigorous measure that captures two aspects of strategic interest strategic: military and political. We do so by first measuring the latent space of three different dyadic variables: dyadic alliances, UN voting and joint membership in an intergovernmental organizations (IGOs). Each of these variables captures to different degrees, the political and military relationship between two countries, with dyadic alliances better capturing the military aspect of the relationship and UN voting and joint membership in IGOs better capturing the political aspect of the relationship. 

To calculate the latent distance between each dyad for each variable, we use a bilinear mixed effects model \citep{hoff:2005}. Finally, we combine the latent distances for each variable through a principal components analysis (PCA). As such, our political strategic relationship measure is the first principal component that results from the PCA of the latent distance between each variable.  

The main advantage of calculating the latent space of different dyadic variables as opposed to using alternative specifications such as the S Score algorithm\footnote{\citet{leeds:2007}, for example, measure a states ``threat environment'' as the set of all states for which ones is contiguous with or which is a major power and with an S score below the population median.} is that it allows us to account for third order dependencies within the data. To review, first order dependency refers to the propensity for some actors to send or receive more ties than others, second-order dependency refers to reciprocity of exchange between actors, and a third-order dependency refers to interactions among three or more actors. Dyadic data are rife with these types of dependencies, and aside from first-order dependencies, they pose serious challenges to the basic assumption of independence between observations \citep{poast2010, hoff:2004}. 




More precisely, third order dependency includes the concepts of (a) transitivity, (b) balance and (c) clusterability. Formally, a triad $ijk$ is said to be transitive if for whenever $y_{ij} = 1$ and $y_{jk} = 1$, we also observe that $y_{ik}$. This follows the logic of `` a friend of a friend is a friend''. Meanwhile,  a triad $ijk$ is said to be balanced if $y_{ij} \times y_{jk} \times y_{ki} >0$. Conceptually, if the relationship between $i$ and $j$ is `positive', then both will relate to another unit $k$ identically, either both positive or both negative. Finally a triad $ijk$ is said to be clusterable if it is balanced or all the relations are all negative. It is a relaxation of the concept of balance and seeks to capture groups where the measurements are positive within groups and negative between groups.

In other words, third order dependencies suggest that ``knowing something about the relationship between $i$ and $j$ as well as between $i$ and $k$ may reveal something about the relationship between $i$ and $k$, even when we do not directly observe it'' \citep{hoff:2004}. Such a dependency is especially important to capture with regards to strategic relationships as dyadic relationships between two particular countries cannot help but be understood in the context of their relationship with other countries. The importance for accounting for these dynamics have long been acknowledged in the foreign aid literature. \citet{trumbull:1994} for example, note that, ``donors do make their decisions with knowledge of what each other are doing, and may actually act cooperatively. Any study that ignores the interrelationship of donor behavior risks problems with simultaneity bias.'' However, we find that until now, this critique has largely gone by unaddressed by the existing literature. 

 To do so, we run a null generalized bilinear mixed effects model (gbme)\footnote{Code for running the gbme can be found from Hoff's website at \url{http://www.stat.washington.edu/hoff/Code/hoff_2005_jasa/}.} to estimate the latent space for each component of our strategic interest variables. Formally, it is represented as follows:

\begin{align*}
\theta_{i,j} = a_i + b_j + \gamma_{i,j} + z_i'z_j
\end{align*}

\noindent where $\theta_{i,j}$ is the dyadic variable of interest (e.g., alliances), $a_i$ estimates `sender' effects, $b_j$ estimates `receiver' effects'  and $z_i'z_j$ is the bilinear effect which estimates the latent space and accounts for third order dependencies common in dyadic data. We estimate the model via Gibbs sampling of full conditionals of the parameters. For a more detailed discussion of this model, see \citet{hoff:2005}.  In Figure \ref{fig:polLat}, we present a visualization of the resultant latent space we calculated for each variable for the year 2005.

\begin{figure}[h!] 
\caption{Latent Spaces for components of Political Strategic Interest Measure during 2005}
\label{fig:polLat}
\centering
\begin{minipage}{.33\linewidth}
\centering
\label{fig:ally}
\includegraphics[width = 2.3in]{\detokenize{ally_2005.pdf}}
\caption*{(a) Alliances}
\end{minipage}
\hspace{-.2in}
\begin{minipage}{.33\linewidth}
\centering
\label{fig:un}
\includegraphics[width = 2.3in]{\detokenize{un_2005.pdf}}
\caption*{(b) UN voting}
\end{minipage}
\hspace{-.2in}
\begin{minipage}{.33\linewidth}
\centering
\label{fig:igo}
\includegraphics[width = 2.3in]{\detokenize{igo_2005.pdf}}
\caption*{(c) IGO membership}
\end{minipage}
\end{figure}

\indent\indent Countries that cluster together in this two-dimensional latent space are more likely to interact with each other. The plots for alliances, UN voting and IGO membership suggest that there is distinct clustering among countries. Moreover, these clusters are different across the different measures, suggesting that each variable is indeed capturing different aspects of strategic interest. % This is also true for latent space of arms transfers, a component of the military strategic relationship variable. However, there is much less clustering for the other two components of the military strategic relationship variable, suggesting that many conflicts involve more than one or two dyads and further that military conflicts can have broad ramifications beyond the dyads involved.\\

\indent\indent After estimating the latent spaces for these components, we calculate the euclidean distance between each dyad for each component. We then combine them in a principal components analysis (PCA) to reduce the dimensionality of our measure while retaining as much variance as possible. That is, alliances, UN voting and joint membership in IGOs all capture certain aspects of political strategic interest. Instead of choosing only one of them as our measure of strategic interest as other papers have done, we combine them in order to increase our explanatory power. We estimate the PCA of these variables for each year separately\footnote{For each year, we conduct a bootstrap PCA of 1000 subsamples each} and use the first principal component for each year as our measure of strategic interest. On average over all the years, we find that the first component of our PCA of alliances, UN voting and joint membership in IGOs, which we use as our measure of strategic interest, explains about 51\% of its variance. % Meanwhile we find that the our military strategic interest variable, that is the first component of our PCA of the combination of arms transfers, MIDs and war incidence, explains about 66\% of its variance.\\


%for our strateci and in Figure \ref{fig:milPCA} for the military strategic measure for the year 2005. 
% \indent\indent  A visualization of the resultant dyadic PCA is shown below in Figures \ref{fig:polPCA}. Along the x and y axes are the countries included in our analysis; any point within the plot represents the dyadic relationship between a country on the x and y axis with darker shading representing a stronger relationship and lighter shading representing a weaker relationship. 

%\indent\indent  These plots suggest that there is much more variation in political strategic relationships than there are military strategic relationships, perhaps because the number of issues spaces in the political arena are much greater. They also suggest that on average, countries have a greater political strategic relationships than military strategic relationships. Since the military strategic relationships data is composed largely of actual military events, this makes sense as on average, conflict between any two countries is much rarer than diplomatic negotiations. \\

% \begin{figure}[h!]
% \centering
% \caption{Dyadic PCA for Political Strategic Interests for year 2005}
% \label{fig:polPCA}
% \includegraphics[width = 5in]{\detokenize{dyadViz_allyIGOUN.pdf}}
% \caption*{\small{Along the x and y axes are the countries included in our analyses for the year 2005. The color gradient reflects the strength of the strategic relationship between any two dyads, with dark colors reflecting a stronger relationship and light colors reflecting a weaker relationship. Note that because the PCA is of latent distances between any two dyads, dyads that are closer in space and thus stronger strategic relationships will have smaller values.}}
% \end{figure}


\subsection*{Validating our measure of strategic interest}

\indent\indent We further conduct a series of post-estimation validation tests for our resulting strategic variable. In particular, we (1) evaluate the relationship between our political strategic interest variable  against S scores and Kendall's $\tau_b$ for alliances and (2) investigate how our measure of strategic interest describe well-known dyadic relationships. 

First, we perform a simple bivariate OLS with and with year fixed effects to evaluate how our measures compare to S scores and Kendall's $\tau_b$.\footnote{Note for comparison that the bivariate relationship of S scores on Kendall's $\tau_b$ is statistically significant with a coefficient of 0.62 while the bivariate relationship of Kendall's $\tau_b$ on S Scores is statistically significant with a coefficient of 0.31.} Note in order to make our strategic measures somewhat interpretable, for the validation we scale our strategic measures to be between 0 and 1 just as S scores and Kendall $\tau_b$ is scaled. The results are shown in Table \ref{table:polval}. % for political strategic interest and Table \ref{table:milval} for military strategic interest. \\

\begin{table}[h!]
\small
\caption{Validation of Political Strategic Interest Variable against S scores and Kendall's $\tau_b$}
\begin{center}
\begin{tabular}{l c c c c c c }
\hline
                    & Unweighted   & Unweighted & Weighted  & Weighted  & Tau-B & Tau-B \\
                   &   S Scores &   S Scores &  S Scores &  S Scores &  &   \\
\hline
(Intercept)         & $0.97^{***}$  & $1.03^{***}$  & $1.01^{***}$  & $1.02^{***}$  & $0.29^{***}$  & $0.25^{***}$  \\
                    & $(0.00)$      & $(0.00)$      & $(0.00)$      & $(0.00)$      & $(0.00)$      & $(0.00)$      \\
Strategic Interest             & $-0.80^{***}$ & $-0.84^{***}$ & $-1.22^{***}$ & $-1.26^{***}$ & $-0.89^{***}$ & $-0.87^{***}$ \\
                    & $(0.00)$      & $(0.00)$      & $(0.00)$      & $(0.00)$      & $(0.00)$      & $(0.00)$      \\
Year FE? 	   & No 		& Yes 		& No		& Yes	& No		& Yes\\
% \hline
% R$^2$               & 0.28          & 0.32          & 0.32          & 0.34          & 0.17          & 0.17          \\
% Adj. R$^2$          & 0.28          & 0.32          & 0.32          & 0.34          & 0.17          & 0.17          \\
% Num. obs.           & 824426        & 824426        & 824426        & 824426        & 824148        & 824148        \\
\hline
\multicolumn{7}{l}{\scriptsize{$^{***}p<0.001$, $^{**}p<0.01$, $^*p<0.05$}}
\end{tabular}
\label{table:polval}
\end{center}
\end{table}

\indent\indent  In brief, we find that our political strategic measure performs well against S scores and Kendall's $\tau_b$ for alliances  with and without fixed effects. Note that because the PCA is of latent distances between any two dyads, dyads that are closer in space will have smaller values and therefore represent a stronger strategic relationship. Therefore the negative relationship we find between the political strategic measure and S scores and $\tau_b$ are interpreted to mean the greater the foreign policy similarity as measured by the S score or Kendal's $\tau_b$ , the smaller the latent distance or the greater the political strategic relationship between a dyad. \\

% \indent\indent   Our military strategic measure performs will mixed results with respect to S scores and Kendall's $\tau_b$ for alliances . It has a negative and statistically significant relationship between S scores with year fixed effects. It in fact has a positive and statistically significant relationship between S scores and Kendall's $\tau_b$ without year fixed effects These mixed results suggest that the military strategic measure is perhaps measuring something qualitatively different than S scores.\\

Second, we also investigate how our measure performs relative to well known dyadic relationships. In the figures below, we plot the dyadic relationships between countries that are well-known to have friendly or antagonistic relationships. For example, Figure \ref{fig:USIsraelIran_p} shows the dyadic relationship between Iran and Israel, the US and Israel, and the US and Iran. The plot suggests that the US and Israel have consistently had a stronger political strategic relationship throughout time except for the early 1970's when Iran and Israel is shown to have had a stronger political strategic relationship. This is in fact consistent with historical evidence which suggests that Iran and Israel enjoyed close ties before the Iranian revolution. 

Meanwhile the plot of the dyadic relationships between China and Japan and, North Korea and China and North Korea and Japan suggest more or less indifferent relations among the three before 1990 after which the political strategic relationship between China and North Korea becomes markedly stronger. This is also consistent with the disappearance of Soviet support for North Korea following the end of the Cold War and the emergence of China as North Korea's new protector. 

Finally, the plot of the dyadic relationship between India and Pakistan, India and the US and Pakistan and the US suggest in fact that India and Pakistan have a much stronger political strategic relationship than either do with the US. Given the history of antagonism between India and Pakistan, this is a rather surprising result; it also suggests however that a dyad's political relationship and military relationship may be quite different and indeed as two large bordering countries, cooperation between India and Pakistan is important to the security of both. 

\begin{figure}[h!] 
\caption{Dyadic relationships over time as measured by the political strategic interest variable}
\centering
\begin{minipage}{.33\linewidth}
\centering
\caption*{}
\label{fig:USIsraelIran_p}
\includegraphics[width = 2.3in]{\detokenize{dyadic_USIsraelIran_allyIGOUN.pdf}}
\end{minipage}
\hspace{-.2in}
\begin{minipage}{.33\linewidth}
\centering
\caption*{}
\label{fig:ChinaJapNK_p}
\includegraphics[width = 2.3in]{\detokenize{dyadic_ChinaJapNK_allyIGOUN.pdf}}
\end{minipage}
\hspace{-.2in}
\begin{minipage}{.33\linewidth}
\caption*{}
\centering
\label{fig:USIndPak_p}
\includegraphics[width = 2.3in]{\detokenize{dyadic_USIndPak_allyIGOUN.pdf}}
\end{minipage}
\end{figure}

% \indent\indent   We plot the same dyadic relationships using our military strategic interest variable. Here, variation between different dyadic relationships is much more difficult to tease out, perhaps a function of the fact that military events are much more rare. There are two points of interest about these plots (i) they have large degrees of variation over time, suggesting that while military events may be rare, they also have a large influence on a dyad's military strategic relationship (ii) they dyadic relationships plotted here seem to be very similar over time potentially suggesting that third order dependencies are very strong with regards to military strategic relationships.

% \begin{figure}[h!] 
% \caption{Dyadic relationships over time as measured by the military strategic interest variable}
% \centering
% \begin{minipage}{.33\linewidth}
% \centering
% \label{fig:ally}
% \includegraphics[width = 2.3in]{\detokenize{dyadic_USIsraelIran_midWarArmsSum.pdf}}
% \end{minipage}
% \hspace{-.2in}
% \begin{minipage}{.33\linewidth}
% \centering
% \label{fig:un}
% \includegraphics[width = 2.3in]{\detokenize{dyadic_ChinaJapNK_midWarArmsSum.pdf}}
% \end{minipage}
% \hspace{-.2in}
% \begin{minipage}{.33\linewidth}
% \centering
% \label{fig:igo}
% \includegraphics[width = 2.3in]{\detokenize{dyadic_USIndPak_midWarArmsSum.pdf}}
% \end{minipage}
% \end{figure}

%\indent\indent  Before moving on to the next section, we note that it is possible to do a PCA on all different of the components of strategic interest --- alliances, UN voting, joint IGO membership, arms transfers, mids and wars --- combined. If we were to take this approach,  we could run our models using the largest components of the resulting PCA. As discussed above, while we argue that political and military strategic interest are qualitatively different, we do acknowledge that both can inform each other and so taking such a course of action would be theoretically logical. \\

%\noindent\noindent While we considered employing this approach, we decided to make the trade-off for better interpretability of our measure over increased precision of our strategic interest measure as the interpretation of different components of a PCA measure is generally not straightforward as it is. For example, we could end up with a first principal component that is explained by alliances \%50 of the time, IGOs \%40 of the time, arms transfers \% 5 of the time and the rest of the components a combined \%5 of the time and a second component that is explained by  MIDS \%60 of the time, alliances 30\% of the time, and the rest of the components a combined \%10 of the time. While we may be able to say that strategic interest matters, it would be more difficult to say in what way. In separating out the variables before hand for theoretical reasons, we increase the interpretability of any of our subsequent results while sacrificing some explanatory power. At the same time, whatever results we do find should represent a harder test for the importance of political or military strategic interest because of this trade-off.

% multilateral aid
% news reports