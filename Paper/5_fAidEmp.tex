\section*{Analysis}
\label{empirics}

%% just taking this out because this model equation doesnt account for the hierarchical structure
% \begin{align*}
%   Log(Aid_{sr,t}) &= \beta_{1}(Pol. \; Strat.  \; Distance_{sr,t-1}) \\
%   & \;+\; \beta_{2}(Colony_{sr,t-1}) \;+\; \beta_{3}(Polity_{r,t-1}) \\
%   & \;+\; \beta_{4}Log(GDP \; per \; capita_{r,t-1}) \;+\; \beta_{5}(Life \; Expect_{r,t-1}) \\  
%   & \;+\; \beta_{6}(No. \; Disasters_{r,t-1}) \;+\; \beta_{7}(Civil \; War_{r,t-1}) 
% \end{align*}

\subsection*{Estimation Method}

To model aid flows using our directed-dyadic panel dataset, we utilize a hierarchical model. To implement this, we nest aid receivers within aid senders and aid senders within years. We include random intercepts in our model for every sender and year. More concretely, we fit the following model: 

The results of this analysis are shown below in a coefficient plot in Figure \ref{fig:intCoef}.\footnote{Note, to examine the model results without the interaction effects, please see Figure \ref{fig:nointCoef} in Appendix \ref{app:rawModels}} We test Hypothesis 1A, 1B and 1C using the model with `Humanitarian Aid' as the dependent variable. The results show a positive and statistically significant relationship between the interaction of \textit{Strategic Distance} and the \textit{No. Disasters}. To interpret these results, we turn to Figure \ref{fig:simEffects} (`Humanitarian Aid' panel) where we plot the substantive effect of this interaction term on humanitarian aid over the range of \textit{Strategic Distance} for different levels of natural disaster severity. 

\begin{align*}
  Log(Aid)_{sr,t} &= \beta_{1}(Pol. \; Strat.  \; Distance_{sr,t-1})  \\
  & \;+\; \beta_{2}(Colony_{sr,t-1}) \;+\; \beta_{3}(Polity_{r,t-1}) \\
  & \;+\; \beta_{4}Log(GDP \; per \; capita_{r,t-1}) \;+\; \beta_{5}(Life \;Expect_{r,t-1}) \\  
  & \;+\; \beta_{6}(No. \; Disasters_{r,t-1}) \;+\; \beta_{7}(Civil \; War_{r,t-1}) \\
  & \;+\; \beta_{8}(Pol. \; Strat.  \; Interest_{sr,t-1} \times No. \; Disasters_{r,t-1}) \\
   & \;+\; \delta_s  \;+\; \rho_r \\
\end{align*}

\noindent Where $\delta_s$ and $\rho_r$ are the sender and receiver random effects respectively. 

\begin{figure}
	\centering
	\includegraphics[width=1\textwidth]{intCoef.pdf}
	\caption{Coefficient plots for the main analyses with interaction terms across each dependent variable, humanitarian aid, civil society aid and development aid.  Coefficients that are significant at the 5\% level are shaded in blue if the coefficient is positive and red if the coefficient is negative. Coefficients that are not significant at the 5\% level are shaded in gray. }
	\label{fig:intCoef}
\end{figure}
\FloatBarrier

These results suggest that the greater the number of natural disasters a country experiences, the more likely it is to receive humanitarian aid from a strategic adversary. This is apparent in the rising slope of the relationship between strategic interest and humanitarian aid as the number of natural disasters increases. As such, these results are consistent with H1C, which suggests that donors may be more likely to dispense humanitarian aid to their strategic adversaries because such disasters present unique opportunities to improve bilateral relations. Notably when natural disasters are particularly severe, donors may dispense a great deal more aid to strategic opponents compared to strategic allies to further their strategic interests.  

Conversely, to find support for H1A, which hypothesizes that donors are more likely to give to their strategic allies in the wake of a natural disaster to further their own self-interest, we would have expected the parameter estimate for the interaction term between strategic interest and natural disasters to be negative, which it is not. Moreover, we would have expected there to be a downward sloping relationship between strategic interest and humanitarian interest as the number of natural disasters increases. This is clearly not evidenced in the ``Humanitarian Aid'' panel in Figure \ref{fig:simEffects}. 

Meanwhile, support for H1B is also missing. In particular, we would have expected there be a downward sloping relationship between strategic interest and humanitarian aid when there are no natural disasters. However, if natural disasters had a humanizing effect on strategic opponents, then we would have expected the slope between strategic interest and humanitarian aid to flatten as the number of natural disasters increased, which we do not find.  




\begin{figure}
	\centering
	\includegraphics[width=1\textwidth]{simComboPlot.pdf}
	\caption{Simulated substantive effect plots for each dependent variable (humanitarian aid, civil society aid, and development aid) for different levels of natural disaster severity across the range of the strategic distance measure. A rug plot is provided below each panel.}
	\label{fig:simEffects}
\end{figure}

Meanwhile, we test H2 by examining the effect of the interaction between strategic interest and natural disasters on civil society aid. In Figure \ref{fig:intCoef} we similarly find a positive and significant relationship between this interaction and civil society aid. The substantive effects plot (in the `Civil Society Aid' panel in Figure \ref{fig:simEffects}) meanwhile also suggests that donors are more likely to target aid to civil society in their strategic adversaries the more natural disasters that country experiences, supporting H2. These results are somewhat suggestive of the idea that donors may be acting to take advantage of vulnerable recipients to mold the relationship to their interests. 

Finally, we test H3 by analyzing how the interaction between strategic interest and natural disasters affects development aid allocation. From, Figure  \ref{fig:intCoef}, we can see that this coefficient is not statistically significant. Examining the substantive significance in Figure \ref{fig:simEffects} (`Development Aid' panel) we can see that while the level of development aid does increase as the number of natural disasters increases, the slope between strategic interest and development aid is only minimally affected, suggesting little support for H3. These results suggest that whatever humanitarian impulse donors may have felt toward their strategic adversaries (which H1B suggests exists), this effect only applies to short-term humanitarian aid and not long-term development aid. 

\subsection*{Persistence of foreign aid allocation over time}

How persistent are these changes that we observe? To answer this question, we re-estimate the original models for different lag lengths for the main interaction terms\footnote{The covariates are measured using a one-year lag throughout}. These models are estimated separately for each lag length (lags of 1, 3, and 5 years). The simulation results when using different lags for the interactions and constituent terms are shown in Figures \ref{fig:humanIntCoef}, \ref{fig:civIntCoef}, and \ref{fig:devIntCoef} for the outcome variables humanitarian aid, civil society aid and development aid, respectively.

\begin{figure}[h!]
	\centering
	\includegraphics[width=1\textwidth]{simComboPlot_lag_hAid.pdf}
	\caption{Simulated substantive effect plots for humanitarian aid for varying lags of variables of interest and different levels of natural disaster severity across the range of the strategic distance measure.}
	\label{fig:humanIntCoef}
\end{figure}

From Figure \ref{fig:humanIntCoef}, we can see that the interaction between strategic interest and natural disasters is rather persistent until approximately five years after a natural disaster. This suggests that donors are more likely to allocate humanitarian aid to their strategic adversaries for some time following a natural disaster, suggesting that donors seek to use natural disasters as a tactic to improve relations with strategic opponents for quite some time after the initial disaster  (supporting H1C).

\begin{figure}[h!]
	\centering
	\includegraphics[width=1\textwidth]{simComboPlot_lag_cAid.pdf}
	\caption{Simulated substantive effect plots for civil society aid for varying lags of variables of interest and different levels of natural disaster severity across the range of the strategic distance measure.}
	\label{fig:civIntCoef}
\end{figure}


% sm: this sentence is not clear tome: Another interpretation is that civil society aid is actually rather effective and as such, recipients governments are likely to push back against allowing it in fairly short order.
Figure \ref{fig:civIntCoef} shows that while the interaction between strategic interest and natural disasters positively affects the allocation of civil society aid, this effect is only consistent for a short time following a natural disaster (supporting H2). One way to interpret these results is that donors recognize the difficulty of trying to influence domestic politics through civil society aid relatively quickly, and, as a consequence, waste relatively little time in pursuing such attempts. Another interpretation is that civil society aid is actually rather effective and as such, recipients governments are likely to push back against allowing it in fairly short order. Teasing out the exact mechanism would be a fruitful area for future research.

Last, Figure \ref{fig:devIntCoef} extends the earlier finding that the interaction between strategic interest and natural disasters has little effect on development aid across a variety of different lags. This result further suggests that there is little support for H3, that is natural disasters do not seem to prompt donors to care concern themselves with recipient's long-term interests.

\begin{figure}[h!]
	\centering
	\includegraphics[width=1\textwidth]{simComboPlot_lag_dAid.pdf}
	\caption{Simulated substantive effect plots for development aid for varying lags of variables of interest and different levels of natural disaster severity across the range of the strategic distance measure.}
	\label{fig:devIntCoef}
\end{figure}