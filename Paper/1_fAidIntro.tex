\section*{Introduction}
\label{intro} 

In the early morning hours of December 26, 2003, a massive earthquake measuring 6.3 on the Richter scale struck the city of Bam, Iran. Its effects were devastating.  Out of Bam's 100,000 residents, approximately 26,000 to 40,000 were killed while those who survived were left to grapple with the destruction of 70 to 90 percent of the city's housing infrastructure \citep{montazeri:2005}.\footnote{Fathi, Nazila. ``Deadly Earthquake Jolts City in Southeast Iran.'' \textit{The New York Times.} 26 December 2003. Accessed October 2017: \url{https://web.archive.org/web/20090620230700/http://www.nytimes.com/2003/12/26/international/26CND-QUAKE.html?ex=1225166400&en=c550b50a2ad59dd6&ei=5070}} As part of the international response that followed, more than 44 countries sent aid, including the United States, which contributed eight plane loads of medical and humanitarian supplies as well as several dozen teams of experts to the relief effort. 

However, while the US response to the 2003 Bam earthquake was seemingly analogous to that of any foreign actor offering aid and support, \textit{a priori}, it was not obvious whether the US would send any humanitarian aid at all, to say nothing of whether Iran would accept it. Just the year prior, then-President George W. Bush had famously anointed Iran membership in the ``Axis of Evil'' \citep{heradstveit:2007}. Meanwhile, at the time of the earthquake US-Iran relations were particularly delicate as the countries navigated the issue of nuclear weapons in Iran.\footnote{\url{http://news.bbc.co.uk/2/hi/middle_east/3362443.stm}} Indeed, given the broader context of contentious bilateral relations, the process of transferring aid from the US to Iran entailed greater intentionality than normal. To initiate the flow of any aid, President Bush was obliged to institute a special 90-day measure to ease US sanctions on Iran\footnote{``US eases Iran sanctions to speed earthquake relief.'' \textit{China Daily}. 1 January 2004. Accessed October 2017: \url{http://www.chinadaily.com.cn/en/doc/2004-01/01/content_295063.htm}} -- these had been in place since 1979 and continue to be enforced to this day.\footnote{The US first imposed sanctions against Iran in 1979 during the US-Iran hostage crisis. While many assets have since been unfrozen, sanctions on a number of items, including military sales, financial assets, and real estate holdings remain in place \citep{katzman:2018} } For Iran's part, accepting US aid meant allowing US military planes to land in Iran, which had not happened in over 20 years.\footnote{``Iran Quake Toll May Hit 50,000.'' \textit{China Daily } 31 December, 2003. Accessed October 2017: \url{http://www.chinadaily.com.cn/en/doc/2003-12/31/content_294833.htm}} For a country that had undergone a revolution in part because the US military was perceived to have had too strong a domestic influence, it was far from obvious that such an act would be perceived as benign.\footnote{\url{https://www.stratfor.com/geopolitical-diary/geopolitical-diary-tuesday-dec-30-2003}} 

\begin{figure}
  \centering
  \includegraphics[width = .9\textwidth]{US_Iran_aid}
  \caption{US aid commitments to Iran, 2002 - 2013}
  \label{fig:us_iran}
\end{figure}

Intriguingly, the Bam earthquake led not only to an increase, albeit temporarily, in US humanitarian aid to Iran, but was followed by other types of aid as well. Figure \ref{fig:us_iran} shows that after 2004, aid commitments to  "strengthen civil society" increased markedly and consistently, reaching its apex with the creation of the 2006 "Iran Democracy Fund" to promote democracy in Iran.\footnote{Carpenter, J. Scott. ``After the Crackdown: The Iran Democracy Fund.'' \textit{The Washington Institute for Near East Policy, PolicyWatch 1576} 8 September 2009. Accessed May 2018: \url{http://www.washingtoninstitute.org/policy-analysis/view/after-the-crackdown-the-iran-democracy-fund}} Meanwhile, US aid for a variety of developmental purposes, (i.e. economic and development policy and planning, infectious disease control, social/welfare services) also increased sporadically following 2003. This is particularly noteworthy given that Iran has generally been barred from receiving US foreign aid since  the US State Department designated it a ``state sponsor of terrorism'' in 1984 \citep{samore:2015}.\footnote{Available data from AidData and the OECD suggest that the US did not commit any aid to Iran from 1974 to 2001.}   

What to make of this remarkable turn of events?  Why did the US send humanitarian aid to Iran despite objectively hostile extant relations? And what drove Iran to accept it?  Was this event \textit{sui generis} or is it possible to observe other dyadic pairs acting in a similar fashion? If so, does the occurrence of a natural disaster also lead donors to distribute other types of aid to strategic opponents?   

Answering these questions has important implications for our understanding of how donors seek to use foreign aid. Furthering such an understanding is important as the occurrences of natural disasters are likely to increase with changing climate conditions. Meanwhile, given an existing literature that emphasizes that donors are more likely to give aid to strategic allies, a more nuanced understanding of what motivates donor aid allocations is necessary to answer these questions.

In this paper, we show that following a natural disaster donor countries actually give more humanitarian aid to strategic \textit{opponents}. We argue that this is because donors use natural disasters as an opportunity to ingratiate themselves with countries that they have historically shared hostile relations with.

While humanitarian aid, is generally intended as a stop-gap to help recipient countries return to their status quo, civil society and development aid are targeted toward catalyzing long term change. Civil society aid is often used to improve governance outcomes,\footnote{More specifically, some argue that the lack of good governance and state capacity in developing countries have stymied the ability for foreign aid to promote development. As such, the promotion of civil society is seen as important to the successful implementation of foreign aid projects.} which provides donors an avenue through which to wade into the domestic politics of recipient states \citep{ottaway2000funding, henderson:2002, resnick2012foreign, spina:2014}. Meanwhile, development aid is primarily focused on economic development. As such, we show that donor motivations to disperse these type of aid are distinct. Specifically, we argue that donors further seek to forward their long-term strategic interests as they tend to increase civil society aid, rather than development aid, following a natural disaster in recipient states with which they do not share positive relations.  We evaluate these claims using a new measure of strategic interest that: 1) captures indirect ties states share 2) and incorporates a variety of dimensions of strategic interest. 

In what follows, we first give a brief overview of the existing literature on natural disasters and foreign aid allocations before outlining our hypotheses. We then introduce our new measure of strategic interest, and present our empirical analysis of how natural disasters condition foreign aid allocation decisions. 