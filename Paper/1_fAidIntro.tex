\section*{Introduction}
\label{intro} 

In the early morning hours of December 26, 2003, a massive earthquake measuring 6.3 on the Richter scale struck the city of Bam, Iran. Its effects were devastating.  Out of Bam's 100,000 residents, approximately 26,000 to 40,000 were killed. Its remaining residents had to further grapple with the destruction of 70 to 90 percent of the city's housing infrastructure \citep{montazeri:2005}.\footnote{Fathi, Nazila. ``Deadly Earthquake Jolts City in Southeast Iran.'' \textit{The New York Times.} 26 December 2003. Accessed October 2017: \url{https://web.archive.org/web/20090620230700/http://www.nytimes.com/2003/12/26/international/26CND-QUAKE.html?ex=1225166400&en=c550b50a2ad59dd6&ei=5070}} As part of the international response that followed, more than 44 countries sent aid, including the United States, which contributed eight planeloads of medical and humanitarian supplies as well as several dozen teams of experts to the relief effort. 

However, while the US response to the 2003 Bam earthquake was seemingly analogous to that of any foreign actor offering aid and support, \textit{a priori}, it was not obvious whether the US would decide to send any humanitarian aid, to say nothing of whether Iran would accept it. Just the year prior, then-President George W. Bush had famously labeled Iran as being part of the ``Axis of Evil'' \citep{heradstveit:2007} and at the time of the earthquake, US-Iran relations were particularly delicate as the countries navigated the issue of nuclear weapons in Iran.\footnote{\url{http://news.bbc.co.uk/2/hi/middle_east/3362443.stm}} Indeed, given the broader context of contentious relations between US and Iran, the process of transferring aid from the US to Iran entailed greater intentionality than normal.  For example, to facilitate aid flows to Iran, President Bush was obliged to institute a special 90-day measure to ease US sanctions on Iran  (which had been in place since 1979 and which continue to be enforced to this day).\footnote{``US eases Iran sanctions to speed earthquake relief.'' \textit{China Daily}. 1 January 2004. Accessed October 2017: \url{http://www.chinadaily.com.cn/en/doc/2004-01/01/content_295063.htm}} For Iran's part, the U.S. military planes that flew in aid to Iran were the first to have landed  there in over 20 years.\footnote{``Iran Quake Toll May Hit 50,000.'' \textit{China Daily } 31 December, 2003. Accessed October 2017: \url{http://www.chinadaily.com.cn/en/doc/2003-12/31/content_294833.htm}} For a country that had undergone a revolution in part because the US military was perceived to have had too strong a domestic influence, such an act was far from benign.\footnote{\url{https://www.stratfor.com/geopolitical-diary/geopolitical-diary-tuesday-dec-30-2003}} 

%The existing literature advises us that donors are largely driven by strategic interest when allocating foreign aid. Yet why then,
What to make of this remarkable turn of events?  Why did the US send aid to Iran despite its objectively chilly extant relations? And what drove Iran to accept it? Was this event \textit{sui generis} or is it possible to observe other dyadic pairs acting in a similar fashion more generally? Answering these questions  has important implications for our understanding of how donors allocate aid more broadly. Such an understanding is perhaps more important than ever given that the incidence of natural disasters is likely to increase with changing climate conditions. Meanwhile, given an existing literature that emphasizes strategic factors as driving foreign aid, a more nuanced understanding of what motivates donor's aid allocations is needed to answer these questions.

%  SM: i think we should be careful about this kind of language since we cant exactly parse out the mechanism using hte data that we have. specifically, that we know after a disaster a country is more likely to commit aid irrespective of their strategic relationshp, but we dont know if it's because they want to do it for humanitraian reasons or because they want to do it to make the government of the host country look bad to its citizens for needing help.
% 
In this paper, we argue that natural disasters can drastically shift the social context of a dyadic relationship toward emphasizing its altruistic or humanitarian dimensions. 

In this paper, we show that 

This shift is greater depending on the severity of the natural disaster and the extent to which the baseline dyadic relationship can be characterized as contentious. We evaluate our argument using a new measure of strategic interest that we believe greatly improves on existing measures of strategic interest as it is able to: 1) capture the third-order effects that the broader literature on donor motivations generally neglects 2) and incorporate many different dimensions of strategic interest more efficiently than existing variables. 

In what follows, we first give a brief overview of the existing literature on natural disasters and foreign aid allocations before outlining our theory and hypotheses. We then present and validate our new measure of strategic interest, and close with our empirical analysis of how natural disasters condition foreign aid allocation decisions. 

% SM not sure if we need this.
% To preview, our results suggest that aid allocations increase with both the severity of the natural disaster and the previous level of antagonism in a dyadic pair. Our analysis is based on a panel of the 18 donor countries and 167 recipient countries from 1975 and 2006. 