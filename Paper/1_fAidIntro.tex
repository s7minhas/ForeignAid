\section*{Introduction}
\label{intro} 

In the early morning hours of December 26, 2003, a massive earthquake measuring 6.3 on the Richter scale struck the city of Bam, Iran. Its effects were devastating.  Out of Bam's 100,000 residents, approximately 26,000 to 40,000 were killed. Its remaining residents had to further grapple with the destruction of 70 to 90 percent of the city's housing infrastructure \citep{montazeri:2005}.\footnote{Fathi, Nazila. ``Deadly Earthquake Jolts City in Southeast Iran.'' \textit{The New York Times.} 26 December 2003. Accessed October 2017: \url{https://web.archive.org/web/20090620230700/http://www.nytimes.com/2003/12/26/international/26CND-QUAKE.html?ex=1225166400&en=c550b50a2ad59dd6&ei=5070}} As part of the international response that followed, more than 44 countries sent aid, including the United States, which contributed eight planeloads of medical and humanitarian supplies as well as several dozen teams of experts to the relief effort. 

However, while the US response to the 2003 Bam earthquake was seemingly analogous to that of any foreign actor offering aid and support, \textit{a priori}, it was not obvious whether the US would decide to send any humanitarian aid, to say nothing of whether Iran would accept it. Just the year prior, then-President George W. Bush had famously labeled Iran as being part of the ``Axis of Evil'' \citep{heradstveit:2007} and at the time of the earthquake, US-Iran relations were particularly delicate as the countries navigated the issue of nuclear weapons in Iran.\footnote{\url{http://news.bbc.co.uk/2/hi/middle_east/3362443.stm}} Indeed, given the broader context of contentious relations between US and Iran, the process of transferring aid from the US to Iran entailed greater intentionality than normal.  For example, to facilitate aid flows to Iran, President Bush was obliged to institute a special 90-day measure to ease US sanctions on Iran  (which had been in place since 1979 and which continue to be enforced to this day).\footnote{``US eases Iran sanctions to speed earthquake relief.'' \textit{China Daily}. 1 January 2004. Accessed October 2017: \url{http://www.chinadaily.com.cn/en/doc/2004-01/01/content_295063.htm}} For Iran's part, the U.S. military planes that flew in aid to Iran were the first to have landed  there in over 20 years.\footnote{``Iran Quake Toll May Hit 50,000.'' \textit{China Daily } 31 December, 2003. Accessed October 2017: \url{http://www.chinadaily.com.cn/en/doc/2003-12/31/content_294833.htm}} For a country that had undergone a revolution in part because the US military was perceived to have had too strong a domestic influence, such an act was far from benign.\footnote{\url{https://www.stratfor.com/geopolitical-diary/geopolitical-diary-tuesday-dec-30-2003}} 

%The existing literature advises us that donors are largely driven by strategic interest when allocating foreign aid. Yet why then,
What to make of this remarkable turn of events?  Why did the US send aid to Iran despite its objectively chilly extant relations? And what drove Iran to accept it? Was this event \textit{sui generis} or is it possible to observe other dyadic pairs acting in a similar fashion more generally? Answering these questions  has important implications for our understanding of how donors allocate aid more broadly. Such an understanding is perhaps more important than ever given that the incidence of natural disasters is likely to increase with changing climate conditions. Meanwhile, given an existing literature that emphasizes strategic factors as driving foreign aid, a more nuanced understanding of what motivates donor's aid allocations is needed to answer these questions.

In this paper, we argue that natural disasters can drastically shift the social context of a dyadic relationship toward emphasizing its altruistic or humanitarian dimensions. This shift is greater depending on the severity of the natural disaster and the extent to which the baseline dyadic relationship can be characterized as contentious. We evaluate our argument using a new measure of strategic interest that we believe greatly improves on existing measures of strategic interest as it is able to: 1) capture the third-order effects that the broader literature on donor motivations generally neglects 2) and incorporate many different dimensions of strategic interest more efficiently than existing variables. 

In what follows, we first give a brief overview of the existing literature on natural disasters and foreign aid allocations before outlining our theory and hypotheses. We then present and validate our new measure of strategic interest, and close with our empirical analysis of how natural disasters condition foreign aid allocation decisions. 

% SM not sure if we need this.
% To preview, our results suggest that aid allocations increase with both the severity of the natural disaster and the previous level of antagonism in a dyadic pair. Our analysis is based on a panel of the 18 donor countries and 167 recipient countries from 1975 and 2006. 



%%%%%% old notes below
%`International food aid has stimulated private markets, reduced the price of food in the markets 25-35 percent, and undermined central government propaganda concerning South Korea and the United States.' %https://www.usip.org/publications/1999/08/politics-famine-north-korea

%`The food crisis did not begin with the floods in August 1995, as has been commonly understood, but with the sharp reduction in heavily subsidized food, equipment, and crude oil from the Soviet Union and China in the early 1990s. This reduction precipi- tated an agricultural and industrial decline of enormous magnitude....
%The massive floods during August 1995 led to a central government appeal to the World Food Program (WFP) for food aid. This natural disaster and a series of successive droughts and floods over the next three years are responsible for about 15-20 percent of the food deficit facing the country, the rest being attributable to collectivist agricul- tural policies '


%`Diversions undermine regime support. Refugees from South Hamgyong know that countries they have been taught are their enemies are giving food aid. As one refugee from Hamhung City told me, “We were taught all these years that the South Koreans and Americans were our enemies. Now we see they are trying to feed us. We are wondering who our real enemies are.” When asked why they had not received the food aid free through the public distribution system, refugees repeatedly replied, “The corrupt cadres [or bureaucracy] are stealing the food and selling it on the markets for their own profit while we starve. We see it there for sale.” Medicins Sans Frontieres refugee interviews provide similar testimony as does the Washington Post investigation based on refugee interviews (February 11, 1999). Thus, the inability of the regime to feed its population and the presence of food aid in the markets are undermining popular support among those without political power. Given the nature of the regime, this public dissatisfaction is not reflected in overt opposition but in growing corruption, black market activities, sabotage, and other antisystem behavior that reflects public cynicism and anger'

% While the question `Who gives foreign aid to who and why?' had been asked long before \citet{alesina:2000} posed it \citep{mckinlay:1977,mckinlay:1978,mckinley:1979,maizels:1984,schraeder.etal:1998}, they were among the first to use qualitative methods to answer it. In so doing, they inspired a spate subsequent researchers, using a variety of different methods and samples, to arrive at their own answers \citep{neumayer:2003,berthelemy:2004,berthelemy:2006,bermeo:2008,easterly:2008,dreher:2012}. 


% Understanding what factors motivate foreign aid distribution is crucial to understanding its subsequent effect on domestic economic development. To that end, scholars have sought to evaluate the degree to which aid is dispersed primarily for the benefit of the donor or for the benefit of the recipient.\footnote{To that end, researchers have used a number of framing devices to evaluate the drivers of foreign aid, including donor interest vs. recipient interest \citep{lumsdaine:1993}, realist vs idealist motivations \citep{schraeder.etal:1998}, donors who give to recipients who practice bad governance or good governance \citep{dollar:2006,neumayer:2005}, donors who give to recipients who implement ``bad polices'' or ``good policies'' \citep{alesina:2000}.} Yet for all the effort that has since been expended on this question, we contend that three important methodological considerations: the 1) operationalization of potential drivers of foreign aid 2) selection of appropriate modeling strategies, and 3) handling of missing data,  should give us pause as to the validity of any subsequent findings 

% In our paper, we develop solutions for each of these issues in order to provide reliable evidence of the determinants of foreign aid. To wit, we not only seek to bring coherence to the variegated operationalization of `strategic interest' across the foreign aid literature, but, more importantly, are able to account for the strategic dimension of strategic interest. Existing measures of strategic interest are generally only able to capture each donor-recipient relationship separately, as opposed to considering the network of donor-recipient relationships as a whole.  We create a new measure of strategic interest that is able to account for these interdependencies by estimating a latent relational space underlying these three dyadic variables: alliances, UN voting, and IGO membership. %In our preliminary models using DAC data only, we find that our approach, which builds off of Hoff (2008), significantly outperforms the raw variables in capturing the effect of strategic interest on foreign aid allocation. 

% Moreover, while humanitarian interest is generally also seen as an important determinant of foreign aid, surprisingly no existing models seem to account for natural disasters, which anecdotally, elicit a great deal of foreign aid funding. We rectify this gap by accounting for natural disasters using data from the EM-DAT dataset. By doing so, we find that countries recently facing disasters do indeed receive greater levels of foreign aid even after accounting for factors such as strategic interest. Moreover, we also find that a recipient country is more likely to receive aid after a natural disaster if a donor county has a greater strategic interest in it.

% Methodologically, we also make two additional contributions to our understanding of the determinants of aid flows. First, while researchers recognize that there may be significant variation in aid allocation across space and time, most modeling strategies only consider one dimension in turn. We develop a multi-level hierarchical model that enables us to better capture variation between dyads over time. More significantly, while a number of studies have pointed to the issues raised by employing list-wise deletion strategies (e.g., \citet{honaker_king:2010}), these concerns have not been addressed in the context of dyadic analyses and the foreign aid literature. We extend the semiparametric copula estimation procedure developed by \citep{hoff:2007} to handle imputation in the context of relational data such as foreign aid.

% In what follows....

%Understanding whether and how non-DAC countries may differ from DAC countries in distributing foreign aid can greatly contribute to our understanding of aid effectiveness. To that end, if our proposal is accepted, we would request access to the New Donors Database currently being coded at Heidelberg University.



% Human and economic catastrophes associated with natural hazards are obviously not new, even if new media have changed the way we are aware of them. The January 2010 earthquake in Haiti and the Indian Ocean tsunami of 2004 both generated much international media attention and unprecedented amounts of international pledges of aid from private charities, non-governmental organizations, governments, and multilateral organizations.1 Nonetheless, aid pledges made while media attention is at its peak may not always be disbursed, could take a long time to arrive, or may replace previously pledged aid. This raises the following questions: how much does foreign aid really increase in the aftermath of large disasters? Are aid surges sizable in relation to the estimated economic damages caused by disasters? And what determines the actual size of the surges?

%\textcolor{red}{sm: Still using quotes here right?}

%The U.S. response to the 2003 Bam earthquake follows a well-worn pattern in which the . Iran, for example, refused similar offers of help from Israel. 

%Of the more than 44 countries that sent aid in response,\footnote{\url{https://web.archive.org/web/20081017173307/http://www.usaid.gov/iran/}} perhaps the most notable donor was the United States. Just the year prior, then-President George W. Bush had famously labeled Iran as being part of the, ``Axis of Evil'' \citep{heradstveit:2007} and at the time of the earthquake, relations were strained over the issue of nuclear weapons in Iran in particular.\footnote{\url{http://news.bbc.co.uk/2/hi/middle_east/3362443.stm}} Despite these tensions, the US provided medical and humanitarian supplies and several dozen relief experts by landing U.S. military planes in Iran, the first to do so in over 20 years. U.S.\footnote{``Iran Quake Toll May Hit 50,000.'' \textit{China Daily } 31 December, 2003. Accessed October 2017: \url{http://www.chinadaily.com.cn/en/doc/2003-12/31/content_294833.htm}}



%Donor countries may also provide relief with an eye to their own economic or geostrategic political interest (for example, Alesina and Dollar, 2000, and the references therein). Large disasters may destabilize governments. Aid to friendly governments could help these stay in power; withholding aid from not-so-friendly governments could destabilize them (Drury, Olson, and Van Belle, 2005). Disaster relief may also be used to protect investments in foreign countries, driving relief towards countries where the donors have large economic stakes.

% United States - Iran earthquake - 2003

% http://www.nytimes.com/2012/08/15/world/middleeast/us-vows-to-speed-aid-to-iran-earthquake-victims.html


%Due to the earthquake, relations between the United States and Iran thawed. The U.S. usually treated Iran as part of the "axis of evil", as its President George W. Bush called the nation.[16] However, following the tremor White House spokesman Scott McClellan spoke on behalf of President Bush: "Our thoughts and prayers are with those who were injured and with the families of those who were killed."[5]




% Fairfax County Urban Search and Rescue squad inspect earthquake damage in Bam
% The U.S. offered direct humanitarian assistance to Iran. Iran initially declined this offer,[17] though later accepted it. On December 30 an 81-member emergency response team was deployed to Iran via U.S. military aircraft, consisting of search and rescue squads, aid coordinators, and medical support.[21] These were the first U.S. military airplanes to land in Iran for more than 20 years.[12]

% In return, the state promised to comply with an agreement with the International Atomic Energy Agency which supports better monitoring of its nuclear interests. This led U.S. Secretary of State Colin Powell to suggest direct talks in the future.[16] Sanctions were temporarily relieved to help the rescue effort.[20] However, he also said that the U.S. was still concerned on other Iranian issues, such as the prospect of terrorism and the country's support of Hamas.[16]


% \indent\indent  This seeming consensus belies the inconsistency with which scholars conceptualize and measure strategic considerations, which have variously included bilateral trade intensity, UN voting scores, colonial legacies and regional dummies among others. In this paper we seek to rectify in fragmentation: First, we create an original measure of bilateral strategic interest that measures the latent distance between countries across the strategic policy space.  In doing so, we seek to provide a more coherent measure of strategic interest which incorporates many of the measures that previous papers have used. Further this measure improves upon existing measures of strategic interest in that it maps strategic interest onto a ``social space'', through which we can account for third order relationships between states \citep{hoff:2002}. \\ 

% %%Scholars have variously used bilateral trade intensity \citep{berthelemy:2006, berthelemy:2004}, colonial legacies \citep{berthelemy:2004},  UN voting \citep{alesina:2000}, political orientation of the recipient country \citep{easterly:2008} to measure strategic interest. Other scholars take a different approach and investigate how much aid allocations can be ascribed to humanitarian reasons. These are broadly split along economic need \citep{collier:2002,nunnenkamp:2006,thiele:2007} and the quality of political governance \citep{neumayer:2005,dollar:2006}, the implication being that countries that fail to give aid along these criterion are acting in their strategic interest.\\

% %\indent %These measures are at best imperfect and at worst, uninterpretable. As \citet{bermeo:2008} states,
% %\begin{quote} `Perhaps the most puzzling conclusion of the existing literature is that a focus on trade partners, former colonies, and allies is somehow evidence against a development focus of aid. Instead, one could interpret this as evidence that donors give aid to the countries in which they most wish to pursue development. In this sense, donor interests and recipient needs are not mutually exclusive categories.'
% %\end{quote} Meanwhile some have argued that donors who give to poor countries may not do it out of a %humanitarian impulse but because it is cheaper to buy interest in poorer countries \citep{demesquita:2007,stone:2006}. Conversely, a donor country may give to a poorly governed, undemocratic country for humanitarian reasons as well, North Korea being a prominent example.

% \indent\indent  The existing lack of coherence in evaluating strategic interest extends to model specification. Papers which have empirically evaluated the dominance of strategic over humanitarian motives with some exceptions \citep{berthelemy:2006}, have done so by specifying models which pool all donors together or by running models for each donor country separately. In our model specification, we use a hierarchical random effects model with countries receiving aid nested in senders and senders nested in time. Applying this method will enable us to explicitly model the drivers of aid in an aggregate sense and to also explore how those drivers vary between senders. For now we present preliminary results that show our strategic interest variable does play a positive role in predicting aid flows between countries.

% %%%%%%%%%%%%%%%%%%%%%%
% % SM note: So we don't resolve these issues with our current model specification either. We could have tried to get at these issues using a network modeling approach, but as we discussed the bipartite nature of our data took away that option. In the next iteration of this paper, we can take a look into adding spatially lagged covariates to get at this issue. Hmmm, it might also be interesting to weight aid flows by our strategic interest variable. This would help us to test whether or not states follow their strategic partners in giving aid flows. 
% %%%%%%%%%%%%%%%%%%%%%%

% % We find this empirical choice puzzling - if foreign aid is indeed given for strategic reasons then surely a donor country should account for the foreign aid given by other countries when making their own allocations. The same should be equally true if foreign aid is given for humanitarian reasons - if a very needy country is already receiving an abundant amount of foreign aid from other countries, a particular donor country may decide to dispense aid to a less needy but overlooked recipient country. Pooled models do not address this issue as they do not distinguish between donor countries while donor by donor regressions cannot address this issue because by construction they do not account for the allocations of other donor countries. \\

% % \indent\indent With Interestingly, we also find meaningful variation between countries in the relevance of that strategic interest variable in directing aid flows, and the effect of that variable has a noticeable upward trend over time. Indicating that in recent years more and more countries are directing aid to those countries that are most relevant to their strategic interests. 

% \indent\indent In what follows, we first give a brief overview of the literature before introducing our new measure of strategic interest. We then run our analyses of the motivations for foreign aid with our new measure using a hierarchical random effects model. We discuss the implications of our results before concluding. 


% One way to think about these matrices is that they provide data on N(N − 1) dyadic relationships among the N actors. A simpler way to think about them is that they de- scribe a single network of interactions, and can therefore be summarized using network modeling techniques. I do so by estimating general bilinear mixed effects (GBME) mod- els. They produce a simple, low-dimensional representation of the public relationships among a large number of individuals and groups. Actors that have a high probability of cooperative interactions are placed closely together in the latent network space, whereas those that have a high probability of conflictual interactions are located far away from each other. The positions are thus determined by condensing dyadic links into a hypothetical low-dimenstional space. As such, my approach is conceptually related (although using a different statistical approach) to recent projects that estimate ideal points for politicians and societal actors using campaign donations or Twitter follower patterns (Bonica, 2014; Barberá, 2014).

% Key findings from faid paper. Countries provide foreign aid as a result of strategic interest. 