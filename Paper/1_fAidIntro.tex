\section*{Introduction}
\label{intro} 

Human and economic catastrophes associated with natural hazards are obviously not new, even if new media have changed the way we are aware of them. The January 2010 earthquake in Haiti and the Indian Ocean tsunami of 2004 both generated much international media attention and unprecedented amounts of international pledges of aid from private charities, non-governmental organizations, governments, and multilateral organizations.1 Nonetheless, aid pledges made while media attention is at its peak may not always be disbursed, could take a long time to arrive, or may replace previously pledged aid. This raises the following questions: how much does foreign aid really increase in the aftermath of large disasters? Are aid surges sizable in relation to the estimated economic damages caused by disasters? And what determines the actual size of the surges?

Natural disasters are one of the major problems facing humankind. Be- tween 1980 and 2004, two million people were reported killed and five billion people cumulatively affected by around 7,000 natural disasters, according to the dataset maintained by the Centre for Research on the Epidemi- ology of Disasters (CRED) at University of Louvain (Belgium). The economic costs are considerable and rising. The direct economic damage from natural disasters between 1980–2004 is estimated at around \$1 trillion.

Donor countries may also provide relief with an eye to their own economic or geostrategic political interest (for example, Alesina and Dollar, 2000, and the references therein). Large disasters may destabilize governments. Aid to friendly governments could help these stay in power; withholding aid from not-so-friendly governments could destabilize them (Drury, Olson, and Van Belle, 2005). Disaster relief may also be used to protect investments in foreign countries, driving relief towards countries where the donors have large economic stakes.

United States - Iran earthquake - 2003

http://www.nytimes.com/2012/08/15/world/middleeast/us-vows-to-speed-aid-to-iran-earthquake-victims.html

Due to the earthquake, relations between the United States and Iran thawed. The U.S. usually treated Iran as part of the "axis of evil", as its President George W. Bush called the nation.[16] However, following the tremor White House spokesman Scott McClellan spoke on behalf of President Bush: "Our thoughts and prayers are with those who were injured and with the families of those who were killed."[5]


Fairfax County Urban Search and Rescue squad inspect earthquake damage in Bam
The U.S. offered direct humanitarian assistance to Iran. Iran initially declined this offer,[17] though later accepted it. On December 30 an 81-member emergency response team was deployed to Iran via U.S. military aircraft, consisting of search and rescue squads, aid coordinators, and medical support.[21] These were the first U.S. military airplanes to land in Iran for more than 20 years.[12]

In return, the state promised to comply with an agreement with the International Atomic Energy Agency which supports better monitoring of its nuclear interests. This led U.S. Secretary of State Colin Powell to suggest direct talks in the future.[16] Sanctions were temporarily relieved to help the rescue effort.[20] However, he also said that the U.S. was still concerned on other Iranian issues, such as the prospect of terrorism and the country's support of Hamas.[16]

\indent\indent  Foreign aid describes the transfer of resources from one government to another. Although the term itself suggests a humanitarian motive, scholars and experts have long debated whether it would be more accurate to ascribe foreign aid a strategic motive instead. With some exceptions \citep{bermeo:2008}, most scholars have found that donors prioritize strategic considerations when dispensing aid \citep{alesina:2000, berthelemy:2006}. \\

\indent\indent  This seeming consensus belies the inconsistency with which scholars conceptualize and measure strategic considerations, which have variously included bilateral trade intensity, UN voting scores, colonial legacies and regional dummies among others. In this paper we seek to rectify in fragmentation: First, we create an original measure of bilateral strategic interest that measures the latent distance between countries across the strategic policy space.  In doing so, we seek to provide a more coherent measure of strategic interest which incorporates many of the measures that previous papers have used. Further this measure improves upon existing measures of strategic interest in that it maps strategic interest onto a ``social space'', through which we can account for third order relationships between states \citep{hoff:2002}. \\ 

%%Scholars have variously used bilateral trade intensity \citep{berthelemy:2006, berthelemy:2004}, colonial legacies \citep{berthelemy:2004},  UN voting \citep{alesina:2000}, political orientation of the recipient country \citep{easterly:2008} to measure strategic interest. Other scholars take a different approach and investigate how much aid allocations can be ascribed to humanitarian reasons. These are broadly split along economic need \citep{collier:2002,nunnenkamp:2006,thiele:2007} and the quality of political governance \citep{neumayer:2005,dollar:2006}, the implication being that countries that fail to give aid along these criterion are acting in their strategic interest.\\

%\indent %These measures are at best imperfect and at worst, uninterpretable. As \citet{bermeo:2008} states,
%\begin{quote} `Perhaps the most puzzling conclusion of the existing literature is that a focus on trade partners, former colonies, and allies is somehow evidence against a development focus of aid. Instead, one could interpret this as evidence that donors give aid to the countries in which they most wish to pursue development. In this sense, donor interests and recipient needs are not mutually exclusive categories.'
%\end{quote} Meanwhile some have argued that donors who give to poor countries may not do it out of a %humanitarian impulse but because it is cheaper to buy interest in poorer countries \citep{demesquita:2007,stone:2006}. Conversely, a donor country may give to a poorly governed, undemocratic country for humanitarian reasons as well, North Korea being a prominent example.

\indent\indent  The existing lack of coherence in evaluating strategic interest extends to model specification. Papers which have empirically evaluated the dominance of strategic over humanitarian motives with some exceptions \citep{berthelemy:2006}, have done so by specifying models which pool all donors together or by running models for each donor country separately. In our model specification, we use a hierarchical random effects model with countries receiving aid nested in senders and senders nested in time. Applying this method will enable us to explicitly model the drivers of aid in an aggregate sense and to also explore how those drivers vary between senders. For now we present preliminary results that show our strategic interest variable does play a positive role in predicting aid flows between countries.

%%%%%%%%%%%%%%%%%%%%%%
% SM note: So we don't resolve these issues with our current model specification either. We could have tried to get at these issues using a network modeling approach, but as we discussed the bipartite nature of our data took away that option. In the next iteration of this paper, we can take a look into adding spatially lagged covariates to get at this issue. Hmmm, it might also be interesting to weight aid flows by our strategic interest variable. This would help us to test whether or not states follow their strategic partners in giving aid flows. 
%%%%%%%%%%%%%%%%%%%%%%

% We find this empirical choice puzzling - if foreign aid is indeed given for strategic reasons then surely a donor country should account for the foreign aid given by other countries when making their own allocations. The same should be equally true if foreign aid is given for humanitarian reasons - if a very needy country is already receiving an abundant amount of foreign aid from other countries, a particular donor country may decide to dispense aid to a less needy but overlooked recipient country. Pooled models do not address this issue as they do not distinguish between donor countries while donor by donor regressions cannot address this issue because by construction they do not account for the allocations of other donor countries. \\

% \indent\indent With Interestingly, we also find meaningful variation between countries in the relevance of that strategic interest variable in directing aid flows, and the effect of that variable has a noticeable upward trend over time. Indicating that in recent years more and more countries are directing aid to those countries that are most relevant to their strategic interests. 

\indent\indent In what follows, we first give a brief overview of the literature before introducing our new measure of strategic interest. We then run our analyses of the motivations for foreign aid with our new measure using a hierarchical random effects model. We discuss the implications of our results before concluding. 


% One way to think about these matrices is that they provide data on N(N − 1) dyadic relationships among the N actors. A simpler way to think about them is that they de- scribe a single network of interactions, and can therefore be summarized using network modeling techniques. I do so by estimating general bilinear mixed effects (GBME) mod- els. They produce a simple, low-dimensional representation of the public relationships among a large number of individuals and groups. Actors that have a high probability of cooperative interactions are placed closely together in the latent network space, whereas those that have a high probability of conflictual interactions are located far away from each other. The positions are thus determined by condensing dyadic links into a hypothetical low-dimenstional space. As such, my approach is conceptually related (although using a different statistical approach) to recent projects that estimate ideal points for politicians and societal actors using campaign donations or Twitter follower patterns (Bonica, 2014; Barberá, 2014).

% Key findings from faid paper. Countries provide foreign aid as a result of strategic interest. 