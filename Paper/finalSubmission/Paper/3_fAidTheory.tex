\section*{How Natural Disasters Affect Foreign Aid Allocations}

Only in the twentieth century has expending public resources to relieve the human suffering of foreigners shifted from being virtually inconceivable to relatively commonplace. The devastation wrought by the two world wars was particularly instrumental in bringing about this change. However, such aid was strictly intended to serve as temporary transfers that would facilitate a return to the previous status quo, rather than a long-term commitment to ``development'' as such. The turn toward promoting development was instead fostered by ongoing Cold War hostilities, which simultaneously promoted the use of aid to further donor's strategic goals while also building a new norm of rich countries aiding poor countries \citep{lancaster:2008}.

The role of mitigating disaster and suffering on the one hand and furthering strategic interest on the other are thus baked into the modern conception of foreign aid. This history also suggests that humanitarian aid, even if only initially meant to serve as a temporary expedient, may lead to the establishment of aid with longer-term strategic goals. Whether this pattern exists more generally and if so, whether it is driven primarily by strategic or humanitarian concerns is unclear however. The role of the Cold War in foreign aid's origin story  dictated that recipients of humanitarian aid were generally within a particular strategic bloc, making it difficult to untangle strategic from humanitarian drivers.  t.

As such, looking at how natural disasters affect foreign aid allocation is not only interesting in its own right but also provides an exogenous factor with which to identify the role of donor interest and recipient need in explaining patterns of aid commitments. To that end, we develop a set of hypotheses as to how natural disasters affect foreign aid allocations. Further, to better untangle the varying potential drivers, we disaggregate foreign aid into three types: humanitarian, civil society, and development aid. In doing so, we seek to offer a more nuanced understanding of the principle drivers of foreign aid allocations. 

\subsection*{Short-term Humanitarian Response to Natural Disasters}

Responding to natural disasters quickly and efficiently is often crucial to saving lives and alleviating human suffering as services like electricity, gas, water, and telecommunications may all be disrupted in the immediate period following a natural disaster. The timely deployment of humanitarian aid is the first response that donors can extend to countries struck by natural disaster. In what follows, we develop three hypotheses as to how the interaction between strategic interests and natural disaster severity can affect humanitarian aid allocation. 

We draw first from recent research in behavioral economics, which underscores the idea that different social contexts lead to varying behavior in identical situations \citep{kahneman:2003,do:2011}.\footnote{While there is evidence that non-governmental organizations are driven by the norms of humanitarian discourse when allocating aid \citep{buthe:2012}, evidence for similar behavior in governments has been mixed at best.} Natural disasters may reorient the social context of a dyadic relationship to encourage donors to increase aid to their strategic opponents. That is, the loss of human life and destruction of infrastructure, which natural disasters provoke, can temporarily serve to emphasize the human aspect of the bilateral relationship as opposed to the  political, economic, and military aspects that generally define foreign relations between two countries.

Moreover, if natural disasters do have a humanizing effect, then we might expect strategic opponents to be particularly sensitive to it. That is is, given that strategic opponents are more likely to ``otherize'' each other, then dyadic opponents must traverse a greater gap to humanize the other compared to dyadic allies \citep{de:2012}. On balance then, we would expect that donors do not discriminate between strategic opponents or strategic allies when dispensing aid. For example, historically hostile relations between the US and Cuba may mean that the baseline extent to which they ``otherize'' each other is much greater than in the US-Japan relationship, increasing the potential for Cubans to be humanized in American eyes. As such, we might expect American aid to Cuba rise to the level they would provide to Japan in the event of similar natural disasters. 

That is not to say that natural disasters can always bridge the divide among strategic opponents. For example, India and Pakistan have had an uneasy history of accepting aid from each other following natural disasters.\footnote{Ravishankar, Siddharth. ``Cooperation between India and Pakistan after Natural Disasters.'' \textit{Stimson Center}. 9 January 2015. Accessed September 2017: \url{https://www.stimson.org/content/cooperation-between-india-and-pakistan-after-natural-disasters}} In general, we contend only that natural disasters may make it more \textit{likely} that a strategic adversary will contribute aid because the humanitarian disaster temporarily reframes the context of bilateral relations. An understanding of the interaction between natural disasters and strategic interests affects humanitarian aid allocations based on social context thus leads us to the following hypothesis:
\begin{inparaenum}
  \item   Donors who are strategic opponents of the recipient are more likely than strategic allies to be sensitive to the humanizing effect of natural disasters. As such, following natural disasters, donors are likely to send \textbf{similar amounts of humanitarian aid to strategic allies and strategic opponents.}
\end{inparaenum}
 

Realist scholars offer an alternative perspective which proclaims that, ``foreign aid is today and will remain for some time an instrument of political power'' \citep{liska:1960}. Under this logic, donors commit aid primarily to recipient countries to further their own strategic interests. Extant literature on the drivers of foreign aid have put forward strong substantive evidence to support this viewpoint \citep{mckinley:1979, maizels:1984, schraeder.etal:1998, alesina:2000, berthelemy:2006, stone:2006, demesquita:2007, bermeo:2008, dreher:2015}. With regards to the interaction between natural disasters and strategic interests, it is in donor's self-interest to commit greater amounts of humanitarian aid to their strategic allies rather than opponents in the event of a natural disaster. A naive reading of the logic of realism would lead to the following hypothesis as to how the interaction between natural disasters and strategic aid affects humanitarian aid allocations: 
\begin{inparaenumb} 
  \item  Donors are driven by self-interest and in the event of natural disasters, donors are \textbf{likely to send less humanitarian aid to their strategic opponents vs their strategic allies}
\end{inparaenumb}
 
A more sophisticated realist perspective, however, suggests that natural disasters may present donors with a strategic opportunity to improve relations with strategic opponents.  As suggested in H1A, social context does matter, but only to the extent  that it limits the acceptable set of responses to natural disasters to the allocation of humanitarian aid (as opposed to, for example, the use of hostile overtures). However, donors may still seek to work within this framework of humanitarian altruism to forward their own interests. 

Indeed, disaster-afflicted countries appear to be sensitive to the possibility that accepting humanitarian aid from strategic opponents may come with ulterior motives. In 1999 for example, Venezuela experienced catastrophic flash floods and debris flows in Vargas State, which left as much as 10\% of the Vargas population dead \citep{wieczorek:2001}. US troops helped in the relief efforts by running helicopter rescue missions and working to provide clean water. However, consistent with his antagonism toward US hegemony in the region, President Hugo Chavez declined US assistance in rebuilding a critical highway, saying that while, ``he would accept American equipment if Venezuelan soldiers operated it...he did not want US troops in his country.''\footnote{Brand, Richard. ``Chavez assailed on handling of Venezuelan flood disaster.'' \textit{The Miami Herald}. 5 August 2001. Accessed September 2017: \url{http://www.latinamericanstudies.org/venezuela/venezuela-disaster.htm}.} Meanwhile, Iran categorically refused any aid from Israel following the 2003 Bam earthquake, though the Israeli government still encouraged its citizens to donate privately.\footnote{Popper, Nathaniel. ``Israelis Help Iran Victims Despite Rebuff.'' \textit{The Forward}. 2 January 2004. Accessed September 2017: \url{http://forward.com/news/6059/israelis-help-iran-victims-despite-rebuff/}} Indeed, even the US first turned down Russian aid for Hurricane Katrina before ultimately accepting it.\footnote{``U.S. accepts Russian Katrina aid.'' \textit{UPI}. 2 September 2005. Accessed September 2017. \url{https://www.upi.com/US-accepts-Russian-Katrina-aid/39221125680989/}.} Most recently, Venezuelan leader Nicolas Maduro's refused humanitarian aid to alleviate its food crisis under the reasoning that such aid is ``merely a pretext for regime change,'' demonstrating that i) some political actors also suspect that humanitarian aid may be strategically driven and that ii) the use of humanitarian aid for strategic purposes may extend beyond natural disasters (as this particular crisis was largely a function of political missteps).  \footnote{Taladrid, Stephania. ``Venezuela's Food Crisis Reaches A Breaking Point.'' \textit{The New Yorker.} 22 February 2019. Accessed March 2019: \url{https://www.newyorker.com/news/news-desk/venezuelas-food-crisis-reaches-a-breaking-point}}

 
There is also anecdotal evidence to suggest that aid given under such circumstances can serve to humanize and improve public perceptions of donors as well. For example, in the wake of US and South Korean aid for the North Korean famine, one refugee summarized his reaction to the US Institute for Peace this way: ``We were taught all these years that the South Koreans and Americans were our enemies. Now we see they are trying to feed us. We are wondering who our real enemies are'' \citep[9]{natsios:1999}. \citet{andrabi:2017} moreover, find that following the inflow of international aid sent to alleviate the damage done during an earthquake in Pakistan 2005, trust in Euoropeans and Americans was much higher in the affected population.  This evidence suggests that, at least in certain contexts, humanitarian aid can serve to improve relations with strategic opponents.

Note that the underlying assumption is that the donor country is ultimately motivated to further its own strategic interests. A priori, the donor country cannot know whether such overtures of humanitarian aid will improve relations with the recipient government, improve perceptions among the recipient population, or both. The incidence of a natural disaster merely provides the donor country a window of opportunity to do so, thus potentially giving it more latitude to further its strategic goals. Improving relations with the recipient government may, on the margin, deter recipient governments from taking actions that conflict with donor interests. Meanwhile improving perceptions of the donor country among the recipient population may also limit the extent to which a still hostile recipient government may enact policies that directly counter donor interests. Here, however, we are primarily interested in investigating whether donors are driven by this possibility when allocating aid, leading to our third hypotheses:


\begin{inparaenumc}
  \item   Donors see natural disasters as a strategic opportunity to improve their relations with strategic opponents and are thus are  likely to send \textbf{more humanitarian aid to strategic opponents versus allies.}
\end{inparaenumc}
 
\subsection*{Long-term Responses to Natural Disasters}

Donor countries may dispense aid that not only serves to immediately address the natural disaster at hand, but also  to further longer-term objectives. Here, we make a distinction between civil society aid and development aid. Civil society aid is aimed at supporting non-governmental organizations (NGOs) and their programs. The stated purpose of such aid is to empower grass-roots advocacy and improve governance and government accountability. Meanwhile, development aid is targeted toward promoting long-term economic development in a recipient country often through the building of infrastructure like roads and hospitals as well as through the promotion of human capital via technical training and education. In what follows, we develop hypotheses as to how the interaction between strategic interest and natural disasters can affect the allocation of these two different types of aid.

\subsubsection*{Natural Disasters as Strategic Opportunities}

Donors generally distribute aid to civil society not only for its intrinsic value but also, and arguably primarily, for its perceived instrumental value in either promoting democratization \citep{robinson:1995,ottaway:2000} or economic development \citep{kral:2013}. However, we make the distinction between civil society and development aid because while any given donor may commit civil society aid to promote economic development, lending support to civil society at all is an inherently political act.\footnote{Carothers, Thomas and Diane de Gramont. ``The Prickly Politics of Aid.'' \textit{Foreign Policy}. 12 May 2013. Accessed June 2018: \url{http://foreignpolicy.com/2013/05/21/the-prickly-politics-of-aid/}} From supporting the growth of government watch dogs to increasing the domestic capacity for grass roots advocacy, whether it is their intention or not, donors are able to exert influence over a recipients domestic politics by directing funds to civil society.

Thus if, as following the realist logic, foreign aid is used to promote donor interests, then donor governments should be especially inclined to increase the allocation of civil society aid.  With respect to natural disasters, countries may be motivated to give more civil society aid to their strategic opponents because the temporary suspension in the normal dynamics of the relationship represents a unique opportunity to increase civil society aid and initiate a shift in the nature of the bilateral relationship (as in H1C). Donors can seize on a country's inherent vulnerability following a natural disaster to  decide to \textit{strategically} increase their civil society aid so as to increase their chances of exerting domestic influence over the recipient countries. 

To draw a concrete example, following the 2004 Indian Ocean Earthquake and Tsunami, the US began committing aid to civil society groups in Somalia. Though the initial nominal amount was a drop in the bucket in absolute terms, considering that no aid was given to civil society in Somalia prior to the natural disaster and such aid has been steadily growing over the past decade, this represented a substantial change in US aid commitments to Somalia.\footnote{Data collected from USAID from: ``USAID Foreign Aid Explorer''. Accessed January 2019: \url{https://explorer.usaid.gov/}} Given that the U.S. had closed its embassy in Somalia in 1991 and only re-established diplomatic presence in 2018.\footnote{Watkins, Eli and Jennifer Hansler. ``State Department announces re-establishment of `permanent diplomatic presence' in Somalia.'' \textit{CNN.} 4 December 2018. Accessed January 2019: \url{https://edition.cnn.com/2018/12/04/politics/us-somalia-state-department/index.html}}, it seems plausible to interpret this as strategic gambit on the US' part to gain a foothold in Somalia, and if so, a successful one. Before jumping to conclusions however, note that the US also increased civil society aid to Indonesia at the same time.\footnote{Data collected from USAID from: ``USAID Foreign Aid Explorer''. Accessed January 2019: \url{https://explorer.usaid.gov/}} Given that Indonesia was affected much more severely by the earthquake than Somalia\footnote{``India, Indonesia, Maldives, Myanmar, Somalia, Thailand: Earthquake and Tsunami OCHA Situation Report No. 14''. \textit{ReliefWeb.} 7 January 2005. Accessed January 2019: \url{https://reliefweb.int/report/india/india-indonesia-maldives-myanmar-somalia-thailand-earthquake-and-tsunami-ocha-situation}} but had also enjoyed much closer ties to the U.S., it would be difficult to substantiate our proposed mechanism based on anecdotal evidence alone. We thus test the following hypothesis through statistical modelling:  
 

\begin{hypolist}[series=test]
\setcounter{hypolisti}{1}
\item Natural disasters present an opportune window for donors to exert influence over recipients who are their strategic opponents and as such, donors are more likely to send additional \textbf{civil society aid} to their strategic opponents
\end{hypolist}
 
If on the contrary, donors are purely driven by the potential intrinsic or instrumental payoffs of supporting civil society, then donors should be no more motivated to support the civil society of their strategic opponents over that of their strategic allies and we should find no support for this hypothesis. 

\subsubsection*{Natural Disasters and Development Aid}

Whereas humanitarian aid provides stop-gap measures to address the immediate aftermath of a natural disaster, the focus of development aid is to build the conditions for long-term, sustainable economic growth. Here we simply expect that donor countries are more likely to give this type of aid to countries that they want to see economically develop and prosper, namely, their strategic allies. This accords with the more simple notion of realism, similar to H1B, that countries will seek to support allies rather than opponents irrespective of the number of natural disasters. This results in the following hypothesis:


\begin{hypolist}[resume=test]
\item Donors are more likely to send greater \textbf{development aid} to their strategic allies irrespective of the number of natural disasters. 
\end{hypolist}
 

If on the contrary, donors seek only to promote development according to recipient need and without regard to its own potential benefit, then donors should be no more motivated to support the development of  their strategic allies over that of their strategic opponents and we should find no support for this hypothesis. 


\section*{Measuring Strategic Relationships}

One reason for evaluating the \textit{motivations} for aid and not aid \textit{outcomes} is that aid given for strategic reasons may still further development objectives, albeit incidentally, while aid given for humanitarian reasons may also bring unexpected strategic benefits \citep{maizels:1984}. However, evaluating the motivations for aid is not a straightforward process -- any given aid project may work toward providing assistance to a recipient country as well as strategic benefits to a donor country. 

Of critical importance to investigating whether strategic considerations (and by extension, the interaction between strategic considerations and humanitarian need) affects foreign aid considerations then is constructing a reliable measure of strategic interest. Unfortunately, we find that \citet[35]{alesina:2000}'s observation  that,  ``the measurement of what a `strategic interest' is varies from study to study and is occasionally tautological,'' still holds true.  Indeed, strategic interest has alternately been operationalized as: trade intensity \citep{berthelemy:2004,bermeo:2008,hoeffler:2011}, UN voting scores \citep{alesina:2000, alesina:2002,hoeffler:2011,dreher:fuchs:2015}, arms transfers \citep{maizels:1984}, colonial legacy \citep{alesina:2000, bermeo:2008, berthelemy:2004,berthelemy:2006,carnegie:2017}, alliances \citep{bermeo:2008,schraeder.etal:1998}, regional dummies \citep{bermeo:2008,berthelemy:2006, maizels:1984}, bilateral dummies \citep{alesina:2000, berthelemy:2004, berthelemy:2006}\footnote{A US-Egypt or US-Israel dummy seems to be the most common instance of a bilateral dummy.} or some combination of the above.\footnote{Meanwhile other papers take a negative approach and argue that any shortfall between what would theoretically be expected from poverty-efficient aid allocation and actual aid allocation \citep{collier:2002,nunnenkamp:2006,thiele:2007}, or similarly between a theoretical allocation based on good governance and actual aid allocation \citep{dollar:2006,neumayer:2005}, is evidence of strategic interest at play.} 

Such inconsistency in the operationalization of strategic interest is not simply a matter of using different variables to measure the same concept but a matter of using different variables to measure different \textit{aspects} of the underlying concept. However, while a dyad's strategic bilateral relationship is quite multifaceted, to date, there has not been a readily available measure of strategic relationships which captures its various aspects the same way that scholars have done for other complex concepts.\footnote{For example, Polity and Freedom House have provided measures or political institutions while the World Bank's World Governance Indicators (WGI) project provides measures for six dimensions of governance} To address this problem, we create a new measure of strategic interest that is able to account for varying aspects of strategic interest. 

\subsection*{A new measure of strategic relationships}

To generate a measure of strategic relationships we adopt a latent variable approach that enables us to estimate a relational measure of interest between countries by taking into account the direct and indirect ways in which states are connected across a variety of dimensions. Specifically, we utilize three dimensions of state relations to construct our strategic interest measure: dyadic alliances, UN voting, and joint membership in intergovernmental organizations (IGOs). We focus on these dimensions because each provides distinct a representation of the strategic relations between countries in the international system and have been commonly employed in the foreign aid literature. Alliances largely capture the strategic and military aspect of the dyadic relationships.  In contrast, joint membership in IGOs reflects the dyadic relationship across many  diverse issue areas expressed across correspondingly many fora while UN voting is better able to capture this relationship in a centralized forum. 

To estimate a measure of strategic interest across these dimensions, we take a network based approach that allows us to leverage both the direct and indirect ways in which states are connected to one another. To do this we employ a latent factor model as described in \citet{hoff:2005}. The model is structured as follows:

\begin{align}
\begin{aligned}
	Y = \textbf{u}_{i}^{T} & \textbf{u}_{j} + \epsilon_{ij} \text{, where} \\
	&\textbf{u}_{i} \in \mathbb{R}^{R=2}, \; i \in \{1, \ldots, n \} \\
	% &\Lambda \text{ a } K \times K \text{ diagonal matrix}
\end{aligned}
\end{align}

$Y$ here is a $n \times n$ undirected sociomatrix in which $y_{ij}$ designates whether there exists a link (e.g., an alliance) between $i$ and $j$. The goal of the model is to provide a projection of the systematic variation in $Y$ into a two-dimensional social space.\footnote{The latent factor model we utilize here is based on an eigvenvalue decomposition that seeks to represent relations between countries as the weighted inner-product of country-specific vectors of latent characteristics. In this application, we project our $n \times n$ sociomatrix into a $n \times 2$ matrix of country positions in a latent social space.} More precisely, the types of systematic variation that we are interested in include the concepts of (a) transitivity, (b) balance and (c) clusterability. Formally, a set of three countries $ijk$ is said to be transitive, if for whenever $y_{ij} = 1$ and $y_{jk} = 1$, we also observe that $y_{ik} = 1$. This follows the logic of `` a friend of a friend is a friend''. Meanwhile, the relationships between $ijk$ are said to be balanced if $y_{ij} \times y_{jk} \times y_{ki} >0$. Conceptually, if the relationship between $i$ and $j$ is ``positive'', then both will relate to another unit $k$ identically, either both positive or both negative. Finally, relationships between $ijk$ are said to be clusterable if it is balanced or all the relations are all negative. It is a relaxation of the concept of balance and seeks to capture groups where the measurements are positive within groups and negative between groups.

Thus third order dependencies suggest that ``knowing something about the relationship between $i$ and $j$ as well as between $i$ and $k$ may reveal something about the relationship between $i$ and $k$, even when we do not directly observe it'' \citep[141]{hoff_ward:2008}. Such dependences would seem especially relevant for our purposes, as one cannot understand the strategic relationship between two countries without taking into account their respective relationships with other countries. The importance for accounting for these dynamics have long been acknowledged in the foreign aid literature. \citet[877]{trumbull:1994} for example, note that, ``donors do make their decisions with knowledge of what each other are doing, and may actually act cooperatively. Any study that ignores the interrelationship of donor behavior risks problems with simultaneity bias.'' However, we find that until now, this critique has largely gone unaddressed by  existing analyses. 

The main advantage of calculating the latent space of different dyadic variables as opposed to using alternative specifications such as the S Score algorithm\footnote{\citet{leeds:2007}, for example, measure a states ``threat environment'' as the set of all states for which ones is contiguous with or which is a major power and with an S score below the population median.} is that it allows us to better account for indirect ties that states share. Indirect ties are accounted for within this framework because the latent factor model takes patterns such as transitivity into account. As a result, the relation between two actors can be inferred even if no
direct interaction between them is observed.

We employ this latent factor model on every year for each of our three measures.\footnote{The models are estimated via Gibbs sampling from the full conditional distributions of $\textbf{u}_{i}^{T} \textbf{u}_{j}$. For a more detailed discussion of this model, see \citet{hoff:2005}.} In Figure \ref{fig:polLat}, we present a visualization of the resultant latent space we calculated for each variable for the year 2005.

\begin{figure}[h!] 
\centering
	\begin{minipage}{.33\linewidth}
		\centering
		\label{fig:ally}
		\includegraphics[width = 1.1\textwidth]{\detokenize{ally_2005.jpg}}
		\caption*{(a) Alliances}
		\end{minipage}
		\hspace{0.2in}
		\begin{minipage}{.33\linewidth}
		\centering
		\label{fig:un}
		\includegraphics[width = 1.1\textwidth]{\detokenize{un_2005.jpg}}
		\caption*{(b) UN voting}
		\end{minipage}
		%\hspace{.2in}
		\begin{minipage}{.33\linewidth}
		\centering
		\label{fig:igo}
		\includegraphics[width = 1.1\textwidth]{\detokenize{igo_2005.jpg}}
		\caption*{(c) IGO membership}
	\end{minipage}
    \caption{Latent Spaces for components of Strategic Interest Measure during 2005}
\label{fig:polLat}
\end{figure}

\indent\indent Countries that cluster together in this two-dimensional latent space are more likely to interact with each other. The plots for alliances, UN voting and IGO membership suggest that there is distinct clustering among countries. Moreover, these clusters are different across the three measures, suggesting that each variable is indeed capturing different aspects of strategic interest.

\indent\indent After estimating the latent spaces for these components, we estimate the distance between each dyadic pair for the three components for each year. We then combine them in a principal components analysis (PCA) to reduce the dimensionality of our measure while retaining as much variance as possible to maximize our explanatory power. We estimate the PCA of these variables for each year separately\footnote{For each year, we conduct a bootstrap PCA of 1000 subsamples.} and use the first principal component for each year as our measure of strategic interest. For more information about how this PCA was conducted, please see the Online Appendix. The end result of this process is a measure of strategic interest that takes into account indirect ties while also accounting for multiple dimensions in which states interact with one another.\footnote{With regards to the strategic interest measure, we also estimate a model in which we incorporate the uncertainty in the estimation of our latent variable (see Figure \ref{fig:latVarUncert} in the Online Appendix).} 
 