\section*{Discussion}
\label{conclusion}

Our analysis suggests that a more nuanced understanding of the drivers of foreign aid is in order. While recent work has shown that accounting for the channel of aid delivery can go a long way toward understanding aid allocation decisions \citep{dietrich:2013,dietrich:2016}, we show that following natural disasters, donor countries actually direct greater levels of humanitarian aid to strategic opponents rather than allies. We argue that donor countries may allocate foreign aid in this way because they see natural disasters as an opportunity improve relations with their strategic opponents. As shown in our lag models, these findings are surprisingly persistent. 

Moreover, natural disasters not not only affect how donor countries allocate aid for short-term purposes. We find that strategic considerations also reign large when one considers the effect on the distribution of aid with longer-term targets.  Specifically, donor countries are more likely to distribute civil society aid to strategic adversaries as the numbers of natural disasters these countries face increase. Civil society aid inherently involves engagement and intervention in the domestic politics of a recipient country, an increase in civil society aid is indicative of a greater desire to increase donor influence over a recipient country, at least relative to development aid. 

Meanwhile we find that donors are more likely to give development aid to strategic allies irrespective of exogenous shocks such as natural disasters. Why might donors pursue a sophisticated realist strategy for humanitarian and civil society aid but a naive one for development aid? To answer this question, we argue that context matters; what may further strategic interest in one situation may not work for another. It is nevertheless useful to note that almost 60\% of the total aid flowing from donor countries can be categorized as development aid. This suggests that donors who seek to develop better relations with traditional strategic opponents by dispersing humanitarian and civil society aid recognize the inherent risk in this strategy and invest accordingly.  

These results should be of particular interest as climate change continues to increase the incidence and the intensity of natural disasters. They suggest that while countries that experience natural disasters can expect humanitarian aid even from their strategic adversaries, such help can also open the doors to efforts to influence domestic politics in line with the interests of donors who have historically been antagonistic.

 


