\section*{Introduction}
\label{intro} 

In the early morning hours of December 26, 2003, a massive earthquake measuring 6.3 on the Richter scale struck the city of Bam, Iran. Its effects were devastating.  Out of Bam's 100,000 residents, approximately 26,000 to 40,000 were killed. Those  who survived were left to grapple with the destruction of 70 to 90 percent of the city's housing infrastructure \citep{montazeri:2005}.\footnote{Fathi, Nazila. ``Deadly Earthquake Jolts City in Southeast Iran.'' \textit{The New York Times.} 26 December 2003. Accessed October 2017: \url{https://web.archive.org/web/20090620230700/http://www.nytimes.com/2003/12/26/international/26CND-QUAKE.html?ex=1225166400&en=c550b50a2ad59dd6&ei=5070}} As part of the international response that followed, more than 44 countries sent aid, including the United States, which contributed eight plane loads of medical and humanitarian supplies as well as several dozen teams of experts to the relief effort. 

 While the US response to the 2003 Bam earthquake was seemingly analogous to that of any foreign actor offering aid and support, \textit{a priori}, it was not obvious whether the US would send any humanitarian aid at all, to say nothing of whether Iran would accept it. Just the year prior, then-President George W. Bush had famously assigned Iran membership in the ``Axis of Evil'' \citep{heradstveit:2007}. Meanwhile, at the time of the earthquake US-Iranian relations were particularly delicate as the countries navigated the issue of nuclear weapons in Iran.\footnote {``Timeline: US-Iran ties.'' \textit{BBC News.} 16 January 2009. Accessed October 2017: \url{http://news.bbc.co.uk/2/hi/middle_east/3362443.stm}} Indeed, given the broader context of contentious bilateral relations, the process of transferring aid from the US to Iran entailed greater intentionality than normal. To initiate the flow of any aid, President Bush was obliged to institute a special 90-day measure to ease US sanctions on Iran\footnote{``US eases Iran sanctions to speed earthquake relief.'' \textit{China Daily}. 1 January 2004. Accessed October 2017: \url{http://www.chinadaily.com.cn/en/doc/2004-01/01/content_295063.htm}} -- which had been in place since 1979 and continue to be enforced to this day \citep{katzman:2018}. For Iran's part, accepting US aid meant allowing US military planes to land  on its soil, which they had spent the previous 20 years prohibiting.\footnote{``Iran Quake Toll May Hit 50,000.'' \textit{China Daily.} 31 December, 2003. Accessed October 2017: \url{http://www.chinadaily.com.cn/en/doc/2003-12/31/content_294833.htm}} For a country that had undergone a revolution in part because the US military was perceived to have had too strong a domestic influence, it was far from obvious that such an act would be perceived as benign.\footnote{``Geopolitical Diary: Tuesday Dec. 30, 2003.'' \textit{Stratfor}. 31 December 2003. Accessed June 2018: \url{https://www.stratfor.com/geopolitical-diary/geopolitical-diary-tuesday-dec-30-2003}} 
%\footnote{The US first imposed sanctions against Iran in 1979 during the US-Iran hostage crisis. While many assets have since been unfrozen, sanctions on a number of items, including military sales, financial assets, and real estate holdings remain in place }
\begin{figure}
  \centering
  \includegraphics[width = .9\textwidth]{US_Iran_aid}
  \caption{US aid commitments to Iran, 2002 - 2013}
  \label{fig:us_iran}
\end{figure}

Yet, the Bam earthquake led not only to an increase in US humanitarian aid to Iran,  albeit temporarily, it was followed by other types of aid as well. Figure \ref{fig:us_iran} shows that after 2004, aid commitments to  "strengthen civil society" increased markedly and consistently, reaching its apex with the creation of the 2006 "Iran Democracy Fund" to promote democracy in Iran.\footnote{Carpenter, J. Scott. ``After the Crackdown: The Iran Democracy Fund.'' \textit{The Washington Institute for Near East Policy, PolicyWatch 1576.} 8 September 2009. Accessed May 2018: \url{http://www.washingtoninstitute.org/policy-analysis/view/after-the-crackdown-the-iran-democracy-fund}} Meanwhile, US aid for a variety of developmental purposes, (i.e. economic and development policy and planning, infectious disease control, social/welfare services) also increased sporadically following 2003. This is particularly noteworthy given that Iran has generally been barred from receiving US foreign aid since  the US State Department designated it a ``state sponsor of terrorism'' in 1984 \citep{samore:2015}.\footnote{Available data from AidData and the OECD suggest that the US did not commit any aid to Iran from 1974 to 2001.} Why did the US send humanitarian aid to Iran despite objectively hostile extant relations? Was this event \textit{sui generis} or is it possible to observe other dyadic pairs acting in a similar fashion? If so, does the occurrence of a natural disaster also lead donors to distribute other types of aid to strategic opponents?   

Answering these questions has important implications for our understanding of how donors use foreign aid. Furthering such an understanding is especially pressing given that the occurrences of natural disasters are likely to increase with changing climate conditions. Meanwhile, in light of an existing literature that finds that donors are more likely to allocate aid to strategic allies, a more nuanced understanding of what drives foreign aid allocations is necessary to answer these questions. To do this, we begin by first disaggregating foreign aid into three types: humanitarian, civil society, and development aid. Humanitarian aid is meant as a stop-gap measure to help recipient countries return to their status quo, while the latter two types of aid are targeted towards catalyzing long term change. Specifically, civil society aid is often used to improve governance outcomes, which provides donors an avenue through which to wade into the domestic politics of recipient states \citep{ottaway:2000, henderson:2002, resnick2012foreign, spina:2014}. Meanwhile, development aid is primarily focused on promoting economic development.
%\footnote{More specifically, some argue that the lack of good governance and state capacity in developing countries have stymied the ability for foreign aid to promote development. As such, the promotion of civil society is seen as important to the successful implementation of foreign aid projects.}
% should this be "citizens of" countries with which they have ...
% further seek to forward their long-term strategic interests as they 
We show that following a natural disaster donor countries actually give more humanitarian aid to strategic opponents. We argue that this is because donors use natural disasters as an opportunity to ingratiate themselves with countries they have historically shared hostile relations with. Additionally, we find that while natural disasters prompt donors to increase civil society aid to strategic opponents for similar reasons, they conversely push donors to give more development aid to strategic allies. In all, we argue that while donors do use aid to promote their strategic interest, the tactics they employ to do so can depend highly on context.  We evaluate these claims using a new measure of strategic interest that: 1) accounts for the indirect ties states share 2) and incorporates a variety of dimensions of strategic interest. 

In what follows, we first give a brief overview of the existing literature on natural disasters and foreign aid allocations before outlining our hypotheses. We then introduce our new measure of strategic interest, and present our empirical analysis of how natural disasters condition foreign aid allocation decisions. 