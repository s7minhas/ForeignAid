\section*{Data}
\label{data}

\subsection*{Aid flows}

Our data from foreign aid flows is taken from the AidData project \citep{tierney2011more}. This database includes information on over a million aid activities from the 1940s to the present. We use the country level aggregated version of this database to create a directed-dyadic dataset of total aid dollars committed. In this analysis, we focus specifically on OECD donor countries as they both are the best able and have the best incentive to give foreign aid to advance their strategic interests. In the final tally, our dataset includes the 18 most active senders\footnote{More specifically, the included donor countries are: Australia, Belgium, Canada, Denmark, France, Germany, Greece, Iceland, Ireland, Italy, Luxembourg, the Netherlands, Norway, Portugal, Spain, Sweden, the United Kingdom and the United States.} and 167 receivers of aid flows from 1975 to 2006. Accounting for all possible senders of aid during this time frame is difficult because of the amount of missing data. That being said, issues with missingness in our dataset still exist and we deal with them by employing a multiple imputation method developed by \citet{hoff:2007} and shown to have good performance by \citet{hollenbach:2014}. %However, even with the limited number of senders in this version of our analysis we still have approximiately 40,000 observations worth of data to work with.

We use the AidData's Sector coding scheme in order to disaggregate bilateral ODA into humanitarian aid, development aid, and civil society aid.\footnote{``AidData's Sector Coding Scheme.'' \url{http://docs.aiddata.org/ad4/files/aiddata_coding_scheme_0.pdf}}  To that end, our measure of humanitarian aid encompasses the sectors of: 

\begin{quote}
	``Emergency Response'', ``Reconstruction Relief'', and ``Disaster Prevention and Preparedness.''
\end{quote}

\noindent Meanwhile, civil society aid is measured as aid to the sectors of:

\begin{quote}
	``Government and Civil Society',  ``Women'', ``Support to Non-Governmental Organizations and Governmental Organizations''.
\end{quote}

\noindent Finally, development aid is defined as aid given to the following sectors: 

\begin{quote}
	``Education'', ``Health'', ``Water Sanitation'', ``Other Infrastructure and Services'', ``Economic Infrastructure and Services'', 'Environmental Protection', ``Other Social Infrastructure and Services'', ``Agriculture Forestry and Fishing'', ``Industry, Mining and Construction'', ``Other Development Aid'', '``Food Aid'' and ``Debt Relief''. 
\end{quote}

We note that bilateral ODA often represents only one channel through which a donor country may allocate foreign aid and that an increasing number of papers have argued for accounting for the heterogeneity of aid channels donors may use when estimating drivers of foreign aid \citep{nunnenkamp:2011,buthecheng:2013,dietrich:2013}. For our paper, we choose to focus solely on bilateral aid in order to maintain greater comparability with previous studies. % but we can look at the differentiated effects as robustness checks?

\subsection*{Strategic Interest}

As previously stated, for our measure of political strategic relationships, we conducted a PCA on the latent distances for alliances, UN voting and joint IGO membership. Data for alliances was retrieved from the Correlates of War (COW) Formal Alliance dataset \citep{gibler:2009}. Following \citep{demesquita:1975} and \citep{signorino:1999}, we distinguish between different types of alliances with the following weighting scheme: 0 = no alliance, 1 = entente, 2 = neutrality or nonaggression pact, 3 = mutual defense pact. 

\indent\indent UN voting data was obtained from the United Nations General Assembly Data set \citep{strezhnev:2012}. Here we calculate the proportion of times two states agree out of the total number of votes they both voted on. Agreement means either both vote yes, both vote no, or both abstain. This measure is similar to the `voting similarity index' readily available from the dataset except the voting similarity index does not account for mutual abstentions. 

\indent\indent  Meanwhile IGO voting data was obtained from the Correlates of War International Governmental Organizations Data Set. \citep{pevehouse:2004}. Dyads were coded as 1 if they belonged to the same IGO as a full member or an associate member and coded as 0 if one or both of them was an observer, had no membership, was not yet a state or was missing data.\footnote{Note we had attempted to make distinctions between different types of membership much like for alliances but found that very few states were noted to be Associate Members or Observers of an IGO for the time period that we are conducting our analysis. Thus we chose to use the simpler coding scheme.} 

\subsection*{Natural Disasters}

Almost all the empirical work on natural disasters relies on the publicly available Emergency Events Database (EM-DAT) maintained by the Center for Research on the Epidemiology of Disasters (CRED) at the Catholic University of Louvain, Belgium\footnote{\url{http://www.emdat.be/}}. EM- DAT defines a disaster as a natural situation or event which overwhelms local capacity and/or necessitates a request for external assistance. For a disaster to be entered into the EM-DAT database, at least one of the following criteria must be met: i) 10 or more people are reported killed; ii) 100 people are reported affected; iii) a state of emergency is declared; or iv) a call for international assistance is issued.  We  use a count of the number of natural disasters a country has experienced a year as our measure of natural disaster severity. Disasters can be hydro-meteorological, including floods, wave surges, storms, droughts, landslides and avalanches; geophysical, including earthquakes, tsunamis and volcanic eruptions; and biological, covering epidemics and insect infestations (the latter are less frequent).
 %The disaster impact data reported in the EM-DAT database consists of direct damages (e.g., value of damage to infrastructure, crops, and housing in current dollars), the number of people killed, and the number of people affected.

%\footnote{Note that another possible measure that we could use is the total number of people affected by a natural disaster. However, according to the EM-DAT Guidelines (\url{https://www.emdat.be/guidelines}): ``the indicator affected is often reported and is widely used by different actors to convey the extent, impact, or severity of a disaster in non-spatial terms.  The ambiguity in the definitions and the different criteria and methods of estimation produce vastly different numbers, which are rarely comparable.'' Given the difficulty in using this measure to compare across countries, we omit using this as a measure of natural disaster severity. }

\subsection*{Developmental Need}

In addition to our dyadic strategic relationship measures, we include a number of covariates to captute characteristics of the countries receiving aid.

For our measures of developmental need, we use (1) Log GDP per capita and (2) life expectancy at birth. This measure ``indicates the number of years a newborn infant would live if prevailing patterns of mortality at the time of its birth were to stay the same throughout its life.'' Both of these measures are extracted from the World \citet{wb:2013}.  

%As Cavallo and Noy (2011) observe, many of the events reported in this database are quite small and are unlikely to have any significant impact on aid disbursements and on the macro-economy more generally. We therefore limit our investigation to disasters in which the number of people killed is above the mean for the entire dataset (more on this below).

%\indent\indent We also use a count of the number of natural disasters a country has experienced a year from the Emergency Disasters Database (EM-DAT) database \citep{emdat:2009}. For a disaster to be included into the database, at least one of the following criteria must be fulfilled: (a) Ten or more people reported killed (b) A hundred or more people reported affected (c) Declaration of a state of emergency (d) Call for international assistance. 

%\indent\indent These two measures of humanitarian need were chosen to reflect as much as possible the humanitarian need of a particular country. We eschewed using GDP per capita as our measure of humanitarian need in favor of life expectancy, which offers a more holistic measure of the level of health, education and income of a country. Life expectancy in turn was used instead of the UN Human Development Index as it was found that life expectancy is highly correlated with the UN HDI with better coverage.\citep{cahill:2005}. Meanwhile natural disasters were included as the incidence of natural disasters are seen as exogenous to a country's current development (though of course the \textit{impact} of a natural disaster is not). 

\subsection*{Additional Covariates}

We also include a number of covariates in our model, including macroeconomic variables and measures for political institutions. For our macroeconomic indicators, we use GDP per capita, available from the World Bank \citep{wb:2013}. For our measure of political institutions, we use Polity IV data available from the Center for Systemic Peace \citep{gurr:2010}. Polity IV captures differences in regime characteristics on a 21 point scale ranging from -10 (hereditary monarchy) to +10 (consolidated democracy).  Note we rescale Polity IV to range from 1 to 21 for greater ease of interpretation. 

\indent\indent We control for the incidence of civil war as incidence of civil war in a recipient country certainly informs the ability for a donor country to dispense aid. We do so with data retrieved from the Uppsala Conflict Data Program (UCDP)/International Peace Research Institute (PRIO) Armed Conflict Database. \citep{gleditsch:2002}. We code as civil war any armed conflict which either (a) ``Internal armed conflict occurs between the government of a state and one or more internal opposition group(s) without intervention from other states'' or (b) ``Internationalized internal armed conflict occurs between the government of a state and one or more internal opposition group(s) with intervention from other states (secondary parties) on one or both sides.''

\indent\indent Finally for our data on former colonies, we used the Colonial History Data Set from the Issue Correlates of War (ICOW) Project \citep{hensel:2009}. This variable is coded as a one when the receiver in a sender-receiver dyad is a former colony of the sender and zero otherwise. 

% sm: we'll leave this for another paper and just make a footnote about imputing this time around
% \subsection*{Multiple Imputation}

% In our estimation, authors who have investigated the determinants of foreign aid have largely handled missing data problems by using list-wise deletion \footnote{We surmise that a number of authors use list-wise deletion in their datasets as they do not explicitely acknowledge any issues with missing data problems in their papers \citep{alesina:2000,bermeo:2008,hoeffler:2011,dietrich:2013}. Some authors do refer to their use of list-wise deletion more concretely though. For example, With regards to his dataset, \citet{berthelemy:2006} notes that, `Of course the number of observations will depend on the availability of explanatory variables. THis availability being taken into account, I still have approximately 36,000 observations.' Meanwhile, \citet{bermeo:2008} acknowledges that some of her explanatory variables have levels of missingness that could be problematic for the validity of her subsequent analyeses. However, her solution is to use alternate variables with better coverage, which while mitigating the missing data problem, does not completely solve it.} However as \citet{honaker:2010}, employing list-wise deletion to handle missing data problems may seriously confound the validity of any subsequent analysis.

% We deal with the missing data problem by developing a multiple imputation model for dyadic data based on \citep{hoff:2007}.

% Summary statistics

% Robustness checks

% channels of aid
% individual donors
% heckman selection --- decision 