\section*{Data}
\label{data}

\subsection*{Aid flows}

Our data from foreign aid flows is taken from the AidData project \citep{tierney2011more}. This database includes information on over a million aid activities from the 1940s to the present. We use the country level aggregated version of this database to create a directed-dyadic dataset of total aid dollars committed. In this analysis, we focus specifically on OECD donor countries as they both are the best able and have the best incentive to give foreign aid to advance their strategic interests. In the final tally, our dataset includes the 18 most active senders\footnote{More specifically, the included donor countries are: Australia, Belgium, Canada, Denmark, France, Germany, Greece, Iceland, Ireland, Italy, Luxembourg, the Netherlands, Norway, Portugal, Spain, Sweden, the United Kingdom and the United States. These countries were chosen both to maximize comparability with previous work as well as for reasons of data availability. Research on non-DAC donors suggests that like DAC donors, they seem to be primarily driven by strategic motivations in distributing aid \citep{neumayer:2003,dreher_etal:2011,fuchs:Vadlamannati:2013,dreher:2015,dreher:2018}.  Existing evidence suggests that non-DAC donors do seem more likely to give aid following a natural disaster however \citep{dreher_etal:2011}, though they still only account for at most 12\% of humanitarian aid in any given year \citep{harmer:2005}. This research suggests that our findings might be even stronger among non-DAC donors. Future work investigating this possibility will become increasingly important the more foreign aid non-DAC donors distribute. } and 167 receivers of aid flows from 1975 to 2005. Accounting for all possible senders of aid during this time frame is difficult because of the amount of missing data. That being said, issues with missingness in our dataset still exist and we deal with them by employing a multiple imputation method developed by \citet{hoff:2007} and shown to have good performance by \citet{hollenbach:2014}. %However, even with the limited number of senders in this version of our analysis we still have approximiately 40,000 observations worth of data to work with.

We use the AidData's Sector coding scheme in order to disaggregate bilateral ODA into humanitarian aid, development aid, and civil society aid.\footnote{``AidData's Sector Coding Scheme.'' \url{http://docs.aiddata.org/ad4/files/aiddata_coding_scheme_0.pdf}}  To that end, our measure of humanitarian aid encompasses the sectors of:\\ 

% \begin{quote}
% 	``Emergency Response'', ``Reconstruction Relief'', and ``Disaster Prevention and Preparedness''.
% \end{quote}

\begin{tabular}{lll}
	\hline
    \multirow{2}{*}{``Emergency response''} & \multirow{2}{*}{``Reconstruction Relief''} & ``Disaster Prevention  \\[-.5mm]
	~ & ~ &  $\quad$and Preparedness'' \\
    \hline
\end{tabular}\\

\noindent Meanwhile, civil society aid is measured as aid to the sectors of:\\

% \begin{quote}
% 	``Government and Civil Society',  ``Women'', ``Support to Non-Governmental Organizations and Governmental Organizations''.
% \end{quote}

\begin{tabular}{lll}
	\hline
	\multirow{3}{*}{``Government and Civil Society''} & \multirow{3}{*}{``Women''} & ``Support to Non-Governmental  \\[-.5mm]
    ~ & ~ & $\quad$Organizations and Governmental \\[-.5mm]
    ~ & ~ &  $\quad$Organizations'' \\
    \hline
\end{tabular}\\

\noindent Finally, development aid is defined as aid given to the following sectors:\\ 

\begin{tabular}{lll}
  \hline
  ``Education'' & ``Health'' & ``Water Sanitation'' \\
  ``Other Infrastructure & ``Economic Infrastructure & \multirow{2}{*}{``Environmental Protection''} \\[-.5mm]
  $\quad$ and Services'' & $\quad$ and Services'' & ~ \\
  ``Other Social  & ``Agriculture Forestry & ``Industry, Mining \\[-.5mm]
  $\quad$Infrastructure and Services'' & $\quad$ and Fishing'' &  $\quad$and Construction'' \\
  ``Other Development Aid'' & '``Food Aid'' & ``Debt Relief'' \\
  \hline
\end{tabular}\\

We note that bilateral ODA often represents only one channel through which donors may allocate foreign aid and that an increasing number of papers have argued for accounting for the heterogeneity of aid channels donors may use when estimating drivers of foreign aid \citep{nunnenkamp:2011,buthecheng:2013,dietrich:2013}. Here, we choose to focus solely on bilateral aid in order to maintain greater comparability with previous studies. % but we can look at the differentiated effects as robustness checks?

\subsection*{Strategic Interest}

As previously stated, we created our measure of strategic relationships by conducting a PCA on the latent distances for alliances, UN voting and joint IGO membership. Data for alliances was retrieved from the Correlates of War (COW) Formal Alliance dataset \citep{gibler:2009}. Following \citet{demesquita:1975} and \citet{signorino:1999}, we distinguish between different types of alliances with the following weighting scheme: 0 = no alliance, 1 = entente, 2 = neutrality or nonaggression pact, 3 = mutual defense pact.\footnote{Note, as for alliances, we had attempted to distinguish between different types of membership but found that very few states were listed as Associate Members or Observers of an IGO for the time period that we are conducting our analysis. Thus we used the simpler coding scheme.} 

\indent\indent UN voting data was obtained from the United Nations General Assembly Data set \citep{strezhnev:2012}.  We calculate the proportion of times two states agree out of the total number of votes they both voted on. Agreement means either both vote yes, both vote no, or both abstain. This measure is similar to the `voting similarity index' readily available from the dataset except the voting similarity index does not account for mutual abstentions. 

\indent\indent  Meanwhile IGO voting data was obtained from the Correlates of War International Governmental Organizations Data Set \citep{pevehouse:2004}. A total of 529 IGOs across a broad swath of topics, including trade, communications, and health and security,  are represented in this dataset. Dyads were coded as 1 if they belonged to the same IGO as a full member or an associate member and coded as 0 if one or both of them was an observer, had no membership, was not yet a state or was missing data. \footnote{Information on the IGOs included in the dataset are available from the Correlates of War website: \url{http://www.correlatesofwar.org/data-sets/IGOs}} 

\subsection*{Natural Disasters}
%\footnote{\url{http://www.emdat.be/}}
Almost all the empirical work on natural disasters relies on the publicly available Emergency Events Database (EM-DAT) maintained by the Center for Research on the Epidemiology of Disasters (CRED) at the Catholic University of Louvain, Belgium. EM- DAT defines a disaster as a natural situation or event which overwhelms local capacity and/or necessitates a request for external assistance. For a disaster to be entered into the EM-DAT database, at least one of the following criteria must be met: i) 10 or more people are reported killed; ii) 100 people are reported affected; iii) a state of emergency is declared; or iv) a call for international assistance is issued.  We  use a count of the number of natural disasters a country has experienced a year as our measure of natural disaster severity. 

\subsection*{Additional Covariates}

In addition to our dyadic strategic relationship measures, we include a number of covariates to capture characteristics of aid recipients.

For our measure of political institutions, we use Polity IV data available from the Center for Systemic Peace \citep{gurr:2010}. Polity IV captures differences in regime characteristics on a 21 point scale ranging from -10 (hereditary monarchy) to +10 (consolidated democracy), rescaling it to range from 1 to 21 for greater ease of interpretation.   We also controlled for colonial history using the Colonial History Data Set from the Issue Correlates of War Project \citep{hensel:2009}. This variable is coded as a one when the receiver in a sender-receiver dyad is a former colony of the sender and zero otherwise. 

Meanwhile, for our measures of developmental need, we use (1) Log GDP per capita and (2) life expectancy at birth.  Both of these measures are extracted from the World \citet{wb:2013}. Finally, we control for the incidence of civil war in a recipient country as it informs the ability for a donor country to dispense aid. We do so with data retrieved from the Uppsala Conflict Data Program (UCDP)/International Peace Research Institute (PRIO) Armed Conflict Database. \citep{gleditsch:2002}. We code as civil war any armed conflict which either (a) ``Internal armed conflict occurs between the government of a state and one or more internal opposition group(s) without intervention from other states'' or (b) ``Internationalized internal armed conflict occurs between the government of a state and one or more internal opposition group(s) with intervention from other states (secondary parties) on one or both sides.''