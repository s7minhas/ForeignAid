\documentclass[12pt]{article} 

%%%%%%%%%%%%%%%%%%%%%%%%%%%%%%%%%%%%%%%%%%%%%%%%%%
%%%%%%%%%%%%%%%%%%%% PREAMBLE %%%%%%%%%%%%%%%%%%%%
%%%%%%%%%%%%%%%%%%%%%%%%%%%%%%%%%%%%%%%%%%%%%%%%%%


% -------------------- defaults -------------------- %
% load lots o' packages

% layout control
\usepackage[paper=a4paper,left=25mm,right=25mm,top=20mm,bottom=25mm]{geometry}
\usepackage[onehalfspacing]{setspace}
\setlength{\parskip}{.5em}
\usepackage{rotating}
\usepackage{setspace}
\usepackage{fancyhdr}
\usepackage{parallel}
\usepackage{parcolumns}
\usepackage{pdflscape}
% math typesetting
\usepackage{array}
\usepackage{amsmath}
\usepackage{amssymb}
\usepackage{amsfonts}

% tables
\usepackage{tabularx}
\usepackage{booktabs}
\usepackage{multicol}
\usepackage{multirow}
\usepackage{longtable}

\usepackage[%
decimalsymbol=.,
digitsep=fullstop
]{siunitx}

% to adapt caption style
\usepackage[font={small},labelfont=bf]{caption}

% references
\usepackage[longnamesfirst]{natbib}
\bibpunct{(}{)}{;}{a}{}{,}

% footnotes at bottom
\usepackage[bottom]{footmisc}

% to change enumeration symbols begin{enumerate}[(a)]
\usepackage{enumerate}

% to make enumerations and itemizations within paragraphs or
% lines. f.i. begin{inparaenum} for (a) is (b) and (c)
\usepackage{paralist}

% to colorize links in document. See color specification below
\usepackage[x11names]{xcolor}

% for multiple references and insertion of the word "figure" or "table"
% \usepackage{cleveref}

% load the hyper-references package and set document info
\usepackage[pdftex]{hyperref}

% graphics stuff
\usepackage{subfig}
\usepackage{graphicx}
\usepackage[space]{grffile} % allows us to specify directories that have spaces
\usepackage[section]{placeins} % prevents floats from moving past a \FloatBarrier or section
\usepackage{tikz}
% \usepackage{pgfplots}

% define clickable links and their colors
\hypersetup{
	unicode=false,          % non-Latin characters in Acrobat's bookmarks
	pdftoolbar=true,        % show Acrobat's toolbar?
	pdfmenubar=true,        % show Acrobat's menu?
	pdffitwindow=false,     % window fit to page when opened
	pdfstartview={FitH},    % fits the width of the page to the window
	pdfnewwindow=true,%
	pdfauthor={Cassy Dorff and Shahryar Minhas},%
	pdftitle={Title},%
	colorlinks,%
	citecolor=black,%
	filecolor=black,%
	linkcolor=black,%
	urlcolor=RoyalBlue4%
	}

% Including External Code
\usepackage{verbatim}
\usepackage{listings}
\lstset{
	language=R,
	basicstyle=\scriptsize\ttfamily,
	commentstyle=\ttfamily\color{gray},
	numbers=left,
	numberstyle=\ttfamily\color{gray}\footnotesize,
	stepnumber=1,
	numbersep=5pt,
	backgroundcolor=\color{white},
	showspaces=false,
	showstringspaces=false,
	showtabs=false,
	frame=single,
	tabsize=2,
	captionpos=b,
	breaklines=true,
	breakatwhitespace=false,
	title=\lstname,
	escapeinside={},
	keywordstyle={},
	morekeywords={}
	}

% -------------------------------------------------- %


% -------------------- title -------------------- %

\title{Something, something, something about Foreign Aid}
\author{	Cindy Cheng \\
	\texttt{cindy.cheng@duke.edu}
	\and
	Shahryar Minhas \\
	\texttt{shahryar.minhas@duke.edu}}
\date{\today}

% \setlength{\headheight}{15pt}
% \setlength{\headsep}{20pt}
% \pagestyle{fancyplain}
 
% \fancyhf{}
 
% \lhead{\fancyplain{}{}}
% \chead{\fancyplain{}{}}
% \rhead{\fancyplain{}{}}
% \rfoot{\fancyplain{}{}}

% ----------------------------------------------- %


% -------------------- customizations -------------------- %

% define the includegraphics search path
% \graphicspath{{Graphics/}}

% easy commands for number propers
\newcommand{\first}{$1^{\text{st}}$}
\newcommand{\second}{$2^{\text{nd}}$}
\newcommand{\third}{$3^{\text{rd}}$}
\newcommand{\nth}[1]{${#1}^{\text{th}}$}

% easy command for boldface math symbols
\newcommand{\mbs}[1]{\boldsymbol{#1}}

% define bibliography style

\graphicspath{{/Users/janus829/Desktop/Research/ButheProjects/ForeignAid/Graphics/}}

% -------------------------------------------------------- %


%%%%%%%%%%%%%%%%%%%%%%%%%%%%%%%%%%%%%%%%%%%%%%%%%%
%%%%%%%%%%%%%%%%%%%% DOCUMENT %%%%%%%%%%%%%%%%%%%%
%%%%%%%%%%%%%%%%%%%%%%%%%%%%%%%%%%%%%%%%%%%%%%%%%%

\doublespacing 

\begin{document}

\maketitle

\begin{abstract}

\singlespacing{
	
	Many studies have come out in recent years trying to predict foreign aid flows \citep{alesina:2000,dreher:2012,werker:2012}. To that end, scholars have advanced and tested a number of hypotheses with regards to two related questions: i) why some countries recieve foriegn aid and others do not ii) why some countries recieve certain levels of foreign aid given that they recieve foreign aid at all.  These range from the economic need of the recipient country, the strategic interest of the donor country or the cultural affinity between the recipient and donor country, among others. \\

\indent\indent	Despite the scholarly attention devoted to this subject, there is not only little consensus from paper to paper as to which motivations are more important in determining foreign aid allocation but little consensus within papers as well. Many scholars find for example that the factors which strongly determine foreign aid flows when they analyze all donor countries together are often inconsistent with the motivations they find to be important when they investigate the determinants of foreign aid allocations each donor country individually \citep{schraeder.etal:1998,berthelemy:2006}. \\

\indent\indent	Each of these approaches analyzes foreign aid flows in a dyadic context where they assume that the flow of aid between any particular dyad is independent of any other.  Yet, scholars are well aware that there may be dependence in foreign aid allocations. Some scholars hypothesize for example that donors exhibit 'herding' \citep{frot:2011} or 'lead donorship' \citep{steinwand:2014}\footnote{ Wherin a recipient country is likely to recive the majority of its aid from one country, the assumption being that other donor countries are dissuaded from giving aid to that recipient country precisely because it already has a major donor} behavior.
 



	As many studies (Ward, Siverson, XAo xxxx; Cranmer, Desmareis, Menninga xxxx; others) have shown this assumption is often untenable within IR structures. 

	We propose an analysis of foreign aid in a network context where

	Controlling for network effects will allow for a more precise test of even teh monadic and dyadic level coefficient estimates that have been proposed

}

\end{abstract}

\newpage

%%%%%%%% INTRO %%%%%%%%
\section*{Introduction}
 

\indent Foreign aid describes the transfer of resources from one government to another. Although the very term suggests a humanitarian motive, scholars and experts have long debated whether it would be more accurate to ascribe foreign aid a strategic motive instead. With some exceptions \citep{bermeo:2008}, most scholars have found that donors prioritize strategic considerations when dispensing aid \citep{alesina:2000, berthelemy:2006}. \\
\indent This seeming consensus belies the fact that how scholars conceputalize and measure strategic considerations is quite inconsistent from paper to paper. Scholars have variously used bilateral trade intensity \citep{berthelemy:2006, berthelemy:2004}, colonial legacies \citep{berthelemy:2004},  UN voting \citep{alesina:2000}, political orientation of the recipient country \citep{easterly:2008} to measure strategic interest. Other scholars take a different approach and investigate how much aid allocations can be ascribed to humanitarian reasons. These are broadly split along economic need \citep{collier:2002,nunnenkamp:2006,thiele:2007} and the quality of political governance \citep{neumayer:2005,dollar:2006}, the implication being that countries that fail to give aid along these criterion are acting in their strategic interest.
\indent These measures are at best imperfect and at worst, uninterpretable. As \citept{bermo:2008} states,
\begin{quote} `Perhaps the most puzzling conclusion of the existing literature is that a focus on trade partners, former colonies, and allies is somehow evidence against a development focus of aid. Instead, one could interpret this as evidence that donors give aid to the countries in which they most wish to pursue development. In this sense, donor interests and recipient needs are not mutually exclusive categories.'
\end{quote} Meanwhile some have argued that donors who give to poor countries may not do it out of a humanitarian impulse but because it is cheaper to buy interest in poorer countries \citep{demesquita:2007,stone:2006}. Conversely, a donor country may give to a poorly governed, undemocratic country for humanitarian reasons as well, North Korea being a prominent example.
\indent  The lack of coherence in evaluating strategic interest extends to model specification. Papers which have empirically evaluated the dominance of strategic over humanitarian motives with some exceptions \citep{berthelemy:2006}, have done so by specifying models which pool all donors together or by running models for each donor country separately. We find this empirical choice puzzling - if foreign aid is indeed given for strategic reasons then surely a donor country should account for the foreign aid given by other countries when making their own allocations. The same should be equally true if foreign aid is given for humanitarian reasons - if a very needy country is already recieving an abundant amount of foreign aid from other countries, a particular donor country may decide to dispense aid to a less needy but overlooked recipient country. Pooled models do not address this issue as they do not distinguish between donor countries while donor by donor regressions cannot address this issue because by construction they do not account for the allocations of other donor countries. \\

In this paper, we seek to more definitevely evaluate the prominence of a strategic or humanitarian motive for foreign aid. We do so by creating a new measure of strategic interest by measuring the latent strategic distance between countries \citep{hoff:2002}. Such a measure improves upon existing measures of strategic interest in that the latent measure is fundamentally a network measure which accounts for third-party relationships. In our model specification, we also use a hierarchical random effects model with panel data\footnote{We can go into greater detail and talk about how it is also a zero-inflated model later in the model specification section, what do you think Shahryar?} to account for the possibility that foreign aid given by one donor is not given without consideration of allocations by other donors. In doing so, we are able to model both the variation that is common among donor countries as well as that which is specific to a particular donor country, combining the best of what a pooled regression or donor by donor regressions can offer. 
\indent When we do so, we find that.... [INCLUDE FINDINGS HERE] Moreover, onor by donor regressions have found that there is wide variations of motivations in allocating foreign aid among donors, we find that when we consider the donor countries together that.... [INCLUDE FINDINGS HERE]. 
In what follows, we first give a brief overview of the literature before blah blah blah. 



\label{intro}
%%%%%%%%%%%%%%%%%%%%%%%

%%%%% Lit Review %%%%%
\subsection*{Literature Review}
\label{lit}
%%%%%%%%%%%%%%%%%%%%%%

%%%%% Theory %%%%%
\label{theory}
\section*{Measuring Strategic Interest}


\noindent\noindent  How one \textit{measures} strategic interest is essential to evaluating the relative importance countries may accord strategic motives when dispensing aid.  However, as argued in the literature review, previous papers have been  inconsistent in how they have measured strategic interest, which in turn produces incoherence as to what exactly is being measured. It is not simply a matter of using different data to measure the same concept but of using different data to measure different aspects of a concept. That is, while UN voting scores and arms transfers may be acceptable measures of strategic interest, surely nobody is arguing that they are conceptually equivalent in the same way as Polity and Freedom House are.\\
\noindent\noindent  A large reason for this inconsistency is that while a dyad's strategic bilateral relationship is quite multifaceted, to date, there has not been a readily available measure of strategic interest which captures its various aspects the same way that scholars have done for other complex concepts.\footnote{For example, Polity and Freedom House have provided measures or political institutions while the World Bank's World Governance Indicators (WGI) project provides measures for six dimensions of governance} The most relevant research to date has been concerned with how to measure foreign policy similarity, starting with \citep{demesquita:1975} Kendall $\tau_b$ measure and then \citet{signorino:1999} \textit{S Scores}, with new work continually being done \citep{gartzke:2006, hage:2011,dorazio:2012}). However, foreign policy similiarity arguably only captures the political dimension of strategic interest, equally relevant is active military cooperation between two countries.\\
\noindent\noindent  While military cooperation certainly has political dimensions, we would argue that it should be considered a separate aspect of strategic interest rather than an subset of political strategic interests. That is, military security is set apart by its capacity to affect a country's security in a manner that is more immediate, concrete and unilateral than other security concerns across countries more generally, for example as compared to access to natural resources, humanitarian sanctions or environmental policy. While military cooperation can certainly be mediated in the political arena, is qualitatively different - that is it is one thing to jointly condem the various atrocities of the North Korean government, it is quite another thing to take joint military action against it.

\subsection*{A new measure of strategic interest}

Our measure of strategic interest attempts to introduce greater coherency to the literature by providing a more rigorous measure of these two aspects of strategic interest, political and military. We do so by first measuring the latent space of different dyadic variables that measure various aspects of the strategic relationshp between countries. We then calculate the latent distance between each dyad for each component. Finally, we combine the latent distances for each component through a principal components analysis (PCA). As such, our political strategic interest is the first principal component that results from the PCA of the latent distance between dyadic alliances, UN voting and membership in an intergovernmental organizations (IGOs). Meanwhile our military strategic interest measure is the first principal component of the PCA that results from the latent distance between dyadic arms transfers, militarized interstate disputes (MIDs), and wars. Note that MIDs and wars are of course, the opposite of military cooperation; for these latent measures we reverse the scale to account for this.  We explain how we constructed these measures of strategic interest in greater detail below while we detail the data sources we relied on in the following section.\\
\noindent\noindent  The main advantage of using calculating the latent space of different dyadic variables as opposed to using alternative specifications such as the \textit{S Score} algorithm\footnote{\citet{leeds:2007} for example creates a measure of a states ``threat environment'' as the set of all states for which ones is contiguous with or which is a major power and with an S score below the population median. } is that we are consequently able to account for third order dependencies within the data. To review, first order dependency refers the propensity for some actors to send or recieve more ties than others, second-order dependency refers to reciprocity of exchange between actors while a third-order dependency refers to interaction among three or more actors. Dyadic data are rife with these types of dependencies, and aside from first-order dependency, they pose serious challenges the basic assumption of independence between observations. \\
In particular, third order dependency includes the concepts of (i) transitivity, (ii) balance and (iii) clusterability. Formally, a triad $ijk$ is said to be transitive if for whenver $y_{ij} = 1$ and $y_{jk} = 1$, we also observe that $y_{ik}$. This follows the logic of `` a friend of a friend is a friend''. Meanwhile,  a triad $ijk$ is said to be balanced if $y_{ij} \times y_{jk} \times y_{ki} >0$. Conceptually, if the relatoionship between $i$ and $j$ is `positive', then both will relate to another unit $k$ identically, either both positive or both negative. Finally a triad $ijk$ is said to be cluserable if it is balanced or all the relations are all negative. It is a relaxation of the concept of balance and seeks to capture groups where the measurements are positive within groups and negative between groups. For a more detailed explanation of these concepts, see \citet{hoff:2002,hoff:2004,hoff:2005}. \\
\noindent\noindent  In other words, third order dependencies suggest that ``knowing something about the relationship between $i$ and $j$ as well as between $i$ and $k$ may reveal something about the relationship between $i$ and $k$, even when we do not directly observe it'' \citep{hoff:2004}. Such a dependency is especially important to capture with regards to strategic relationships as dyadic relationships between two particular countries cannot help but be understood in the context of their relationship with other countries. \\
\noindent\noindent  Following \citep{hoff:2005}, we run a null generalized bilinear mixed effects model (gbme) \footnote{Code for running the gbme can be found from Hoff's website at \url{http://www.stat.washington.edu/hoff/Code/hoff_2005_jasa/} } to estimate the latent space for each component of our strategic interest variables. Below we show a visualization for each component for the year 2005.



\begin{figure}[h!] 
\caption{Latent Spaces for components of Political Strategic Interest Measure for the year 2005}
\centering
\begin{minipage}{.33\linewidth}
\centering
\label{fig:ally}
\includegraphics[width = 2.3in]{\detokenize{ally_2005.pdf}}
\caption*{(a) Alliiances}
\end{minipage}
\hspace{-.2in}
\begin{minipage}{.33\linewidth}
\centering
\label{fig:un}
\includegraphics[width = 2.3in]{\detokenize{un_2005.pdf}}
\caption*{(b) UN voting}
\end{minipage}
\hspace{-.2in}
\begin{minipage}{.33\linewidth}
\centering
\label{fig:igo}
\includegraphics[width = 2.3in]{\detokenize{igo_2005.pdf}}
\caption*{(c) IGO membership}
\end{minipage}
\end{figure}
 
\begin{figure}[h!] 
\caption{Latent Spaces for components of Military Strategic Interest Measure for the year 2005}
\centering
\begin{minipage}{.33\linewidth}
\centering
\label{fig:armsSum}
\includegraphics[width = 2.3in]{\detokenize{armsSum_2005.pdf}}
\caption*{(a) Arms transfers}
\end{minipage}
\hspace{-.2in}
\begin{minipage}{.33\linewidth}
\centering
\label{fig:mid}
\includegraphics[width = 2.3in]{\detokenize{mid_2005.pdf}}
\caption*{(b) MIDs}
\end{minipage}
\hspace{-.2in}
\begin{minipage}{.33\linewidth}
\centering
\label{fig:war}
\includegraphics[width = 2.3in]{\detokenize{warMsum5_2005.pdf}}
\caption*{(c) War}
\end{minipage}
\end{figure}
 




After estimating the latent spaces for these components, we calculate the latent distances between each dyad for each component. We then combine them in a principal components analysis to reduce the dimensionality of our measure while retaining as much variance as possible. That is, for example, alliances, UN voting and joint membership in IGOs all capture certain aspects of political strategic interest, and instead of choosing only one of them as our measure of strategic interest as other papers have done, we combine them in order to increase our explanatory power. A visualization of the resultant dyadic PCA is shown below for the political strategic measure and the military strategic measure for the year 2005. These suggest that there is much more variation in political strategic interests than there are military strategic interests, perhaps because the number of issues spaces in the political arena are much greater. These vizualizations also suggest that on average, countries have a greater political strategic interest than military strategic interest. Since the military strategic interest data is composed largely of actual military events, this makes sense as on average, conflict between any two countries is much rarer than diplomatic negotiations. \\


\begin{figure}[h!]
\centering
\caption{Dyadic PCA for Political Strategic Interests for year 2005}
\includegraphics[width = 5in]{\detokenize{dyadViz_allyIGOUN.pdf}}
\caption*{\small{Along the x and y axes are the countries included in our analyes for the year 2005. The color gradient reflects the strength of the strategic relationship between any two dyads, with dark colors reflecting a stronger relationship and light colors reflecting a weaker relationship. Note that because the PCA is of latent distances between any two dyads, dyads that are closer in space and thus stronger strategic relationships will have smaller values.}}
\end{figure}

\begin{figure}[h!]
\centering
\caption{Dyadic PCA for Military Strategic Interests for year 2005}
\includegraphics[width = 5in]{\detokenize{dyadViz_midWarArmsSum.pdf}}
\caption*{\small{Along the x and y axes are the countries included in our analyes for the year 2005. The color gradient reflects the strength of the strategic relationship between any two dyads, with dark colors reflecting a stronger relationship and light colors reflecting a weaker relationship. Note that because the PCA is of latent distances between any two dyads, dyads that are closer in space and thus stronger strategic relationships will have smaller values.}}
\end{figure}

We also conduct a series of post-estimatation validation tests for our resulting strategic variables. In particular, we (1) evaluate the relationship between our political strategic interest variable and our military strategic interest variable against S scores and Kendall's $\tau_b$ for alliances and (2) investigate how our measures describe well-known dyadic relationships. With perform a simple bivariate OLS with and with year fixed effects to evaluate how our measures compare to S scores and Kendall's $\tau_b$. Note in order to make our strategic measures somewhat interpretable, for the validation we scale our strategic measures to be between 0 and 1 just as S scores and Kendall $\tau_b$ is scaled. The results are shown in Table \ref{table:polval} for political strategic interest and Table \ref{table:milval} for miltiary strategic interest. 

\begin{table}[h!]
\small
\caption{Validation of Political Strategic Interest Variable against S scores and Kendall's $\tau_b$}
\begin{center}
\begin{tabular}{l c c c c c c }
\hline
                    & Unweighted   & Unweighted & Weighted  & Weighted  & Tau-B & Tau-B \\
                   &   S Scores &   S Scores &  S Scores &  S Scores &  &   \\
\hline
(Intercept)         & $0.97^{***}$  & $1.03^{***}$  & $1.01^{***}$  & $1.02^{***}$  & $0.29^{***}$  & $0.25^{***}$  \\
                    & $(0.00)$      & $(0.00)$      & $(0.00)$      & $(0.00)$      & $(0.00)$      & $(0.00)$      \\
Political Strategic Interest             & $-0.80^{***}$ & $-0.84^{***}$ & $-1.22^{***}$ & $-1.26^{***}$ & $-0.89^{***}$ & $-0.87^{***}$ \\
                    & $(0.00)$      & $(0.00)$      & $(0.00)$      & $(0.00)$      & $(0.00)$      & $(0.00)$      \\
Year FE? 	   & No 		& Yes 		& No		& Yes	& No		& Yes\\
%as.factor(year)1971 &               & $-0.01^{***}$ &               & $0.01^{**}$   & $-0.00^{*}$   &               \\
%                    &               & $(0.00)$      &               & $(0.00)$      & $(0.00)$      &               \\
%as.factor(year)1972 &               & $-0.03^{***}$ &               & $-0.01^{***}$ & $-0.02^{***}$ &               \\
%                    &               & $(0.00)$      &               & $(0.00)$      & $(0.00)$      &               \\
%as.factor(year)1973 &               & $-0.02^{***}$ &               & $-0.01^{**}$  & $-0.02^{***}$ &               \\
%                    &               & $(0.00)$      &               & $(0.00)$      & $(0.00)$      &               \\
%as.factor(year)1974 &               & $-0.02^{***}$ &               & $0.00$        & $-0.02^{***}$ &               \\
%                    &               & $(0.00)$      &               & $(0.00)$      & $(0.00)$      &               \\
%as.factor(year)1975 &               & $-0.01^{***}$ &               & $0.02^{***}$  & $-0.02^{***}$ &               \\
%                    &               & $(0.00)$      &               & $(0.00)$      & $(0.00)$      &               \\
%as.factor(year)1976 &               & $-0.03^{***}$ &               & $-0.01^{***}$ & $-0.04^{***}$ &               \\
%                    &               & $(0.00)$      &               & $(0.00)$      & $(0.00)$      &               \\
%as.factor(year)1977 &               & $-0.03^{***}$ &               & $0.01^{**}$   & $-0.03^{***}$ &               \\
%                    &               & $(0.00)$      &               & $(0.00)$      & $(0.00)$      &               \\
%as.factor(year)1978 &               & $-0.05^{***}$ &               & $-0.01^{***}$ & $-0.04^{***}$ &               \\
%                    &               & $(0.00)$      &               & $(0.00)$      & $(0.00)$      &               \\
%as.factor(year)1979 &               & $-0.07^{***}$ &               & $-0.03^{***}$ & $-0.05^{***}$ &               \\
%                    &               & $(0.00)$      &               & $(0.00)$      & $(0.00)$      &               \\
%as.factor(year)1980 &               & $-0.06^{***}$ &               & $-0.00$       & $-0.04^{***}$ &               \\
%                    &               & $(0.00)$      &               & $(0.00)$      & $(0.00)$      &               \\
%as.factor(year)1981 &               & $-0.08^{***}$ &               & $-0.02^{***}$ & $-0.05^{***}$ &               \\
%                    &               & $(0.00)$      &               & $(0.00)$      & $(0.00)$      &               \\
%as.factor(year)1982 &               & $-0.09^{***}$ &               & $-0.02^{***}$ & $-0.05^{***}$ &               \\
%                    &               & $(0.00)$      &               & $(0.00)$      & $(0.00)$      &               \\
%as.factor(year)1983 &               & $-0.09^{***}$ &               & $-0.02^{***}$ & $-0.04^{***}$ &               \\
%                    &               & $(0.00)$      &               & $(0.00)$      & $(0.00)$      &               \\
%as.factor(year)1984 &               & $-0.10^{***}$ &               & $-0.03^{***}$ & $-0.06^{***}$ &               \\
%                    &               & $(0.00)$      &               & $(0.00)$      & $(0.00)$      &               \\
%as.factor(year)1985 &               & $-0.09^{***}$ &               & $-0.03^{***}$ & $-0.05^{***}$ &               \\
%                    &               & $(0.00)$      &               & $(0.00)$      & $(0.00)$      &               \\
%as.factor(year)1986 &               & $-0.10^{***}$ &               & $-0.03^{***}$ & $-0.05^{***}$ &               \\
%                    &               & $(0.00)$      &               & $(0.00)$      & $(0.00)$      &               \\
%as.factor(year)1987 &               & $-0.10^{***}$ &               & $-0.04^{***}$ & $-0.06^{***}$ &               \\
%                    &               & $(0.00)$      &               & $(0.00)$      & $(0.00)$      &               \\
%as.factor(year)1988 &               & $-0.10^{***}$ &               & $-0.04^{***}$ & $-0.06^{***}$ &               \\
%                    &               & $(0.00)$      &               & $(0.00)$      & $(0.00)$      &               \\
%as.factor(year)1989 &               & $-0.12^{***}$ &               & $-0.06^{***}$ & $-0.06^{***}$ &               \\
%                    &               & $(0.00)$      &               & $(0.00)$      & $(0.00)$      &               \\
%as.factor(year)1990 &               & $-0.11^{***}$ &               & $-0.08^{***}$ & $-0.05^{***}$ &               \\
%                    &               & $(0.00)$      &               & $(0.00)$      & $(0.00)$      &               \\
%as.factor(year)1991 &               & $-0.07^{***}$ &               & $-0.03^{***}$ & $-0.04^{***}$ &               \\
%                    &               & $(0.00)$      &               & $(0.00)$      & $(0.00)$      &               \\
%as.factor(year)1992 &               & $-0.04^{***}$ &               & $0.04^{***}$  & $-0.04^{***}$ &               \\
%                    &               & $(0.00)$      &               & $(0.00)$      & $(0.00)$      &               \\
%as.factor(year)1993 &               & $-0.05^{***}$ &               & $0.01^{***}$  & $-0.06^{***}$ &               \\
%                    &               & $(0.00)$      &               & $(0.00)$      & $(0.00)$      &               \\
%as.factor(year)1994 &               & $-0.02^{***}$ &               & $0.06^{***}$  & $-0.03^{***}$ &               \\
%                    &               & $(0.00)$      &               & $(0.00)$      & $(0.00)$      &               \\
%as.factor(year)1995 &               & $-0.02^{***}$ &               & $0.05^{***}$  & $-0.03^{***}$ &               \\
%                    &               & $(0.00)$      &               & $(0.00)$      & $(0.00)$      &               \\
%as.factor(year)1996 &               & $-0.03^{***}$ &               & $0.05^{***}$  & $-0.03^{***}$ &               \\
%                    &               & $(0.00)$      &               & $(0.00)$      & $(0.00)$      &               \\
%as.factor(year)1997 &               & $-0.04^{***}$ &               & $0.03^{***}$  & $-0.03^{***}$ &               \\
%                    &               & $(0.00)$      &               & $(0.00)$      & $(0.00)$      &               \\
%as.factor(year)1998 &               & $-0.04^{***}$ &               & $0.02^{***}$  & $-0.03^{***}$ &               \\
%                    &               & $(0.00)$      &               & $(0.00)$      & $(0.00)$      &               \\
%as.factor(year)1999 &               & $-0.04^{***}$ &               & $0.02^{***}$  & $-0.04^{***}$ &               \\
%                    &               & $(0.00)$      &               & $(0.00)$      & $(0.00)$      &               \\
%as.factor(year)2000 &               & $-0.01^{***}$ &               & $0.06^{***}$  & $-0.01^{***}$ &               \\
%                    &               & $(0.00)$      &               & $(0.00)$      & $(0.00)$      &               \\
\hline
R$^2$               & 0.28          & 0.32          & 0.32          & 0.34          & 0.17          & 0.17          \\
Adj. R$^2$          & 0.28          & 0.32          & 0.32          & 0.34          & 0.17          & 0.17          \\
Num. obs.           & 824426        & 824426        & 824426        & 824426        & 824148        & 824148        \\
\hline
\multicolumn{7}{l}{\scriptsize{$^{***}p<0.001$, $^{**}p<0.01$, $^*p<0.05$}}
\end{tabular}
\label{table:polval}
\end{center}
\end{table}



\begin{table}
\small
\caption{Validation of Military Strategic Interest Variable against S scores and Kendall's $\tau_b$}
\begin{center}
\begin{tabular}{l c c c c c c }
\hline
                    & Unweighted   & Unweighted & Weighted  & Weighted  & Tau-B & Tau-B \\
                   &   S Scores &   S Scores &  S Scores &  S Scores &  &   \\
\hline
(Intercept)         & $0.75^{***}$ & $0.79^{***}$  & $0.68^{***}$ & $0.66^{***}$  & $0.02^{***}$  & $0.02^{***}$ \\
                    & $(0.00)$     & $(0.00)$      & $(0.00)$     & $(0.00)$      & $(0.00)$      & $(0.00)$     \\
Military Strategic Interest              & $0.01^{***}$ & $-0.05^{***}$ & $0.03^{***}$ & $-0.10^{***}$ & $0.02^{***}$  & $-0.00$      \\
                    & $(0.00)$     & $(0.00)$      & $(0.00)$     & $(0.01)$      & $(0.00)$      & $(0.00)$     \\
Year FE? 	   & No 		& Yes 		& No		& Yes	& No		& Yes\\
%as.factor(year)1971 &              & $0.03^{***}$  &              & $0.09^{***}$  & $-0.01^{**}$  &              \\
%                    &              & $(0.00)$      &              & $(0.00)$      & $(0.00)$      &              \\
%as.factor(year)1972 &              & $0.03^{***}$  &              & $0.09^{***}$  & $-0.01^{**}$  &              \\
%                    &              & $(0.00)$      &              & $(0.00)$      & $(0.00)$      &              \\
%as.factor(year)1973 &              & $0.03^{***}$  &              & $0.10^{***}$  & $-0.01^{**}$  &              \\
%                    &              & $(0.00)$      &              & $(0.00)$      & $(0.00)$      &              \\
%as.factor(year)1974 &              & $-0.01^{**}$  &              & $0.03^{***}$  & $0.00$        &              \\
%                    &              & $(0.00)$      &              & $(0.00)$      & $(0.00)$      &              \\
%as.factor(year)1975 &              & $0.04^{***}$  &              & $0.12^{***}$  & $-0.01^{**}$  &              \\
%                    &              & $(0.00)$      &              & $(0.00)$      & $(0.00)$      &              \\
%as.factor(year)1976 &              & $0.03^{***}$  &              & $0.10^{***}$  & $-0.01^{**}$  &              \\
%                    &              & $(0.00)$      &              & $(0.00)$      & $(0.00)$      &              \\
%as.factor(year)1977 &              & $0.04^{***}$  &              & $0.12^{***}$  & $-0.01^{**}$  &              \\
%                    &              & $(0.00)$      &              & $(0.00)$      & $(0.00)$      &              \\
%as.factor(year)1978 &              & $-0.01^{***}$ &              & $0.05^{***}$  & $-0.00$       &              \\
%                    &              & $(0.00)$      &              & $(0.00)$      & $(0.00)$      &              \\
%as.factor(year)1979 &              & $0.01^{***}$  &              & $0.12^{***}$  & $-0.01^{**}$  &              \\
%                    &              & $(0.00)$      &              & $(0.00)$      & $(0.00)$      &              \\
%as.factor(year)1980 &              & $0.02^{***}$  &              & $0.14^{***}$  & $-0.01^{**}$  &              \\
%                    &              & $(0.00)$      &              & $(0.00)$      & $(0.00)$      &              \\
%as.factor(year)1981 &              & $-0.00$       &              & $0.12^{***}$  & $-0.01^{**}$  &              \\
%                    &              & $(0.00)$      &              & $(0.00)$      & $(0.00)$      &              \\
%as.factor(year)1982 &              & $-0.01^{*}$   &              & $0.13^{***}$  & $-0.01^{**}$  &              \\
%                    &              & $(0.00)$      &              & $(0.00)$      & $(0.00)$      &              \\
%as.factor(year)1983 &              & $-0.03^{***}$ &              & $0.08^{***}$  & $-0.00$       &              \\
%                    &              & $(0.00)$      &              & $(0.00)$      & $(0.00)$      &              \\
%as.factor(year)1984 &              & $-0.05^{***}$ &              & $0.05^{***}$  & $0.00$        &              \\
%                    &              & $(0.00)$      &              & $(0.00)$      & $(0.00)$      &              \\
%as.factor(year)1985 &              & $-0.04^{***}$ &              & $0.05^{***}$  & $0.00$        &              \\
%                    &              & $(0.00)$      &              & $(0.00)$      & $(0.00)$      &              \\
%as.factor(year)1986 &              & $-0.01^{**}$  &              & $0.12^{***}$  & $-0.01^{**}$  &              \\
%                    &              & $(0.00)$      &              & $(0.00)$      & $(0.00)$      &              \\
%as.factor(year)1987 &              & $-0.01^{**}$  &              & $0.12^{***}$  & $-0.01^{**}$  &              \\
%                    &              & $(0.00)$      &              & $(0.00)$      & $(0.00)$      &              \\
%as.factor(year)1988 &              & $-0.01^{*}$   &              & $0.13^{***}$  & $-0.01^{**}$  &              \\
%                    &              & $(0.00)$      &              & $(0.00)$      & $(0.00)$      &              \\
%as.factor(year)1989 &              & $-0.02^{***}$ &              & $0.11^{***}$  & $-0.01^{*}$   &              \\
%                    &              & $(0.00)$      &              & $(0.00)$      & $(0.00)$      &              \\
%as.factor(year)1990 &              & $-0.03^{***}$ &              & $0.06^{***}$  & $-0.00$       &              \\
%                    &              & $(0.00)$      &              & $(0.00)$      & $(0.00)$      &              \\
%as.factor(year)1991 &              & $0.01^{**}$   &              & $0.11^{***}$  & $-0.01^{*}$   &              \\
%                    &              & $(0.00)$      &              & $(0.00)$      & $(0.00)$      &              \\
%as.factor(year)1992 &              & $0.03^{***}$  &              & $0.17^{***}$  & $-0.01^{**}$  &              \\
%                    &              & $(0.00)$      &              & $(0.00)$      & $(0.00)$      &              \\
%as.factor(year)1993 &              & $0.04^{***}$  &              & $0.17^{***}$  & $-0.01^{**}$  &              \\
%                    &              & $(0.00)$      &              & $(0.00)$      & $(0.00)$      &              \\
%as.factor(year)1994 &              & $0.04^{***}$  &              & $0.18^{***}$  & $-0.01^{**}$  &              \\
%                    &              & $(0.00)$      &              & $(0.00)$      & $(0.00)$      &              \\
%as.factor(year)1995 &              & $0.04^{***}$  &              & $0.17^{***}$  & $-0.01^{***}$ &              \\
%                    &              & $(0.00)$      &              & $(0.00)$      & $(0.00)$      &              \\
%as.factor(year)1996 &              & $0.04^{***}$  &              & $0.16^{***}$  & $-0.01^{***}$ &              \\
%                    &              & $(0.00)$      &              & $(0.00)$      & $(0.00)$      &              \\
%as.factor(year)1997 &              & $0.03^{***}$  &              & $0.15^{***}$  & $-0.01^{***}$ &              \\
%                    &              & $(0.00)$      &              & $(0.00)$      & $(0.00)$      &              \\
%as.factor(year)1998 &              & $0.03^{***}$  &              & $0.15^{***}$  & $-0.01^{***}$ &              \\
%                    &              & $(0.00)$      &              & $(0.00)$      & $(0.00)$      &              \\
%as.factor(year)1999 &              & $0.03^{***}$  &              & $0.15^{***}$  & $-0.01^{***}$ &              \\
%                    &              & $(0.00)$      &              & $(0.00)$      & $(0.00)$      &              \\
%as.factor(year)2000 &              & $0.03^{***}$  &              & $0.15^{***}$  & $-0.01^{**}$  &              \\
%                    &              & $(0.00)$      &              & $(0.00)$      & $(0.00)$      &              \\
\hline
R$^2$               & 0.00         & 0.01          & 0.00         & 0.01          & 0.00          & 0.00         \\
Adj. R$^2$          & 0.00         & 0.01          & 0.00         & 0.01          & 0.00          & -0.00        \\
Num. obs.           & 824426       & 824426        & 824426       & 824426        & 824148        & 824148       \\
\hline
\multicolumn{7}{l}{\scriptsize{$^{***}p<0.001$, $^{**}p<0.01$, $^*p<0.05$}}
\end{tabular}
\label{table:milval}
\end{center}
\end{table}

\noindent\noindent  In brief, we find that our political strategic measure performs well against S scores and Kendal's $\tau_b$ for alliances  with and without fixed effects. Note that because the PCA is of latent distances between any two dyads, dyads that are closer in space and thus stronger strategic relationships will have smaller values. Therefore the negative relationship we find between the political strategic measure and S scores and $\tau_b$ are interpreted to mean the greater the foreign policy similiarity as measured by the S score or Kendal's $\tau_b$ , the smaller the latent distance or the greater the political strategic interest between a dyad. \\
\noindent\noindent  Our military strategic measure performs will mixed results with respect to S scoresand Kendal's $\tau_b$ for alliances . It has a negative and statistically significant relationship between S scores with year fixed effects. It in fact has a postiive and statistically significant relationship between S scores and Kendal's $\tau_b$ without year fixed effects These mixed results suggest that the military strategic measure is perhaps measuring something qualitatively different than S scores.\\
\noindent\noindent Finally we also investigate how our measure performs relative to well known dyadic relationships. In the figures below, we plot the dyadic relationships between countries that are well-known to have friendly or antagonistic relationships. In Figure \ref{USIsraelIran_p} shows for example the dyadic relationship between Iran and Israel, the US and Israel, and the US and Iran. The plot suggests that the US and Israel have consistently had a stronger political strategic relationship throughought time except for the early 1970s when Iran and Israel is shown to have had a stronger political strategic relationship. This is in fact consistent with historical evidence which suggests that Iran and Israel enjoyed close ties before the Iranian revolution. Meanwhile the plot of the dyadic relationships between China and Japan and, North Korea and China and North Korea and Japan suggest more or less indifferent relations among the three before 1990 after which the political strategic relationship between China and North Korea becomes markedly stronger. This is also consistent with the disappearance of Soviet support for North Korea following the end of the Cold War and the emergence of China as North Korea's new protector. Finally, the plot of the dyadic relationship between India and Pakistan, India and the US and Pakistan and the US suggest in fact that India and Pakistan have a much stronger political strategic relationship than either do with the US. Given the history of antagonism between India and Pakistan, this is a rather suprising result; it also suggests however that a dyad's political relationshp and military relationship may be quite different and indeed as two large bordering countries, cooperation between India and Pakistan is important to the security of both. 


\begin{figure}[h!] 
\caption{Dyadic relationships over time as measured by the political strategic interest variable}
\centering
\begin{minipage}{.33\linewidth}
\centering
\label{fig:USIsraelIran_p}
\includegraphics[width = 2.3in]{\detokenize{dyadic_USIsraelIran_allyIGOUN.pdf}}
\end{minipage}
\hspace{-.2in}
\begin{minipage}{.33\linewidth}
\centering
\label{fig:ChinaJapNK_p}
\includegraphics[width = 2.3in]{\detokenize{dyadic_ChinaJapNK_allyIGOUN.pdf}}
\end{minipage}
\hspace{-.2in}
\begin{minipage}{.33\linewidth}
\centering
\label{fig:USIndPak_p}
\includegraphics[width = 2.3in]{\detokenize{dyadic_USIndPak_allyIGOUN.pdf}}
\end{minipage}
\end{figure}

We plot the same dyadic relationships using our military strategic interest variable. Here, variation between different dyadic relationships is much more difficult to tease out, perhaps a function of the fact that military events are much more rare. There are two points of interest about these plots (i) they have large degrees of variation over time, suggesting that while military events may be rare, they also have a large influence on a dyad's military strategic relationship (ii) they dyadic relationships plotted here seem to be very similar over time potentially suggesting that third order dependencies are very strong with wegards to military strategic relationships.

\begin{figure}[h!] 
\caption{Dyadic relationships over time as measured by the military strategic interest variable}
\centering
\begin{minipage}{.33\linewidth}
\centering
\label{fig:ally}
\includegraphics[width = 2.3in]{\detokenize{dyadic_USIsraelIran_midWarArmsSum.pdf}}
\end{minipage}
\hspace{-.2in}
\begin{minipage}{.33\linewidth}
\centering
\label{fig:un}
\includegraphics[width = 2.3in]{\detokenize{dyadic_ChinaJapNK_midWarArmsSum.pdf}}
\end{minipage}
\hspace{-.2in}
\begin{minipage}{.33\linewidth}
\centering
\label{fig:igo}
\includegraphics[width = 2.3in]{\detokenize{dyadic_USIndPak_midWarArmsSum.pdf}}
\end{minipage}
\end{figure}

Before moving on to the next section, we note that it is possible to do a PCA on all different of these components of strategic interest --- alliances, UN voting, joint IGO membership, arms transfers, mids and wars --- together. If we were to take this approach,  we could run our models using the largest components of the resulting PCA. As discussed above, while we argue that political and military strategic interest are qualititatively different, we do acknowledge that both can inform each other and so taking such a course of action would be theoretically logical. \\
\noindent\noindent While we considered employing this approach, we decided to make the tradeoff for better interpretability of our measure over increased precision of our strategic interest measure as the interpretation of different components of a PCA measure is generally not straightforward as it is. For example, we could end up with a first principal component that is explained by alliances \%50 of the time, IGOs \%40 of the time, arms transfers \% 5 of the time and the rest of the components a combined \%5 of the time and a second component that is explained by  MIDS \%60 of the time, alliances 30\% of the time, and the rest of the components a combined \%10 of the time. While we may be able to say that strategic interest matters, it would be more difficult to say in what way. In separating out the variables before hand for theoretical reasons, we increase the interpretability of any of our subsequent results while sacrificing some explanatory power. At the same time, whatever results we do find should represent a harder test for the importance of political or military strategic interest because of this tradeoff.\\






%%%%%%%%%%%%%%%%%%%%%%

%%%%% Empirics %%%%%
\section*{Analysis}
\label{empirics}

\subsection*{Estimation Method}

To model aid flows using our directed-dyadic panel dataset, we utilize a hierarchical model. To implement this we nest receivers within senders and senders within years. We include random intercepts in our model for every sender and year. The results of this analyis are shown below in table \ref{tab:mainResults}.

\begin{table}[!htbp] \centering 
\begin{tabular}{@{\extracolsep{5pt}}lc} 
\\[-1.8ex]\hline 
\hline \\[-1.8ex] 
 & \multicolumn{1}{c}{\textit{Dependent variable:}} \\ 
\cline{2-2} 
\\[-1.8ex] & Ln(Aid$_{sender-receiver, t}$) \\ 
\hline \\[-1.8ex] 
 Pol. Rel.$_{s-r, t-1}$ & 0.059$^{***}$ \\ 
  & (0.009) \\ 
  & \\ 
 Mil. Rel.$_{s-r, t-1}$ & $-$0.004$^{**}$ \\ 
  & (0.002) \\ 
  & \\ 
Colony$_{s-r, t-1}$ & 1.822$^{***}$ \\ 
  & (0.058) \\ 
  & \\ 
 Polity$_{r, t-1}$ & 0.015$^{***}$ \\ 
  & (0.002) \\ 
  & \\ 
 Ln(GDP Capita)$_{r, t-1}$ & $-$0.580$^{***}$ \\ 
  & (0.017) \\ 
  & \\ 
 Life Expect$_{r, t-1}$ & 0.017$^{***}$ \\ 
  & (0.002) \\ 
  & \\ 
 No. Disasters$_{r, t-1}$ & 0.168$^{***}$ \\ 
  & (0.004) \\ 
  & \\ 
 Constant & 17.463$^{***}$ \\ 
  & (0.107) \\ 
  & \\ 
\hline \\[-1.8ex] 
Observations & 34,486 \\ 
RMSE & 2.09  \\ 
\hline 
\hline \\[-1.8ex] 
  & \multicolumn{1}{r}{$^{*}$p$<$0.1; $^{**}$p$<$0.05; $^{***}$p$<$0.01} \\ 
\end{tabular} 
  \caption{Hierarchical regression results on logged aid flows between senders and receivers from 1975 to 2005.} 
  \label{tab:mainResults} 
\end{table} 

Our strategic interest variables are shown in the first two rows. The results for the political strategic relationship variable align with the extant literature, we can see that countries are likely to send greater levels of aid to those with whom they have strong political relationships. The same does not hold for strong military relationships, the relationship with this variable and aid is actually slightly negative. We also see that countries are more likely to send aid to former colonies, a finding which also has received support in previous literature. 

Next we turn to particular characteristics of receiver countries that are associated with higher levels of aid flows. Our analysis finds that countries which are poorer, more democratic, or that have recently faced a natural disaster receive higher levels of aid on average. Lower levels of life expectancy, however, are not associated with greater levels of aid. 


%%%%%%%%%%%%%%%%%%%%%%

%%%%% Empirics %%%%%
\input{fAidFindings}
%%%%%%%%%%%%%%%%%%%%%%

%%%%% Conclusion %%%%%
\input{fAidConcl}
%%%%%%%%%%%%%%%%%%%%%%

\newpage
\bibliographystyle{APSR}
\bibliography{fAidRefs.bib}

\end{document} 