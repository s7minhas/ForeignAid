\documentclass[doublespaced,bibtex]{cupjournal}
%DIF LATEXDIFF DIFFERENCE FILE
%DIF DEL /Users/cindycheng/Documents/Papers/ForeignAid/Paper/Paper_Revision1/fAid.tex   Sat Aug  3 11:31:30 2019
%DIF ADD /Users/cindycheng/Documents/Papers/ForeignAid/fAid.tex                         Sat Aug  3 12:56:24 2019



%%%%%%%%%%%%%%%%%%%%%%%%%%%%%%%%%%%%%%%%%%%%%%%%%%
%%%%%%%%%%%%%%%%%%%% PREAMBLE %%%%%%%%%%%%%%%%%%%%
%%%%%%%%%%%%%%%%%%%%%%%%%%%%%%%%%%%%%%%%%%%%%%%%%%


% % -------------------- defaults -------------------- %
% % load lots o' packages

% % approx iid
% \newcommand\simiid{\stackrel{\mathclap{\normalfont\mbox{\tiny{iid}}}}{\sim}}

% % define bibliography style
 
% % Fonts
% % at some point figure out bolding ...
% \usepackage[utf8]{inputenc}

 


% % layout control
% %\usepackage[doublespacing]{setspace}
% %\setlength{\parskip}{.5em}
% \usepackage{rotating}
% \usepackage{setspace}
% \usepackage{fancyhdr}
% \usepackage{parallel}
% \usepackage{parcolumns}
% \usepackage{pdflscape}
% % math typesetting
% \usepackage{array}
% \usepackage{amsmath}
% \usepackage{amsfonts}
% \usepackage{amssymb}

% % tables
\usepackage{tabularx}
% \usepackage{booktabs}
% \usepackage{multicol}
\usepackage{multirow}
% \usepackage{longtable}

% \usepackage[%
% decimalsymbol=.,
% digitsep=fullstop
% ]{siunitx}

% % to adapt caption style
% %\usepackage[font={small},labelfont=bf]{caption}

% % footnotes at bottom
% %\usepackage[bottom]{footmisc}

% % to change enumeration symbols begin{enumerate}[(a)]
% \usepackage{enumerate}

% % to make enumerations and itemizations within paragraphs or
% % lines. f.i. begin{inparaenum} for (a) is (b) and (c)
% \usepackage{paralist}

% % to colorize links in document. See color specification below
\usepackage[x11names]{xcolor}

% % for multiple references and insertion of the word "figure" or "table"
% % \usepackage{cleveref}

% % load the hyper-references package and set document info
\usepackage[pdftex]{hyperref}

% % graphics stuff
% \usepackage{subfig}
\usepackage{graphicx}
% \usepackage[space]{grffile} % allows us to specify directories that have spaces
\usepackage[section]{placeins} % prevents floats from moving past a \FloatBarrier or section
% \usepackage{tikz}
% \usepackage{caption}

% % \usepackage{pgfplots}

% % define clickable links and their colors
\hypersetup{
	unicode=false,          % non-Latin characters in Acrobat's bookmarks
	pdftoolbar=true,        % show Acrobat's toolbar?
	pdfmenubar=true,        % show Acrobat's menu?
	pdffitwindow=false,     % window fit to page when opened
	pdfstartview={FitH},    % fits the width of the page to the window
	pdfnewwindow=true,%
	%pdfauthor={Cindy Cheng and Shahryar Minhas},%
	%pdftitle={Aid something},%
	colorlinks,%
	citecolor=black,%
	filecolor=black,%
	linkcolor=black,%
	urlcolor=RoyalBlue4%
	}

% % Including External Code
% \usepackage{verbatim}
% \usepackage{listings}
% \lstset{
% 	language=R,
% 	basicstyle=\scriptsize\ttfamily,
% 	commentstyle=\ttfamily\color{gray},
% 	numbers=left,
% 	numberstyle=\ttfamily\color{gray}\footnotesize,
% 	stepnumber=1,
% 	numbersep=5pt,
% 	backgroundcolor=\color{white},
% 	showspaces=false,
% 	showstringspaces=false,
% 	showtabs=false,
% 	frame=single,
% 	tabsize=2,
% 	captionpos=b,
% 	breaklines=true,
% 	breakatwhitespace=false,
% 	title=\lstname,
% 	escapeinside={},
% 	keywordstyle={},
% 	morekeywords={}
% 	}

% -------------------------------------------------- %

\usepackage{graphicx}
\usepackage{amssymb}
% -------------------- title -------------------- %

\title{Keeping Friends Close, But Enemies Closer: Foreign Aid Responses to Natural Disasters}
% \author{	Cindy Cheng \\
% 	\texttt{cindy.cheng@hfp.tum.de}
% 	\and
% 	Shahryar Minhas \\
% 	\texttt{minhassh@msu.edu}}
% \date{\today}

% \setlength{\headheight}{15pt}
% \setlength{\headsep}{20pt}
% \pagestyle{fancyplain}
 
% \fancyhf{}

% \lhead{\fancyplain{}{}}
% \chead{\fancyplain{}{}}
% \rhead{\fancyplain{}{}}
% \rfoot{\fancyplain{}{}}


 



% ----------------------------------------------- %


% -------------------- customizations -------------------- %

%\graphicspath{{~/Users/cindycheng/Dropbox/ForeignAid/graphics/}}
\graphicspath{{graphics/}}
\makeatletter
\def\input@path{{graphics/}}
 \makeatother

% \makeatletter
%\def\input@path{{/Users/janus829/Dropbox/Research/ForeignAid/Graphics/}}
% \makeatother
%\graphicspath{{/Users/janus829/Dropbox/Research/ForeignAid/Graphics/}}

% easy commands for number propers
\newcommand{\first}{$1^{\text{st}}$}
\newcommand{\second}{$2^{\text{nd}}$}
\newcommand{\third}{$3^{\text{rd}}$}
\newcommand{\nth}[1]{${#1}^{\text{th}}$}

% easy command for boldface math symbols
\newcommand{\mbs}[1]{\boldsymbol{#1}}

% -------------------------------------------------------- %


%%%%%%%%%%%%%%%%%%%%%%%%%%%%%%%%%%%%%%%%%%%%%%%%%%
%%%%%%%%%%%%%%%%%%%% DOCUMENT %%%%%%%%%%%%%%%%%%%%
%%%%%%%%%%%%%%%%%%%%%%%%%%%%%%%%%%%%%%%%%%%%%%%%%%
%\addbibresource{Paper/fAidRefs.bib}

\usepackage{enumitem}
\newlist{inparaenum}{enumerate}{2}% allow two levels of nesting in an enumerate-like environment
\setlist[inparaenum]{itemsep=0pt,leftmargin=10em,label=\textsc{Hypothesis} \arabic{inparaenumi}\Alph*: }% labels for top level
\setlist[inparaenum,2]{label=\arabic{inparaenumi}\alph*:) }% labels for second level


\newlist{inparaenumb}{enumerate}{2}% allow two levels of nesting in an enumerate-like environment
\setlist[inparaenumb]{itemsep=0pt,leftmargin=10em,label=\textsc{Hypothesis} \arabic{inparaenumbi}B: }% labels for top level
\setlist[inparaenumb,2]{label=\arabic{inparaenumbi}\alph*:) }% labels for second level

\newlist{inparaenumc}{enumerate}{2}% allow two levels of nesting in an enumerate-like environment
\setlist[inparaenumc]{itemsep=0pt,leftmargin=10em,label=\textsc{Hypothesis} \arabic{inparaenumci}C: }% labels for top level
\setlist[inparaenumc,2]{label=\arabic{inparaenumci}\alph*:) }% labels for second level
%DIF PREAMBLE EXTENSION ADDED BY LATEXDIFF
%DIF UNDERLINE PREAMBLE %DIF PREAMBLE
\RequirePackage[normalem]{ulem} %DIF PREAMBLE
\RequirePackage{color}\definecolor{RED}{rgb}{1,0,0}\definecolor{BLUE}{rgb}{0,0,1} %DIF PREAMBLE
\providecommand{\DIFaddtex}[1]{{\protect\color{blue}\uwave{#1}}} %DIF PREAMBLE
\providecommand{\DIFdeltex}[1]{{\protect\color{red}\sout{#1}}}                      %DIF PREAMBLE
%DIF SAFE PREAMBLE %DIF PREAMBLE
\providecommand{\DIFaddbegin}{} %DIF PREAMBLE
\providecommand{\DIFaddend}{} %DIF PREAMBLE
\providecommand{\DIFdelbegin}{} %DIF PREAMBLE
\providecommand{\DIFdelend}{} %DIF PREAMBLE
%DIF FLOATSAFE PREAMBLE %DIF PREAMBLE
\providecommand{\DIFaddFL}[1]{\DIFadd{#1}} %DIF PREAMBLE
\providecommand{\DIFdelFL}[1]{\DIFdel{#1}} %DIF PREAMBLE
\providecommand{\DIFaddbeginFL}{} %DIF PREAMBLE
\providecommand{\DIFaddendFL}{} %DIF PREAMBLE
\providecommand{\DIFdelbeginFL}{} %DIF PREAMBLE
\providecommand{\DIFdelendFL}{} %DIF PREAMBLE
%DIF END PREAMBLE EXTENSION ADDED BY LATEXDIFF
%DIF PREAMBLE EXTENSION ADDED BY LATEXDIFF
%DIF HYPERREF PREAMBLE %DIF PREAMBLE
\providecommand{\DIFadd}[1]{\texorpdfstring{\DIFaddtex{#1}}{#1}} %DIF PREAMBLE
\providecommand{\DIFdel}[1]{\texorpdfstring{\DIFdeltex{#1}}{}} %DIF PREAMBLE
%DIF END PREAMBLE EXTENSION ADDED BY LATEXDIFF

\begin{document}


 
\markboth{Author}{Keeping Enemies Close}
\journalname{Research Article Submission to B.J.Pol.S.}
\journalvolume{XX}
\journalyear{2018}
\pagecount{X--XX}
\doinumber{XXX}
%doi:10.1017/XXXX

\maketitle

\begin{abstract}

 
\DIFdelbegin \DIFdel{How can we square an existing literature which shows that bilateral donors primarily allocate aid to strategic allies with strong anecdotal evidence which suggests that }\DIFdelend \DIFaddbegin \DIFadd{If foreign aid is given primarily for strategic reasons, as much of the field finds, how can we explain donor generosity }\DIFaddend following natural disasters\DIFdelbegin \DIFdel{, aid flows to strategic opponents quite generously}\DIFdelend ? In this paper, we address this puzzle by building on the literature in three ways. First, we differentiate between three major types of aid: humanitarian, civil society, and development. Next, we show natural disasters act as an exogenous shock to the strategic calculus donor countries undertake when making foreign aid allocation decisions. Specifically, we argue that donor countries use natural disasters as opportunities to exert influence on strategic opponents through the allocation of humanitarian and civil society aid. However, donors still reserve development aid for strategic allies irrespective of the incidence of natural disasters. Lastly, we substantiate our findings using a new measure of strategic interest that accounts for the indirect ties states share and the multiple dimensions upon which they interact.

%While the existing literature shows that bilateral donors primarily allocate aid to strategic allies, anecdotal evidence suggests that following natural disasters, bilateral aid flows to strategic opponents quite generously. We build on this literature in three ways. First, we differentiate between the three major types of aid: humanitarian, civil society, and development. Next, we argue that donor countries use natural disasters as opportunities to exert influence on strategic opponents through the allocation of humanitarian and civil society aid. However, donors still primarily reserve development aid for strategic allies irrespective of the occurrence of natural disasters . Last, we substantiate our findings using a new measure of strategic interest that accounts for the indirect ties states share and the multiple dimensions upon which they interact.
\DIFdelbegin %DIFDELCMD < 

%DIFDELCMD < %%%
\DIFdelend \end{abstract}

%%%%%%%% INTRO %%%%%%%%
\section*{Introduction}
\label{intro} 

In the early morning hours of December 26, 2003, a massive earthquake measuring 6.3 on the Richter scale struck the city of Bam, Iran. Its effects were devastating.  Out of Bam's 100,000 residents, approximately 26,000 to 40,000 were killed. Those  who survived were left to grapple with the destruction of 70 to 90 percent of the city's housing infrastructure \citep{montazeri:2005}.\footnote{Fathi, Nazila. ``Deadly Earthquake Jolts City in Southeast Iran.'' \textit{The New York Times.} 26 December 2003. Accessed October 2017: \url{https://web.archive.org/web/20090620230700/http://www.nytimes.com/2003/12/26/international/26CND-QUAKE.html?ex=1225166400&en=c550b50a2ad59dd6&ei=5070}} As part of the international response that followed, more than 44 countries sent aid, including the United States, which contributed eight plane loads of medical and humanitarian supplies as well as several dozen teams of experts to the relief effort. 

 While the US response to the 2003 Bam earthquake was seemingly analogous to that of any foreign actor offering aid and support, \textit{a priori}, it was not obvious whether the US would send any humanitarian aid at all, to say nothing of whether Iran would accept it. Just the year prior, then-President George W. Bush had famously assigned Iran membership in the ``Axis of Evil'' \citep{heradstveit:2007}. Meanwhile, at the time of the earthquake US-Iranian relations were particularly delicate as the countries navigated the issue of nuclear weapons in Iran.\footnote {``Timeline: US-Iran ties.'' \textit{BBC News.} 16 January 2009. Accessed October 2017: \url{http://news.bbc.co.uk/2/hi/middle_east/3362443.stm}} Indeed, given the broader context of contentious bilateral relations, the process of transferring aid from the US to Iran entailed greater intentionality than normal. To initiate the flow of any aid, President Bush was obliged to institute a special 90-day measure to ease US sanctions on Iran\footnote{``US eases Iran sanctions to speed earthquake relief.'' \textit{China Daily}. 1 January 2004. Accessed October 2017: \url{http://www.chinadaily.com.cn/en/doc/2004-01/01/content_295063.htm}} -- which had been in place since 1979 and continue to be enforced to this day \citep{katzman:2018}. For Iran's part, accepting US aid meant allowing US military planes to land  on its soil, which they had spent the previous 20 years prohibiting.\footnote{``Iran Quake Toll May Hit 50,000.'' \textit{China Daily.} 31 December, 2003. Accessed October 2017: \url{http://www.chinadaily.com.cn/en/doc/2003-12/31/content_294833.htm}} For a country that had undergone a revolution in part because the US military was perceived to have had too strong a domestic influence, it was far from obvious that such an act would be perceived as benign.\footnote{``Geopolitical Diary: Tuesday Dec. 30, 2003.'' \textit{Stratfor}. 31 December 2003. Accessed June 2018: \url{https://www.stratfor.com/geopolitical-diary/geopolitical-diary-tuesday-dec-30-2003}} 
%\footnote{The US first imposed sanctions against Iran in 1979 during the US-Iran hostage crisis. While many assets have since been unfrozen, sanctions on a number of items, including military sales, financial assets, and real estate holdings remain in place }
\begin{figure}
  \centering
  \includegraphics[width = .9\textwidth]{US_Iran_aid}
  \caption{US aid commitments to Iran, 2002 - 2013}
  \label{fig:us_iran}
\end{figure}

Yet, the Bam earthquake led not only to an increase in US humanitarian aid to Iran,  albeit temporarily, it was followed by other types of aid as well. Figure \ref{fig:us_iran} shows that after 2004, aid commitments to  "strengthen civil society" increased markedly and consistently, reaching its apex with the creation of the 2006 "Iran Democracy Fund" to promote democracy in Iran.\footnote{Carpenter, J. Scott. ``After the Crackdown: The Iran Democracy Fund.'' \textit{The Washington Institute for Near East Policy, PolicyWatch 1576.} 8 September 2009. Accessed May 2018: \url{http://www.washingtoninstitute.org/policy-analysis/view/after-the-crackdown-the-iran-democracy-fund}} Meanwhile, US aid for a variety of developmental purposes, (i.e. economic and development policy and planning, infectious disease control, social/welfare services) also increased sporadically following 2003. This is particularly noteworthy given that Iran has generally been barred from receiving US foreign aid since  the US State Department designated it a ``state sponsor of terrorism'' in 1984 \citep{samore:2015}.\footnote{Available data from AidData and the OECD suggest that the US did not commit any aid to Iran from 1974 to 2001.} Why did the US send humanitarian aid to Iran despite objectively hostile extant relations? Was this event \textit{sui generis} or is it possible to observe other dyadic pairs acting in a similar fashion? If so, does the occurrence of a natural disaster also lead donors to distribute other types of aid to strategic opponents?   

Answering these questions has important implications for our understanding of how donors use foreign aid. Furthering such an understanding is especially pressing given that the occurrences of natural disasters are likely to increase with changing climate conditions. Meanwhile, in light of an existing literature that finds that donors are more likely to allocate aid to strategic allies, a more nuanced understanding of what drives foreign aid allocations is necessary to answer these questions. To do this, we begin by first disaggregating foreign aid into three types: humanitarian, civil society, and development aid. Humanitarian aid is meant as a stop-gap measure to help recipient countries return to their status quo, while the latter two types of aid are targeted towards catalyzing long term change. Specifically, civil society aid is often used to improve governance outcomes, which provides donors an avenue through which to wade into the domestic politics of recipient states \citep{ottaway:2000, henderson:2002, resnick2012foreign, spina:2014}. Meanwhile, development aid is primarily focused on promoting economic development.
%\footnote{More specifically, some argue that the lack of good governance and state capacity in developing countries have stymied the ability for foreign aid to promote development. As such, the promotion of civil society is seen as important to the successful implementation of foreign aid projects.}
% should this be "citizens of" countries with which they have ...
% further seek to forward their long-term strategic interests as they 
We show that following a natural disaster donor countries actually give more humanitarian aid to strategic opponents. We argue that this is because donors use natural disasters as an opportunity to ingratiate themselves with countries they have historically shared hostile relations with. Additionally, we find that while natural disasters prompt donors to increase civil society aid to strategic opponents for similar reasons, they conversely push donors to give more development aid to strategic allies. In all, we argue that while donors do use aid to promote their strategic interest, the tactics they employ to do so can depend highly on context.  We evaluate these claims using a new measure of strategic interest that: 1) accounts for the indirect ties states share 2) and incorporates a variety of dimensions of strategic interest. 

In what follows, we first give a brief overview of the existing literature on natural disasters and foreign aid allocations before outlining our hypotheses. We then introduce our new measure of strategic interest, and present our empirical analysis of how natural disasters condition foreign aid allocation decisions. 
%%%%%%%%%%%%%%%%%%%%%%%

%%%%% Lit Review %%%%%
\section*{Extant Motivations for Foreign Aid}
\label{theory}

Natural disasters can lead to the destruction or impairment of physical and social infrastructure and even more significantly, the devastating loss of human lives. For example, the 1985 Mexico City Earthquake, one of the most catastrophic natural disasters in modern times, killed at least 10,000 people\footnote{The Editors of Encyclopaedia Brittanica. ``Mexico City earthquake of 1985.'' \textit{Encyclopaedia Britannica}. 20 September 2017. Accessed September 2017: \url{https://www.britannica.com/event/Mexico-City-earthquake-of-1985}} and cost around 9 billion dollars.\footnote{\DIFdelbegin \DIFdel{Wiliams}\DIFdelend \DIFaddbegin \DIFadd{Williams}\DIFaddend , Dan. ``Mexico Quake Loss put at \$4 Billion: Report by U.N. Panel Includes Damages to Economy.'' \textit{Los Angeles Times.}  25 October 1985. Accessed September 2017: \url{http://articles.latimes.com/1985-10-25/news/mn-14160_1_mexico-city}\DIFdelbegin \DIFdel{.}\DIFdelend } While the resulting destruction prompted the Mexican government to institute a number of regulatory measures to limit future damage, 32 years later, Mexico City's 2017 earthquake still resulted in the deaths of at least 360\footnote{The Associated Press. ``Death toll rises to 360 in Mexico earthquake.'' \textit{The Denver Post.} 21 September 2017. Accessed October 2017: \url{http://www.denverpost.com/2017/09/30/mexico-earthquake-death-toll-update/}} and the recovery effort could cost more than 2 billion dollars.\footnote{\DIFdelbegin \DIFdel{`}\DIFdelend The Associated Press. \DIFdelbegin \DIFdel{' }\DIFdelend ``Economic Costs of Mexico's Earthquake Could Surpass \$2B.'' \textit{Insurance Journal} 29 September 2017. } Even more devastating was the 2004 Indian Ocean earthquake (the fourth largest since the \DIFdelbegin \DIFdel{1900s}\DIFdelend \DIFaddbegin \DIFadd{1900's}\DIFaddend ) and tsunami which led to the deaths of more than 200,000 people across 13 countries, causing around USD 7.5 billion in damage.\footnote{Pickrell, John. ``Facts and Figures: Asian Tsunami Disaster.'' \textit{New Scientist.} 20 January 2005. Accessed January 2019: \url{https://www.newscientist.com/article/dn9931-facts-and-figures-asian-tsunami-disaster/}}  Meanwhile, the most expensive natural disasters have been in the tens of billions. These  range from  the 2008 Sichuan earthquake (191 billion)\footnote{``Sichuan 2008: A disaster on an immense scale.'' \textit{BBC News.} 9 May 2013. Accessed January 2019: \url{https://www.bbc.com/news/science-environment-22398684}} to the Thai floods in 2011 (45 billion).\footnote{Tang, Alisa. ``Thailand Cleans Up; Area Remain Flooded. \textit{Time.} 2 December 2011. Accessed April 2019, \url{https://web.archive.org/web/20120108085747/http://www.time.com/time/world/article/0,8599,2101273,00.html}} 
%(as well as Hurricane Katrina in 2005 (161 billion dollars).\footnote{``Fast Facts Hurricane Costs''. \textit{Office for Coastal Management} Accessed January 2019: \url{https://coast.noaa.gov/states/fast-facts/hurricane-costs.html}})}

%(include the 2011 Fukushima disaster (202 billion), \footnote{Hornyak, Tim. ``Clearing the Radioactive Rubble Heap That Was Fukushima Daiichi, 7 Years On.'' \textit{Scientific American} 9 March 2018. Accessed January 2019: \url{https://www.scientificamerican.com/article/clearing-the-radioactive-rubble-heap-that-was-fukushima-daiichi-7-years-on/}})}
%which left 82 dead, \footnote{Moravec, Eva Ruth. ``Texas officials: Hurricane Harvey death toll at 82 in 2017, `mass casualties have absolutely not happened.'' \textit{The Washington Post.} 14 September 2017. Accessed September 2017: ' \url{https://www.washingtonpost.com/national/texas-officials-hurricane-harvey-death-toll-at-82-mass-casualties-have-absolutely-not-happened/2017/09/14/bff3ffea-9975-11e7-87fc-c3f7ee4035c9_story.html?utm_term=.f5eecca9ee21}} likely to be the most expensive natural disaster in US history at 180 billion dollars.\footnote{``Hurricane Harvey Damages Could Cost up to \$180 Billion.'' \textit{Fortune}. 3 September 2017. Accessed September 2017: \url{http://fortune.com/2017/09/03/hurricane-harvey-damages-cost/}} 


 % The 2011 Fukushima incident meanwhile, stands out for both its death toll and high cost, leaving nearly 1,600 dead and more than 174,000 displaced.\footnote{Hamilton, Bevan. ``Fukushima 5 years later: 2011 disaster by the numbers.'' \textit{CBC News}. 10 March 2016. Accessed September 2017: \url{http://www.cbc.ca/news/world/5-years-after-fukushima-by-the-numbers-1.3480914}} 2017 projections estimate that it will cost around 187 billion dollars --- double the 2013 estimate.\footnote{McCurry, Justin. ``Possible nuclear fuel find raises hopes of Fukushima plant breakthrough.'' \textit{The Guardian.} 30 January 2017. Accessed September 2017: \url{https://www.theguardian.com/environment/2017/jan/31/possible-nuclear-fuel-find-fukushima-plant}}\url{http://www.insurancejournal.com/news/international/2017/09/29/465995.htm}} 

Few countries are spared the devastation that natural disasters can wreak. Between 1980 and 2004, approximately 7,000 natural disasters led to the deaths of around two million people and further negatively affected the lives of five billion more \citep{emdat:2009}. The economic costs are also considerable and rising, with the direct economic damage from natural disasters between 1980-2012 estimated to be \$3.8 trillion \citep{gitay:2013}.

While dealing with both the immediate and long-term damage wrought by natural disasters can seriously drain existing resources for any country, developing countries find it especially difficult to cope. Often, their existing physical infrastructure is grossly unequal to the task of withstanding natural disasters. Meanwhile, their institutional infrastructure often lacks the resilience or capacity necessary to deal with the often long and complex process of rebuilding. In general, when natural disaster strikes, developing countries are likely to experience more serious physical damage and have less state capacity to recover from it. For example, prior to its 2010 earthquake, Haiti had no building codes and many of its buildings were not designed to withstand even a mild earthquake.\footnote{Watkins, Tom. ``Problems with Haiti building standards outlined.'' \textit{CNN}. 2010 January 14. Accessed September 2017: \url{http://edition.cnn.com/2010/WORLD/americas/01/13/haiti.construction/index.html}} Meanwhile, the lack of governmental leadership and low state capacity, along with other factors, has meant that even 7 years after the disaster, Haiti has yet to fully recover \citep{hartberg:2011}.\footnote{Cook, Jesselyn. ``7 years after Haiti's Earthquake, millions still need aid.'' \textit{Huffington Post}. 13 January 2017. Accessed May 2018: \url{https://www.huffingtonpost.com/entry/haiti-earthquake-anniversary_us_5875108de4b02b5f858b3f9c?guccounter=1}} 
%kobayashi:2014

From a purely tactical perspective then, natural disasters represent an opportune time to inflict harm on a strategic adversary, particularly if it is a developing country, as both  government officials and public resources are fully engaged with responding to the emergency. Yet, anecdotal evidence suggests that strategic adversaries rarely take advantage of this opportunity to overtly initiate hostile actions, at least as far as can be openly observed.\footnote{Note, whether countries take advantage of their strategic opponents using more covert methods during times of natural disaster is a more open question.} Many of the deadliest natural disasters (which should present foreign opponents the best opportunity to inflict harm) do not seem to have been followed up by hostile overtures. For instance, Taiwan did not use the 1976 Tangshan earthquake, believed to be the largest earthquake in the 20th century by death toll, as an opportunity to inflict further harm on China. Similarly the India did not use the occasion of either the 1970 Bhola cyclone in then East Pakistan (the deadliest tropical \DIFdelbegin \DIFdel{cylone }\DIFdelend \DIFaddbegin \DIFadd{cyclone }\DIFaddend ever recorded)\footnote{Halloran, Richard. ``Pakistan Storm Relief a Vast Problem.'' \textit{New York Times.} 30 Nov 1970. Accessed January 2019: \url{https://www.nytimes.com/1970/11/30/archives/pakistan-storm-relief-a-vast-problem-disaster-in-pakistan-created.html}} or the 1991 Bangladesh cyclone, to initiate hostile gestures  %2011 Fukushima disaster was not followed by hostile gestures from China nor did Russia react to Hurricane Harvey with belligerence toward the US.

Context of course matters. There are  different rules of engagement depending on whether one has a contentious versus an actively hostile relationship with another country. In the former context, though taking preemptive action against a strategic opponent may lead to short term gains, it could very well lead to long term losses, especially since such an action would be well out of the realm of socially acceptable behavior in response to a natural disaster. But even by this hard-nosed logic, we might expect countries to simply do nothing when tragedy befalls their strategic opponents. Such behavior would fit well with the larger literature that investigates donor motivations for allocating foreign aid. Indeed, scholars have produced a large body of evidence suggesting that donors overwhelmingly prioritize their own self-interest over recipient need when dispensing aid\footnote{For example, see \citet{mckinlay:1977, mckinlay:1978, mckinley:1979, maizels:1984, schraeder.etal:1998, alesina:2000, berthelemy:2006, stone:2006, demesquita:2007, deMesquita:Smith:2009, bermeo:2008, fleck:kilby:2010, hoeffler:2011, dreher:2015,qian:2015}.} and under certain conditions,  have seen such efforts pay off \citep{deMesquita:Smith:2009, carter:stone:2015, deMesquita:Smith:2016}.

% \citet{deMesquita:Smith:2009}: donors buy policy concessions with aid

Yet, much anecdotal evidence suggests that rather than jockeying for a more favorable strategic perch or doing nothing, natural disasters encourages the flow of \textit{aid} from strategic opponents. For example, during the famine that ravaged North Korea from 1994 to 1998, the United States, South Korea, Japan and the European Union stepped up as the primary donors of food aid \citep{noland:2004}.  Meanwhile, Taiwan was one of the biggest donors to China in the aftermath of the 2008 Sichuan earthquake.\footnote{``FACTBOX-Earthquake aid for China.'' \textit{Reuters.} 14 May 2008. Accessed April 2019: \url{http://uk.reuters.com/article/idUKPEK29448220080514}} Taiwan also actively contributed to the rescue effort,\footnote{French, Howard and Edward Wong. ``In Departure, China Invites Outside Help.'' \textit{The New York Times}. 16 May 2008. Accessed September 2017: \url{http://www.nytimes.com/2008/05/16/world/asia/16china.html}} and further offered to share the technical expertise it developed from its own devastating earthquake experience in 1999.\footnote{Hille, Kathrin. ``Taiwan shares quake lessons with Sichuan.' \textit{Financial Times}. 9 June 2008. Accessed September 2017: \url{https://www.ft.com/content/b0204002-3641-11dd-8bb8-0000779fd2ac}} 

%\citep{buthecheng:2013}
Are these anecdotes of non-strategic behavior indicative of a systemic pattern or one-off exceptions to the rule of strategic self-interest? If the former, what could explain this seemingly humanitarian turn of behavior? Finding an answer to these questions in the current literature is difficult. For one, in evaluating the relative roles that donor interest and recipient need play in foreign aid allocation, what researchers refer to as recipient need may be more precisely understood as ``developmental need'' and as such, targeted towards addressing chronic poverty. To that end, development need is frequently measured using gross domestic product (GDP) or gross national product (GNP) per capita;\footnote{For example, see \citet{mckinlay:1977,mckinlay:1978,mckinley:1979,maizels:1984,alesina:2000,berthelemy:2006,stone:2006,demesquita:2007,bermeo:2008}.} or occasionally with more holistic measures of social outcomes such as the Physical Quality of Life Index,\footnote{See \citet{maizels:1984}.} the average life expectancy,\footnote{See \citet{schraeder.etal:1998}.} or the daily caloric intake.\footnote{See \citet{mckinley:1979,schraeder.etal:1998}.}

Meanwhile, only a small body of research investigates the degree to which aid is given in response to acute crises, such as natural disasters, which will be referred to here as humanitarian need. Considering that around 11\% of official development assistance (ODA) was officially categorized as being given for humanitarian reasons in 2015, the systematic failure to include natural disasters as a potential driver of foreign aid is puzzling.\footnote{Total ODA for DAC countries was 131.6 billion in 2015, 15.6 billion of which was designated as humanitarian assistance \url{http://www.oecd.org/dac/development-aid-rises-again-in-2015-spending-on-refugees-doubles.htm} \url{http://www.oecd.org/dac/stats/humanitarian-assistance.htm}}  What evidence that does exist suggests a null or small effect of humanitarian aid on foreign aid allocations. For instance, \citet{bermeo:2008} finds no relationship between the number of people affected by disasters and the allocation of bilateral aid for France, Japan, the UK and the US.\footnote{Note, \citet{bermeo:2008} also conceptualizes humanitarian aid using measures of the number of refugees and civil war, with mixed effects across countries for both\DIFaddbegin \DIFadd{.}\DIFaddend }  Similarly, \citet{david:2011} finds no statistically significant relationship between development aid flows and climatic or human disasters. David does find evidence for increased development aid following geological disasters, but the effect is only found with a 2 year lag and substantively small.\footnote{\citet{david:2011} defines climatic events as: floods, droughts, extreme temperatures and hurricanes; human disasters as: famines and epidemics; geological events as: earthquakes, landslides, volcano eruptions and tidal waves.} \citet{yang:2008} also finds that ODA increases after a hurricane, but only with a lag of 2 years.\footnote{\citet{stromberg:2007} does find a positive and significant relationship between aid and natural disasters, but his paper is concerned with emergency aid in particular, not foreign aid. Similarly, \citet{olsen:2003} find that donors are more likely to give aid for strategic reasons, though their analysis is confined to emergency aid.} In this paper, we not only seek to investigate \DIFaddbegin \DIFadd{whether }\DIFaddend donors give more aid in response to natural disasters, but to explain why they might do so

%Finally, there appears to be virtually no work that has explored whether there is a conditional relationship between donor's strategic interest and recipient's humanitarian need. To our knowledge, we are first to investigate this question.% whether there may be a conditional relationship between donor's strategic interest and recipient's humanitarian need on foreign aid allocation decisions.  

% One seeming exception is \citet{drury_etal:2005} who find that between 1964 to 1995, the United States made its decision to dispense aid based on strategic considerations, but based the amount given on humanitarian considerations. However, their dependent variable of interest is humanitarian aid, not ODA. 
%%%%%%%%%%%%%%%%%%%%%%

%%%%% Theory %%%%%
\DIFdelbegin %DIFDELCMD < 

%DIFDELCMD < %%%
\DIFdelend \section*{How Natural Disasters Affect Foreign Aid Allocations}

Only in the twentieth century has expending public resources to relieve the human suffering of foreigners shifted from being virtually inconceivable to relatively commonplace. The devastation wrought by the two world wars was particularly instrumental in bringing about this change. However, such aid was strictly intended to serve as temporary transfers that would facilitate a return to the previous status quo, rather than a long-term commitment to ``development'' as such. The turn toward promoting development was instead fostered by ongoing Cold War hostilities, which simultaneously promoted the use of aid to further donor's strategic goals while also building a new norm of rich countries aiding poor countries \citep{lancaster:2008}.

The role of mitigating disaster and suffering on the one hand and furthering strategic interest on the other are thus baked into the modern conception of foreign aid. This history also suggests that humanitarian aid, even if only initially meant to serve as a temporary expedient, may lead to the establishment of aid with longer-term strategic goals. Whether this pattern exists more generally and if so, whether it is driven primarily by strategic or humanitarian concerns is unclear however. The role of the Cold War in foreign aid's origin story  dictated that recipients of humanitarian aid were generally within a particular strategic bloc, making it difficult to untangle strategic from humanitarian drivers. %The vignette of US-Iran aid relations following the 2003 Bam earthquake provides some anecdotal evidence, however, that contrasting examples exist.

As such, looking at how natural disasters affect foreign aid allocation is not only interesting in its own right but also provides an exogenous factor with which to identify the role of donor interest and recipient need in explaining patterns of aid commitments. To that end, we develop a set of hypotheses as to how natural disasters affect foreign aid allocations. Further, to better untangle the varying potential drivers, we disaggregate foreign aid into three types: humanitarian, civil society, and development aid. In doing so, we seek to offer a more nuanced understanding of the principle drivers of foreign aid allocations. 

\subsection*{Short-term Humanitarian Response to Natural Disasters}

Responding to natural disasters quickly and efficiently is often crucial to saving lives and alleviating human suffering as services like electricity, gas, water, and telecommunications may all be disrupted in the immediate period following a natural disaster. The timely deployment of humanitarian aid is the first response that donors can extend to countries struck by natural disaster. In what follows, we develop three hypotheses as to how the interaction between strategic interests and natural disaster severity can affect humanitarian aid allocation. 

We draw first from recent research in behavioral economics, which underscores the idea that different social contexts lead to varying behavior in identical situations \citep{kahneman:2003,do:2011}.\footnote{While there is evidence that non-governmental organizations are driven by the norms of humanitarian discourse when allocating aid \citep{buthe:2012}, evidence for similar behavior in governments has been mixed at best.} Natural disasters may reorient the social context of a dyadic relationship to encourage donors to increase aid to their strategic opponents. That is, the loss of human life and destruction of infrastructure, which natural disasters provoke, can temporarily serve to emphasize the human aspect of the bilateral relationship as opposed to the  political, economic, and military aspects that generally define foreign relations between two countries.

Moreover, if natural disasters do have a humanizing effect, then we might expect strategic opponents to be particularly sensitive to it. That is is, given that strategic opponents are more likely to ``otherize'' each other, then dyadic opponents must traverse a greater gap to humanize the other compared to dyadic allies \citep{de:2012}. On balance then, we would expect that donors do not discriminate between strategic opponents or strategic allies when dispensing aid. For example, historically hostile relations between the US and Cuba may mean that the baseline extent to which they ``otherize'' each other is much greater than in the US-Japan relationship, increasing the potential for Cubans to be humanized in American eyes. As such, we might expect American aid to Cuba rise to the level they would provide to \DIFdelbegin \DIFdel{the }\DIFdelend Japan in the event of similar natural disasters. 

That is not to say that natural disasters can always bridge the divide among strategic opponents. For example, India and Pakistan have had an uneasy history of accepting aid from each other following natural disasters.\footnote{Ravishankar, Siddharth. ``Cooperation between India and Pakistan after Natural Disasters.'' \textit{Stimson Center}. 9 January 2015. Accessed September 2017: \url{https://www.stimson.org/content/cooperation-between-india-and-pakistan-after-natural-disasters}} In general, we contend only that natural disasters may make it more \textit{likely} that a strategic adversary will contribute aid because the humanitarian disaster temporarily reframes the context of bilateral relations. An understanding of the interaction between natural disasters and strategic interests affects humanitarian aid allocations based on social context thus leads us to the following hypothesis:
\begin{inparaenum}
  \item   Donors who are strategic opponents of the recipient are more likely than strategic allies to be sensitive to the humanizing effect of natural disasters. As such, following natural disasters, donors are likely to send \textbf{similar amounts of humanitarian aid to strategic allies and strategic opponents.}
\end{inparaenum}
%\textit{H1A:  Donors who are strategic opponents of the recipient are more likely than strategic allies to be sensitive to the humanizing effect of natural disasters. As such, following natural disasters, donors are likely to send \textbf{similar amounts of humanitarian aid to strategic allies and strategic opponents.}} 

Realist scholars offer an alternative perspective which proclaims that, ``foreign aid is today and will remain for some time an instrument of political power'' \citep{liska:1960}. Under this logic, donors commit aid primarily to recipient countries to further their own strategic interests. Extant literature on the drivers of foreign aid have put forward strong substantive evidence to support this viewpoint \citep{mckinley:1979, maizels:1984, schraeder.etal:1998, alesina:2000, berthelemy:2006, stone:2006, demesquita:2007, bermeo:2008, dreher:2015}. With regards to the interaction between natural disasters and strategic interests, it is in donor's self-interest to commit greater amounts of humanitarian aid to their strategic allies rather than opponents in the event of a natural disaster. A naive reading of the logic of realism would lead to the following hypothesis as to how the interaction between natural disasters and strategic aid affects humanitarian aid allocations: 
\begin{inparaenumb} 
  \item  Donors are driven by self-interest and in the event of natural disasters, donors are \textbf{likely to send less humanitarian aid to their strategic opponents vs their strategic allies}
\end{inparaenumb}
%\textit{H1B: Donors are driven by self-interest and in the event of a natural disaster, donors are \textbf{likely to send less humanitarian aid to their strategic allies opponents vs their strategic allies.}}

A more sophisticated realist perspective, however, suggests that natural disasters may present donors with a strategic opportunity to improve relations with strategic opponents.  As suggested in H1A, social context does matter, but only to the extent  that it limits the acceptable set of responses to natural disasters to the allocation of humanitarian aid (as opposed to, for example, the use of hostile overtures). However, donors may still seek to work within this framework of humanitarian altruism to forward their own interests. 

Indeed, disaster-afflicted countries appear to be sensitive to the possibility that accepting humanitarian aid from strategic opponents may come with ulterior motives. In 1999 for example, Venezuela experienced catastrophic flash floods and debris flows in Vargas State, which left as much as 10\% of the Vargas population dead \citep{wieczorek:2001}. US troops helped in the relief efforts by running helicopter rescue missions and working to provide clean water. However, consistent with his antagonism toward US hegemony in the region, President Hugo Chavez declined US assistance in rebuilding a critical highway, saying that while, ``he would accept American equipment if Venezuelan soldiers operated it...he did not want US troops in his country.''\footnote{Brand, Richard. ``Chavez assailed on handling of Venezuelan flood disaster.'' \textit{The Miami Herald}. 5 August 2001. Accessed September 2017: \url{http://www.latinamericanstudies.org/venezuela/venezuela-disaster.htm}.} Meanwhile, Iran categorically refused any aid from Israel following the 2003 Bam earthquake, though the Israeli government still encouraged its citizens to donate privately.\footnote{Popper, Nathaniel. ``Israelis Help Iran Victims Despite Rebuff.'' \textit{The Forward}. 2 January 2004. Accessed September 2017: \url{http://forward.com/news/6059/israelis-help-iran-victims-despite-rebuff/}} Indeed, even the US first turned down Russian aid for Hurricane Katrina before ultimately accepting it.\footnote{``U.S. accepts Russian Katrina aid.'' \textit{UPI}. 2 September 2005. Accessed September 2017. \url{https://www.upi.com/US-accepts-Russian-Katrina-aid/39221125680989/}.} Most recently, Venezuelan leader Nicolas Maduro's refused humanitarian aid to alleviate its food crisis under the reasoning that such aid is ``merely a pretext for regime change,'' demonstrating that i) some political actors also suspect that humanitarian aid may be strategically driven and that ii) the use of humanitarian aid for strategic purposes may extend beyond natural disasters (as this particular crisis was largely a function of political missteps).  \footnote{Taladrid, Stephania. ``Venezuela's Food Crisis Reaches A Breaking Point.'' \textit{The New Yorker.} 22 February 2019. Accessed March 2019: \url{https://www.newyorker.com/news/news-desk/venezuelas-food-crisis-reaches-a-breaking-point}}

% Note, regardless of the actual motivation of donors when they give aid to their strategic opponents, t
There is also anecdotal evidence to suggest that aid given under such circumstances can serve to humanize and improve public perceptions of donors as well. For example, in the wake of US and South Korean aid for the North Korean famine, one refugee summarized his reaction to the US Institute for Peace this way: ``We were taught all these years that the South Koreans and Americans were our enemies. Now we see they are trying to feed us. We are wondering who our real enemies are'' \citep[9]{natsios:1999}. \citet{andrabi:2017} moreover, find that following the inflow of international aid sent to alleviate the damage done during an earthquake in Pakistan 2005, trust in Euoropeans and Americans was much higher in the affected population.  This evidence suggests that, at least in certain contexts, humanitarian aid can serve to improve relations with strategic opponents.
\DIFaddbegin 

\DIFadd{Note that the underlying assumption is that the donor country is ultimately motivated to further its own strategic interests. A priori, the donor country cannot know whether such overtures of humanitarian aid will improve relations with the recipient government, improve perceptions among the recipient population, or both. The incidence of a natural disaster merely provides the donor country a window of opportunity to do so, thus potentially giving it more latitude to further its strategic goals. Improving relations with the recipient government may, on the margin, deter recipient governments from taking actions that conflict with donor interests. Meanwhile improving perceptions of the donor country among the recipient population may also limit the extent to which a still hostile recipient government may enact policies that directly counter donor interests. }\DIFaddend Here, however, we are primarily interested in investigating whether donors are driven by this possibility when allocating aid, leading to our third hypotheses:


\begin{inparaenumc}
  \item   Donors see natural disasters as a strategic opportunity to improve their relations with strategic opponents and are thus are  likely to send \textbf{more humanitarian aid to strategic opponents versus allies.}
\end{inparaenumc}
%\textit{H1C:  Donors see natural disasters as a strategic opportunity to improve their relations with strategic opponents and are thus are  likely to send \textbf{more humanitarian aid to strategic opponents versus allies.}} 

\subsection*{Long-term Responses to Natural Disasters}

Donor countries may dispense aid that not only serves to immediately address the natural disaster at hand, but also  to further longer-term objectives. Here, we make a distinction between civil society aid and development aid. Civil society aid is aimed at supporting non-governmental organizations (NGOs) and their programs. The stated purpose of such aid is to empower grass-roots advocacy and improve governance and government accountability. Meanwhile, development aid is targeted toward promoting long-term economic development in a recipient country often through the building of infrastructure like roads and hospitals as well as \DIFdelbegin \DIFdel{the growth of human resources }\DIFdelend \DIFaddbegin \DIFadd{through the promotion of human capital }\DIFaddend via technical training and education. In what follows, we develop hypotheses as to how the interaction between strategic interest and natural disasters can affect the allocation of these two different types of aid.

\subsubsection*{Natural Disasters as Strategic Opportunities}

Donors generally distribute aid to civil society not only for its intrinsic value but also, and arguably primarily, for its perceived instrumental value in either promoting democratization \citep{robinson:1995,ottaway:2000} or economic development \citep{kral:2013}. However, we make the distinction between civil society and development aid because while any given donor may commit civil society aid to promote economic development, lending support to civil society at all is an inherently political act.\footnote{Carothers, Thomas and Diane de Gramont. ``The Prickly Politics of Aid.'' \textit{Foreign Policy}. 12 May 2013. Accessed June 2018: \url{http://foreignpolicy.com/2013/05/21/the-prickly-politics-of-aid/}} From supporting the growth of government watch dogs to increasing the domestic capacity for grass roots advocacy, whether it is their intention or not, donors are able to exert influence over a recipients domestic politics by directing funds to civil society.

Thus if, as following the realist logic, foreign aid is used to promote donor interests, then donor governments should be especially inclined to increase the allocation of civil society aid.  With respect to natural disasters, countries may be motivated to give more civil society aid to their strategic opponents because the temporary suspension in the normal dynamics of the relationship represents a unique opportunity to increase civil society aid and initiate a shift in the nature of the bilateral relationship (as in H1C). Donors can seize on a country's inherent vulnerability following a natural disaster to  decide to \textit{strategically} increase their civil society aid so as to increase their chances of exerting domestic influence over the recipient countries. 

To draw a concrete example, following the 2004 Indian Ocean Earthquake and Tsunami, the US began committing aid to civil society groups in Somalia. Though the initial nominal amount was a drop in the bucket in absolute terms, considering that no aid was given to civil society in Somalia prior to the natural disaster and such aid has been steadily growing over the past decade, this represented a substantial change in US aid commitments to Somalia.\footnote{Data collected from USAID from: ``USAID Foreign Aid Explorer''. Accessed January 2019: \url{https://explorer.usaid.gov/}} Given that the U.S. had closed its embassy in Somalia in 1991 and only re-established diplomatic presence in 2018.\footnote{Watkins, Eli and Jennifer Hansler. ``State Department announces re-establishment of `permanent diplomatic presence' in Somalia.'' \textit{CNN.} 4 December 2018. Accessed January 2019: \url{https://edition.cnn.com/2018/12/04/politics/us-somalia-state-department/index.html}}, it seems plausible to interpret this as strategic gambit on the US' part to gain a foothold in Somalia, and if so, a successful one. Before jumping to conclusions however, note that the US also increased civil society aid to Indonesia at the same time.\footnote{Data collected from USAID from: ``USAID Foreign Aid Explorer''. Accessed January 2019: \url{https://explorer.usaid.gov/}} Given that Indonesia was affected much more severely by the earthquake than Somalia\footnote{``India, Indonesia, Maldives, Myanmar, Somalia, Thailand: Earthquake and Tsunami OCHA Situation Report No. 14''. \textit{ReliefWeb.} 7 January 2005. Accessed January 2019: \url{https://reliefweb.int/report/india/india-indonesia-maldives-myanmar-somalia-thailand-earthquake-and-tsunami-ocha-situation}} but had also enjoyed much closer ties to the U.S., it would be difficult to substantiate our proposed mechanism based on anecdotal evidence alone. We thus test the following hypothesis through statistical modelling:  

%As such, we derive the following hypothesis:


\begin{hypolist}[series=test]
\setcounter{hypolisti}{1}
\item Natural disasters present an opportune window for donors to exert influence over recipients who are their strategic opponents and as such, donors are more likely to send additional \textbf{civil society aid} to their strategic opponents
\end{hypolist}
%\textit{H2: Natural disasters present an opportune window for donors to exert influence over recipients who are their strategic opponents and as such, donors are more likely to send additional \textbf{civil society aid} to their strategic opponents.}  

If on the contrary, donors are purely driven by the potential intrinsic or instrumental payoffs of supporting civil society, then donors should be no more motivated to support the civil society of their strategic opponents over that of their strategic allies and we should find no support for this hypothesis. 

\subsubsection*{Natural Disasters and Development Aid}

Whereas humanitarian aid provides stop-gap measures to address the immediate aftermath of a natural disaster, the focus of development aid is to build the conditions for long-term, sustainable economic growth. Here we simply expect that donor countries are more likely to give this type of aid to countries that they want to see economically develop and prosper, namely, their strategic allies. This accords with the more simple notion of realism, similar to H1B, that countries will seek to support allies rather than opponents irrespective of the number of natural disasters. This results in the following hypothesis:


\begin{hypolist}[resume=test]
\item Donors are more likely to send greater \textbf{development aid} to their strategic allies irrespective of the number of natural disasters. 
\end{hypolist}
%\textit{H3: Donors are more likely to send greater \textbf{development aid} to their strategic opponents irrespective of the number of natural disasters. }  

If on the contrary, donors seek only to promote development according to recipient need and without regard to its own potential benefit, then donors should be no more motivated to support the development of  their strategic allies over that of their strategic opponents and we should find no support for this hypothesis. 


\section*{Measuring Strategic Relationships}

One reason for evaluating the \textit{motivations} for aid and not aid \textit{outcomes} is that aid given for strategic reasons may still further development objectives, albeit incidentally, while aid given for humanitarian reasons may also bring unexpected strategic benefits \citep{maizels:1984}. However, evaluating the motivations for aid is not a straightforward process -- any given aid project may work toward providing assistance to a recipient country as well as strategic benefits to a donor country. 

Of critical importance to investigating whether strategic considerations (and by extension, the interaction between strategic considerations and humanitarian need) affects foreign aid considerations then is constructing a reliable measure of strategic interest. Unfortunately, we find that \citet[35]{alesina:2000}'s observation  that,  ``the measurement of what a `strategic interest' is varies from study to study and is occasionally tautological,'' still holds true.  Indeed, strategic interest has alternately been operationalized as: trade intensity \citep{berthelemy:2004,bermeo:2008,hoeffler:2011}, UN voting scores \citep{alesina:2000, alesina:2002,hoeffler:2011,dreher:fuchs:2015}, arms transfers \citep{maizels:1984}, colonial legacy \citep{alesina:2000, bermeo:2008, berthelemy:2004,berthelemy:2006,carnegie:2017}, alliances \citep{bermeo:2008,schraeder.etal:1998}, regional dummies \citep{bermeo:2008,berthelemy:2006, maizels:1984}, bilateral dummies \citep{alesina:2000, berthelemy:2004, berthelemy:2006}\footnote{A US-Egypt or US-Israel dummy seems to be the most common instance of a bilateral dummy.} or some combination of the above.\footnote{Meanwhile other papers take a negative approach and argue that any shortfall between what would theoretically be expected from poverty-efficient aid allocation and actual aid allocation \citep{collier:2002,nunnenkamp:2006,thiele:2007}, or similarly between a theoretical allocation based on good governance and actual aid allocation \citep{dollar:2006,neumayer:2005}, is evidence of strategic interest at play.} 

Such inconsistency in the operationalization of strategic interest is not simply a matter of using different variables to measure the same concept but a matter of using different variables to measure different \textit{aspects} of the underlying concept. However, while a dyad's strategic bilateral relationship is quite multifaceted, to date, there has not been a readily available measure of strategic relationships which captures its various aspects the same way that scholars have done for other complex concepts.\footnote{For example, Polity and Freedom House have provided measures or political institutions while the World Bank's World Governance Indicators (WGI) project provides measures for six dimensions of governance} To address this problem, we create a new measure of strategic interest that is able to account for varying aspects of strategic interest. 

\subsection*{A new measure of strategic relationships}

To generate a measure of strategic relationships we adopt a latent variable approach that enables us to estimate a relational measure of interest between countries by taking into account the direct and indirect ways in which states are connected across a variety of dimensions. Specifically, we utilize three dimensions of state relations to construct our strategic interest measure: dyadic alliances, UN voting, and joint membership in intergovernmental organizations (IGOs). We focus on these dimensions because each provides distinct a representation of the strategic relations between countries in the international system and have been commonly employed in the foreign aid literature. Alliances largely capture the strategic and military aspect of the dyadic relationships.  In contrast, joint membership in IGOs reflects the dyadic relationship across many  diverse issue areas expressed across correspondingly many fora while UN voting is better able to capture this relationship in a centralized forum. 

To estimate a measure of strategic interest across these dimensions, we take a network based approach that allows us to leverage both the direct and indirect ways in which states are connected to one another. To do this we employ a latent factor model as described in \citet{hoff:2005}. The model is structured as follows:

\begin{align}
\begin{aligned}
	Y = \textbf{u}_{i}^{T} & \textbf{u}_{j} + \epsilon_{ij} \text{, where} \\
	&\textbf{u}_{i} \in \mathbb{R}^{R=2}, \; i \in \{1, \ldots, n \} \\
	% &\Lambda \text{ a } K \times K \text{ diagonal matrix}
\end{aligned}
\end{align}

$Y$ here is a $n \times n$ undirected sociomatrix in which $y_{ij}$ designates whether there exists a link (e.g., an alliance) between $i$ and $j$. The goal of the model is to provide a projection of the systematic variation in $Y$ into a two-dimensional social space.\footnote{The latent factor model we utilize here is based on an eigvenvalue decomposition that seeks to represent relations between countries as the weighted inner-product of country-specific vectors of latent characteristics. In this application, we project our $n \times n$ sociomatrix into a $n \times 2$ matrix of country positions in a latent social space.} More precisely, the types of systematic variation that we are interested in include the concepts of (a) transitivity, (b) balance and (c) clusterability. Formally, a set of three countries $ijk$ is said to be transitive, if for whenever $y_{ij} = 1$ and $y_{jk} = 1$, we also observe that $y_{ik} = 1$. This follows the logic of `` a friend of a friend is a friend''. Meanwhile, the relationships between $ijk$ are said to be balanced if $y_{ij} \times y_{jk} \times y_{ki} >0$. Conceptually, if the relationship between $i$ and $j$ is ``positive'', then both will relate to another unit $k$ identically, either both positive or both negative. Finally, relationships between $ijk$ are said to be clusterable if it is balanced or all the relations are all negative. It is a relaxation of the concept of balance and seeks to capture groups where the measurements are positive within groups and negative between groups.

Thus third order dependencies suggest that ``knowing something about the relationship between $i$ and $j$ as well as between $i$ and $k$ may reveal something about the relationship between $i$ and $k$, even when we do not directly observe it'' \citep[141]{hoff_ward:2008}. Such dependences would seem especially relevant for our purposes, as one cannot understand the strategic relationship between two countries without taking into account their respective relationships with other countries. The importance for accounting for these dynamics have long been acknowledged in the foreign aid literature. \citet[877]{trumbull:1994} for example, note that, ``donors do make their decisions with knowledge of what each other are doing, and may actually act cooperatively. Any study that ignores the interrelationship of donor behavior risks problems with simultaneity bias.'' However, we find that until now, this critique has largely gone unaddressed by  existing analyses. 

The main advantage of calculating the latent space of different dyadic variables as opposed to using alternative specifications such as the S Score algorithm\footnote{\citet{leeds:2007}, for example, measure a states ``threat environment'' as the set of all states for which ones is contiguous with or which is a major power and with an S score below the population median.} is that it allows us to better account for indirect ties that states share. Indirect ties are accounted for within this framework because the latent factor model takes patterns such as transitivity into account. As a result, the relation between two actors can be inferred even if no
direct interaction between them is observed.

We employ this latent factor model on every year for each of our three measures.\footnote{The models are estimated via Gibbs sampling from the full conditional distributions of $\textbf{u}_{i}^{T} \textbf{u}_{j}$. For a more detailed discussion of this model, see \citet{hoff:2005}.} In Figure \ref{fig:polLat}, we present a visualization of the resultant latent space we calculated for each variable for the year 2005.

\begin{figure}[h!] 
\centering
	\begin{minipage}{.33\linewidth}
		\centering
		\label{fig:ally}
		\includegraphics[width = 1.1\textwidth]{\detokenize{ally_2005.jpg}}
		\caption*{(a) Alliances}
		\end{minipage}
		\hspace{0.2in}
		\begin{minipage}{.33\linewidth}
		\centering
		\label{fig:un}
		\includegraphics[width = 1.1\textwidth]{\detokenize{un_2005.jpg}}
		\caption*{(b) UN voting}
		\end{minipage}
		%\hspace{.2in}
		\begin{minipage}{.33\linewidth}
		\centering
		\label{fig:igo}
		\includegraphics[width = 1.1\textwidth]{\detokenize{igo_2005.jpg}}
		\caption*{(c) IGO membership}
	\end{minipage}
    \caption{Latent Spaces for components of Strategic Interest Measure during 2005}
\label{fig:polLat}
\end{figure}

\indent\indent Countries that cluster together in this two-dimensional latent space are more likely to interact with each other. The plots for alliances, UN voting and IGO membership suggest that there is distinct clustering among countries. Moreover, these clusters are different across the three measures, suggesting that each variable is indeed capturing different aspects of strategic interest.

\indent\indent After estimating the latent spaces for these components, we estimate the distance between each dyadic pair for the three components for each year. We then combine them in a principal components analysis (PCA) to reduce the dimensionality of our measure while retaining as much variance as possible to maximize our explanatory power. We estimate the PCA of these variables for each year separately\footnote{For each year, we conduct a bootstrap PCA of 1000 subsamples.} and use the first principal component for each year as our measure of strategic interest. For more information about how this PCA was conducted, please see the Online Appendix. The end result of this process is a measure of strategic interest that takes into account indirect ties while also accounting for multiple dimensions in which states interact with one another.\footnote{With regards to the strategic interest measure, we also estimate a model in which we incorporate the uncertainty in the estimation of our latent variable (see Figure \ref{fig:latVarUncert} in the Online Appendix).} 

% Our measure of strategic relationships introduces greater coherency to the literaturex by providing a more rigorous measure that captures aspects of strategic interest. We do so by providing a latent space representation of indirect ties that states share on three dimensions: dyadic alliances, UN voting and joint membership in an intergovernmental organizations (IGOs). After generating each latent space representation, we calculate the distances between countries in that space, and then utilize a principal components analysis (PCA) across the distances to generate a single measure.\footnote{As such, our political strategic relationship measure is the first principal component that results from the PCA of the latent distance between each variable.} 
%%%%%%%%%%%%%%%%%%%%%%


%%%%% Theory %%%%%
\section*{Data}
\label{data}

\subsection*{Aid flows}

Our data \DIFdelbegin \DIFdel{from }\DIFdelend \DIFaddbegin \DIFadd{for }\DIFaddend foreign aid flows is taken from the AidData project \citep{tierney2011more}. This database includes information on over a million aid activities from the 1940s to the present. We use the country level aggregated version of this database to create a directed-dyadic dataset of total aid dollars committed. In this analysis, we focus specifically on OECD donor countries as they both are the best able and have the best incentive to give foreign aid to advance their strategic interests. In the final tally, our dataset includes the 18 most active senders\footnote{More specifically, the included donor countries are: Australia, Belgium, Canada, Denmark, France, Germany, Greece, Iceland, Ireland, Italy, Luxembourg, the Netherlands, Norway, Portugal, Spain, Sweden, the United Kingdom and the United States. These countries were chosen both to maximize comparability with previous work as well as for reasons of data availability. Research on non-DAC donors suggests that like DAC donors, they seem to be primarily driven by strategic motivations in distributing aid \citep{neumayer:2003,dreher_etal:2011,fuchs:Vadlamannati:2013,dreher:2015,dreher:2018}.  Existing evidence suggests that non-DAC donors do seem more likely to give aid following a natural disaster however \citep{dreher_etal:2011}, though they still only account for at most 12\% of humanitarian aid in any given year \citep{harmer:2005}. This research suggests that our findings might be even stronger among non-DAC donors. Future work investigating this possibility will become increasingly important the more foreign aid non-DAC donors distribute. } and 167 receivers of aid flows from 1975 to 2005. Accounting for all possible senders of aid during this time frame is difficult because of the amount of missing data. That being said, issues with missingness in our dataset still exist and we deal with them by employing a multiple imputation method developed by \citet{hoff:2007} and shown to have good performance by \citet{hollenbach:2014}. %However, even with the limited number of senders in this version of our analysis we still have approximiately 40,000 observations worth of data to work with.

We use the AidData's Sector coding scheme in order to disaggregate bilateral ODA into humanitarian aid, development aid, and civil society aid.\footnote{``AidData's Sector Coding Scheme.'' \url{http://docs.aiddata.org/ad4/files/aiddata_coding_scheme_0.pdf}}  To that end, our measure of humanitarian aid encompasses the sectors of:\\ 

% \begin{quote}
% 	``Emergency Response'', ``Reconstruction Relief'', and ``Disaster Prevention and Preparedness''.
% \end{quote}

\begin{tabular}{lll}
	\hline
    \multirow{2}{*}{``Emergency response''} & \multirow{2}{*}{``Reconstruction Relief''} & ``Disaster Prevention  \\[-.5mm]
	~ & ~ &  $\quad$and Preparedness'' \\
    \hline
\end{tabular}\\

\noindent Meanwhile, civil society aid is measured as aid to the sectors of:\\

% \begin{quote}
% 	``Government and Civil Society',  ``Women'', ``Support to Non-Governmental Organizations and Governmental Organizations''.
% \end{quote}

\begin{tabular}{lll}
	\hline
	\multirow{3}{*}{``Government and Civil Society''} & \multirow{3}{*}{``Women''} & ``Support to Non-Governmental  \\[-.5mm]
    ~ & ~ & $\quad$Organizations and Governmental \\[-.5mm]
    ~ & ~ &  $\quad$Organizations'' \\
    \hline
\end{tabular}\\

\noindent Finally, development aid is defined as aid given to the following sectors:\\ 

\begin{tabular}{lll}
  \hline
  ``Education'' & ``Health'' & ``Water Sanitation'' \\
  ``Other Infrastructure & ``Economic Infrastructure & \multirow{2}{*}{``Environmental Protection''} \\[-.5mm]
  $\quad$ and Services'' & $\quad$ and Services'' & ~ \\
  ``Other Social  & ``Agriculture Forestry & ``Industry, Mining \\[-.5mm]
  $\quad$Infrastructure and Services'' & $\quad$ and Fishing'' &  $\quad$and Construction'' \\
  ``Other Development Aid'' & '``Food Aid'' & ``Debt Relief'' \\
  \hline
\end{tabular}\\

We note that bilateral ODA often represents only one channel through which donors may allocate foreign aid and that an increasing number of papers have argued for accounting for the heterogeneity of aid channels donors may use when estimating drivers of foreign aid \citep{nunnenkamp:2011,buthecheng:2013,dietrich:2013}. Here, we choose to focus solely on bilateral aid in order to maintain greater comparability with previous studies. % but we can look at the differentiated effects as robustness checks?

\subsection*{Strategic Interest}

As previously stated, we created our measure of strategic relationships by conducting a PCA on the latent distances for alliances, UN voting and joint IGO membership. Data for alliances was retrieved from the Correlates of War (COW) Formal Alliance dataset \citep{gibler:2009}. Following \citet{demesquita:1975} and \citet{signorino:1999}, we distinguish between different types of alliances with the following weighting scheme: 0 = no alliance, 1 = entente, 2 = neutrality or nonaggression pact, 3 = mutual defense pact.\footnote{Note, as for alliances, we had attempted to distinguish between different types of membership but found that very few states were listed as Associate Members or Observers of an IGO for the time period that we are conducting our analysis. Thus we used the simpler coding scheme.} 

\indent\indent UN voting data was obtained from the United Nations General Assembly Data set \citep{strezhnev:2012}.  We calculate the proportion of times two states agree out of the total number of votes they both voted on. Agreement means either both vote yes, both vote no, or both abstain. This measure is similar to the `voting similarity index' readily available from the dataset except the voting similarity index does not account for mutual abstentions. 

\indent\indent  Meanwhile IGO voting data was obtained from the Correlates of War International Governmental Organizations Data Set \citep{pevehouse:2004}. A total of 529 IGOs across a broad swath of topics, including trade, communications, and health and security,  are represented in this dataset. Dyads were coded as 1 if they belonged to the same IGO as a full member or an associate member and coded as 0 if one or both of them was an observer, had no membership, was not yet a state or was missing data.\footnote{Information on the IGOs included in the dataset are available from the Correlates of War website: \url{http://www.correlatesofwar.org/data-sets/IGOs}} 

\subsection*{Natural Disasters}
%\footnote{\url{http://www.emdat.be/}}
Almost all the empirical work on natural disasters relies on the publicly available Emergency Events Database (EM-DAT) maintained by the Center for Research on the Epidemiology of Disasters (CRED) at the Catholic University of Louvain, Belgium. EM- DAT defines a disaster as a natural situation or event which overwhelms local capacity and/or necessitates a request for external assistance. For a disaster to be entered into the EM-DAT database, at least one of the following criteria must be met: i) 10 or more people are reported killed; ii) 100 people are reported affected; iii) a state of emergency is declared; or iv) a call for international assistance is issued.  We  use a count of the number of natural disasters a country has experienced a year as our measure of natural disaster severity. 

\subsection*{Additional Covariates}

In addition to our dyadic strategic relationship measures, we include a number of covariates to capture characteristics of aid recipients.

For our measure of political institutions, we use Polity IV data available from the Center for Systemic Peace \citep{gurr:2010}. Polity IV captures differences in regime characteristics on a 21 point scale ranging from -10 (hereditary monarchy) to +10 (consolidated democracy), rescaling it to range from 1 to 21 for greater ease of interpretation.   We also controlled for colonial history using the Colonial History Data Set from the Issue Correlates of War Project \citep{hensel:2009}. This variable is coded as a one when the receiver in a sender-receiver dyad is a former colony of the sender and zero otherwise. 

Meanwhile, for our measures of developmental need, we use (1) Log GDP per capita and (2) life expectancy at birth.  Both of these measures are extracted from the World \citet{wb:2013}. Finally, we control for the incidence of civil war in a recipient country as it informs the ability for a donor country to dispense aid. We do so with data retrieved from the Uppsala Conflict Data Program (UCDP)/International Peace Research Institute (PRIO) Armed Conflict Database. \citep{gleditsch:2002}. We code as civil war any armed conflict which either (a) ``Internal armed conflict occurs between the government of a state and one or more internal opposition group(s) without intervention from other states'' or (b) ``Internationalized internal armed conflict occurs between the government of a state and one or more internal opposition group(s) with intervention from other states (secondary parties) on one or both sides.''
%%%%%%%%%%%%%%%%%%%%%%

%%%%% Empirics %%%%%
\section*{Analysis}
\label{empirics}

\subsection*{Estimation Method}

To model aid flows using our directed-dyadic panel dataset, we utilize a hierarchical model. We include random intercepts in our model for every dyad and year. More concretely, we fit the following model: 

\begin{align*}
  Log(Aid)_{sr,t} &= \beta_{1}(Pol. \; Strat.  \; Distance_{sr,t-1})  \\
  & \;+\; \beta_{2}(Colony_{sr,t-1}) \;+\; \beta_{3}(Polity_{r,t-1}) \\
  & \;+\; \beta_{4}Log(GDP \; per \; capita_{r,t-1}) \;+\; \beta_{5}(Life \;Expect_{r,t-1}) \\  
  & \;+\; \beta_{6}(No. \; Disasters_{r,t-1}) \;+\; \beta_{7}(Civil \; War_{r,t-1}) \\
  & \;+\; \beta_{8}(Pol. \; Strat.  \; Interest_{sr,t-1} \times No. \; Disasters_{r,t-1}) \\
   & \;+\; \delta_{s,r}  \;+\; \rho_{t} \\
\end{align*}
\FloatBarrier

\noindent Where $\delta_{s,r}$ and $\rho_t$ are the sender-receiver and year random effects respectively.\footnote{In terms of the model, we find that our results hold when we estimating the model with donor and year fixed effects, though the results from a Hausman test suggest that a random effects model is still a better fit for our data (see Figure \ref{fig:devIntCoef_fixed}).} We use one year lags because while our natural disaster data is pinpointed to the day, we do not have correspondingly fine-grained data on foreign aid distributions. Thus we take a conservative approach and lag by one year to guarantee that the aid is committed after the incidence of a natural disaster.

The results of this analysis are shown below in a coefficient plot in Figure \ref{fig:intCoef}. We test Hypotheses 1A, 1B and 1C using ``Humanitarian Aid'' as the dependent variable. The results show a positive and statistically significant relationship between the interaction of \textit{Strategic Distance} and the \textit{No. Disasters}. To interpret these results, we turn to Figure \ref{fig:simEffects} (``Humanitarian Aid'' panel) where we plot the substantive effect of this interaction term on humanitarian aid over the range of \textit{Strategic Distance} for different levels of natural disaster severity. 

\begin{figure}
	\centering
	\includegraphics[width=1\textwidth]{intCoef.pdf}
	\caption{Coefficient plots for the main analyses with interaction terms across each dependent variable \DIFdelbeginFL \DIFdelFL{, }\DIFdelendFL \DIFaddbeginFL \DIFaddFL{(}\DIFaddendFL humanitarian aid, civil society aid and development aid\DIFaddbeginFL \DIFaddFL{)}\DIFaddendFL .  Coefficients that are significant at the 5\% level are shaded in blue if the coefficient is positive and red if the coefficient is negative. Coefficients that are not significant at the 5\% level are shaded in gray. }
	\label{fig:intCoef}
\end{figure}
\FloatBarrier

\begin{figure}
	\centering
	\includegraphics[width=1\textwidth]{simComboPlot.pdf}
	\caption{Simulated substantive effect plots for each dependent variable (humanitarian aid, civil society aid, and development aid) for different levels of natural disaster severity across the range of the strategic distance measure. A rug plot is provided below each panel.}
	\label{fig:simEffects}
\end{figure}
\FloatBarrier

These results suggest that the greater the number of natural disasters a country experiences, the more likely it is to receive humanitarian aid from a strategic adversary. This is apparent in the rising slope between strategic interest and humanitarian aid as the number of natural disasters increases. As such, these results are consistent with H1C, which suggests that donors may be more likely to dispense humanitarian aid to their strategic adversaries because such disasters present unique opportunities to improve bilateral relations. Notably when natural disasters are particularly severe, donors may dispense a great deal more aid to strategic opponents compared to strategic allies to further their strategic interests.  

Conversely, support for H1A is missing. In particular, we would have expected there be a downward sloping relationship between strategic interest and humanitarian aid when there are no natural disasters. However, if natural disasters had a humanizing effect on strategic opponents, then we would have expected the slope between strategic interest and humanitarian aid to flatten as the number of natural disasters increased, which we do not find. 

Support for H1B is also lacking.  To find support for H1B, which hypothesizes that donors are more likely to give to their strategic allies in the wake of a natural disaster to further their own self-interest, we would have expected the parameter estimate for the interaction term between strategic interest and natural disasters to be negative, which it is not. Moreover, we would have expected to observe a downward sloping relationship between strategic interest and humanitarian interest as the number of natural disasters increases. This is clearly not evidenced in the ``Humanitarian Aid'' panel in Figure \ref{fig:simEffects}. 

Meanwhile, we test H2 by examining the effect of the interaction between strategic interest and natural disasters on civil society aid. In Figure \ref{fig:intCoef}, we find a positive and significant relationship between this interaction and civil society aid. The substantive effects plot (in the ``Civil Society Aid'' panel in Figure \ref{fig:simEffects}) meanwhile also suggests that donors are more likely to target aid to civil society in their strategic adversaries the more natural disasters that country experiences, supporting H2. These results provide support for the idea that donors may be acting to take advantage of vulnerable recipients to mold the relationship to their interests. 

Finally, we test H3 by analyzing how the interaction between strategic interest and natural disasters affects development aid allocation. From, Figure  \ref{fig:intCoef}, we can see that this coefficient is not statistically significant. However, examining the substantive significance in Figure \ref{fig:simEffects} (``Development Aid'' panel) we can see that the relationship between strategic interest and development aid allocation is consistently downward sloping. This suggests that donors tend to give more development aid to strategic allies rather than strategic opponents, showing strong support for H3.  These results indicate that irrespective of natural disaster intensity, development aid is reserved for strategic allies of donor countries and does not alter the strategic calculus donor countries undertake. 

Overall, we believe that we have found strong evidence showing how the role of strategic interest can be heavily conditioned by context. That is, we find donors are more likely to give both more humanitarian and civil society aid to strategic opponents in the face of natural disasters. Our findings are consistent with the argument that they do so in order to take advantage of the opportunities natural disasters provide to improve their relationships with strategic opponents. These results are all the more interesting given that, consistent with the existing literature, we also find that donors are more likely to give development aid to strategic allies irrespective of the number of natural disasters a recipient country experiences. This suggests that donor countries strategically use different types of aid to further their interests in different contexts. 

\subsection*{Persistence of foreign aid allocation over time}

How persistent are these estimated effects? To answer this question, we re-estimate the original models for different lag lengths of the main interaction and constituent terms\footnote{The controls are measured using a one-year lag throughout.}. These models are estimated separately for each lag length (lags of 1, 3, and 5 years). The simulation results when using different lags for the interactions and constituent terms are shown in Figures \ref{fig:humanIntCoef}, \ref{fig:civIntCoef}, and \ref{fig:devIntCoef} for the outcome variables humanitarian aid, civil society aid and development aid, respectively.

\begin{figure}[h!]
	\centering
	\includegraphics[width=1\textwidth]{simComboPlot_lag_hAid.pdf}
	\caption{Simulated substantive effect plots for humanitarian aid for varying lags of variables of interest and different levels of natural disaster severity across the range of the strategic distance measure.}
	\label{fig:humanIntCoef}
\end{figure}

From Figure \ref{fig:humanIntCoef}, we can see that the interaction between strategic interest and natural disasters is persistent until approximately five years after a natural disaster. This suggests that donors seek to use natural disasters as a tactic to improve relations with strategic opponents for a number of years after the initial disaster  (supporting H1C).

\begin{figure}[h!]
	\centering
	\includegraphics[width=1\textwidth]{simComboPlot_lag_cAid.pdf}
	\caption{Simulated substantive effect plots for civil society aid for varying lags of variables of interest and different levels of natural disaster severity across the range of the strategic distance measure.}
	\label{fig:civIntCoef}
\end{figure}

Figure \ref{fig:civIntCoef} shows that while the interaction between strategic interest and natural disasters positively affects the allocation of civil society aid, this effect is only consistent for a short time following a natural disaster. One way to interpret these results is that donors recognize the difficulty of trying to influence domestic politics through civil society aid relatively quickly, and thus waste relatively little time in pursuing such attempts. Another interpretation is that civil society aid is actually rather effective and as such, recipients governments are likely to push back against allowing it in fairly short order. Teasing out the exact mechanism would be a fruitful area for future research.

Last, Figure \ref{fig:devIntCoef} extends the earlier finding that the interaction between strategic interest and natural disasters has little effect on development aid across a variety of different lags. This result further suggests that there is strong support for H3, that is donor counties focus on reserving development aid for strategic allies.

\begin{figure}[h!]
	\centering
	\includegraphics[width=1\textwidth]{simComboPlot_lag_dAid.pdf}
	\caption{Simulated substantive effect plots for development aid for varying lags of variables of interest and different levels of natural disaster severity across the range of the strategic distance measure.}
	\label{fig:devIntCoef}
\end{figure}

\subsection{Robustness Checks}

We also ran a number of checks to test the robustness of our findings. We discuss these checks briefly here and invite readers to learn more in the Online Appendix. In particular, we find that our findings are robust to different operationalizations of the disaster variable, including when using a binary variable for the number of disasters (see Figure~\ref{fig:binaryDisasterSimulation} in the Online Appendix) as well as when using the number killed in natural disasters (see Figure~\ref{fig:killedSim}). \DIFaddbegin \DIFadd{We also run model specifications with lagged dependent variables for both our main models (see Figure~\ref{fig:simEffects_lagDV}) as well as for models that analyze the persistence of foreign aid over time (see  Figures ~\ref{fig:humanIntCoef_lagDV}, \ref{fig:civIntCoef_lagDV}, and \ref{fig:devIntCoef_lagDV}) and find that our findings remain robust and substantively unchanged.  
}\DIFaddend 

\DIFdelbegin \DIFdel{Finally, we }\DIFdelend \DIFaddbegin \DIFadd{We }\DIFaddend also examined whether our results hold across different sub-samples of our data. For instance, \citet{bermeo:2017,bermeo:2018}'s recent work suggests that \DIFdelbegin \DIFdel{in the post 2001 era, }\DIFdelend it is increasingly in the donor's self interest to promote development against negative spillovers from developing countries \DIFaddbegin \DIFadd{in the post 2001 era}\DIFaddend . If a similar logic predominates in the event of a natural disaster, then we would expect it to wash out any consideration of more traditional self interest on the part of the donor. That is, if the prevention of negative spillovers were donor's only concerns, then we would expect donors to give most to countries for whom the potential negative spillovers from natural disasters would be greatest and thus would not expect to find any statistically significant relationship between more traditional strategic interest concerns and humanitarian aid in the event of a natural disaster. Given that, we test whether our findings hold when we restrict our sample to the post 2001 era and find that they do (see Figure \ref{fig:post2001Sim} in the Online Appendix). Numerous works also suggest that aid became less tied to security concerns after the end of the Cold War \citep{fleck:kilby:2010, clist:2009}. As such, we also investigate whether our results are based on a similar dynamic by examining how our findings fare when we restrict our analysis to after the Cold War and find them to be robust (see Figure \ref{fig:postColdWarSim} in the Online Appendix).

\DIFaddbegin \DIFadd{Finally, we also explored whether donors may find it easier to funnel aid to strategic allies through pre-existing channels in the wake of natural disasters. If so, then our findings that donors give more to strategic opponents through humanitarian aid channels may simply be an artifact of of how different aid pipelines are labeled. We investigated whether this was a possibility by both exploring how aid composition changes depending on whether a recipient country is a strategic ally or opponent as well as whether strategic allies are more likely to get more overall aid. We do not find evidence that our findings are a function of a labeling effect (for more details see Figures \ref{fig:aidtype}, \ref{fig:coefPlot_totAidv2}) and \ref{fig:simEffects_totAid} in the Online Appendix). 
}

\DIFaddend Note, a potentially important covariate that we do not control for in our analysis is the role of media and public opinion. While \citet{eisensee:2007} and \citet{stromberg:2007} find that news coverage of a natural disaster is a big factor in shaping US humanitarian aid allocation, \citet{olsen:2003} find that media coverage has only a limited effect on shaping humanitarian aid  across a larger cross-section of donors. Other studies suggest that public opinion can help shape aid allocation \citep{bryant_etal:2018}, including whether aid is given bilaterally or multilaterally \citep{milner:2013}.  This work strengthens our findings to the extent they suggest that increased media coverage and public opinion pushes donors to give purely for humanitarian motivations, in line with H1A. If so then this should have made it our findings in support of H1C less likely, as opposed to more likely, to have been found. All of these studies have either been conducted on select countries or select cross sections of time however; the relevant data to test these propositions over a large panel of countries over time is unfortunately not available and prohibitively costly to collect. Further research on how media coverage and public opinion affect aid allocation following a natural disaster however, will be an important avenue for future research.  
%%%%%%%%%%%%%%%%%%%%%%

%%%%% Conclusion %%%%%
\section*{Discussion}
\label{conclusion}

Our analysis suggests that a more nuanced understanding of the drivers of foreign aid is in order. While recent work has shown that accounting for the channel of aid delivery can go a long way toward understanding aid allocation decisions \citep{dietrich:2013,dietrich:2016}, we show that following natural disasters, donor countries actually direct greater levels of humanitarian aid to strategic opponents rather than allies. We argue that donor countries may allocate foreign aid in this way because they see natural disasters as an opportunity improve relations with their strategic opponents. As shown in our lag models, these findings are surprisingly persistent. 

Moreover, natural disasters not not only affect how donor countries allocate aid for short-term purposes. We find that strategic considerations also reign large when one considers the effect on the distribution of aid with longer-term targets.  Specifically, donor countries are more likely to distribute civil society aid to strategic adversaries as the numbers of natural disasters these countries face increase. Civil society aid inherently involves engagement and intervention in the domestic politics of a recipient country, an increase in civil society aid is indicative of a greater desire to increase donor influence over a recipient country, at least relative to development aid. 

Meanwhile \DIFdelbegin \DIFdel{, in the wake of a natural disaster, }\DIFdelend we find that donors are more likely to give development aid to strategic allies irrespective of exogenous shocks such as natural disasters. Why might donors pursue a sophisticated realist strategy for humanitarian and civil society aid but a naive one for development aid? To answer this question, we argue that context matters; what may further strategic interest in one situation may not work for another. It is nevertheless useful to note that almost 60\% of the total aid flowing from donor countries can be categorized as development aid. This suggests that donors who seek to develop better relations with traditional strategic opponents by dispersing humanitarian and civil society aid recognize the inherent risk in this strategy and invest accordingly.  

These results should be of particular interest as climate change continues to increase the incidence and the intensity of natural disasters. They suggest that while countries that experience natural disasters can expect humanitarian aid even from their strategic adversaries, such help can also open the doors to efforts to influence domestic politics in line with the interests of donors who have historically been antagonistic.

%During this hurricane season alone, residents in the United States have faced the wrath of Hurricane Harvey, Hurricane Irma . Meanwhile, wildfires continue to rage in Northern California. Neither is the rest of the world untouched, as the Mexico City earthquake, flooding in South Asia. \footnote{\url{https://www.nytimes.com/2017/08/29/world/asia/floods-south-asia-india-bangladesh-nepal-houston.html?_r=0}}

%Meanwhile, to revisit the original illustrative example presented in the introduction, note that though the US' offer and Iran's acceptance of humanitarian aid was a head-turning deviation from the status quo, the swiftness with which both countries reverted back to it was equally remarkable.  Indeed,  following the immediate fallout of the earthquake a couple of weeks later, Iran declined US offers of further humanitarian aid \footnote{``Iran to prosecute over building law breaches in Bam.'' \textit{China Daily.} 3 January 2004. Accessed October 2017: \url{https://web.archive.org/web/20090619204216/http://www.chinadaily.com.cn/en/doc/2004-01/03/content_295446.htm}}. Meanwhile, President Bush denied attempts to interpret US aid as evidence of thaw in US-Iran relations.\footnote{\url{http://news.bbc.co.uk/2/hi/middle_east/3362443.stm}}

%In this particular case then, the exchange of aid led to only a temporary reprieve from the generally contentious bilateral relations. However, whether aid given exchanged between historically contentious dyads can lead to a more permanent softening of relations remains an open question. We plan to explore this question more fully in future work. 



%%%%%%%%%%%%%%%%%%%%%%

\newpage
\bibliographystyle{chicago}
\bibliography{Paper/fAidRefs.bib}

%%%%% Appendix %%%%%
\clearpage

\renewcommand{\thefigure}{A\arabic{figure}}
\setcounter{figure}{0}
\renewcommand{\thetable}{A.\arabic{table}}
\setcounter{table}{0}
\renewcommand{\thesection}{A.\arabic{section}}
\setcounter{section}{0}

\appendix
\section{Appendix}
\label{sec:appendix}

\subsection*{Using PCA of latent distance between dyadic pairs to construct measure of strategic interest}

After having first estimated the latent space and then subsequently calculating the latent distance between each dyadic pair for each of our three variables, dyadic alliances, UN voting and joint membership in intergovernmental organizations, we then \DIFdelbegin \DIFdel{needed to combine }\DIFdelend \DIFaddbegin \DIFadd{combined }\DIFaddend these separate distances into one coherent measure.

To do so, we built off of the work of \citet{chen:2012}. They developed a measure of relation strength similarity (RSS) which facilitates the discovery of relationships in complex networks. It allows for the combination of multiple-relationship networks (for the purposes of our paper, these are the latent distances between dyads as measured through alliances, UN voting scores and IGO membership), into a single weighted network (our measure of strategic interest). It does so using a principle components analysis (PCA) for each dyad. To that end, \citet{chen:2012} developed an R package \textit{dils} to calculate the RSS. 

We identified a number of issues with the original coding that we have adapted for our analysis. In particular, we adjusted the code \DIFdelbegin \DIFdel{to }\DIFdelend : i) \DIFaddbegin \DIFadd{to }\DIFaddend scale and center the data as PCA analysis is sensitive to relative scaling of data ii) \DIFaddbegin \DIFadd{to }\DIFaddend sample with replacement as best practice with bootstrapping would seem to indicate that the sample size of each bootstrapped sample should be the same as the size of the original sample iii) \DIFdelbegin \DIFdel{adjusted the code so }\DIFdelend \DIFaddbegin \DIFadd{such }\DIFaddend that the directions of the eigenvectors are consistent across the dyads. We then use the adapted version of this code to calculate the PCA for each dyad pair for a given year and then used the first principle component as our measure of strategic interest, which on average explains 42\% of the variability across the three original measures. 


\subsection*{Validating our measure of strategic interest}

\indent\indent We further conduct a series of post-estimation validation tests for our resulting strategic variable. In particular, we (1) evaluate the relationship between our political strategic interest variable  against S scores and Kendall's $\tau_b$ for alliances and (2) investigate how our measure of strategic interest describe well-known dyadic relationships. 

First, we perform a simple bivariate OLS with and with year fixed effects to evaluate how our measures compare to S scores and Kendall's $\tau_b$.\footnote{Note for comparison that the bivariate relationship of S scores on Kendall's $\tau_b$ is statistically significant with a coefficient of 0.62 while the bivariate relationship of Kendall's $\tau_b$ on S Scores is statistically significant with a coefficient of 0.31.} Note in order to make our strategic measures somewhat interpretable, for the validation we scale our strategic measures to be between 0 and 1 just as S scores and Kendall $\tau_b$ is scaled. The results are shown in Table \ref{table:polval}. % for political strategic interest and Table \ref{table:milval} for military strategic interest. \\

\begin{table}[h!]
\small
\caption{Validation of Political Strategic Interest Variable against S scores and Kendall's $\tau_b$}
\begin{center}
\begin{tabular}{l c c c c c c }
\hline
                    & Unweighted   & Unweighted & Weighted  & Weighted  & Tau-B & Tau-B \\
                   &   S Scores &   S Scores &  S Scores &  S Scores &  &   \\
\hline
(Intercept)         & $0.97^{***}$  & $1.03^{***}$  & $1.01^{***}$  & $1.02^{***}$  & $0.29^{***}$  & $0.25^{***}$  \\
                    & $(0.00)$      & $(0.00)$      & $(0.00)$      & $(0.00)$      & $(0.00)$      & $(0.00)$      \\
Strategic Interest             & $-0.80^{***}$ & $-0.84^{***}$ & $-1.22^{***}$ & $-1.26^{***}$ & $-0.89^{***}$ & $-0.87^{***}$ \\
                    & $(0.00)$      & $(0.00)$      & $(0.00)$      & $(0.00)$      & $(0.00)$      & $(0.00)$      \\
Year FE? 	   & No 		& Yes 		& No		& Yes	& No		& Yes\\
% \hline
% R$^2$               & 0.28          & 0.32          & 0.32          & 0.34          & 0.17          & 0.17          \\
% Adj. R$^2$          & 0.28          & 0.32          & 0.32          & 0.34          & 0.17          & 0.17          \\
% Num. obs.           & 824426        & 824426        & 824426        & 824426        & 824148        & 824148        \\
\hline
\multicolumn{7}{l}{\scriptsize{$^{***}p<0.001$, $^{**}p<0.01$, $^*p<0.05$}}
\end{tabular}
\label{table:polval}
\end{center}
\end{table}
\FloatBarrier

\indent\indent  In brief, we find that our political strategic measure performs well against S scores and Kendall's $\tau_b$ for alliances  with and without fixed effects. Note that because the PCA is of latent distances between any two dyads, dyads that are closer in space will have smaller values and therefore represent a stronger strategic relationship. Therefore the negative relationship we find between the political strategic measure and S scores and $\tau_b$ are interpreted to mean the greater the foreign policy similarity as measured by the S score or Kendal's $\tau_b$ , the smaller the latent distance or the greater the political strategic relationship between a dyad.

\DIFaddbegin \clearpage
\subsection*{\DIFadd{Incorporating Lagged Dependent Variable}}

\DIFadd{We evaluate the robustness of our key findings when including a lagged dependent variable in our model specifications:
}

\begin{align*}
  \DIFadd{Log(Aid)_{sr,t} }&\DIFadd{= \beta_{1}(Aid_{sr,t-1})  }\\
  & \DIFadd{\;+\; \beta_{2}(Pol. \; Strat.  \; Distance_{sr,t-1})  }\\
  & \DIFadd{\;+\; \beta_{3}(Colony_{sr,t-1}) \;+\; \beta_{4}(Polity_{r,t-1}) }\\
  & \DIFadd{\;+\; \beta_{5}Log(GDP \; per \; capita_{r,t-1}) \;+\; \beta_{6}(Life \;Expect_{r,t-1}) }\\  
  & \DIFadd{\;+\; \beta_{7}(No. \; Disasters_{r,t-1}) \;+\; \beta_{8}(Civil \; War_{r,t-1}) }\\
  & \DIFadd{\;+\; \beta_{9}(Pol. \; Strat.  \; Interest_{sr,t-1} \times No. \; Disasters_{r,t-1}) }\\
   & \DIFadd{\;+\; \delta_{s,r}  \;+\; \rho_{t} }\\
\end{align*}
\FloatBarrier

\DIFadd{When including the lagged dependent variable we see no change in the substantive implications of the model for humanitarian, civil society, or development aid. The results including the lagged dependent variable in the specification are shown in Figure~\ref{fig:simEffects_lagDV} and those without are shown in Figure~\ref{fig:simEffects} of the paper.
}

\begin{figure}
	\centering
	\includegraphics[width=1\textwidth]{simComboPlot_lagDV.pdf}
	\caption{\DIFaddFL{Simulated substantive effect plots when including lagged dependent variable in the model for each dependent variable (humanitarian aid, civil society aid, and development aid) for different levels of natural disaster severity across the range of the strategic distance measure. A rug plot is provided below each panel.}}
	\label{fig:simEffects_lagDV}
\end{figure}

\DIFadd{Within the paper we also analyzed the persistence of foreign aid allocations over time. We do so here as well when including a lagged dependent variable in the specification. The results for each of the dependent variables are shown in Figures~\ref{fig:humanIntCoef_lagDV}, \ref{fig:civIntCoef_lagDV}, and \ref{fig:devIntCoef_lagDV} below. The findings from this analysis match with those shown in the paper where a lagged dependent variable is not included in the specification.
}

\begin{figure}[h!]
	\centering
	\includegraphics[width=1\textwidth]{simComboPlot_lag_hAid_lagDV.pdf}
	\caption{\DIFaddFL{Simulated substantive effect plots when including lagged dependent variable in the model for humanitarian aid for varying lags of variables of interest and different levels of natural disaster severity across the range of the strategic distance measure.}}
	\label{fig:humanIntCoef_lagDV}
\end{figure}

\begin{figure}[h!]
	\centering
	\includegraphics[width=1\textwidth]{simComboPlot_lag_cAid_lagDV.pdf}
	\caption{\DIFaddFL{Simulated substantive effect plots when including lagged dependent variable in the model for civil society aid for varying lags of variables of interest and different levels of natural disaster severity across the range of the strategic distance measure.}}
	\label{fig:civIntCoef_lagDV}
\end{figure}

\begin{figure}[h!]
	\centering
	\includegraphics[width=1\textwidth]{simComboPlot_lag_dAid_lagDV.pdf}
	\caption{\DIFaddFL{Simulated substantive effect plots when including lagged dependent variable in the model for development aid for varying lags of variables of interest and different levels of natural disaster severity across the range of the strategic distance measure.}}
	\label{fig:devIntCoef_lagDV}
\end{figure}
\FloatBarrier

\clearpage
\DIFaddend \subsection*{Alternative Parameterization of Disaster Severity}

We have also run our analysis using a dummy variable for whether a natural disaster occurred instead of a count. We show the substantive results of this analysis below in Figure~\ref{fig:binaryDisasterSimulation}. The findings from this analysis reflect those that we observe when we use the count variable. However, given the variation in relationships that we observe when using a count of the number of natural disasters, we choose to focus on that in the main portion of our paper.

\begin{figure}[h!]
	\centering
	\includegraphics[width=.9\textwidth]{graphics/simComboPlot_bin_disaster.pdf}
	\caption{Simulated substantive effect plots for development aid for varying lags of variables of interest and whether or not a recipient country experienced a natural disaster across the range of the strategic distance measure.}
	\label{fig:binaryDisasterSimulation}
\end{figure}			
\FloatBarrier

We have also run our analysis using the number killed from a natural disaster instead of a count of the number of natural disasters. We show the substantive results of this analysis in Figure~\ref{fig:killedSim}.

\begin{figure}[h!]
	\centering
	\includegraphics[width=1\textwidth]{graphics/simComboPlot_no_killed.pdf}
	\caption{Simulated substantive effect plots for development aid for varying lags of variables of interest and different levels of natural disaster severity (specifically, the log of the number killed) across the range of the strategic distance measure.}
	\label{fig:killedSim}
\end{figure}	
\FloatBarrier		

The substantive trends with respect to humanitarian aid and development aid are notably similar to results that rely on a count of the number of natural disasters. There is a difference, however, with respect to the finding for the civil society aid dependent variable. In our analysis with the count of the number of natural disasters we saw that at higher counts of natural disasters the slope between the amount of civil society aid given and strategic distance became positive. Here we see a less pronounced change in the slope between strategic distance when there are a higher number of deaths. This is perhaps explained by the fact that this measure has a missingness rate of 10.8\%.

With regards to other potential measures, the EM-DAT database provides the data on number people injured, homeless, or affected and the dollar amount of the disaster. However such data has a high degree of missingness and, by their own admission, frequently imprecise or under-reported. For instance\DIFaddbegin \DIFadd{, }\DIFaddend there is 79\% missingness for the number of injured, 36\% missingness for the total number of homeless and 33\% for the total damages. The number of affected has comparatively less missingness, with 9.6\%\DIFdelbegin \DIFdel{, however }\DIFdelend \DIFaddbegin \DIFadd{. However }\DIFaddend the EM-DAT \DIFdelbegin \DIFdel{Gudelines }\DIFdelend \DIFaddbegin \DIFadd{Guidelines }\DIFaddend note that, ``The indicator affected is often reported and is widely used by different actors to convey the extent, impact, or severity of a disaster in non-spatial terms.  The ambiguity in the definitions and the different criteria and methods of estimation produce vastly different numbers, which are rarely comparable.'' Generally all the indicators have varying degrees of imprecision. For instance, the guidelines further state, ``Any related word like \DIFdelbegin \DIFdel{'}\DIFdelend \DIFaddbegin \DIFadd{`}\DIFaddend hospitalized' is considered as injured. If there is no precise number is given, such as \DIFdelbegin \DIFdel{'}\DIFdelend \DIFaddbegin \DIFadd{`}\DIFaddend hundreds of injured', 200 injured will be entered (although it is probably underestimated).'' Given these problems with these other potential measures, we decided to focus on the number of disasters as our measure of disaster intensity.

\DIFaddbegin \clearpage
\DIFaddend \subsection*{\DIFaddbegin \DIFadd{Substantive Results when using }\DIFaddend Fixed \DIFdelbegin \DIFdel{versus random effects}\DIFdelend \DIFaddbegin \DIFadd{Effects}\DIFaddend }

In Figure~\ref{fig:devIntCoef_fixed} below we present the results of our analysis when using fixed effects. The results remain broadly the same \DIFaddbegin \DIFadd{as those shown in Figure~\ref{fig:simEffects} of the paper}\DIFaddend . Additionally, when running a Hausman specification test for our models we fail to reject the null hypothesis at both the 90 and 95\% confidence intervals, providing at least some initial evidence that we are justified in our choice \citep{greene:2008}.

\begin{figure}[h!]
	\centering
	\DIFdelbeginFL %DIFDELCMD < \includegraphics[width=1\textwidth]{graphics/intCoef_fe_re_compare.pdf}
%DIFDELCMD < 	%%%
\DIFdelendFL \DIFaddbeginFL \includegraphics[width=1\textwidth]{graphics/simComboPlot_fe.pdf}
	\DIFaddendFL \caption{\DIFdelbeginFL \DIFdelFL{Comparison between parameter estimates }\DIFdelendFL \DIFaddbeginFL \DIFaddFL{Simulated substantive effect plots when estimating }\DIFaddendFL using \DIFaddbeginFL \DIFaddFL{a }\DIFaddendFL fixed \DIFdelbeginFL \DIFdelFL{and random }\DIFdelendFL effects \DIFaddbeginFL \DIFaddFL{framework for each dependent variable (humanitarian aid, civil society aid, and development aid) for different levels of natural disaster severity across the range of the strategic distance measure}\DIFaddendFL . \DIFaddbeginFL \DIFaddFL{A rug plot is provided below each panel.}\DIFaddendFL }
	\label{fig:devIntCoef_fixed}
\end{figure}
\FloatBarrier

\DIFaddbegin \clearpage
\DIFaddend \subsection*{Temporal variation in patterns of aid}

A limitation of our study is that it ends in 2005 because we face the constraint that the IGO data, an important component of our strategic interest measure, is simply not available past 2005. However, to show the potential relevance of our findings for more recent periods we have run our models using only data from the post Cold War period. The results are presented in Figure~\ref{fig:postColdWarSim} below and mirror the findings presented in the paper. 

\begin{figure}[h!]
	\centering
	\includegraphics[width=1\textwidth]{graphics/simComboPlot_post_coldwar.pdf}
	\caption{Simulated substantive effect plots for development aid for varying lags of variables of interest and different levels of natural disaster severity across the range of the strategic distance measure for the post Cold War period.}
	\label{fig:postColdWarSim}
\end{figure}	
\FloatBarrier

Additionally, we also run our models using only data from 2002-2005 (post-2001 period in our sample). The results are presented in Figure~\ref{fig:post2001Sim} below and mirror the findings presented in the paper. 

\begin{figure}[h!]
	\centering
	\includegraphics[width=1\textwidth]{graphics/simComboPlot_post_2001.pdf}
	\caption{Simulated substantive effect plots for development aid for varying lags of variables of interest and different levels of natural disaster severity across the range of the strategic distance measure for 2001-2005.}
	\label{fig:post2001Sim}
\end{figure}	
\FloatBarrier

\DIFaddbegin \clearpage
\DIFaddend \subsection*{Accounting for \DIFdelbegin \DIFdel{uncertainty }\DIFdelend \DIFaddbegin \DIFadd{Uncertainty }\DIFaddend in \DIFdelbegin \DIFdel{strategic interest measure}\DIFdelend \DIFaddbegin \DIFadd{Strategic Interest Measure}\DIFaddend }

One methodological concern about our strategic interest measure is that since it is estimated from a model it comes with uncertainty. In Figure~\ref{fig:latVarUncert}, we show results when taking into account uncertainty in the latent variable compared with our original estimates. We do this by simulating 1000 values of each latent variable estimate from the underlying distribution. From this we create 1000 versions of our dataset in which for each dataset we have a different sampled value of the strategic interest variable. We then run each of our models on those 1000 datasets and combine the parameter estimates using Rubin's rules \citep{rubin:1987}. We present the results of this analysis juxtaposed against our original model where we just use the average value of the strategic uncertainty variable. 

\begin{figure}[h!]
	\centering
	\includegraphics[width=1\textwidth]{graphics/intCoef_latVarUncert.pdf}
	\caption{Effect of accounting for uncertainty in latent variable.}
	\label{fig:latVarUncert}
\DIFaddbeginFL \end{figure}		
\FloatBarrier

\clearpage
\subsection*{\DIFadd{Investigating Whether Donor Labeling of Aid Channels is Driving Model Results}}

\DIFadd{One point of concern is that donors may find it easier to channel aid to strategic allies through pre-existing development channels in the wake of natural disasters. Since donors cannot rely on such pre-existing channels to funnel aid to strategic opponents, our finding that donors give more humanitarian aid to strategic opponents in the wake of natural disasters may simply be an artifact. That is, it is possible that strategic allies and opponents both get more aid following natural disasters; strategic allies just get tend to get it through development aid and strategic opponents tend to get it through humanitarian aid. 
}

\DIFadd{We explore the possibility first through Figure \ref{fig:aidtype}. Here we break down how much development aid was given to countries experiencing that experienced either 0 disasters or 1-3 disasters across different types of development aid. The comparison between 0 or 1-3 disasters was used to maximized comparability, as around 40 percent of the country-years in the dataset had 0 disasters, while 43 percent experienced 1-3 disasters. This figure suggests that i) countries that are strategic allies (located at low levels of strategic distance) are more likely to get more aid related to economic infrastructure and services while ii) countries that experience disasters are much more likely to get debt relief when the they are strategic opponents (that is at high levels of strategic distance), compared to countries that experience 0 natural disasters. However, while Figure \ref{fig:aidtype} does seem to be consistent the hunch that it may be easier to distribute different types of aid depending on whether one is a strategic opponent or strategic ally, there does not seem to be much in the way of strategic labeling going on. That is, the additional aid for economic infrastructure services to strategic allies and the additional aid for debt relief appear to be given in on top of existing levels of aid; neither appear to be offsetting other types of development aid. 
}

\DIFadd{We also investigate whether strategic allies are more likely to get a greater total amount of aid following a natural disaster.  For instance, it is  possible that strategic allies may receive more non-humanitarian or non-civil society aid, like budget aid following a natural disaster. To find evidence against this possibility and in support of the main argument in the paper, that natural disasters are indeed prompting more giving in strategic opponents compared to allies, we would expect a null effect of the interaction term between natural disasters and strategic distance on total aid (that is, subtracting out humanitarian aid and civil society aid from total aid). We indeed find that there is a null effect on the interaction between natural disasters and our measure of strategic interest on total aid (as defined as all aid minus total humanitarian aid and minus total civil society aid; see Figure \ref{fig:coefPlot_totAidv2}). When we plot the interaction effects, we verify this null finding (see Figure \ref{fig:simEffects_totAid}).
          }

\begin{figure}[h!]
	\centering
	\includegraphics[width = 1\textwidth]{graphics/developmentAidValueByType.pdf}
	\caption{\DIFaddFL{Value of Aid Commitments categorized by type of development aid and by the number of disasters}}
	\label{fig:aidtype}
\end{figure}

\begin{figure}
	\centering
	\includegraphics[width=1\textwidth]{totAidv2_Coef.pdf}
	\caption{\DIFaddFL{Coefficient plot for total aid model.  Coefficients that are significant at the 5\% level are shaded in blue if the coefficient is positive and red if the coefficient is negative. Coefficients that are not significant at the 5\% level are shaded in gray.}}
	\label{fig:coefPlot_totAidv2}
\end{figure}

\begin{figure}
	\centering
	\includegraphics[width=1\textwidth]{simTotAidPlot_lagDV.pdf}
	\caption{\DIFaddFL{Simulated substantive effect plots for total aid for different levels of natural disaster severity across the range of the strategic distance measure. A rug plot is provided below each panel.}}
	\label{fig:simEffects_totAid}
\DIFaddendFL \end{figure}
\FloatBarrier
%%%%%%%%%%%%%%%%%%%%%%

\end{document} 