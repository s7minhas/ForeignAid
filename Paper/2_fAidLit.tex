\section*{Accounting for Natural Disasters in Determining Donor Motivations for Foreign Aid}
\label{theory}
%[why do we care about looking at the interaction between natural disasters and strategic interest]
% [NEED BETTER TRANSITION HERE FROM PREVIOUS SECTION]

% In contrast to the relative lack of attention devoted to it in the literature,  substantial media attention is generally paid toward the amount of foreign aid given in the face of natural disasters. Consequently, public awareness about foreign aid
% Despite the relative lack of attention that the literature has devoted toward it, foreign aid given in times of natural disasters are often the most visible 
%\footnote{Resnick, Brian. `The deadly earthquake that rocked Mexico City: what we know.' \textit{Vox}. 20 September 2017. Accessed September 2017: \url{https://www.vox.com/science-and-health/2017/9/19/16335714/mexico-city-earthquake-7-1-puebla-mexico}}
Natural disasters can lead to the calamitous destruction or impairment of physical and social infrastructures to say nothing of provoking the devastating loss or disruption of human lives. For example, the 1985 Mexico City Earthquake killed at least 10,000 people\footnote{The Editors of Encyclopaedia Brittanica.``Mexico City earthquake of 1985.'' \textit{Encyclopaedia Britannica}. 20 September 2017. Accessed September 2017: \url{https://www.britannica.com/event/Mexico-City-earthquake-of-1985}} and cost around 4 billion in 1985 dollars (around 9 billion in 2017 dollars).\footnote{Wiliams, Dan. `Mexico Quake Loss put at \$4 Billion: Report by U.N. Panel Includes Damages to Economy.' \textit{Los Angeles Times.}  25 October 1985. Accessed September 2017: \url{http://articles.latimes.com/1985-10-25/news/mn-14160_1_mexico-city}.} While the resulting devastation prompted the Mexican government to institute a number of regulatory measures to limit future damage, 32 years later, Mexico City's 2017 earthquake still resulted in a death toll of at least 360\footnote{The Associated Press. `Death toll rises to 360 in Mexico earthquake.' \textit{The Denver Post.} 21 September 2017. Accessed October 2017: \url{http://www.denverpost.com/2017/09/30/mexico-earthquake-death-toll-update/}} and the recovery effort could cost more than 2 billion dollars.\footnote{`The Associated Press.' ``Economic Costs of Mexico’s Earthquake Could Surpass \$2B.'' \textit{Insurance Journal} 29 September 2017. \url{http://www.insurancejournal.com/news/international/2017/09/29/465995.htm}}  The 2011 Fukushima incident meanwhile, stands out for both its death toll and high cost, leaving nearly 1,600 dead and more than 174,000 displaced.\footnote{Hamilton, Bevan. `Fukushima 5 years later: 2011 disaster by the numbers.' \textit{CBC News}. 10 March 2016. Accessed September 2017: \url{http://www.cbc.ca/news/world/5-years-after-fukushima-by-the-numbers-1.3480914}} Recent 2017 projections estimate that it will cost around 187 billion dollars, double the 2013 estimate.\footnote{McCurry, Justin. `Possible nuclear fuel find raises hopes of Fukushima plant breakthrough.' \textit{The Guardian.} 30 January 2017. Accessed September 2017: \url{https://www.theguardian.com/environment/2017/jan/31/possible-nuclear-fuel-find-fukushima-plant}}  Similarly, estimates put the cost of responding to Hurricane Harvey, which left 82 dead,\footnote{Moravec, Eva Ruth. ``Texas officials: Hurricane Harvey death toll at 82, `mass casualties have absolutely not happened.'' \textit{The Washington Post.} 14 September 2017. Accessed September 2017: ' \url{https://www.washingtonpost.com/national/texas-officials-hurricane-harvey-death-toll-at-82-mass-casualties-have-absolutely-not-happened/2017/09/14/bff3ffea-9975-11e7-87fc-c3f7ee4035c9_story.html?utm_term=.f5eecca9ee21}} at around 180 billion dollars, likely to be the most expensive natural disaster in US history.\footnote{`Hurricane Harvey Damages Could Cost up to \$180 Billion.' \textit{Fortune}. 3 September 2017. Accessed September 2017: \url{http://fortune.com/2017/09/03/hurricane-harvey-damages-cost/}} 

Indeed, few countries are spared the devastation that natural disasters can wreak. Between 1980 and 2004, two million people were reported killed and five billion people cumulatively affected by around 7,000 natural disasters, according to the dataset maintained by the Centre for Research on the Epidemiology of Disasters (CRED) at University of Louvain (Belgium). The economic costs are considerable and rising, with the direct economic damage from natural disasters between 1980-2012 estimated to be around \$3.8 trillion \citep{gitay:2013}.



While dealing with both the immediate and long-term damage wrought by natural disasters can seriously drain existing resources for any country, developing countries generally find it especially difficult to cope. Often, the existing physical infrastructure in developing countries is grossly unequal to the task of withstanding natural disasters. Meanwhile, the resilience and capacity of their institutional infrastructure is often ill-suited for dealing with the often long and complicated process of rebuilding. In general, when natural disaster strikes, developing countries are both more likely to experience more serious damage and have less state capacity to recover from it. For example, prior to its 2010 earthquake, Haiti had no building codes and many of its buildings were not designed to withstand even a mild earthquake.\footnote{Watkins, Tom. `Problems with Haiti building standards outlined.' \textit{CNN}. 2010 January 14. Accessed September 2017: \url{http://edition.cnn.com/2010/WORLD/americas/01/13/haiti.construction/index.html}} Meanwhile, the lack of governmental leadership and low state capacity, along with other factors, has meant that even 7 years after the disaster, Haiti has yet to fully recover \citep{hartberg:2011}. 
%kobayashi:2014

From a purely strategic perspective then, natural disasters represent an opportune time to inflict harm on a strategic adversary, especially if it is a developing country, as both  government officials and government resources are fully engaged with responding to the emergency. Yet, anecdotal evidence suggests that strategic adversaries rarely take advantage of this tactical opportunity. That is, at least as far as can be openly observed, the most famous  deadliest natural disasters (which, incidentally, should present foreign opponents the best opportunity to inflict harm) do not seem to have been followed up by hostile overtures. For instance, Taiwan did not use the 1976 Tangshan earthquake, believed to be the largest earthquake in the 20th century by deathtoll, as an opportunity to attack China. Similarly the 2011 Fukushima disaster was not followed by hostile gestures from China nor did Russia react to Hurricane Harvey with belligerence toward the US.\footnote{Note, the question of whether countries take advantage of their strategic opponents using more covert methods during times of natural disaster is a more open question.} 

Context of course matters. There is a sizable difference between taking advantage of a country that one has contentious relationship with and taking advantage of a country with which one is actively engaged in outright conflict. In the former context, while taking pre-emptive action against a strategic opponent might lead to short term gains, it could very well lead to long term losses. By this logic, we might expect countries to simply do nothing when tragedy befalls their strategic opponents. Such behavior would fit well with the larger literature that investigates donor motivations for allocating foreign aid. Indeed, scholars have produced a large body of evidence which suggests that donors overwhelmingly prioritize strategic considerations over developmental or humanitarian ones when dispensing aid \citep{mckinlay:1977,mckinlay:1978,mckinley:1979,maizels:1984,schraeder.etal:1998,alesina:2000,berthelemy:2006,stone:2006,demesquita:2007,bermeo:2008,hoeffler:2011,dreher:2015}. 

Yet, there is much anecdotal evidence to suggest that rather than taking pre-emptive strikes or doing nothing, natural disasters encourages the flow of \textit{aid} from strategic opponents. For example, during the famine that ravaged North Korea from 1994 to 1998, the United States, South Korea, Japan and the European Union were the primary donors of food aid \citep{noland:2004}.  Meanwhile Taiwan was one of the biggest donors to China in the aftermath of the 2008 Sichuan earthquake.\footnote{`FACTBOX-Earthquake aid for China.' 14 May 2008.  \url{http://uk.reuters.com/article/idUKPEK29448220080514}} Taiwan also actively contributed to the rescue effort\footnote{French, Howard and Edward Wong. `In Departure, China Invites Outside Help.' \textit{The New York Times}. 16 May 2008. Accessed September 2017: \url{http://www.nytimes.com/2008/05/16/world/asia/16china.html}} and further offered to share the technical expertise it learned from its own devastating earthquake experience in 1999.\footnote{Hille, Kathrin. `Taiwan shares quake lessons with Sichuan.' \textit{Financial Times}. 9 June 2008. Accessed September 2017: \url{https://www.ft.com/content/b0204002-3641-11dd-8bb8-0000779fd2ac}} Similarly, following Hurricane Katrina, the United States accepted Russian aid, despite frosty relations.\footnote{`U.S. accepts Russian Katrina aid.' \textit{UPI}. 2 September 2005. Accessed September 2017. \url{https://www.upi.com/US-accepts-Russian-Katrina-aid/39221125680989/}.} 
 
What explains this decidedly non-strategic behavior? Finding an answer to this question in the current literature is difficult. For one, in evaluating the relative roles that donor interest and recipient need play in foreign aid allocation, what researchers refer to as recipient need may be more precisely understood as `developmental need' and as such, targeted towards addressing chronic poverty. To that end, development need is frequently measured using GDP or GNP per capita \citep{mckinlay:1977,mckinlay:1978,mckinley:1979,maizels:1984,alesina:2000,berthelemy:2006,stone:2006,demesquita:2007,bermeo:2008} and occasionally a more holistic measures of of social outcomes such as the Physical Quality of Life Index \citep{maizels:1984}, the average life expectancy \citep{schraeder.etal:1998} or the daily caloric intake \citep{mckinley:1979,schraeder.etal:1998} % and occasionally population size \citep{bermeo:2008}. 

% sm: just taking out the below lit ref because someone might ask us to then also account for aid flows as a proportion of recipient economy. 
% Meanwhile using an event-study approach, \citet{becerra:2014} finds that while ODA increases on average 18\% following a natural disaster, the substantive increase is small relative to the size of the recipient's economy. 

Meanwhile, a much smaller body of research investigates the degree to which aid is given in response to acute crises, such as natural disasters, which will be referred to here as humanitarian need, at all. Considering that around 11\% of ODA was officially categorized as being given for humanitarian reasons in 2015, the systematic failure to include natural disasters as a potential driver of foreign aid is puzzling.\footnote{Total ODA for DAC countries was 131.6 billion in 2015, 15.6 billion of which was designated as humanitarian assistance \url{http://www.oecd.org/dac/development-aid-rises-again-in-2015-spending-on-refugees-doubles.htm} \url{http://www.oecd.org/dac/stats/humanitarian-assistance.htm}}  What evidence that does exist suggests a null or small effect of humanitarian aid on foreign aid allocations. For instance, \citet{bermeo:2008} finds no relationship between the number of people affected by disasters and the allocation of bilateral aid for France, Japan, the UK and the US.\footnote{Note, \citet{bermeo:2008} also conceptualizes humanitarian aid using measures of the number of refugees and civil war, with mixed effects across countries for both}  Similarly, \citet{david:2011} finds no statistically significant relationship between foreign aid flows and climatic or human disasters. Though he finds evidence for increased foreign aid following geological disasters, the effect is lagged by 2 years and substantively small.\footnote{\citet{david:2011} defines climatic events as `floods, droughts, extreme temperatures and hurricanes'; human disasters as: famines and epidemics; geological events as: earthquakes, landslides, volcano eruptions and tidal waves} \citet{yang:2008} also finds that ODA increases after a hurricane, but only with a lag of 2 years. Finally, though \citet{stromberg:2007} does find a positive and significant relationship between aid and natural disasters, his paper is concerned with emergency aid in particular, not foreign aid, that is official development assistance, more generally. Similarly, \citet{olsen:2003} find that donors are more likely to give aid for strategic reasons, though their analysis is confined to emergency aid. 

Similarly, there is relatively little work that has explored whether there are interactive effects between donor's strategic interest and recipient's humanitarian need. One seeming exception is \citet{drury_etal:2005} who find that between 1964 to 1995, the United States made its decision to dispense aid based on strategic considerations, but based the amount given on humanitarian considerations. However, their dependent variable of interest is humanitarian aid, not ODA. To our knowledge, our paper is to first to explore whether there may be an interactive relationship between donor's strategic interest and recipient's humanitarian need on foreign aid allocation decisions. 




% At the same time, note that within the general consensus that donors are more driven by their strategic interests rather than recipient need however, scholars have found that there is significant variation in the strength of this relationship \textit{across different donor countries}. To that end, many papers endeavor to provide an analysis of aid allocation on a cross national level and for individual donor countries \citep{alesina:2000,stone:2006,berthelemy:2006,bermeo:2008} while other papers choose to focus on the aid allocation strategies of one donor country at a time \citep{mckinlay:1977,mckinlay:1978,mckinley:1979, dreher:2012,neumayer:2003,fleck:2010}. 


%\indent\indent To the extent that the current literature explores variation across donor countries, the literature suggests that different donors may be more motivated by some goals relative to others. \citet{schraeder.etal:1998} find for example that Japanese aid is more likely to be motivated by economic and trade interests, Swedish aid supports progressive, socialist-minded regimes and France's aid is almost exclusively targeted toward francophone countries \citep{werker:2012}. \citet{berthelemy:2006} finds that the Noridc countries, along with Switzerland and Ireland are more prone to be altruistic than other donors. \\
%\indent\indent Focusing solely on aid allocation in the Arab world, \citet{neumayer:2003} finds that OPEC donors favor other Arab and non-Arab Muslim recipients.  Countries that do not maintain diplomatic relations with Israel or with voting patterns similar to Saudi Arabia are also more likely to get more aid. He finds that donor interest, in particular Arab solidarity, is a strong determinant in whether a country gets aid and how much while recipient need (as measured by a country's level of income) only affects if the country gets aid, not how much aid they get. \\
%\indent\indent Focusing solely on China, \citet{dreher:2012} find that political and commercial considerations shape China's aid allocation. However they find that compared to other countries, China does not pay substantially more attention to politics nor do thy find evidence that China's aid allocation is motivated by natural resource concerns.\footnote{Note however that many projects that could be categorized as foreign aid is often doled out in the form of FDI.} In general they argue that China's aid allocation seems to be independent of democracy and governance considerations in recipient countries.\\
%\indent\indent \citet{fleck:2010} investigate how the motivations for US aid have evolved over time. They find evidence to suggest that i) the probability of receiving US foreign aid before and following the War on terror increased with the level of development. Higher-developed countries were more likely to receive aid at all, but this did not necessarily come at the expense of lower developed countries as the overall aid budget also increased  following the War on Terror ii) the level of US aid for the US' core foreign aid countries has been less a function of need following the War on Terror than before. 

%Similarly, the strength of the relationship between donor's strategic interest and foreign aid allocation also possesses significant variation \textit{across time}. In their analysis of 22 donor countries and 137 recipient countries from 1980-1999, \citet{berthelemy:2004} finds that following the Cold War, foreign aid allocation has been more responsive to a good governance and good economic policy in recipient countries, a result that \citet{bermeo:2008} and \citet{dollar:2006} echoes. This suggests not only that recipient need has become more important in recent years but that the relative balance between strategic and developmental considerations are not fixed over time. This claim is disputed by \citet{nunnenkamp:2006}'s findings however, whose analysis suggests that foreign aid dispersed from 1981 to 2002 has been \textit{less} targeted to needy countries over time. What these studies hold in common however, is a lack of a \textit{measure} of what they think may be affecting aid allocation across time, only an \textit{interpretation} of what these time effects might mean given their knowledge of the different time periods. \\


%. Note here that they explicitely look at humanitarian aid, while we look at how nomral ODA aid might be driven by strategic or humanitarian concerns. They also control for disaster type, their strategic variable is alliance data, and control for disaster salience using a count of NYtimes stories. They use a Heckman selection model but also model each stage separately.

%Using data from 1964 to 1995,



%In none of the papers we have encountered however, have scholars sought to evaluate differences in aid allocation in a multi-level hierarchical model.


 

 
% \indent\indent  Scholars sought to provide empirical evidence for answer or another since at least as far back as the late 1970's \citep{ mckinlay:1977,mckinlay:1978,mckinley:1979} and onward to the 1980's \citep{maizels:1984} and 1990's \citep{lumsdaine:1993, schraeder.etal:1998}, with scholars finding evidence to suggest that foreign aid allocation is driven by strategic concerns much more than humanitarian ones.  \citet{alesina:2000} were among the first to extend this finding across a large panel of countries, that is to 21 donor countries and 181 recipient countries from 1970-1994. They find that countries that votes relatively more similarly to Japan in the UN are 172\% more likely to receive more aid while Egypt and Israel receive upwards of 400\% more foreign aid than other countries. Ceteris paribus they argue that inefficient, economically closed non-democratic former colonies are much more likely to receive aid than countries that had not been formerly colonized with similar poverty levels, a finding that \citet{alesina:2002} echoe when they find that the US is more likely to give corrupt governments more aid. \citet{berthelemy:2006} reaches a similar conclusion, noting that donor countries are generally much more likely to act based off of egotistic motivations than altruistic ones, while \citet{stone:2006} and \citet{demesquita:2007} find evidence to suggest that donor countries are more likely to use foreign aid to ``buy influence''.  \\ 

%scholars have certainly found variation in their results, the general consensus seems to be that strategic interest largely takes precedence over humanitarian ones in foreign aid allocation.  However, such findings should be taken with a grain of salt as 

% We seek to account for both variation of foreign aid allocations in time and variation across countries in our model specification, which we detail further in the empirical section of this paper. \\

 %\citep{stromberg:2007} `finds that ODA given after a disaster is influenced by news coverage of the disaster (the answer: yes); and whether a potential donor country is more likely to give aid if it has a well-established connection with the affected country (the answer is again: yes). Our approach is different methodologically, and our answers are correspondingly different.'
 
%`Beyond these supply factors guiding aid allocations, Olsen, Carstensen and Høyen (2003) note that demand factors (i.e., the receiving country’s characteristics), and in particular its readiness to absorb new flows through NGOs, are important in determining aid inflows. On the other hand, they find little evidence that policy effectiveness by the receiving government and the presence of efficient institutional capacity to implement aid matter for the magnitude of aid

% implications: http://www.hbs.edu/faculty/Publication%20Files/08-040.pdf
% http://library1.nida.ac.th/worldbankf/fulltext/wps04952.pdf

%\indent\indent  What's more, what some scholars measure as strategic interest other scholars interpret as a measure of humanitarian interest. As \citet{bermeo:2008} notes for example, there is some controversy in interpreting GDP per capita as a measure of humanitarian aid, as ``the poorer a country is, the more it needs aid, and the easier it might be for donors to use aid to influence decisions in the recipient''. She further notes that colonial legacy, a factor that some scholars see as evidence of strategic interest, may not necessarily be an appropriate measure of strategic interest but instead of ``strategic development''. In this sense, humanitarian and strategic interests are mutually complementary motivations as donor countries seek to further the development of countries that they have a self-interest in seeing develop. We would further add that increased aid among countries with former colonial ties could also be interpreted as a measurement of the greater degree of cultural understanding between these countries, which has long been argued to be a cornerstone of effective aid. In order to properly evaluate the motivations for foreign aid, what is needed is a better and clearer measure of strategic interest, something which we take up in the next section.\\

% \begin{table}
% \scriptsize
% \caption{ Covariates}
% \begin{tabular}{l|c c c c}
%  				&  																					 				& Regional 		 		& UN					& Alliance \\
%  				&																					 				& Dummies				& Voting				& Dummy \\
%  			\hline
%  Positive  		& \multirow{2}{*}{\parbox[c]{1em}{\includegraphics{flags/flags-iso/flat/24/US.png}}}				&						&\citet{alesina:2000} 	&\citet{schraeder.etal:1998} \\
% 	and   		& 	 																								&						& (1970-1994)			&(1980-1989)\\
% Significant  	& 	 																								&						& 						&\\
% 				\cline{2-3}\\
%  				& \multirow{2}{*}{\parbox[c]{1em}{\includegraphics{flags/flags-iso/flat/24/GB.png}}}				& 						& \citet{alesina:2000}	&\citet{bermeo:2008} \\
%  				& 	 																 								& 						& (1970-1994)			&(1984-1988)\\
% 				& 	 																								& 						&						&\\

% 				& \multirow{2}{*}{\parbox[c]{1em}{\includegraphics{flags/flags-iso/flat/24/FR.png}}}				& 						& \citet{alesina:2000}	&\citet{bermeo:2008} \\
%  				& 	 																 								& 						& (1970-1994)			&	(1984-1988)\\
% 				& 	 																								& 						& 						&\\
%  				& \multirow{2}{*}{\parbox[c]{1em}{\includegraphics{flags/flags-iso/flat/24/JP.png}}}				& 						&\citet{alesina:2000} 	&\citet{bermeo:2008} \\
%  				& 	 																 								& 						& (1970-1994)			&(1984-1988/2000-2005)\\
% 				& 	 																								& 						&						&\\
% \hline
%  Not 			& \multirow{2}{*}{\parbox[c]{1em}{\includegraphics{flags/flags-iso/flat/24/US.png}}}				&\citet{mcgillivray:2003}& 						& \citet{bermeo:2008}\\
% Significant   	& 	 																								& (1980)				& 						&(1984-1988/2000-2005)\\
% 				& 	 																								& 						&						&\\
% 				& \multirow{2}{*}{\parbox[c]{1em}{\includegraphics{flags/flags-iso/flat/24/GB.png}}} 				& 						& 						&\citet{bermeo:2008} \\
%  				& 	 																 								& 						& 						&(2000-2005)\\
% 				& 	 																								& 						&						&\\
% 				& \multirow{4}{*}{\parbox[c]{1em}{\includegraphics{flags/flags-iso/flat/24/FR.png}}}				& 						& 						&\citet{schraeder.etal:1998} \\
% 	 			& 	 																 								& 						& 						&(1980-1989)\\
% 	 			& 	 																 								& 						& 						&\citet{bermeo:2008} \\
% 	 			& 	 																 								& 						& 						&(1984-1988) \\
% 				& 	 																								& 						&						&\\
% 				&\multirow{2}{*}{\parbox[c]{.8em}{\includegraphics[width = 1cm]{flags/flags-iso/flat/24/globe.jpg}}}& 						&\citet{alesina:2002} 	& \\
% 				& 	 																								& 						& (1970-1995)			&\\
% \hline
% Negative 		&\multirow{2}{*}{\parbox[c]{.8em}{\includegraphics[width = 1cm]{flags/flags-iso/flat/24/globe.jpg}}}&					&\citet{alesina:2002} 	& \\
% Significant		& 	 																								& & (1970-1995)			&\\


% \end{tabular}
% \end{table}


%  %\includegraphics{flags/flags-iso/flat/24/US.png}

% \begin{landscape}
% \begin{table}
% \scriptsize
% \caption{ Covariates}
% \begin{tabular}{l|c|c|c|}
% \hline
% 								& Scope  				&Stragegic/Donor 					& Recipient Need/ Humanitarian\\
% 								&    		  			&Interest$^a$ 					& Interest \\
% \hline\hline
% \citet{maizels:1984} 			& 79(recip)  			&Arms transfers 			&GNP per capita\\
% 								& /DAC(donor)  			&Regional dummies 			&PQLI\\
% 								&1969-70				& 	 						&GNP growth rate\\
% 								&1978-80 				& 							&Balance-of-payments current account\\
% \hline
% \citet{trumbull:1994} 			&86 Countries  			&  							& L(Infant  Mortality)\\
% 								&1984-89  				& 							&L(GNP pc) \\
% \hline
% \citet{schraeder.etal:1998} 	& 36(recip)/			&Alliance Dummy   			&Avg. life expectancy\\
%              			 		& 4(donor)*  			&Military Spending (\% GNP)	&Caloric intake;    		     \\
% 				 				& 1980-89 				&Military Pop (\% Tot Pop)    			                      \\
% \hline
% \citet{alesina:2000}  			& 181(recip)/ 			&UN Voting;					& L(GDP pc)      \\
%              			 		& 21(donor) 			&Colonial Past   \\
% 				 				& 1970-94 (5yr agg) 	&    	  \\
% \hline 
% \citet{alesina:2002} 			&  						&UN Voting 			& GDP pc			  	 \\
% 								& 13 (donors)  			&Colonial Past  &     \\
% 								& 1970-95 (5 yr avg)	&  		 \\
% 								& 1970-95 (5 yr avg)	& 	   \\
% \hline
% \citet{mcgillivray:2003}		& US						& Arms transfers &  GNP per capita\\
% 								& 						& Regional interests &  Population size\\
% 								& 						& Special relationships & infant mortality rate\\
% 								&						&						& annual GDP growth\\	
% \hline
% \citet{neumayer:2003}  			& Arab countries (donors)&Regional Dummies	 		& L(GDP pc)\\
%   								& 				  		&Trade Imports 				& Life Expectancy\\
%  				 				& 1974-1997 (3y avg)	&Socialist Dummy   			& Infant Mortality\\
%  				 				& 						&S-Scores  					& Literacy Rates\\
%  				 				& 						&UN Voting  				\\
%  				 				& 						&Relations with Israel Dummy  \\

% \hline
% \citet{dollar:2006}  			& $\approx$ 100 (recip)& 							& L(GDP pc)\\
% 				 				& 22 donors 			&    \\
% 							 	& 1984-2003 (5yr avg) 	&     \\
% \hline
% \citet{kuziemko:2006}  			& 137 (recip)  			&UN Security  				&L(GDP pc)	 \\
% 			 	  				& US (donor) 			&Council Member 	 \\
%  				 				&  1946-2001			& 	 \\
% \hline
% \citet{berthelemy:2005}  		&  137 (recip)  		&Colonial Past 	 			&L(GDP pc)\\
% 								& 22 donor 				&Regional Dummies  	 		&GDP growth\\
% 								&  1980-99  			& 	 						&openness, gov deficit\\
% 								&    					& 	  						&inflation; life expectancy\\
% 								&   					& 	  						&child mortality\\
% 								&   					& 	 						&lit rate, school enrollment\\
% \hline
% \citet{bermeo:2008} 			& 106 ( recip) 			&Trade  	            	&L(Pop) \\ 
% 								& 4 (donor)  			&L(Oil Production)    		&L(GDP pc)\\ 
% 								& 1984-1988; 			&L(US Military Assistance)    \\
% 								& 2000-2005				&Defensive Alliance Dummy 	 	\\
% 								& 						&Colony Dummy   \\
% 								& 						&L(Distance) 	  \\
% 								& 						&L(Immigrants) 	 \\
% \hline
% \citet{fleck:2010}  			&119 (recip)			&Dummy for receiving US military aid   	&Population\\ 
% 								&US (donor)				&    									& real PPP GDP per capita\\ 
% 								&1955-2006				&    									& Polity \\ 
% 								&1955-2006				&    \\ 

% \hline
% \citet{dreher:2012}  			& 						&UN Voting;   				&L(GDP pc)\\
% 								&  China (donor)	    & 							& Total Number of People Affected by Natural Disaster \\
% 								& 1956-2006  			&   \\
% \hline
% \hline
% \multicolumn{4}{l}{\tiny{PGLI: Physical Quality of LIfe Index}}\\
% \multicolumn{4}{l}{\tiny{* The 36 recipient countries were solely African countries and the 4 donor countries were the US, Japan, Sweden, France }}\\
% \multicolumn{4}{l}{\tiny{** Note that somewhat bizarrely, \citet{maizels:1984} runs separate regressions for recipient needs (objective need) and donor interests (security and commercial interests) }}\\
% \multicolumn{4}{l}{\tiny{Abbreviations: ICRG - International Country Risk Guide; FH -  Freedom House ; PQLI: Physical Quality of Life Index; ACP -  Associated states from Africa, the Caribbean and the Pacific Ocean; MAA- Multilateral Assistance Acts}}
% \end{tabular}
% \end{table}

%Another is that it is somewhat unclear what they mean - recipient need is defined not in terms of the economic needs of that particular society but the needs of that country relative to the needs of other countries. 
%
%
% 
%Of course two motivations are not necessarily mutually exclusive. By the humanitarian need criterion, donor countries should give exclusively to the poorest countries in the world, and for some scholars, the reason that they do not can only be evidence of strategic tomfoolery. Yet the very measures that scholars have used to measure strategic interest may provide a reasonable explanation to why donor countries do not all give to the poorest recipient countries -physical distance from a particular donor country may mean that it is more efficient to give to countries that are closer 
%
%
%
%\subsection*{Variation across countries}
%
%
%
%% A way to see if donors allocation is not independent is to look at how much total aid a country gets per capita? 
%
%\subsection*{Dependence across countries}
%Of course the presumption in these models is that donor giving is completely independent; whether a country receives aid from one donor does not affect whether it receives aid from another donor.  To that end, the presumption  seems to be little coordination among aid donors with better coordination being perceived as enhancing aid effectiveness.  In their investigation of aid allocation from 1956-2006, \citet{aldasoro:2010} find little evidence of increased coordination among aid donors despite rhetoric to improve coordination. \citet{frot:2011} also find that aid donors are likely to 'herd' aid, but find that this effect is substantively small. Exploring a single country, Cambodia, \citet{ohler:2013} also find a lack of evidence for aid coordination among donors. \\ 
%\indent\indent \citet{mascarenhas:2006} find similar evidence of non-cooperation among donors but view this as a positive development. Under their framework, foreign aid is a public good and the extent to which donors derive donor-specific benefit from giving foreign aid (and thus give more than what one might expect given the possibility to free-ride), non cooperation among aid donors boosts much need foreign aid for recipients (instead of increasing redundancy and reducing efficiency as implied by \citet{aldasoro:2010} and \citet{frot:2011} )\\
%
% \indent\indent Meanwhile \citet{steinwand:2014} argue that donors do in fact coordinate under some conditions. Namely they present the idea of 'lead donorship', wherein a recipient has one major donor country and argue that this is more likely to occur when there are large oil exports to donor countries, large imports from donor countries and a colonial history.\footnote{\citet{steinwand:2014} also develop a framework wherein the cross tabs between lead donorship and the existence of coordination indicates whether aid is seen primarily as a private good or public good in a particular situation. } \\
%\indent\indent  \citet{fuchs:2013} explore why donors do not better coordinate the allocation of foreign aid. They find evidence to suggest that competition for export markets and political supports prevents coordination and thus implicitly make an argument for why foreign aid allocation is indeed independent across donor countries.  \citet{lawson:2013} also argues competing objectives as well as division of labor problems among donors prevents coordination.
%\indent\indent  Moreover \citet{frot:2010} find high fragmentation of aid across sectors. They find that in general, countries that are poor, democratic and have a large population are likely to have more fragmented aid though this is only because they are likely to attract more donors. Once these effects are controlled for, democratic and poor countries are no more likely to recieve aid then their authoritarian and rich counterparts. \\
%
%
%
%
%
%
%\subsection*{Bypassing bilateral aid}
%
%
%
%
%\subsection*{Methods and Models} 
%
%\subsubsection*{Operationalizing the DV}
%\citet{trumbull:1994} operationalize the dependent variable as $log ( \frac{i\text{'s per capita ODA in year } t}{\text{sample average ODA for year }t})$ where $i$ denotes the recipient country.\\
%
%\citet{dreher:2012} operationalize the dependent variable as $\frac{i\text{'s ODA recieved from China in year} t}{\text{China's total aid in year }t}$ where $i$ denotes the recipient country.
%
%
%Committment or Disbursement?
%A number of authors \citep{aldasoro:2010,neumayer:2005,berthelemy:2006} use aid committments over actual disbursements of aid reasoning that donors have full control of committments only. Disbursements are seen as partly depending on whether the prospective recipient country actually requests the commitment, which for a number of reasons (that they don't give examples of) sometimes is not the case. Disbursements may also in part rely on the recipient country's capacity to recieve aid. 
%
%
%
%Meanwhile other authors \citep{alesina:2002,alesina:2000,kuziemko:2006,fleck:2010} use various measures of disbursed ODA though none seem to give a particular justification why. 
%
%
%\citet{alesina:2002} use net ODA per capita\citet{alesina:2000} uses log of net ODA, though neither give a particular justification as to why.  \citet{acharya:2006} also uses aid disbursements but they argue from the point of view of the recipients.
%
%Gross or net?
%
%\citet{neumayer:2005} and \citet{mcgillivray:1992} argue that one should use gross ODA to a recipient country as a percentage of total gross ODA dispersed by the donor country as this more accurately reflects donor the weight that the donor gives to the recipient country. They argue that while the donor may take into account the population of the recipient country in its disbursements, this is merely one outcome of the process of dividing the aid pie, not its main consideration.
%
%\subsubsection*{Common Covariates}
%
% 
%





% \begin{table}
% \scriptsize
% \caption{ Covariates}
% \begin{tabular}{l|c|c|c|c|c|c|c}
% \hline
% 								& Scope  			& L(Pop)& Political   		& Humanitarian   			& Strategic   		& Commerc. 							& Other \\
% 								&    		  		&  	   	& Factors 			&   Need 					&   Interests 		&   Interests 						& \\
% \hline\hline
% \citet{maizels:1984} 			& 79(recip)  		& X  	& 					& L(GNP pc); 			&  arms transfers 	& stock of private direct investm.;	&\\
% 								& /DAC(donor)  		& 		&  					& BOP (current accts)	& regional dummy  	& \# TNC subsidiaries/affiliates;  	& \\
% 								&1969-70			& 		&  					& GNP growth rather 	&   				&Availability of   					&\\
% 								&1978-80 			& 		&  					& PQLI				 	&   				& strategic materials 				&\\
% \hline
% \citet{trumbull:1994} 			&86 Countries  		& X  	& FH 				& L(Infant  Mortality); &   				&  									&\\
% 								&1984-89  			& 		&  					& L(GNP pc) 			&   				&  									&\\
% \hline
% \citet{schraeder.etal:1998} 	& 36(recip)/		& 	 	&Marxist dum		& Avg. life expectancy  & Alliance Dummy & L(GDP pc);	& GNP per cap & colonial dummy \\
%              			 		& 4(donor)*  		&   	&Socialist dum 		& Caloric intake; 	    & Military Spending (\% GNP) 	&L(imports from donor) 	& region dummy		     \\
% 				 				& 1980-89 			&     	&Capitalist dum 	&						& Military Pop (\% Tot Pop)		&  				                      \\
% \hline
% \citet{alesina:2000}  			& 181(recip)/ 		&   	&Rule of Law (PRS) 	& L(GDP pc) 			& UN Voting; 		& proportion of years 				& Egypt/Israel dummy ;\\
%              			 		& 21(donor) 		&  X 	& /Pol \&Civil Rights & 				& colony dummy;		& country is open ;					& \% Catholic, Muslim \\
% 				 				& 1970-94 (5yr agg) &   	& (FH)				&						&  years as Colony 	& Net FDI inflows/GDP \\
% \hline 
% \citet{alesina:2002} 			&  					&   	& FH  				& GDP pc 				&  CORBI; CORRICRG; & Debt relief/ capita 				& yrs as colony\\
% 								& 13 (donors)  		&    	& 					& 						& CORRIMD; CORRS\&P & Net FDI and portfolio  			& Egypt/Israel dum\\
% 								& 1970-95 (5 yr avg)&  		& 					&    					&; CORRTI ; CORRWDR1 & investment; net private \\
% 								& 1970-95 (5 yr avg)&	  	&					&    					&CORRWDR2; UN Voting & capital flows\\
% \hline
% \citet{neumayer:2003}  			& ?					& 		& Socialist dummy;	& L(GDP pc) 			&   				&  l(imports from 					& Arab dummy \\
%   								& Arab aid, agg  	& X 	& UN Voting;Israel 	& 					&  					& Kuwait, Saudi 					& African dummy \\
%  				 				& 1974-1997 (3y avg)& 		& diplomatic relations&   				&    				&Arabia, UAE) 						& Islamic dummy \\
% \hline
% \citet{dollar:2006}  			& $\approx$ 100 (recip)& X  &L(ICRG)  			& L(GDP pc) 			&   				&  \% donor exports to& colonial dummy \\
% 				 				& 22 donors 			&  	&  L(FH)  			&  						&   				&   recipients 						& L(distance) \\
% 							 	& 1984-2003 (5yr avg) 	&  	&  					&   					&   				&     								&   \\
% \hline
% \citet{kuziemko:2006}  			& 137 (recip)  			& 	&  Polity 2  		& L(GDP pc) 			& UN Security  		& 									& war ($>$1000 deaths) ;\\
% 			 	  				& US (donor) 			& 	&  					&   					& Council Member 	&   								& NYT articles \\
%  				 				&  1946-2001			& 	&    				&  					 	&   				&  									& distance \\
% \hline
% \citet{berthelemy:2006}  		&  137 (recip)  		& 	& FH  ;				& L(GDP pc) 			& colony dummy  	& lag  imports + 					& total donor\\
% 								& 22 donor 				& 	& conflicts(PRIO) 	&GDP growth 			& US-Egypt dummy 	&exports/GDP; 						&  aid\\
% 								&  1980-99  			& 	& MAA ; 			&openness, gov deficit & US-Latin Dummy 	& net debt/exports	& ;\\
% 								&    					& 	& aid by other 		&inflation; life expectancy & Japan-Asia dummy & 			&  \\
% 								&   					& 	& donors  			&child mortality 		& EU- ACP dummy 	&  mil/GDP&  \\
% 								&   					& 	&     				&lit rate, school enroll&  					& & \\
% \hline
% \citet{bermeo:2008} 			& 106 ( recip) 			& 	& Law \& Order		& L(Pop)				&Trade 				&   								&Total People affected by Disaster              	\\ 
% 								& 4 (donor)  			&   & FH				& L(GDP pc) 			&L(Oil Production)	&   								& Civil War dummy\\ 
% 								& 1984-1988; 			&   & Trade Openness	& 						&L(US Military Assistance)& 							& Refugees \\
% 								& 2000-2005				& 	& 					&						&Defensive Alliance Dummy  & 							& 			\\
% 								& 						& 	& 					&						&Colony Dummy  		& 									& \\
% 								& 						& 	& 					&						&L(Distance)  		& 									& \\
% 								& 						& 	& 					&						&L(Immigrants)  		& 								& \\
% \hline
% \citet{fleck:2010}  			& 119 (recip)			& X & lag polity ;		& L(GDP pc) 			& US mil budget ;	& l(exports)& interwar dum \\ 
% 								&US (donor)				&  	& polity transition; &  					& recieved US mil 	&   & war terror dum \\ 
% 								&1955-2006				&  	& political location &   					& aid dummy & &   \\ 
% 								&1955-2006				&  	& Pres, Congress   	&   					&  & &   \\ 

% \hline
% \citet{dreher:2012}  			&? & X 					&  Chiebub et al.  		& L(GDP pc) 			& UN Voting; 	& log total & distance \\
% 								&  China (donor)	    & 						&   (2010)'s  Dummy 	& &    diplomatic relations & exports to China;\\
% 								& 1956-2006  			&  						& & &  with Taiwan 		& log oil production\\
% \hline
% \hline
% \multicolumn{8}{l}{\tiny{* The 36 recipient countries were solely African countries and the 4 donor countries were the US, Japan, Sweden, France }}\\
% \multicolumn{8}{l}{\tiny{** Note that somewhat bizarrely, \citet{maizels:1984} runs separate regressions for recipient needs (objective need) and donor interests (security and commercial interests) }}\\
% \multicolumn{8}{l}{\tiny{Abbreviations: ICRG - International Country Risk Guide; FH -  Freedom House ; PQLI: Physical Quality of Life Index; ACP -  Associated states from Africa, the Caribbean and the Pacific Ocean; MAA- Multilateral Assistance Acts}}
% \end{tabular}
% \end{table}

%\end{landscape}

%
%\subsubsection*{Models}
%
%\citet{neumayer:2003} uses a two-step Heckman selection model. The first stage is the gate-keeping stage where it is determined which countries recieve aid; the second stage is a level stage where it is determined how much aid a country receives given that they receive any aid. In the Heckman two step estimator, the error terms of both stages are allowed to be correlated however it requries an exclusionary variable that is associated with the gatekeeping stage but not the level stage. \citet{neumayer:2003} uses the total amount of aid allocated in any given year as his exclusionary variable, arguing that the higher amount of total aid allocation increases the chances of receiving any aid at all. He notes that this variable is an imperfect one at best. 
%
%%\clearpage
%%\singlespacing
%%% run  pdfLaTex then Bibtex then pdflaTex again
%%\pagestyle{empty} % This tells LaTeX to make the bibliography without numbers.
%%%\singlespace      %Makes the references single spaced.
%%\bibliographystyle{apsr}
%%%\bibliography{/Users/cindycheng/Documents/Dissertation/Writing/Prospectus/masterlist}
%%\bibliography{/Users/cindycheng/Documents/Dissertation/Writing/masterlist}
%%%\bibliography{masterlist}
%%%\bibliography{~/Users/cindycheng/Documents/DissertationProspectus/latexprospectus/masterlist.bib}
%% 
%
%
%% but have there does not seem to be a clearly defined line in the literature between humanitarian and economic motivations. While giving foreign aid in the wake of a natural disaster can be relatively cleanly classified as being a result of a humanitarian impulse rather than an economic one, giving foreign aid to build schools or hospitals can be equally interpreted as either being motivated by humanitarian/moral reasons or being motivated by developmental/economic ones.
%
%% 
%%\begin{itemize}
%%\item Log Distance between Recipient and Donor Country
%%\begin{itemize}
%%\item \citet{dreher:2012} 
%%\end{itemize}
%%\item Log of population
%%\begin{itemize}
%%\item \citet{dreher:2012} ;\citet{trumbull:1994} 
%%\end{itemize}
%%\item Log GDP per capita
%%\begin{itemize}
%%\item \citet{dreher:2012}; \citet{trumbull:1994} (though note they use GNP per capita)
%%\end{itemize}
%%\item Political Covariates
%%\begin{itemize}
%%\item \citet{dreher:2012} (\citet{cheibub:2010}'s democracy dummy) ;\citet{trumbull:1994} (Freedom House)
%%\end{itemize}
%%\item Objective Need
%%\begin{itemize}
%%\item  ;\citet{trumbull:1994} (log of infant mortality)
%%\end{itemize}
%%\item Strategic Interests
%%\begin{itemize}
%%\item \citet{dreher:2012} (UNGA voting; diplomatic relations with Taiwan) ;
%%\end{itemize}
%%\item Commercial Interests
%%\begin{itemize}
%%\item \citet{dreher:2012} (log total exports from China to country j; recipients log oil production per day) ;
%%\end{itemize}
%%\end{itemize}
%%
%%
%%\subsection*{How does Foreign Aid Effect X?} 
%%Jepma (1997) [haven't been able to find online yet but referenced in \citet{alesina:2000}] concludes that foreign aid crowds out private saving, supports public consumption and has no significant impact on recipients macroeconomic policies and growth.
%%
%%•Cross country growth empirics: \citep{boone:1994,boone:1996} argues that foreign aid does not affect investment and growth; 
%%
%%Burnside and Dollar find that aid is beneficial to countries that adopt appropriate and stable policies and is otherwise wasted and no evidence of endogeneity. 
%%
%%Collier and Dollar (1998) show that under certain assumtptions, the allocation of aid that has the maximum effect on poverty reduction is a function of the recipeients level of poverty and quality of economic policies
%
%
%%
%
%%
%%
%% Leaving aside this distinction for now and categorizing the above as instances of 'objective need', the below gives an overview of the evidence thus far that suggests that foreign aid might (or might not be) a result of these humanitarian/developmental impulses.\\
%
%
%\citep{alesina:2000} do pooled regressions and donor by donor regressions. They find that while humanitarian motivations do matter, strategic ones are far more important
%
%
%Perhaps the largest hurdle is agreeing on a measure for strategic interest. 
%
%
%However seemingly without exception,
%
%
%
% Such research has explored how actual aid allocations differ from theoretical `poverty-efficient' allocation of aid, defined as \citep{collier:2002}. Other papers have investigated the extent to which donor countries take into account good governance when distributing aid, with researching suggesting that aid to countries with better governance are more likely to be growth-promoting \citep{dollar:2006} 
%
%
%1. these are pretty poor proxies of strategic interest
%1. these things are not mutually exculsive - we need to know the relative importance of each
% 
%
%\citep{berthelemy:2006} does control for the total amount of aid given by other aid donors. However if there is a strategic dynamic to aid giving, it should matter not just that other aid was given but which countries gave additional aid, which his measure does not differentiate between.
%When comparing among different donors, \citep{berthelemy:2006}  meausres the strategic to humanitarian scale with a measure of bilateral trade intensity, with higher trade intensity being equated with greater egotistical behavior. It is not clear why this is a good measure as donors may have strategic motives in giving to countries with which it has weak trade ties while they may also have humanitarian motives in giving to countries with which it has strong trade ties donors may have humanitarian motives 
%
%We address this question in this paper both with regards to model specification and variable specification. 
%
%
%arms transfers : maizels
% \section{robustness checks}
%
%channel of delivery -  donor countries bypass governmental channels to deliver foreign aid when they think the government is corrupt, gives credence to the humanitarian angle \citep{dietrich:2013}
%
%A great number of scholars have puzzled over the motivations for foreign aid. Some exceptions withstanding (Bermeo 2006), most scholars have found that 
%
%
%Strategic and humanitarian motives are hard to parse out because they are so idiosyncratic. For example, a not inconsiderable number of papers include a Egypt or Israel dummy (Alesina and Dollar 2000, Berthelemy 2005) in their models to capture the strategic importance of these countrie. While this variable is often to be significant, it is unclear why these particular sets of countries are chosen when other donor-recipient relations may be described in equally idiosyncratic terms (insert example).  Scholar also insert other variables such as colonial history or UN voting to capture strategic motives 
% 
%We propose to assess the respective importance of strategic or humanitarian motives by investigating whether the global level of military tension is influential in a donor country's foreign aid decisions. In doing so, we introduce a novel variable to measure strategic interest. Should donor countries increase foreign aid giving in response to rising levels of strategic tensions, this provides strong evidence that countries are motivated by strategic interests when giving aid. Meanwhile, a null relationship between strategic aid and foreign aid would provide some, though not definitive evidence that donor countries ignore strategic considerations when dispersing aid. 
%
%
% 
%
% 
%
%In 2013, OECD donors disbursed a total of \$134.8 billion in foreign aid, the highest level ever recorded since numbers were recorded in 1962.
%\footnote{http://www.oecd.org/newsroom/aid-to-developing-countries-rebounds-in-2013-to-reach-an-all-time-high.htm} Since its early inception, foreign aid has been steadily increasing every year [To Shahryar - 1) Any idea why we have this big drop-off for 2010? Here's the code I used to generate the plot:
%\begin{verbatim} pdf("foreignAidYrTrend.pdf")
%plot(1962:2010, tapply(aidData$commitUSD09, aidData$year, sum, na.rm = T), type = "l", xlab = "year", ylab = "Foreign Aid Disbursement")
%dev.off() 
%\end{verbatim}
%2) Should we get the most recent data for the last three years to put into the plot even though we'll only be going up to 2010 in our empirics? If so, would you mind getting it, sorry, it takes forever for me to download stuff here]
%
%
%\begin{figure}[h!]
%\includegraphics[width = 3in]{foreignAidYrTrend.pdf}
%\end{figure}
%
%
%\subsection*{Why and When do Countries Give Foreign Aid?} 
%In this literature review, I give a brief overview of the different potential reasons that donor countries distribute foreign aid as explored by existing literature. Note that for the most part, these reasons tend to be tied to characteristics of the recipient country rather than characteristics of the donor countries. I then attempt to document how different scholars have operationalized these different potential motivations and their modelling strategy (to the extent that the modeling strategy may be methodologically interesting to consider for our purposes).\\
%\indent\indent In general, it seems that there is no clear delineation between the different possible motivations for foreign aid. For example, in their review of the foreign aid literature, \citet{schraeder.etal:1998} identify 6 possible motivations for foreign aid giving: humanitarian reasons (as measured by average life expectancy and daily caloric intake) , strategic importance, economic potential (that is, if the recipient country is deemed to be most economically strong in the region, the implication being that the better their economy does, the better the donor economy's does), cultural similarity (as measured by colonial history), ideological stance, regional similarity (if countries are similar to each other, donors may use a similar foreign aid strategy for all of them). However it may be impossible to uniquely identify any one of these different measures from the other. Cultural similarity can bleed into regional similarity for example, or strategic importance and economic potential may be endogenous to each other. \\
%\indent\indent  What this means in practice is that it is often not clearly defined what the authors are seeking to explain when they throw in different variables into their regressions. They way in which the work below is categorized below then is not strictly how the authors themselves would categorize them, but how they might reasonably fit together given the vagueness and variation across the field. 
% 
%
%\subsubsection*{Objective Need:Humanitarian/Moral Motivations/Economic Motivations}
% 
%\citet{lumsdaine:1993} makes a strong case for the `moral vision' being a driving factor in aid distribution. He also makes a case for foreign aid being a function of the colonial history, democratic status, and the income levels of a country.\footnote{ However, as \citet{alesina:2000} note, he presents only simple correlations in his analysis and as such does not provide robust evidence for his theory}  \\
%\indent\indent Meanwhile, using a cross national panel analysis, \citet{trumbull:1994} explore the extent to which foreign aid is given as a function of material physical security. They find that infant mortality (their measure of physical security) is associated with increased foreign aid. They further find that GNP per capita  (their measure of material security) is \textit{not} associated with increased foreign aid. 
%
%
%
%\subsubsection*{Strategic Motivations}
%
%\citet{maizels:1984}  find evidence to suggest that countries give aid for strategic interests.  Restricting their analysis to African countries, \citet{schraeder.etal:1998} also reject an altruistic motivation for donors.
%To that end, \citet{alesina:2000} also find that countries with a colonial past and political alliances\footnote{As measured by UN voting alliances}  to the sender are much more likely to receive foreign aid
%
%\citet{kuziemko:2006} find that nonpermanent members of the UN Security Council\footnote{10 of the 15 seats on the UNSC are held for non renewable 2 year terms} experience temporary increases in foreign aid when they serve on the Council.\\
%
%
%\noindent Potential other variables: terrorism/terrorist threats?
%\subsubsection*{Ideological Reasons}
%
%Ideological motivations for sending foreign aid can in some ways be seen as a combination of humanitarian and strategic reasons. For example, if one frames the conflict between the US and USSR during the Cold War as an ideological struggle, the respective desires of each side to ensure the ideological allegiance of third party countries was as much a function of strategic reasons (i.e as illustrated in the policy of containment) as well as a function of humanitarian ones (i.e. the idea that capitalism/democracy was an inherently morally and politically superior system to communism or vice versa). \\
%\indent\indent To this end, \citet{alesina:2000} find that countries that democratize  (as measured by Freedom House) receive more aid with relatively more democratic countries receive 39 percent more aid. They find evidence to suggest that countries pay more attention to democracy strictly defined then broader definition of civil rights or law enforcement.\footnote{They find that US, Dutch, UK, the Nordic countries and Canada conform to this description while Germany and Japan only weakly conform to this description; France does not} This provides an interesting contrast to their finding for FDI which concentrates on rule of law but insensitive to democratic institutions They argue that in general FDI is more sensitive to economic incentives than foreign aid which responds more the political variablesUsing a cross national panel analysis, \citet{trumbull:1994} also explore the extent to which foreign aid is given as a function of political rights. They find that an increase the political and civil rights (as measured by a precursor to the Freedom House index) is associated with an increase in foreign aid. \\
%\indent\indent  \citet{alesina:1999} do not find that less corrupt governments receive more foreign aid. However they do find that individual donor countries do vary in terms of how much they give to corrupt governments. They find that Scandinavian countries and Australia give more to less corrupt governments while the US gives more to more corrupt governments. \citet{berthelemy:2006} finds that on average, donors act for political or commercial reasons. However, contrary to \citet{alesina:1991}, they  also tend to give aid to countries with better governance indicators, less conflict and higher growth.
%
%
%
%\citet{boutton:2013}
%
%\citet{barthel:2013}
%\citet{heinrich:2013}
%
%
%No shortage of scholars have sought to understand a donor country's motivation for giving foreign aid. Are donor countries primarily motivated by strategic interest? Or is foreign aid better characterized as being given for humanitarian reasons? Perhaps one reason why no satisfactory answer to this question has been found is because these motivations are not necessarily mutually exclusive. Take the recent outbreak of ebola in West Africa for instance -  while increasing foreign assistance to recipient countries undoubtedly assists West African countries in combatting a humanitarian crisis, it also brings strategic benefit to donor countries in that it increases the likelihood of containing ebola to recipient countries.
%
%Given the mutability of these two motivations, the best scholars can do is to determine whether on average,  countries are more motivated by strategic or humanitarian reasons. Yet the literature to date has not asked this question. Instead, scholars have sought to ask whether individual donor countries are more motivated to give to donor countries for strategic or humanitarian reasons. While the US may be more motivated by strategic interests, Sweden and Norway may not be, the question is on the whole which way does the scale tip?  Morever in evaluating these different motivations, scholars have largely used variables which can be interpreted as being done for both. For example, while some scholars argue that giving to poorer recipient countries is an indication of a humanitarian impulse, other scholars argue that this can be just as likely to be interpreted as given for strategic reasons as influence is likewise `cheaper' in poorer countries.  
%
%
%1. There has been a lot of literature trying to parse out the reasons why countries give foreign aid. The two opposing motivations can be broadly described as realist/strategic or idealist/humanitarian.
%2. However the consensus as to which motivation is more dominant is lacking. Literature promoting the realist view includes - Alesina and Dollar (2000) Stone (2006), Bueno de Mesquita and Smith (2007) Shraeder. Literature promoting the idealist view includes Bermeo (2008)
%Shortcomings - empirically - individual donor countries versus panel data - by conducting regressions of each donor country separately, you're not accounting for the fact that the foreign aid given by other countries may influence the foreign aid you give. Certainly this is something important for any country to consider whether they are dispersing foreign aid for strategic or humanitarian reasons. 
%
%
% This paper seeks to address this question in a much more comprehensive way by i) allowing our empirical model to consider the motivations of all donor countries via panel data, as opposed to breaking the model down by individual model and ii) evaluating whether the global level of strategic tension affects foreign aid giving as opposed to only considering bilateral strategic interests. 

 
%