\section*{Accounting for Natural Disasters in Determining Donor Motivations for Foreign Aid}
\label{theory}

Natural disasters can lead to the destruction or impairment of physical and social infrastructure and even more significantly, the devastating loss of human lives. For example, the 1985 Mexico City Earthquake, one of the most catastrophic natural disasters in modern times, killed at least 10,000 people\footnote{The Editors of Encyclopaedia Brittanica. ``Mexico City earthquake of 1985.'' \textit{Encyclopaedia Britannica}. 20 September 2017. Accessed September 2017: \url{https://www.britannica.com/event/Mexico-City-earthquake-of-1985}} and cost around 4 billion 1985 dollars (around 9 billion in 2017 dollars).\footnote{Wiliams, Dan. `Mexico Quake Loss put at \$4 Billion: Report by U.N. Panel Includes Damages to Economy.' \textit{Los Angeles Times.}  25 October 1985. Accessed September 2017: \url{http://articles.latimes.com/1985-10-25/news/mn-14160_1_mexico-city}.} While the resulting destruction prompted the Mexican government to institute a number of regulatory measures to limit future damage, 32 years later, Mexico City's 2017 earthquake still resulted in a death toll of at least 360\footnote{The Associated Press. `Death toll rises to 360 in Mexico earthquake.' \textit{The Denver Post.} 21 September 2017. Accessed October 2017: \url{http://www.denverpost.com/2017/09/30/mexico-earthquake-death-toll-update/}} and the recovery effort could cost more than 2 billion dollars.\footnote{`The Associated Press.' ``Economic Costs of Mexico’s Earthquake Could Surpass \$2B.'' \textit{Insurance Journal} 29 September 2017. \url{http://www.insurancejournal.com/news/international/2017/09/29/465995.htm}}  The 2011 Fukushima incident meanwhile, stands out for both its death toll and high cost, leaving nearly 1,600 dead and more than 174,000 displaced.\footnote{Hamilton, Bevan. `Fukushima 5 years later: 2011 disaster by the numbers.' \textit{CBC News}. 10 March 2016. Accessed September 2017: \url{http://www.cbc.ca/news/world/5-years-after-fukushima-by-the-numbers-1.3480914}} Recent 2017 projections estimate that it will cost around 187 billion dollars -- double the 2013 estimate.\footnote{McCurry, Justin. `Possible nuclear fuel find raises hopes of Fukushima plant breakthrough.' \textit{The Guardian.} 30 January 2017. Accessed September 2017: \url{https://www.theguardian.com/environment/2017/jan/31/possible-nuclear-fuel-find-fukushima-plant}}  Similarly, estimates put the cost of responding to Hurricane Harvey, which left 82 dead,\footnote{Moravec, Eva Ruth. ``Texas officials: Hurricane Harvey death toll at 82 in 2017, `mass casualties have absolutely not happened.'' \textit{The Washington Post.} 14 September 2017. Accessed September 2017: ' \url{https://www.washingtonpost.com/national/texas-officials-hurricane-harvey-death-toll-at-82-mass-casualties-have-absolutely-not-happened/2017/09/14/bff3ffea-9975-11e7-87fc-c3f7ee4035c9_story.html?utm_term=.f5eecca9ee21}} at around 180 billion dollars, likely to be the most expensive natural disaster in US history.\footnote{`Hurricane Harvey Damages Could Cost up to \$180 Billion.' \textit{Fortune}. 3 September 2017. Accessed September 2017: \url{http://fortune.com/2017/09/03/hurricane-harvey-damages-cost/}} 

Few countries are spared the devastation that natural disasters can wreak. Between 1980 and 2004, approximately 7,000 natural disasters led to the deaths of around two million people and further negatively affected the lives of five billion more \citep{emdat:2009}. The economic costs are also considerable and rising, with the direct economic damage from natural disasters between 1980-2012 estimated to be around \$3.8 trillion \citep{gitay:2013}.

While dealing with both the immediate and long-term damage wrought by natural disasters can seriously drain existing resources for any country, developing countries generally find it especially difficult to cope. Often, their existing physical infrastructure is grossly unequal to the task of withstanding natural disasters. Meanwhile, their institutional infrastructure often lacks the resilience or capacity necessary to deal with the often long and complex process of rebuilding. In general, when natural disaster strikes, developing countries are likely to experience more serious physical damage and have less state capacity to recover from it. For example, prior to its 2010 earthquake, Haiti had no building codes and many of its buildings were not designed to withstand even a mild earthquake.\footnote{Watkins, Tom. `Problems with Haiti building standards outlined.' \textit{CNN}. 2010 January 14. Accessed September 2017: \url{http://edition.cnn.com/2010/WORLD/americas/01/13/haiti.construction/index.html}} Meanwhile, the lack of governmental leadership and low state capacity, along with other factors, has meant that even 7 years after the disaster, Haiti has yet to fully recover \citep{hartberg:2011}.\footnote{Cook, Jesselyn. ``7 years after Haiti's Earthquake, millions still need aid.'' \textit{Huffington Post}. 13 January 2017. Accessed May 2018: \url{https://www.huffingtonpost.com/entry/haiti-earthquake-anniversary_us_5875108de4b02b5f858b3f9c?guccounter=1}} 
%kobayashi:2014

From a purely tactical perspective then, natural disasters represent an opportune time to inflict harm on a strategic adversary, particularly, if it is a developing country, as both  government officials and public resources are fully engaged with responding to the emergency. Yet, anecdotal evidence suggests that strategic adversaries rarely take advantage of this opportunity, at least as far as can be openly observed. Many of the deadliest natural disasters (which should present foreign opponents the best opportunity to inflict harm) do not seem to have been followed up by hostile overtures. For instance, Taiwan did not use the 1976 Tangshan earthquake, believed to be the largest earthquake in the 20th century by death toll, as an opportunity to improve its strategic position vis-a-vis China. Similarly the 2011 Fukushima disaster was not followed by hostile gestures from China nor did Russia react to Hurricane Harvey with belligerence toward the US.\footnote{Note, whether countries take advantage of their strategic opponents using more covert methods during times of natural disaster is a more open question.} 

Context of course matters. There are  different rules of engagement when dealing with a country that one has contentious relationship with and taking advantage of a country with which one is actively engaged in outright conflict. In the former context, though taking preemptive action against a strategic opponent may lead to short term gains, it could very well lead to long term losses, especially since such an action would be well out of the realm of socially acceptable behavior in response to a natural disaster. But even by this hard-nosed logic, we might expect countries to simply do nothing when tragedy befalls their strategic opponents. Such behavior would fit well with the larger literature that investigates donor motivations for allocating foreign aid. Indeed, scholars have produced a large body of evidence suggesting that donors overwhelmingly prioritize their own self-interest over recipient need  when dispensing aid.\footnote{For example, see \citet{mckinlay:1977,mckinlay:1978,mckinley:1979,maizels:1984,schraeder.etal:1998,alesina:2000,berthelemy:2006,stone:2006,demesquita:2007,bermeo:2008,hoeffler:2011,dreher:2015}.}

Yet, much anecdotal evidence suggests that rather than jockeying for a more favorable strategic perch or doing nothing, natural disasters encourages the flow of \textit{aid} from strategic opponents. For example, during the famine that ravaged North Korea from 1994 to 1998, the United States, South Korea, Japan and the European Union stepped up as the primary donors of food aid \citep{noland:2004}.  Meanwhile, Taiwan was one of the biggest donors to China in the aftermath of the 2008 Sichuan earthquake.\footnote{`FACTBOX-Earthquake aid for China.' 14 May 2008.  \url{http://uk.reuters.com/article/idUKPEK29448220080514}} Taiwan also actively contributed to the rescue effort,\footnote{French, Howard and Edward Wong. `In Departure, China Invites Outside Help.' \textit{The New York Times}. 16 May 2008. Accessed September 2017: \url{http://www.nytimes.com/2008/05/16/world/asia/16china.html}} and further offered to share the technical expertise it developed from its own devastating earthquake experience in 1999.\footnote{Hille, Kathrin. `Taiwan shares quake lessons with Sichuan.' \textit{Financial Times}. 9 June 2008. Accessed September 2017: \url{https://www.ft.com/content/b0204002-3641-11dd-8bb8-0000779fd2ac}} Similarly, following Hurricane Katrina, the United States accepted Russian aid, despite frosty relations.\footnote{`U.S. accepts Russian Katrina aid.' \textit{UPI}. 2 September 2005. Accessed September 2017. \url{https://www.upi.com/US-accepts-Russian-Katrina-aid/39221125680989/}.} 

%\citep{buthecheng:2013}
Are these anecdotes of non-strategic behavior indicative of a systemic pattern or one-off exceptions to the rule of strategic self-interest? If the former, what could explain this seemingly humanitarian turn of behavior? Finding an answer to these questions in the current literature is difficult. For one, in evaluating the relative roles that donor interest and recipient need play in foreign aid allocation, what researchers refer to as recipient need may be more precisely understood as ``developmental need'' and as such, targeted towards addressing chronic poverty. To that end, development need is frequently measured using gross domestic product (GDP) or gross national product (GNP) per capita;\footnote{For example, see \citet{mckinlay:1977,mckinlay:1978,mckinley:1979,maizels:1984,alesina:2000,berthelemy:2006,stone:2006,demesquita:2007,bermeo:2008}.} or occasionally with more holistic measures of social outcomes such as the Physical Quality of Life Index,\footnote{See \citet{maizels:1984}.} the average life expectancy,\footnote{See \citet{schraeder.etal:1998}.} or the daily caloric intake.\footnote{See \citet{mckinley:1979,schraeder.etal:1998}.}

Meanwhile, only a small body of research investigates the degree to which aid is given in response to acute crises, such as natural disasters, which will be referred to here as humanitarian need. Considering that around 11\% of official development assistance (ODA) was officially categorized as being given for humanitarian reasons in 2015, the systematic failure to include natural disasters as a potential driver of foreign aid is puzzling.\footnote{Total ODA for DAC countries was 131.6 billion in 2015, 15.6 billion of which was designated as humanitarian assistance \url{http://www.oecd.org/dac/development-aid-rises-again-in-2015-spending-on-refugees-doubles.htm} \url{http://www.oecd.org/dac/stats/humanitarian-assistance.htm}}  What evidence that does exist suggests a null or small effect of humanitarian aid on foreign aid allocations. For instance, \citet{bermeo:2008} finds no relationship between the number of people affected by disasters and the allocation of bilateral aid for France, Japan, the UK and the US.\footnote{Note, \citet{bermeo:2008} also conceptualizes humanitarian aid using measures of the number of refugees and civil war, with mixed effects across countries for both}  Similarly, \citet{david:2011} finds no statistically significant relationship between development aid flows and climatic or human disasters. David does find evidence for increased development aid following geological disasters, but the effect is only found with a 2 year lag and substantively small.\footnote{\citet{david:2011} defines climatic events as `floods, droughts, extreme temperatures and hurricanes'; human disasters as: famines and epidemics; geological events as: earthquakes, landslides, volcano eruptions and tidal waves.} \citet{yang:2008} also finds that ODA increases after a hurricane, but only with a lag of 2 years.\footnote{\citet{stromberg:2007} does find a positive and significant relationship between aid and natural disasters, but his paper is concerned with emergency aid in particular, not foreign aid. Similarly, \citet{olsen:2003} find that donors are more likely to give aid for strategic reasons, though their analysis is confined to emergency aid.} 

Finally, there appears to be virtually no work that has explored whether there is a conditional relationship between donor's strategic interest and recipient's humanitarian need. To our whether there may be a conditional relationship between donor's strategic interest and recipient's humanitarian need on foreign aid allocation decisions.  

% One seeming exception is \citet{drury_etal:2005} who find that between 1964 to 1995, the United States made its decision to dispense aid based on strategic considerations, but based the amount given on humanitarian considerations. However, their dependent variable of interest is humanitarian aid, not ODA. 