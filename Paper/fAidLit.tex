 \section*{Lit Review}
\label{lit}



\subsection*{Why and When do Countries Give Foreign Aid?} 
In this literature review, I give a brief overview of the different potential reasons that donor countries distribute foreign aid as explored by existing literature. Note that for the most part, these reasons tend to be tied to characteristics of the recipient country rather than characteristics of the donor countries. I then attempt to document how different scholars have operationalized these different potential motivations and their modelling strategy (to the extent that the modeling strategy may be methodologically interesting to consider for our purposes).\\
\indent\indent In general, it seems that there is no clear delineation between the different possible motivations for foreign aid. For example, in their review of the foreign aid literature, \citet{schraeder.etal:1998} identify 6 possible motivations for foreign aid giving: humanitarian reasons (as measured by average life expectancy and daily caloric intake) , strategic importance, economic potential (that is, if the recipient country is deemed to be most economically strong in the region, the implication being that the better their economy does, the better the donor economy's does), cultural similarity (as measured by colonial history), ideological stance, regional similarity (if countries are similar to each other, donors may use a similar foreign aid strategy for all of them). However it may be impossible to uniquely identify any one of these different measures from the other. Cultural similarity can bleed into regional similarity for example, or strategic importance and economic potential may be endogenous to each other. \\
\indent\indent  What this means in practice is that it is often not clearly defined what the authors are seeking to explain when they throw in different variables into their regressions. They way in which the work below is categorized below then is not strictly how the authors themselves would categorize them, but how they might reasonably fit together given the vagueness and variation across the field. 
% Log GDP per capita has been alternatively used as a measure of objective need or as a measure of the economic potential of a country while the colonial dummy has been interpreted. 




\subsubsection*{Objective Need:Humanitarian/Moral Motivations/Economic Motivations}
 
\citet{lumsdaine:1993} makes a strong case for the `moral vision' being a driving factor in aid distribution. He also makes a case for foreign aid being a function of the colonial history, democratic status, and the income levels of a country.\footnote{ However, as \citet{alesina:2000} note, he presents only simple correlations in his analysis and as such does not provide robust evidence for his theory}  \\
\indent\indent Meanwhile, using a cross national panel analysis, \citet{trumbull:1994} explore the extent to which foreign aid is given as a function of material physical security. They find that infant mortality (their measure of physical security) is associated with increased foreign aid. They further find that GNP per capita  (their measure of material security) is \textit{not} associated with increased foreign aid. 



\subsubsection*{Strategic Motivations}

\citet{maizels:1984}  find evidence to suggest that countries give aid for strategic interests.  Restricting their analysis to African countries, \citet{schraeder.etal:1998} also reject an altruistic motivation for donors.
To that end, \citet{alesina:2000} also find that countries with a colonial past and political alliances\footnote{As measured by UN voting alliances}  to the sender are much more likely to receive foreign aid

\citet{kuziemko:2006} find that nonpermanent members of the UN Security Council\footnote{10 of the 15 seats on the UNSC are held for non renewable 2 year terms} experience temporary increases in foreign aid when they serve on the Council.\\


\noindent Potential other variables: terrorism/terrorist threats?
\subsubsection*{Ideological Reasons}

Ideological motivations for sending foreign aid can in some ways be seen as a combination of humanitarian and strategic reasons. For example, if one frames the conflict between the US and USSR during the Cold War as an ideological struggle, the respective desires of each side to ensure the ideological allegiance of third party countries was as much a function of strategic reasons (i.e as illustrated in the policy of containment) as well as a function of humanitarian ones (i.e. the idea that capitalism/democracy was an inherently morally and politically superior system to communism or vice versa). \\
\indent\indent To this end, \citet{alesina:2000} find that countries that democratize  (as measured by Freedom House) receive more aid with relatively more democratic countries receive 39 percent more aid. They find evidence to suggest that countries pay more attention to democracy strictly defined then broader definition of civil rights or law enforcement.\footnote{They find that US, Dutch, UK, the Nordic countries and Canada conform to this description while Germany and Japan only weakly conform to this description; France does not} This provides an interesting contrast to their finding for FDI which concentrates on rule of law but insensitive to democratic institutions They argue that in general FDI is more sensitive to economic incentives than foreign aid which responds more the political variablesUsing a cross national panel analysis, \citet{trumbull:1994} also explore the extent to which foreign aid is given as a function of political rights. They find that an increase the political and civil rights (as measured by a precursor to the Freedom House index) is associated with an increase in foreign aid. \\
\indent\indent  \citet{alesina:1999} do not find that less corrupt governments receive more foreign aid. However they do find that individual donor countries do vary in terms of how much they give to corrupt governments. They find that Scandinavian countries and Australia give more to less corrupt governments while the US gives more to more corrupt governments. \citet{berthelemy:2006} finds that on average, donors act for political or commercial reasons. However, contrary to \citet{alesina:1991}, they  also tend to give aid to countries with better governance indicators, less conflict and higher growth.



\citet{boutton:2013}

\citet{barthel:2013}
\citet{heinrich:2013}


\subsection*{Variation across countries}

\indent\indent To the extent that the current literature explores variation across donor countries, the literature suggests that different donors may be more motivated by some goals relative to others. \citet{schraeder.etal:1998} find for example that Japanese aid is more likely to be motivated by economic and trade interests, Swedish aid supports progressive, socialist-minded regimes and France's aid is almost exclusively targeted toward francophone countries \citep{werker:2012}. \citet{berthelemy:2006} finds that the Noridc countries, along with Switzerland and Ireland are more prone to be altruistic than other donors. \\
\indent\indent Focusing solely on aid allocation in the Arab world, \citet{neumayer:2003} finds that OPEC donors favor other Arab and non-Arab Muslim recipients.  Countries that do not maintain diplomatic relations with Israel or with voting patterns similar to Saudi Arabia are also more likely to get more aid. He finds that donor interest, in particular Arab solidarity, is a strong determinant in whether a country gets aid and how much while recipient need (as measured by a country's level of income) only affects if the country gets aid, not how much aid they get. \\
\indent\indent Focusing solely on China, \citet{dreher:2012} find that political and commercial considerations shape China's aid allocation. However they find that compared to other countries, China does not pay substantially more attention to politics nor do thy find evidence that China's aid allocation is motivated by natural resource concerns.\footnote{Note however that many projects that could be categorized as foreign aid is often doled out in the form of FDI.} In general they argue that China's aid allocation seems to be independent of democracy and governance considerations in recipient countries.\\
\indent\indent \citet{fleck:2010} investigate how the motivations for US aid have evolved over time. They find evidence to suggest that i) the probability of receiving US foreign aid before and following the War on terror increased with the level of development. Higher-developed countries were more likely to receive aid at all, but this did not necessarily come at the expense of lower developed countries as the overall aid budget also increased  following the War on Terror ii) the level of US aid for the US' core foreign aid countries has been less a function of need following the War on Terror than before. 



% A way to see if donors allocation is not independent is to look at how much total aid a country gets per capita? 

\subsection*{Dependence across countries}
Of course the presumption in these models is that donor giving is completely independent; whether a country receives aid from one donor does not affect whether it receives aid from another donor.  To that end, the presumption in the literature seems to be that this is in fact what is going on; there seems to be little coordination among aid donors with better coordination being perceived as enhancing aid effectiveness. In their investigation of aid allocation from 1956-2006, \citet{aldasoro:2010} find little evidence of increased coordination among aid donors despite rhetoric to improve coordination. \citet{frot:2011} also find that aid donors are likely to 'herd' aid, but find that this effect is substantively small. Exploring a single country, Cambodia, \citet{ohler:2013} also find a lack of evidence for aid coordination among donors. \\ 
\indent\indent \citet{mascarenhas:2006} find similar evidence of non-cooperation among donors but view this as a positive development. Under their framework, foreign aid is a public good and the extent to which donors derive donor-specific benefit from giving foreign aid (and thus give more than what one might expect given the possibility to free-ride), non cooperation among aid donors boosts much need foreign aid for recipients (instead of increasing redundancy and reducing efficiency as implied by \citet{aldasoro:2010} and \citet{frot:2011} )\\

 \indent\indent Meanwhile \citet{steinwand:2014} argue that donors do in fact coordinate under some conditions. Namely they present the idea of 'lead donorship', wherein a recipient has one major donor country and argue that this is more likely to occur when there are large oil exports to donor countries, large imports from donor countries and a colonial history.\footnote{\citet{steinwand:2014} also develop a framework wherein the cross tabs between lead donorship and the existence of coordination indicates whether aid is seen primarily as a private good or public good in a particular situation. } \\
\indent\indent  \citet{fuchs:2013} explore why donors do not better coordinate the allocation of foreign aid. They find evidence to suggest that competition for export markets and political supports prevents coordination and thus implicitly make an argument for why foreign aid allocation is indeed independent across donor countries.  \citet{lawson:2013} also argues competing objectives as well as division of labor problems among donors prevents coordination.
\indent\indent  Moreover \citet{frot:2010} find high fragmentation of aid across sectors. They find that in general, countries that are poor, democratic and have a large population are likely to have more fragmented aid though this is only because they are likely to attract more donors. Once these effects are controlled for, democratic and poor countries are no more likely to recieve aid then their authoritarian and rich counterparts. 





\subsection*{Bypassing bilateral aid}




\subsection*{Methods and Models} 

\subsubsection*{Operationalizing the DV}
\citet{trumbull:1994} operationalize the dependent variable as $log ( \frac{i\text{'s per capita ODA in year } t}{\text{sample average ODA for year }t})$ where $i$ denotes the recipient country.\\

\citet{dreher:2012} operationalize the dependent variable as $\frac{i\text{'s ODA recieved from China in year} t}{\text{China's total aid in year }t}$ where $i$ denotes the recipient country.


\subsubsection*{Common Covariates}

 
\begin{landscape}

\begin{table}
\scriptsize
\caption{ Covariates}
\begin{tabular}{l|c|c|c|c|c|c|c}
\hline
 & Scope  & L(Pop)  & Political   & Objective   & Strategic   & Commerc. & Other \\
 &    		  &  	   & Interests &   Need 	&   Interests 		&   Interests & \\
\hline\hline
\citet{dreher:2012}  &? & X &  Chiebub et al.  & L(GDP pc) & UN Voting; & log total & distance \\
				&  China (donor) & &   (2010)'s  dummy& &    diplomatic relations & exports to China;\\
				& 1956-2006  &  & & &  with Taiwan & log oil production\\
\hline
\citet{fleck:2010}  & 119 (recip)& X &  lag polity ;& L(GDP pc) & US mil budget ;& l(exports)& interwar dum \\ 
				&US (donor) &  & polity transition ;&  & recieved US mil &   & war terror dum \\ 
				&1955-2006&  & political location    &   & aid dummy & &   \\ 
				&1955-2006&  & Pres, Congress   &   &  & &   \\ 
\hline
\citet{dollar:2006}  & $\approx$ 100 (recip) & X &  L(ICRG)  & L(GDP pc) &   &  \% donor exports to& colonial dummy \\
 				 &  22 donors &  &  L(FH)  &  &   &   recipients & L(distance) \\
 				 &  1984-2003 (5yr avg) &  &  &   &   &     &   \\
\hline
\citet{kuziemko:2006}  & 137 (recip)  & &  Polity 2  & L(GDP pc) & UN Security  & & war ($>$1000 deaths) ;\\
 			 	  & US (donor) & &  &   & Council Member &   & NYT articles \\
  				 &  1946-2001& &    &   &   &  & distance \\
\hline
\citet{berthelemy:2006}  &  137 (recip)  & & FH  ;& L(GDP pc) & colony dummy  & lag  imports + & total donor\\
					& 22 donor & &  conflicts(PRIO) ; &GDP growth & US-Egypt dummy &exports/GDP; &  aid\\
					&  1980-99  & &  MAA ; & openness, gov deficit & US-Latin dummy & net debt/exports& ;\\
					&    & &  aid by other  & inflation; life expectancy & Japan-Asia dummy & &  \\
					&   & &  donors  & child mortality & EU- ACP dummy &  mil/GDP&  \\
					&   & &     & lit rate, school enroll&  & & \\
\hline
\citet{neumayer:2003}  & ?& &  Socialist dummy; & L(GDP pc) &   &  l(imports from & Arab dummy \\
   				& Arab aid, agg  & X 	&  UN Voting;  Israel  & &  & Kuwait, Saudi & African dummy \\
  				 & 1974-1997 (3y avg) & 	&    diplomatic relations &   &    &Arabia, UAE) & Islamic dummy \\
\hline
\citet{alesina:2000}  & 181(recip)/ &   &  Rule of Law (PRS)  & L(GDP pc) & UN Voting; & proportion of  years & Egypt/Israel dummy ;\\
              			 & 21(donor) &  X & /Pol \&Civil Rights  & & colony dummy;& country is open ;& \% 		
				Catholic, Muslim \\
				 & 1970-94 (5yr agg) &    & (FH)&&  years as colony& Net FDI inflows/GDP \\
\hline 
\citet{alesina:1999} &  &    & FH  & GDP pc&  CORBI; CORRICRG; & Debt relief/ capita & yrs as colony\\
				& 13 (donors)  &    & && CORRIMD; CORRS\&P & Net FDI and portfolio  & Egypt/Israel dum\\
				& 1970-95 (5 yr avg)  &    &&    &; CORRTI ; CORRWDR1 & investment; net private \\
& 1970-95 (5 yr avg)  &    &&    &CORRWDR2; UN voting& capital flows\\

\hline
\citet{schraeder.etal:1998}  	  & 36(recip)/	& 	 			&  Marxist dum	 &   caloric intake; & mil spending (\% GDP);& L(GDP pc);& lagged DV ;\\
              			 & 4(donor)*  &   	& Socialist dum  & avg life expectancy & mil force (\% pop);&L(imports from donor& colonial dummy  \\
				 & 1980-89 &     &  Capitalist dum &&  ally dumy&  \ total imports ) & region dummy \\
\hline
\citet{trumbull:1994} & 86 countries   & X  & FH & L(Infant  Mortality); &   &  \\
				&1984-89  & &  &  L(GNP pc) &   &  \\
\hline
\citet{maizels:1984} & 79(recip)  & X  &  &L(GNP pc); &  arms transfers &   stock of private direct investm.;\\
				& /DAC(donor)  & &  &    BOP (current accts)& regional dummy  &  \# TNC subsidiaries/affiliates;  \\
				&1969-70& &  &    GNP growth rate &   &Availability of   \\
				&1978-80 & &  &   PQLI &   & strategic materials \\
\hline
\multicolumn{8}{l}{\tiny{* The 36 recipient countries were solely African countries and the 4 donor countries were the US, Japan, Sweden, France }}\\
\multicolumn{8}{l}{\tiny{** Note that somewhat bizarrely, \citet{maizels:1984} runs separate regressions for recipient needs (objective need) and donor interests (security and commercial interests) }}\\
\multicolumn{8}{l}{\tiny{Abbreviations: ICRG - International Country Risk Guide; FH -  Freedom House ; PQLI: Physical Quality of Life Index; ACP -  Associated states from Africa, the Caribbean and the Pacific Ocean; MAA- Multilateral Assistance Acts}}
\end{tabular}
\end{table}

\end{landscape}


\subsubsection*{Models}

\citet{neumayer:2003} uses a two-step Heckman selection model. The first stage is the gate-keeping stage where it is determined which countries recieve aid; the second stage is a level stage where it is determined how much aid a country receives given that they receive any aid. In the Heckman two step estimator, the error terms of both stages are allowed to be correlated however it requries an exclusionary variable that is associated with the gatekeeping stage but not the level stage. \citet{neumayer:2003} uses the total amount of aid allocated in any given year as his exclusionary variable, arguing that the higher amount of total aid allocation increases the chances of receiving any aid at all. He notes that this variable is an imperfect one at best. 

%\clearpage
%\singlespacing
%% run  pdfLaTex then Bibtex then pdflaTex again
%\pagestyle{empty} % This tells LaTeX to make the bibliography without numbers.
%%\singlespace      %Makes the references single spaced.
%\bibliographystyle{apsr}
%%\bibliography{/Users/cindycheng/Documents/Dissertation/Writing/Prospectus/masterlist}
%\bibliography{/Users/cindycheng/Documents/Dissertation/Writing/masterlist}
%%\bibliography{masterlist}
%%\bibliography{~/Users/cindycheng/Documents/DissertationProspectus/latexprospectus/masterlist.bib}
% 


% but have there does not seem to be a clearly defined line in the literature between humanitarian and economic motivations. While giving foreign aid in the wake of a natural disaster can be relatively cleanly classified as being a result of a humanitarian impulse rather than an economic one, giving foreign aid to build schools or hospitals can be equally interpreted as either being motivated by humanitarian/moral reasons or being motivated by developmental/economic ones.

% 
%\begin{itemize}
%\item Log Distance between Recipient and Donor Country
%\begin{itemize}
%\item \citet{dreher:2012} 
%\end{itemize}
%\item Log of population
%\begin{itemize}
%\item \citet{dreher:2012} ;\citet{trumbull:1994} 
%\end{itemize}
%\item Log GDP per capita
%\begin{itemize}
%\item \citet{dreher:2012}; \citet{trumbull:1994} (though note they use GNP per capita)
%\end{itemize}
%\item Political Covariates
%\begin{itemize}
%\item \citet{dreher:2012} (\citet{cheibub:2010}'s democracy dummy) ;\citet{trumbull:1994} (Freedom House)
%\end{itemize}
%\item Objective Need
%\begin{itemize}
%\item  ;\citet{trumbull:1994} (log of infant mortality)
%\end{itemize}
%\item Strategic Interests
%\begin{itemize}
%\item \citet{dreher:2012} (UNGA voting; diplomatic relations with Taiwan) ;
%\end{itemize}
%\item Commercial Interests
%\begin{itemize}
%\item \citet{dreher:2012} (log total exports from China to country j; recipients log oil production per day) ;
%\end{itemize}
%\end{itemize}
%
%
%\subsection*{How does Foreign Aid Effect X?} 
%Jepma (1997) [haven't been able to find online yet but referenced in \citet{alesina:2000}] concludes that foreign aid crowds out private saving, supports public consumption and has no significant impact on recipients macroeconomic policies and growth.
%
%•Cross country growth empirics: \citep{boone:1994,boone:1996} argues that foreign aid does not affect investment and growth; 
%
%Burnside and Dollar find that aid is beneficial to countries that adopt appropriate and stable policies and is otherwise wasted and no evidence of endogeneity. 
%
%Collier and Dollar (1998) show that under certain assumtptions, the allocation of aid that has the maximum effect on poverty reduction is a function of the recipeients level of poverty and quality of economic policies


%

%
%
% Leaving aside this distinction for now and categorizing the above as instances of 'objective need', the below gives an overview of the evidence thus far that suggests that foreign aid might (or might not be) a result of these humanitarian/developmental impulses.\\
