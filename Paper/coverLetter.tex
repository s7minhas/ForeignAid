\documentclass[letterpaper]{article}
\usepackage{graphicx,fullpage}
\usepackage{hyperref}
\usepackage{geometry}
\usepackage[T1]{fontenc}
\usepackage[sc,osf]{mathpazo}

\geometry{
  body={6.5in, 8.5in},
  left=1.0in,
  top=1.25in
}
\usepackage{sectsty}
\sectionfont{\rmfamily\mdseries\Large}
\subsectionfont{\rmfamily\mdseries\itshape\large}
\setlength\parindent{0em}

% Make lists without bullets
\renewenvironment{itemize}{
  \begin{list}{}{
    \setlength{\leftmargin}{1.5em}
  }
}{
  \end{list}
}


\begin{document}
\thispagestyle{empty}
  
\begin{minipage}{0.64\linewidth}
Cindy Cheng \\
Bavarian School of Public Policy \\
Technical University of Munich \\
Munich, Germany
\end{minipage}
\begin{minipage}{0.45\linewidth}
  \begin{tabular}{lr}
    Email: & \href{mailto:cindy.cheng@hfp.tum.de}{\tt cindy.cheng@hfp.tum.de}  \\
    Website:& \href{http://cindyyawencheng.com/}{\tt \url{http://cindyyawencheng.com/}}
  \end{tabular}
\end{minipage}
  
\vspace{1.5in}

{Editorial Team of British Journal of Political Science via submission portal}

\vspace{0.5in}

Dear Colleagues:\\[1ex]

This letter accompanies our submission of a manuscript for your consideration. The manuscript ``Keeping Friends Close, But Enemies Closer: Foreign Aid Responses to Natural Disasters'' seeks to explain the role that natural disasters play in shaping the strategic calculus of donor countries as it pertains to the allocation of foreign aid. While much of the extant literature has focused on highlighting the important role that strategic interests play in shaping aid allocations, little work has been done in exploring on how natural disasters can serve as an exogenous shock to strategic considerations. \\[1ex]

We provide such an analysis that differentiates between three major types of aid: humanitarian, civil society, and development. We develop a set of hypotheses that specify how aid allocation decisions for each of these channels are shaped by natural disasters. In sum, we show that natural disasters are seen by donor countries as an opportunity to exert influence on strategic opponents when it comes to the distribution of humanitarian and civil society aid. We substantiate these findings using a novel measure of strategic interest that accounts for the indirect ties states share and the multiple dimensions upon which they interact. \\[1ex]

We look forward to your evaluation of this paper.\\[1ex]

Respectfully submitted,

\vspace{.1in}

The Authors

\vskip 0.5in
\hrule

\end{document}\bye
