\section*{Reviewer 3}

\subsection*{Major Comments}

\begin{enumerate}
	\item This paper argues that the extent to which natural disasters lead to an increase in bilateral aid allocation is a dependent upon the strategic relations between countries. They examine this relationship using dyadic data, and a novel measure of strategic interest. I find the discussion of potential theoretical mechanisms surrounding the impact of natural disasters upon aid allocation to be well executed and rather common sense. Therefore I think the main contribution of the paper is its empirical analysis, particularly with respect to the effect of strategic interest. Thus I would suggest a revise and resubmit. However, as a primarily empirical contribution the empirical analysis needs to be expanded considerably to be suitable for publication.
	\begin{itemize}
		\item \textcolor{blue}{ \emph{
		Insert great response.
		}}
	\end{itemize}	
	\item One key empirical part of the paper is the introduction of the new measure of strategic interest. I think this measure is a good addition to the literature. However there is one methodological concern that it raises, which is that this measure of strategic interest is an estimate with uncertainty. This introduces a statistical bias akin to measurement error. Therefore the statistical models need to take this into account.
	\begin{itemize}
		\item \textcolor{blue}{ \emph{
		Insert great response.
		}}
	\end{itemize}
	\item Another concern is that strategic interest as operationalised is correlated with factors such as trade and FDI, i.e. economic interests. In the case of the impact of natural disasters on aid, and development assistance in particular, I would think it's exactly such economic linkages that would be relevant for donors. For example to help rebuild infrastructure that facilitates trade between them. Therefore this economic interdependence needs to be incorporated in the empirical specification.
	\begin{itemize}
		\item \textcolor{blue}{ \emph{
		With regards to FDI data, dyadic data does exist from OECD and UNCTAD but there is a lot of missing data in these datasets and they start, at the very earliest, around 2001. As such, it unfortunately doesn't make sensse to use dyadic FDI data for our analysis.
		}}
	\end{itemize}
	\item I also have other comments/suggestions related to the use of this measure of strategic interest:
	\begin{itemize}
		\item It would be interesting to see how much better this measure is at explaining variation in aid commitments compared to existing bilateral approaches. e.g. compare the latent space measure of alliances to the simple measure of whether the countries share an alliance or not.
		\begin{itemize}
			\item \textcolor{blue}{ \emph{
			Insert great response.
			}}
		\end{itemize}
		\item  I'd be interested to see which aspects of strategic interest drive the results. Therefore it would be nice to see the results from simply including each of the latent space measures in the model, to potentially see their relative importance.
		\begin{itemize}
			\item \textcolor{blue}{ \emph{
			Insert great response.
			}}
		\end{itemize}				
	\end{itemize}
	\item I have further concerns about the empirical specification used in the paper:
	\begin{itemize}
		\item Aid tends to have pretty strong temporal dependence, particularly within dyads. However no efforts are taken to model this dependence. Therefore it's important to ensure the results are robust to models that take this into account, such as including a lagged dependent variable or more elaborate specifications such as an Error Correction Model.
		\begin{itemize}
			\item \textcolor{blue}{ \emph{
			Insert great response.
			}}
		\end{itemize}
		\item There should be discussion of why fixed effects models aren't estimated, given their common use in the aid literature and given that unobserved unit heterogeneity is likely correlated with the right hand side variables. At least FE models should be estimated as a robustness test.
		\begin{itemize}
			\item \textcolor{blue}{ \emph{
			Insert great response.
			}}
		\end{itemize}	
		\item Any justification for why the independent variables are lagged by one year?			
		\begin{itemize}
			\item \textcolor{blue}{ \emph{
			Insert great response.
			}}
		\end{itemize}			
	\end{itemize}	
	\item Regarding the dependent variable could there be an issue of countries committing more to non-strategically aligned countries, with the expectation that they will not accept all of this money? Some information on how the relationship between commitments and disbursements varies according to strategic interest would be useful.
	\begin{itemize}
		\item \textcolor{blue}{ \emph{
		Insert great response.
		}}
	\end{itemize}	
	\item Finally as a presentational issue I think that presenting the marginal effects in addition to the predicted values would be useful, at least in an appendix.
	\begin{itemize}
		\item \textcolor{blue}{ \emph{
		Insert great response.
		}}
	\end{itemize}	
\end{enumerate}