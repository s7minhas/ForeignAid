\section*{Reviewer 1}

\begin{enumerate}
	\item Overall, I still think this manuscript is a strong submission which promises to contribute to our understanding of the strategic logic of foreign aid allocation in a novel way. However, I have three remaining reservations that I would like to see addressed before moving to publications.
	\begin{itemize}
		\item \textcolor{blue}{ \emph{
		We thank the reviewer for the comment and for their additional suggestions.
		}}
	\end{itemize}
	\item  First, I think the authors should move the specification with the lagged dependent variable into the main text and present this as the main model in the paper. The time dependency of aid giving is a well-established pattern in the aid literature, and it is the norm for the literature to model this time dependency in the main specifications. 
	\begin{itemize}
		\item \textcolor{blue}{ \emph{
		Insert great response.
	}}

	\textcolor{red}{q for cc: i'm really not in favor of rerunning everything with a lagged dv and recentering all the analyses around that given one reviewer's request. i suggest htat we just say we will expand the appendix section with the lagged dv framework to show that it is consistent. the plot below is the sub effects plot for all three dvs using a lagged dv and it's basically the same. what do you think? }

	\begin{figure}
		\centering
		\includegraphics[width=1\textwidth]{simComboPlot_lagDV.pdf}
		\caption{Simulated substantive effect plots for each dependent variable (humanitarian aid, civil society aid, and development aid) for different levels of natural disaster severity across the range of the strategic distance measure. A rug plot is provided below each panel.}
		\label{fig:simEffects_lagDV}
	\end{figure}

	\begin{figure}[h!]
		\centering
		\includegraphics[width=1\textwidth]{simComboPlot_lag_hAid_lagDV.pdf}
		\caption{Simulated substantive effect plots for humanitarian aid for varying lags of variables of interest and different levels of natural disaster severity across the range of the strategic distance measure.}
		\label{fig:humanIntCoef}
	\end{figure}

	\begin{figure}[h!]
		\centering
		\includegraphics[width=1\textwidth]{simComboPlot_lag_cAid_lagDV.pdf}
		\caption{Simulated substantive effect plots for civil society aid for varying lags of variables of interest and different levels of natural disaster severity across the range of the strategic distance measure.}
		\label{fig:civIntCoef}
	\end{figure}

	\begin{figure}[h!]
		\centering
		\includegraphics[width=1\textwidth]{simComboPlot_lag_dAid_lagDV.pdf}
		\caption{Simulated substantive effect plots for development aid for varying lags of variables of interest and different levels of natural disaster severity across the range of the strategic distance measure.}
		\label{fig:devIntCoef}
	\end{figure}
	\FloatBarrier

	\end{itemize}

	\clearpage
	\item Relatedly, I think the authors should provide the donor fixed effects specifications in the appendix and report the coefficients and standard errors in a table. It is difficult to tell for sure from the coefficient plot, but it looks like the coefficient on the main interaction term is no longer significant at the 95\% level in this specification. That's not necessarily a problem - it would just be nice to know for sure what the magnitude and uncertainty are in this specification.
	\begin{itemize}
		\item \textcolor{blue}{ \emph{
		Insert great response.
		}}

\textcolor{red}{comment for cc: what an annoying reviewer, we can clean these tables up and put them in the appendix as well}

humanitarianTotal
% latex table generated in R 3.6.0 by xtable 1.8-4 package
% Mon Jul 29 22:38:35 2019
\begin{table}[ht]
\centering
\begin{tabular}{rrrrlr}
  \hline
 & beta & se & t & var & pval \\ 
  \hline
1 & 0.81 & 0.10 & 8.11 & LstratMu & 0.00 \\ 
  2 & -0.12 & 0.07 & -1.74 & Lno\_disasters & 0.08 \\ 
  3 & 9.56 & 1.66 & 5.78 & colony & 0.00 \\ 
  4 & 0.00 & 0.01 & 0.31 & Lpolity2 & 0.76 \\ 
  5 & -1.08 & 0.08 & -12.80 & LlnGdpCap & 0.00 \\ 
  6 & -0.03 & 0.01 & -2.30 & LlifeExpect & 0.02 \\ 
  7 & 1.41 & 0.11 & 12.99 & Lcivwar & 0.00 \\ 
  2410 & 0.08 & 0.02 & 4.11 & LstratMu:Lno\_disasters & 0.00 \\ 
   \hline
\end{tabular}
\end{table}

developTotal
% latex table generated in R 3.6.0 by xtable 1.8-4 package
% Mon Jul 29 22:38:35 2019
\begin{table}[ht]
\centering
\begin{tabular}{rrrrlr}
  \hline
 & beta & se & t & var & pval \\ 
  \hline
1 & -0.76 & 0.10 & -7.45 & LstratMu & 0.00 \\ 
  2 & -0.04 & 0.07 & -0.50 & Lno\_disasters & 0.61 \\ 
  3 & 9.70 & 1.67 & 5.81 & colony & 0.00 \\ 
  4 & 0.04 & 0.01 & 4.96 & Lpolity2 & 0.00 \\ 
  5 & -0.63 & 0.09 & -7.29 & LlnGdpCap & 0.00 \\ 
  6 & 0.10 & 0.01 & 8.63 & LlifeExpect & 0.00 \\ 
  7 & -0.73 & 0.11 & -6.67 & Lcivwar & 0.00 \\ 
  2410 & 0.02 & 0.02 & 1.20 & LstratMu:Lno\_disasters & 0.23 \\ 
   \hline
\end{tabular}
\end{table}

civSocietyTotal
% latex table generated in R 3.6.0 by xtable 1.8-4 package
% Mon Jul 29 22:38:35 2019
\begin{table}[ht]
\centering
\begin{tabular}{rrrrlr}
  \hline
 & beta & se & t & var & pval \\ 
  \hline
1 & -0.18 & 0.09 & -2.03 & LstratMu & 0.04 \\ 
  2 & -0.19 & 0.06 & -3.11 & Lno\_disasters & 0.00 \\ 
  3 & 2.00 & 1.44 & 1.39 & colony & 0.16 \\ 
  4 & 0.02 & 0.01 & 2.52 & Lpolity2 & 0.01 \\ 
  5 & -0.62 & 0.07 & -8.48 & LlnGdpCap & 0.00 \\ 
  6 & 0.02 & 0.01 & 2.32 & LlifeExpect & 0.02 \\ 
  7 & -0.34 & 0.09 & -3.66 & Lcivwar & 0.00 \\ 
  2410 & 0.10 & 0.02 & 5.83 & LstratMu:Lno\_disasters & 0.00 \\ 
   \hline
\end{tabular}
\end{table}
\FloatBarrier

	\end{itemize}
	\clearpage
	\item Second, I think the authors need to show us what is happening with total aid. It was not clear from their first submission that governance aid, budget aid, and technical assistance are all excluded categories. My concern, which I tried to express in the first referee report, is that when a strategic ally has a natural disaster, donors will have pre-existing aid relationships with the strategic ally and be able to amp up aid through these pre-existing channels without needing the category of humanitarian aid at all in order to provide support. A major way this could be done is by allocating budget aid, which is a channel that donors are more likely to use for strategic allies. With strategic opponents, by contrast, donors need the category of humanitarian aid to provide the same support because (i) there are not pre-existing aid channels that donors can use to increase aid flows and (ii) because donors need to be more specific about what aid should be used for when giving to strategic opponents compared to allies in order to convince skeptical institutions within their own governments to grant aid to an opponent. So, I would like to see 4 columns in the specifications of the main models - humanitarian aid, civil society aid, development aid, and total aid (subtracting out humanitarian aid and civil society aid from total aid). If the authors are correct that natural disasters are prompting more giving in strategic opponents compared to allies, then I would expect a null effect of their interaction term on total aid (subtracting out humanitarian aid and civil society aid from total aid).
	\begin{itemize}
		\item \textcolor{blue}{ \emph{
		Insert great response.
		}}

\textcolor{red}{q for cc: what do you think of this result? this basically seems like a null effect to me but what do you think? also coef plot looks kinda silly might just do a table here}

	\FloatBarrier
	\begin{figure}
		\centering
		\includegraphics[width=.5\textwidth]{totAidv2_Coef.pdf}
		\caption{coefficient plot for total aid.}
		\label{fig:coefPlot_totAidv2}
	\end{figure}

	\begin{figure}
		\centering
		\includegraphics[width=1\textwidth]{simTotAidPlot_lagDV.pdf}
		\caption{Simulated substantive effect plots for total aid for different levels of natural disaster severity across the range of the strategic distance measure. A rug plot is provided below each panel.}
		\label{fig:simEffects_totAid}
	\end{figure}
	\FloatBarrier

	\end{itemize}

	\clearpage
	\item Third, given that the primary contribution of the paper is to test the strategic logic behind humanitarian aid, I think the authors can still do more to make this strategic logic clearer. In Hypothesis 1C, the authors claim that the strategic logic of humanitarian aid is about improving strategic relations with opponents. They repeat this a few times in the paper (e.g., pg. 35 of resubmission). However, this doesn't match the strategic logic in some of their own examples. For example, the U.S. giving humanitarian aid to Venezuela probably isn't to improve relations with the Maduro regime, but more about laying the groundwork for regime change (possibly by winning over the population). Is what 1C means that donors are trying to improve relations with the populations within strategic opponents, possibly at the sitting regime's expense, which remains their strategic opponent? Or, do the authors mean that donors are trying to improve relations with the regimes of strategic opponents? The authors need to be clear about what the strategic logic is throughout the paper and make sure it's framed consistently. 
	\begin{itemize}
		\item \textcolor{blue}{ \emph{
		Insert great response.
		}}
	\end{itemize}
	\item  Finally, a small point, but I found the first sentence of the new abstract very confusing. Can the authors rephrase this?
	\begin{itemize}
		\item \textcolor{blue}{ \emph{
		Insert great response.
		}}
	\end{itemize}
\end{enumerate}
