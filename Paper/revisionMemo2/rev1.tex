\section*{Reviewer 1}

\begin{enumerate}
	\item Overall, I still think this manuscript is a strong submission which promises to contribute to our understanding of the strategic logic of foreign aid allocation in a novel way. However, I have three remaining reservations that I would like to see addressed before moving to publications.
	\begin{itemize}
		\item \textcolor{blue}{ \emph{
		We thank the reviewer for the comment and for their additional suggestions.
		}}
	\end{itemize}
	\item  First, I think the authors should move the specification with the lagged dependent variable into the main text and present this as the main model in the paper. The time dependency of aid giving is a well-established pattern in the aid literature, and it is the norm for the literature to model this time dependency in the main specifications. 
	\begin{itemize}
		\item \textcolor{blue}{ \emph{
		We have rerun each of the main analyses presented in the paper with a lagged dependent variable in the specification. To show the robustness of our results when a lagged dependent variable is included in the specification we have created a new section in the Appendix titled, ``Reanalysis when Including Lagged Dependent Variable''. In this section we have reconstructed each of the key analyses in the main version of our paper: Figures 4-7. The interpretations of each of the analyses with and without the lagged dependent variable essentially mimic each other. As a result, we favor the more parismonious model specification be presented in the main paper, but agree with the importance of showcasing the robustness of our findings to the inclusion of a lagged dependent variable. We would also note that not including a lagged dependent variable in the main paper is not atypical in the recent literature on foreign aid. For example, the following are a small sampling of papers published since 2017, cited in our paper, that do not include a lagged dependent variable: Bermeo 2017 (published in International Organization), Carnegie \& Marinov 2017 (published in American Journal of Political Science), and Dreher et al. 2018 (published in International Studies Quarterly).
	}}

	\end{itemize}

	\clearpage
	\item Relatedly, I think the authors should provide the donor fixed effects specifications in the appendix and report the coefficients and standard errors in a table. It is difficult to tell for sure from the coefficient plot, but it looks like the coefficient on the main interaction term is no longer significant at the 95\% level in this specification. That's not necessarily a problem - it would just be nice to know for sure what the magnitude and uncertainty are in this specification.
	\begin{itemize}
		\item \textcolor{blue}{ \emph{
		We have included tabular versions of the results with fixed effects in the appendix. 
		}}

	\end{itemize}
	\clearpage
	\item Second, I think the authors need to show us what is happening with total aid. It was not clear from their first submission that governance aid, budget aid, and technical assistance are all excluded categories. My concern, which I tried to express in the first referee report, is that when a strategic ally has a natural disaster, donors will have pre-existing aid relationships with the strategic ally and be able to amp up aid through these pre-existing channels without needing the category of humanitarian aid at all in order to provide support. A major way this could be done is by allocating budget aid, which is a channel that donors are more likely to use for strategic allies. With strategic opponents, by contrast, donors need the category of humanitarian aid to provide the same support because (i) there are not pre-existing aid channels that donors can use to increase aid flows and (ii) because donors need to be more specific about what aid should be used for when giving to strategic opponents compared to allies in order to convince skeptical institutions within their own governments to grant aid to an opponent. So, I would like to see 4 columns in the specifications of the main models - humanitarian aid, civil society aid, development aid, and total aid (subtracting out humanitarian aid and civil society aid from total aid). If the authors are correct that natural disasters are prompting more giving in strategic opponents compared to allies, then I would expect a null effect of their interaction term on total aid (subtracting out humanitarian aid and civil society aid from total aid).
	\begin{itemize}
		\item \textcolor{blue}{ \emph{
		This is indeed important to consider and we thank the reviewer for providing a concrete suggestion for how to test for this possibility. We have run the model as suggested and have found that there is a null effect on the interaction between natural disasters and our measure of strategic interest on total aid (as defined as all aid minus total humanitarian aid and minus total civil society aid) (see Figure \ref{fig:coefPlot_totAidv2} . When we plot the interaction effects, we verify this null finding (see Figure \ref{fig:simEffects_totAid}fig:simEffects_totAid). We have opted to present our main results as straightforwardly as possible but we will include this model in the appendix and have adjusted the language in our robsutness check section to note that we have explored and have not found evidence that donors engage in strategic labelling in this context. 
		}}

 
	\FloatBarrier
	\begin{figure}
		\centering
		\includegraphics[width=.5\textwidth]{totAidv2_Coef.pdf}
		\caption{coefficient plot for total aid.}
		\label{fig:coefPlot_totAidv2}
	\end{figure}

	\begin{figure}
		\centering
		\includegraphics[width=1\textwidth]{simTotAidPlot_lagDV.pdf}
		\caption{Simulated substantive effect plots for total aid for different levels of natural disaster severity across the range of the strategic distance measure. A rug plot is provided below each panel.}
		\label{fig:simEffects_totAid}
	\end{figure}
	\FloatBarrier

	\end{itemize}

	\clearpage
	\item Third, given that the primary contribution of the paper is to test the strategic logic behind humanitarian aid, I think the authors can still do more to make this strategic logic clearer. In Hypothesis 1C, the authors claim that the strategic logic of humanitarian aid is about improving strategic relations with opponents. They repeat this a few times in the paper (e.g., pg. 35 of resubmission). However, this doesn't match the strategic logic in some of their own examples. For example, the U.S. giving humanitarian aid to Venezuela probably isn't to improve relations with the Maduro regime, but more about laying the groundwork for regime change (possibly by winning over the population). Is what 1C means that donors are trying to improve relations with the populations within strategic opponents, possibly at the sitting regime's expense, which remains their strategic opponent? Or, do the authors mean that donors are trying to improve relations with the regimes of strategic opponents? The authors need to be clear about what the strategic logic is throughout the paper and make sure it's framed consistently. 
	\begin{itemize}
		\item \textcolor{blue}{ \emph{
		Thanks for pointing out where we could be clearer in our writing. We agree that, a priori, it is unclear whether donors might give humanitarian aid to strategic opponents in order to improve relations with the existing government or to improve relations with those who would oppose the existing government. The motivation may be both dependent on both the recipient. Moreover, arguably, it may be to some extent unknowbale to the donor itself (a priori the donor may not be 100 percent certain whether a gesture of humanitarian aid will be received warmly or coldly by different domestic actors). The aim of H1C however is more modest than this --- we simply wish to test whether donors may be motivated to improve relations with a strategic opponent --- what causal pathway this may happen through (via improving relations with the domestic government or other actors) is not something that we try to test here. To that end, we have added a paragraph to clarify this goal in the paper. 
		}}
	\end{itemize}
	\item  Finally, a small point, but I found the first sentence of the new abstract very confusing. Can the authors rephrase this?
	\begin{itemize}
		\item \textcolor{blue}{ \emph{
		Thank you for this comment, we have revised the sentence, though we're open to further suggestions on how it might be better worded. 
		}}
	\end{itemize}
\end{enumerate}
