\documentclass[doublespaced,bibtex]{cupjournal}

%%%%%%%%%%%%%%%%%%%%%%%%%%%%%%%%%%%%%%%%%%%%%%%%%%
%%%%%%%%%%%%%%%%%%%% PREAMBLE %%%%%%%%%%%%%%%%%%%%
%%%%%%%%%%%%%%%%%%%%%%%%%%%%%%%%%%%%%%%%%%%%%%%%%%


% % -------------------- defaults -------------------- %
% % load lots o' packages

% % approx iid
% \newcommand\simiid{\stackrel{\mathclap{\normalfont\mbox{\tiny{iid}}}}{\sim}}

% % define bibliography style
 
% % Fonts
% % at some point figure out bolding ...
% \usepackage[utf8]{inputenc}

 


% % layout control
% %\usepackage[doublespacing]{setspace}
% %\setlength{\parskip}{.5em}
% \usepackage{rotating}
% \usepackage{setspace}
% \usepackage{fancyhdr}
% \usepackage{parallel}
% \usepackage{parcolumns}
% \usepackage{pdflscape}
% % math typesetting
% \usepackage{array}
% \usepackage{amsmath}
% \usepackage{amsfonts}
% \usepackage{amssymb}

% % tables
\usepackage{tabularx}
% \usepackage{booktabs}
% \usepackage{multicol}
\usepackage{multirow}
% \usepackage{longtable}

% \usepackage[%
% decimalsymbol=.,
% digitsep=fullstop
% ]{siunitx}

% % to adapt caption style
% %\usepackage[font={small},labelfont=bf]{caption}

% % footnotes at bottom
% %\usepackage[bottom]{footmisc}

% % to change enumeration symbols begin{enumerate}[(a)]
% \usepackage{enumerate}

% % to make enumerations and itemizations within paragraphs or
% % lines. f.i. begin{inparaenum} for (a) is (b) and (c)
% \usepackage{paralist}

% % to colorize links in document. See color specification below
\usepackage[x11names]{xcolor}

% % for multiple references and insertion of the word "figure" or "table"
% % \usepackage{cleveref}

% % load the hyper-references package and set document info
\usepackage[pdftex]{hyperref}

% % graphics stuff
% \usepackage{subfig}
\usepackage{graphicx}
% \usepackage[space]{grffile} % allows us to specify directories that have spaces
\usepackage[section]{placeins} % prevents floats from moving past a \FloatBarrier or section
% \usepackage{tikz}
% \usepackage{caption}

% % \usepackage{pgfplots}

% % define clickable links and their colors
\hypersetup{
	unicode=false,          % non-Latin characters in Acrobat's bookmarks
	pdftoolbar=true,        % show Acrobat's toolbar?
	pdfmenubar=true,        % show Acrobat's menu?
	pdffitwindow=false,     % window fit to page when opened
	pdfstartview={FitH},    % fits the width of the page to the window
	pdfnewwindow=true,%
	%pdfauthor={Cindy Cheng and Shahryar Minhas},%
	%pdftitle={Aid something},%
	colorlinks,%
	citecolor=black,%
	filecolor=black,%
	linkcolor=black,%
	urlcolor=RoyalBlue4%
	}

% % Including External Code
% \usepackage{verbatim}
% \usepackage{listings}
% \lstset{
% 	language=R,
% 	basicstyle=\scriptsize\ttfamily,
% 	commentstyle=\ttfamily\color{gray},
% 	numbers=left,
% 	numberstyle=\ttfamily\color{gray}\footnotesize,
% 	stepnumber=1,
% 	numbersep=5pt,
% 	backgroundcolor=\color{white},
% 	showspaces=false,
% 	showstringspaces=false,
% 	showtabs=false,
% 	frame=single,
% 	tabsize=2,
% 	captionpos=b,
% 	breaklines=true,
% 	breakatwhitespace=false,
% 	title=\lstname,
% 	escapeinside={},
% 	keywordstyle={},
% 	morekeywords={}
% 	}

% -------------------------------------------------- %

\usepackage{graphicx}
\usepackage{amssymb}
% -------------------- title -------------------- %

\title{Keeping Friends Close, But Enemies Closer: Foreign Aid Responses to Natural Disasters}
% \author{	Cindy Cheng \\
% 	\texttt{cindy.cheng@hfp.tum.de}
% 	\and
% 	Shahryar Minhas \\
% 	\texttt{minhassh@msu.edu}}
% \date{\today}

% \setlength{\headheight}{15pt}
% \setlength{\headsep}{20pt}
% \pagestyle{fancyplain}
 
% \fancyhf{}

% \lhead{\fancyplain{}{}}
% \chead{\fancyplain{}{}}
% \rhead{\fancyplain{}{}}
% \rfoot{\fancyplain{}{}}


 



% ----------------------------------------------- %


% -------------------- customizations -------------------- %

%\graphicspath{{~/Users/cindycheng/Dropbox/ForeignAid/graphics/}}
\graphicspath{{graphics/}}
\makeatletter
\def\input@path{{graphics/}}
 \makeatother

% \makeatletter
%\def\input@path{{/Users/janus829/Dropbox/Research/ForeignAid/Graphics/}}
% \makeatother
%\graphicspath{{/Users/janus829/Dropbox/Research/ForeignAid/Graphics/}}

% easy commands for number propers
\newcommand{\first}{$1^{\text{st}}$}
\newcommand{\second}{$2^{\text{nd}}$}
\newcommand{\third}{$3^{\text{rd}}$}
\newcommand{\nth}[1]{${#1}^{\text{th}}$}

% easy command for boldface math symbols
\newcommand{\mbs}[1]{\boldsymbol{#1}}

% -------------------------------------------------------- %


%%%%%%%%%%%%%%%%%%%%%%%%%%%%%%%%%%%%%%%%%%%%%%%%%%
%%%%%%%%%%%%%%%%%%%% DOCUMENT %%%%%%%%%%%%%%%%%%%%
%%%%%%%%%%%%%%%%%%%%%%%%%%%%%%%%%%%%%%%%%%%%%%%%%%
%\addbibresource{Paper/fAidRefs.bib}

\usepackage{enumitem}
\newlist{inparaenum}{enumerate}{2}% allow two levels of nesting in an enumerate-like environment
\setlist[inparaenum]{itemsep=0pt,leftmargin=10em,label=\textsc{Hypothesis} \arabic{inparaenumi}\Alph*: }% labels for top level
\setlist[inparaenum,2]{label=\arabic{inparaenumi}\alph*:) }% labels for second level


\newlist{inparaenumb}{enumerate}{2}% allow two levels of nesting in an enumerate-like environment
\setlist[inparaenumb]{itemsep=0pt,leftmargin=10em,label=\textsc{Hypothesis} \arabic{inparaenumbi}B: }% labels for top level
\setlist[inparaenumb,2]{label=\arabic{inparaenumbi}\alph*:) }% labels for second level

\newlist{inparaenumc}{enumerate}{2}% allow two levels of nesting in an enumerate-like environment
\setlist[inparaenumc]{itemsep=0pt,leftmargin=10em,label=\textsc{Hypothesis} \arabic{inparaenumci}C: }% labels for top level
\setlist[inparaenumc,2]{label=\arabic{inparaenumci}\alph*:) }% labels for second level


\begin{document}


 
\markboth{Author}{Keeping Enemies Close}
\journalname{Research Article Submission to B.J.Pol.S.}
\journalvolume{XX}
\journalyear{2018}
\pagecount{X--XX}
\doinumber{XXX}
%doi:10.1017/XXXX

\maketitle

\begin{abstract}

 
If foreign aid is given primarily for strategic reasons, as much of the field finds, how can we explain donor generosity following natural disasters? In this paper, we address this puzzle by building on the literature in three ways. First, we differentiate between three major types of aid: humanitarian, civil society, and development. Next, we show natural disasters act as an exogenous shock to the strategic calculus donor countries undertake when making foreign aid allocation decisions. Specifically, we argue that donor countries use natural disasters as opportunities to exert influence on strategic opponents through the allocation of humanitarian and civil society aid. However, donors still reserve development aid for strategic allies irrespective of the incidence of natural disasters. Lastly, we substantiate our findings using a new measure of strategic interest that accounts for the indirect ties states share and the multiple dimensions upon which they interact.

%While the existing literature shows that bilateral donors primarily allocate aid to strategic allies, anecdotal evidence suggests that following natural disasters, bilateral aid flows to strategic opponents quite generously. We build on this literature in three ways. First, we differentiate between the three major types of aid: humanitarian, civil society, and development. Next, we argue that donor countries use natural disasters as opportunities to exert influence on strategic opponents through the allocation of humanitarian and civil society aid. However, donors still primarily reserve development aid for strategic allies irrespective of the occurrence of natural disasters . Last, we substantiate our findings using a new measure of strategic interest that accounts for the indirect ties states share and the multiple dimensions upon which they interact.
\end{abstract}

%%%%%%%% INTRO %%%%%%%%
\section*{Introduction}
\label{intro} 

Human and economic catastrophes associated with natural hazards are obviously not new, even if new media have changed the way we are aware of them. The January 2010 earthquake in Haiti and the Indian Ocean tsunami of 2004 both generated much international media attention and unprecedented amounts of international pledges of aid from private charities, non-governmental organizations, governments, and multilateral organizations.1 Nonetheless, aid pledges made while media attention is at its peak may not always be disbursed, could take a long time to arrive, or may replace previously pledged aid. This raises the following questions: how much does foreign aid really increase in the aftermath of large disasters? Are aid surges sizable in relation to the estimated economic damages caused by disasters? And what determines the actual size of the surges?

Natural disasters are one of the major problems facing humankind. Be- tween 1980 and 2004, two million people were reported killed and five billion people cumulatively affected by around 7,000 natural disasters, according to the dataset maintained by the Centre for Research on the Epidemi- ology of Disasters (CRED) at University of Louvain (Belgium). The economic costs are considerable and rising. The direct economic damage from natural disasters between 1980–2004 is estimated at around \$1 trillion.

Donor countries may also provide relief with an eye to their own economic or geostrategic political interest (for example, Alesina and Dollar, 2000, and the references therein). Large disasters may destabilize governments. Aid to friendly governments could help these stay in power; withholding aid from not-so-friendly governments could destabilize them (Drury, Olson, and Van Belle, 2005). Disaster relief may also be used to protect investments in foreign countries, driving relief towards countries where the donors have large economic stakes.

United States - Iran earthquake - 2003

http://www.nytimes.com/2012/08/15/world/middleeast/us-vows-to-speed-aid-to-iran-earthquake-victims.html

Due to the earthquake, relations between the United States and Iran thawed. The U.S. usually treated Iran as part of the "axis of evil", as its President George W. Bush called the nation.[16] However, following the tremor White House spokesman Scott McClellan spoke on behalf of President Bush: "Our thoughts and prayers are with those who were injured and with the families of those who were killed."[5]


Fairfax County Urban Search and Rescue squad inspect earthquake damage in Bam
The U.S. offered direct humanitarian assistance to Iran. Iran initially declined this offer,[17] though later accepted it. On December 30 an 81-member emergency response team was deployed to Iran via U.S. military aircraft, consisting of search and rescue squads, aid coordinators, and medical support.[21] These were the first U.S. military airplanes to land in Iran for more than 20 years.[12]

In return, the state promised to comply with an agreement with the International Atomic Energy Agency which supports better monitoring of its nuclear interests. This led U.S. Secretary of State Colin Powell to suggest direct talks in the future.[16] Sanctions were temporarily relieved to help the rescue effort.[20] However, he also said that the U.S. was still concerned on other Iranian issues, such as the prospect of terrorism and the country's support of Hamas.[16]

\indent\indent  Foreign aid describes the transfer of resources from one government to another. Although the term itself suggests a humanitarian motive, scholars and experts have long debated whether it would be more accurate to ascribe foreign aid a strategic motive instead. With some exceptions \citep{bermeo:2008}, most scholars have found that donors prioritize strategic considerations when dispensing aid \citep{alesina:2000, berthelemy:2006}. \\

\indent\indent  This seeming consensus belies the inconsistency with which scholars conceptualize and measure strategic considerations, which have variously included bilateral trade intensity, UN voting scores, colonial legacies and regional dummies among others. In this paper we seek to rectify in fragmentation: First, we create an original measure of bilateral strategic interest that measures the latent distance between countries across the strategic policy space.  In doing so, we seek to provide a more coherent measure of strategic interest which incorporates many of the measures that previous papers have used. Further this measure improves upon existing measures of strategic interest in that it maps strategic interest onto a ``social space'', through which we can account for third order relationships between states \citep{hoff:2002}. \\ 

%%Scholars have variously used bilateral trade intensity \citep{berthelemy:2006, berthelemy:2004}, colonial legacies \citep{berthelemy:2004},  UN voting \citep{alesina:2000}, political orientation of the recipient country \citep{easterly:2008} to measure strategic interest. Other scholars take a different approach and investigate how much aid allocations can be ascribed to humanitarian reasons. These are broadly split along economic need \citep{collier:2002,nunnenkamp:2006,thiele:2007} and the quality of political governance \citep{neumayer:2005,dollar:2006}, the implication being that countries that fail to give aid along these criterion are acting in their strategic interest.\\

%\indent %These measures are at best imperfect and at worst, uninterpretable. As \citet{bermeo:2008} states,
%\begin{quote} `Perhaps the most puzzling conclusion of the existing literature is that a focus on trade partners, former colonies, and allies is somehow evidence against a development focus of aid. Instead, one could interpret this as evidence that donors give aid to the countries in which they most wish to pursue development. In this sense, donor interests and recipient needs are not mutually exclusive categories.'
%\end{quote} Meanwhile some have argued that donors who give to poor countries may not do it out of a %humanitarian impulse but because it is cheaper to buy interest in poorer countries \citep{demesquita:2007,stone:2006}. Conversely, a donor country may give to a poorly governed, undemocratic country for humanitarian reasons as well, North Korea being a prominent example.

\indent\indent  The existing lack of coherence in evaluating strategic interest extends to model specification. Papers which have empirically evaluated the dominance of strategic over humanitarian motives with some exceptions \citep{berthelemy:2006}, have done so by specifying models which pool all donors together or by running models for each donor country separately. In our model specification, we use a hierarchical random effects model with countries receiving aid nested in senders and senders nested in time. Applying this method will enable us to explicitly model the drivers of aid in an aggregate sense and to also explore how those drivers vary between senders. For now we present preliminary results that show our strategic interest variable does play a positive role in predicting aid flows between countries.

%%%%%%%%%%%%%%%%%%%%%%
% SM note: So we don't resolve these issues with our current model specification either. We could have tried to get at these issues using a network modeling approach, but as we discussed the bipartite nature of our data took away that option. In the next iteration of this paper, we can take a look into adding spatially lagged covariates to get at this issue. Hmmm, it might also be interesting to weight aid flows by our strategic interest variable. This would help us to test whether or not states follow their strategic partners in giving aid flows. 
%%%%%%%%%%%%%%%%%%%%%%

% We find this empirical choice puzzling - if foreign aid is indeed given for strategic reasons then surely a donor country should account for the foreign aid given by other countries when making their own allocations. The same should be equally true if foreign aid is given for humanitarian reasons - if a very needy country is already receiving an abundant amount of foreign aid from other countries, a particular donor country may decide to dispense aid to a less needy but overlooked recipient country. Pooled models do not address this issue as they do not distinguish between donor countries while donor by donor regressions cannot address this issue because by construction they do not account for the allocations of other donor countries. \\

% \indent\indent With Interestingly, we also find meaningful variation between countries in the relevance of that strategic interest variable in directing aid flows, and the effect of that variable has a noticeable upward trend over time. Indicating that in recent years more and more countries are directing aid to those countries that are most relevant to their strategic interests. 

\indent\indent In what follows, we first give a brief overview of the literature before introducing our new measure of strategic interest. We then run our analyses of the motivations for foreign aid with our new measure using a hierarchical random effects model. We discuss the implications of our results before concluding. 


% One way to think about these matrices is that they provide data on N(N − 1) dyadic relationships among the N actors. A simpler way to think about them is that they de- scribe a single network of interactions, and can therefore be summarized using network modeling techniques. I do so by estimating general bilinear mixed effects (GBME) mod- els. They produce a simple, low-dimensional representation of the public relationships among a large number of individuals and groups. Actors that have a high probability of cooperative interactions are placed closely together in the latent network space, whereas those that have a high probability of conflictual interactions are located far away from each other. The positions are thus determined by condensing dyadic links into a hypothetical low-dimenstional space. As such, my approach is conceptually related (although using a different statistical approach) to recent projects that estimate ideal points for politicians and societal actors using campaign donations or Twitter follower patterns (Bonica, 2014; Barberá, 2014).

% Key findings from faid paper. Countries provide foreign aid as a result of strategic interest. 
%%%%%%%%%%%%%%%%%%%%%%%

%%%%% Lit Review %%%%%
\section*{Accounting for Natural Disasters in Determining Donor Motivations for Foreign Aid}
\label{theory}

Natural disasters can lead to the destruction or impairment of physical and social infrastructure and even more significantly, the devastating loss of human lives. For example, the 1985 Mexico City Earthquake, one of the most catastrophic natural disasters in modern times, killed at least 10,000 people\footnote{The Editors of Encyclopaedia Brittanica. ``Mexico City earthquake of 1985.'' \textit{Encyclopaedia Britannica}. 20 September 2017. Accessed September 2017: \url{https://www.britannica.com/event/Mexico-City-earthquake-of-1985}} and cost around 4 billion 1985 dollars (around 9 billion in 2017 dollars).\footnote{Wiliams, Dan. `Mexico Quake Loss put at \$4 Billion: Report by U.N. Panel Includes Damages to Economy.' \textit{Los Angeles Times.}  25 October 1985. Accessed September 2017: \url{http://articles.latimes.com/1985-10-25/news/mn-14160_1_mexico-city}.} While the resulting destruction prompted the Mexican government to institute a number of regulatory measures to limit future damage, 32 years later, Mexico City's 2017 earthquake still resulted in a death toll of at least 360\footnote{The Associated Press. `Death toll rises to 360 in Mexico earthquake.' \textit{The Denver Post.} 21 September 2017. Accessed October 2017: \url{http://www.denverpost.com/2017/09/30/mexico-earthquake-death-toll-update/}} and the recovery effort could cost more than 2 billion dollars.\footnote{`The Associated Press.' ``Economic Costs of Mexico’s Earthquake Could Surpass \$2B.'' \textit{Insurance Journal} 29 September 2017. \url{http://www.insurancejournal.com/news/international/2017/09/29/465995.htm}}  The 2011 Fukushima incident meanwhile, stands out for both its death toll and high cost, leaving nearly 1,600 dead and more than 174,000 displaced.\footnote{Hamilton, Bevan. `Fukushima 5 years later: 2011 disaster by the numbers.' \textit{CBC News}. 10 March 2016. Accessed September 2017: \url{http://www.cbc.ca/news/world/5-years-after-fukushima-by-the-numbers-1.3480914}} Recent 2017 projections estimate that it will cost around 187 billion dollars -- double the 2013 estimate.\footnote{McCurry, Justin. `Possible nuclear fuel find raises hopes of Fukushima plant breakthrough.' \textit{The Guardian.} 30 January 2017. Accessed September 2017: \url{https://www.theguardian.com/environment/2017/jan/31/possible-nuclear-fuel-find-fukushima-plant}}  Similarly, estimates put the cost of responding to Hurricane Harvey, which left 82 dead,\footnote{Moravec, Eva Ruth. ``Texas officials: Hurricane Harvey death toll at 82 in 2017, `mass casualties have absolutely not happened.'' \textit{The Washington Post.} 14 September 2017. Accessed September 2017: ' \url{https://www.washingtonpost.com/national/texas-officials-hurricane-harvey-death-toll-at-82-mass-casualties-have-absolutely-not-happened/2017/09/14/bff3ffea-9975-11e7-87fc-c3f7ee4035c9_story.html?utm_term=.f5eecca9ee21}} at around 180 billion dollars, likely to be the most expensive natural disaster in US history.\footnote{`Hurricane Harvey Damages Could Cost up to \$180 Billion.' \textit{Fortune}. 3 September 2017. Accessed September 2017: \url{http://fortune.com/2017/09/03/hurricane-harvey-damages-cost/}} 

Few countries are spared the devastation that natural disasters can wreak. Between 1980 and 2004, approximately 7,000 natural disasters led to the deaths of around two million people and further negatively affected the lives of five billion more \citep{emdat:2009}. The economic costs are also considerable and rising, with the direct economic damage from natural disasters between 1980-2012 estimated to be around \$3.8 trillion \citep{gitay:2013}.

While dealing with both the immediate and long-term damage wrought by natural disasters can seriously drain existing resources for any country, developing countries generally find it especially difficult to cope. Often, their existing physical infrastructure is grossly unequal to the task of withstanding natural disasters. Meanwhile, their institutional infrastructure often lacks the resilience or capacity necessary to deal with the often long and complex process of rebuilding. In general, when natural disaster strikes, developing countries are likely to experience more serious physical damage and have less state capacity to recover from it. For example, prior to its 2010 earthquake, Haiti had no building codes and many of its buildings were not designed to withstand even a mild earthquake.\footnote{Watkins, Tom. `Problems with Haiti building standards outlined.' \textit{CNN}. 2010 January 14. Accessed September 2017: \url{http://edition.cnn.com/2010/WORLD/americas/01/13/haiti.construction/index.html}} Meanwhile, the lack of governmental leadership and low state capacity, along with other factors, has meant that even 7 years after the disaster, Haiti has yet to fully recover \citep{hartberg:2011}.\footnote{Cook, Jesselyn. ``7 years after Haiti's Earthquake, millions still need aid.'' \textit{Huffington Post}. 13 January 2017. Accessed May 2018: \url{https://www.huffingtonpost.com/entry/haiti-earthquake-anniversary_us_5875108de4b02b5f858b3f9c?guccounter=1}} 
%kobayashi:2014

From a purely tactical perspective then, natural disasters represent an opportune time to inflict harm on a strategic adversary, particularly, if it is a developing country, as both  government officials and public resources are fully engaged with responding to the emergency. Yet, anecdotal evidence suggests that strategic adversaries rarely take advantage of this opportunity, at least as far as can be openly observed. Many of the deadliest natural disasters (which should present foreign opponents the best opportunity to inflict harm) do not seem to have been followed up by hostile overtures. For instance, Taiwan did not use the 1976 Tangshan earthquake, believed to be the largest earthquake in the 20th century by death toll, as an opportunity to improve its strategic position vis-a-vis China. Similarly the 2011 Fukushima disaster was not followed by hostile gestures from China nor did Russia react to Hurricane Harvey with belligerence toward the US.\footnote{Note, whether countries take advantage of their strategic opponents using more covert methods during times of natural disaster is a more open question.} 

Context of course matters. There are  different rules of engagement when dealing with a country that one has contentious relationship with and taking advantage of a country with which one is actively engaged in outright conflict. In the former context, though taking preemptive action against a strategic opponent may lead to short term gains, it could very well lead to long term losses, especially since such an action would be well out of the realm of socially acceptable behavior in response to a natural disaster. But even by this hard-nosed logic, we might expect countries to simply do nothing when tragedy befalls their strategic opponents. Such behavior would fit well with the larger literature that investigates donor motivations for allocating foreign aid. Indeed, scholars have produced a large body of evidence suggesting that donors overwhelmingly prioritize their own self-interest over recipient need  when dispensing aid.\footnote{For example, see \citet{mckinlay:1977,mckinlay:1978,mckinley:1979,maizels:1984,schraeder.etal:1998,alesina:2000,berthelemy:2006,stone:2006,demesquita:2007,bermeo:2008,hoeffler:2011,dreher:2015}.}

Yet, much anecdotal evidence suggests that rather than jockeying for a more favorable strategic perch or doing nothing, natural disasters encourages the flow of \textit{aid} from strategic opponents. For example, during the famine that ravaged North Korea from 1994 to 1998, the United States, South Korea, Japan and the European Union stepped up as the primary donors of food aid \citep{noland:2004}.  Meanwhile, Taiwan was one of the biggest donors to China in the aftermath of the 2008 Sichuan earthquake.\footnote{`FACTBOX-Earthquake aid for China.' 14 May 2008.  \url{http://uk.reuters.com/article/idUKPEK29448220080514}} Taiwan also actively contributed to the rescue effort,\footnote{French, Howard and Edward Wong. `In Departure, China Invites Outside Help.' \textit{The New York Times}. 16 May 2008. Accessed September 2017: \url{http://www.nytimes.com/2008/05/16/world/asia/16china.html}} and further offered to share the technical expertise it developed from its own devastating earthquake experience in 1999.\footnote{Hille, Kathrin. `Taiwan shares quake lessons with Sichuan.' \textit{Financial Times}. 9 June 2008. Accessed September 2017: \url{https://www.ft.com/content/b0204002-3641-11dd-8bb8-0000779fd2ac}} Similarly, following Hurricane Katrina, the United States accepted Russian aid, despite frosty relations.\footnote{`U.S. accepts Russian Katrina aid.' \textit{UPI}. 2 September 2005. Accessed September 2017. \url{https://www.upi.com/US-accepts-Russian-Katrina-aid/39221125680989/}.} 

%\citep{buthecheng:2013}
Are these anecdotes of non-strategic behavior indicative of a systemic pattern or one-off exceptions to the rule of strategic self-interest? If the former, what could explain this seemingly humanitarian turn of behavior? Finding an answer to these questions in the current literature is difficult. For one, in evaluating the relative roles that donor interest and recipient need play in foreign aid allocation, what researchers refer to as recipient need may be more precisely understood as ``developmental need'' and as such, targeted towards addressing chronic poverty. To that end, development need is frequently measured using gross domestic product (GDP) or gross national product (GNP) per capita;\footnote{For example, see \citet{mckinlay:1977,mckinlay:1978,mckinley:1979,maizels:1984,alesina:2000,berthelemy:2006,stone:2006,demesquita:2007,bermeo:2008}.} or occasionally with more holistic measures of social outcomes such as the Physical Quality of Life Index,\footnote{See \citet{maizels:1984}.} the average life expectancy,\footnote{See \citet{schraeder.etal:1998}.} or the daily caloric intake.\footnote{See \citet{mckinley:1979,schraeder.etal:1998}.}

Meanwhile, only a small body of research investigates the degree to which aid is given in response to acute crises, such as natural disasters, which will be referred to here as humanitarian need. Considering that around 11\% of official development assistance (ODA) was officially categorized as being given for humanitarian reasons in 2015, the systematic failure to include natural disasters as a potential driver of foreign aid is puzzling.\footnote{Total ODA for DAC countries was 131.6 billion in 2015, 15.6 billion of which was designated as humanitarian assistance \url{http://www.oecd.org/dac/development-aid-rises-again-in-2015-spending-on-refugees-doubles.htm} \url{http://www.oecd.org/dac/stats/humanitarian-assistance.htm}}  What evidence that does exist suggests a null or small effect of humanitarian aid on foreign aid allocations. For instance, \citet{bermeo:2008} finds no relationship between the number of people affected by disasters and the allocation of bilateral aid for France, Japan, the UK and the US.\footnote{Note, \citet{bermeo:2008} also conceptualizes humanitarian aid using measures of the number of refugees and civil war, with mixed effects across countries for both}  Similarly, \citet{david:2011} finds no statistically significant relationship between development aid flows and climatic or human disasters. David does find evidence for increased development aid following geological disasters, but the effect is only found with a 2 year lag and substantively small.\footnote{\citet{david:2011} defines climatic events as `floods, droughts, extreme temperatures and hurricanes'; human disasters as: famines and epidemics; geological events as: earthquakes, landslides, volcano eruptions and tidal waves.} \citet{yang:2008} also finds that ODA increases after a hurricane, but only with a lag of 2 years.\footnote{\citet{stromberg:2007} does find a positive and significant relationship between aid and natural disasters, but his paper is concerned with emergency aid in particular, not foreign aid. Similarly, \citet{olsen:2003} find that donors are more likely to give aid for strategic reasons, though their analysis is confined to emergency aid.} 

Finally, there appears to be virtually no work that has explored whether there is a conditional relationship between donor's strategic interest and recipient's humanitarian need. To our whether there may be a conditional relationship between donor's strategic interest and recipient's humanitarian need on foreign aid allocation decisions.  

% One seeming exception is \citet{drury_etal:2005} who find that between 1964 to 1995, the United States made its decision to dispense aid based on strategic considerations, but based the amount given on humanitarian considerations. However, their dependent variable of interest is humanitarian aid, not ODA. 
%%%%%%%%%%%%%%%%%%%%%%

%%%%% Theory %%%%%
\section*{How Natural Disasters Affect Foreign Aid Allocations}

Only in the twentieth century has expending public resources to relieve the human suffering of foreigners shifted from being a virtually inconceivable act to relatively commonplace. The devastation wrought by the two world wars was particularly instrumental in bringing about this change. However, such aid was strictly intended to serve as temporary transfers that would facilitate a return to the previous status quo, rather than a long-term commitment to ``development'' as such. The turn toward development aid was instead fostered by ongoing Cold War hostilities, which simultaneously promoted the use of aid to further donor's strategic goals while also building a new norm of rich countries aiding poor countries \citep{lancaster:2008}.

The role of mitigating disaster and suffering on the one hand and furthering strategic interest on the other are baked into the modern conception of foreign aid. This history also suggests that initial humanitarian aid, though meant to serve as a temporary expedient, may lead to the establishment of aid with longer-term strategic goals. Whether this pattern exists more generally and if so, whether it is driven primarily by strategic or humanitarian concerns is unclear however. The role of the Cold War in foreign aid's origin story  dictated that recipients of humanitarian aid were generally within a particular strategic bloc, making it difficult to untangle strategic from humanitarian drivers. The vignette of US-Iran aid relations following the 2003 Bam earthquake provides some anecdotal evidence, however, that contrasting examples exist.

As such, looking at how natural disasters affect foreign aid allocation is not only interesting in its own right but also provides an exogenous instrument with which to identify the role of donor interest and recipient need in explaining patterns of aid commitments. To that end, we derive and test four hypotheses as to how natural disasters affect foreign aid allocations. Further to better entangle the varying strategic motivations, we disaggregate foreign aid into three types: humanitarian, civil society, and development aid. In doing so, we seek to offer a more nuanced understanding of the principle drivers of foreign aid allocations. 

\subsection*{Short-term Humanitarian Response to Natural Disasters}

Responding to natural disasters quickly and efficiently is often crucial to saving lives and alleviating human suffering. The first 72 hours after a natural disaster are often critical as services like electricity, gas, water, and telecommunications may all be disrupted. The timely deployment of humanitarian aid is the first response that donors can extend to countries struck by natural disaster. In what follows, we develop two hypotheses as to how the interaction between strategic interests and natural disaster severity can affect humanitarian aid allocation. 

\subsubsection*{Realist Imperviousness to Natural Disasters}

The logic of realism dictates that countries are driven by self-interest to amass power in a zero-sum game. Thus, realist scholars expound the view that ``foreign aid is today and will remain for some time an instrument of political power'' \citep{liska:1960}. Under this logic, donors commit aid to recipient countries primarily to further their own strategic interests. Extant literature on the drivers of foreign aid have put forward strong substantive evidence to support this viewpoint.

With regards to the interaction between natural disasters and strategic interests, the realist perspective suggests that donors do not send more humanitarian aid to their strategic opponents in the event of a natural disaster as it is against their self-interest to do so. Rather, it would be in donor self-interest to send humanitarian aid to their strategic allies in the event of a natural disaster. This does not imply that donors never send humanitarian aid to their strategic opponents, only that they are more likely to send aid to their strategic allies. 

Indeed, disaster-afflicted countries appear to be sensitive to the possibility that accepting humanitarian aid from strategic opponents may come with strings attached. In 1999 for example, Venezuela experienced catastrophic flash floods and debris flows in Vargas State, which left as much as 10\% of the Vargas population dead \citep{wieczorek:2001}. US troops helped in the relief efforts by running helicopter rescue missions and working to provide clean water. However, consistent with his antagonism toward US hegemony in the region, President Hugo Chavez declined US assistance in rebuilding a critical highway, saying that while, ``he would accept American equipment if Venezuelan soldiers operated it...he did not want US troops in his country.''\footnote{Brand, Richard. `Chavez assailed on handling of Venezuelan flood disaster.' \textit{The Miami Herald}. 5 August 2001. Accessed September 2017: \url{http://www.latinamericanstudies.org/venezuela/venezuela-disaster.htm}.} Meanwhile, Iran categorically refused any aid from Israel following the 2003 Bam earthquake, though the Israeli government still encouraged its citizens to donate privately.\footnote{Popper, Nathaniel. ``Israelis Help Iran Victims Despite Rebuff.'' \textit{The Forward}. 2 January 2004. Accessed September 2017: \url{http://forward.com/news/6059/israelis-help-iran-victims-despite-rebuff/}} %Indeed, even the US first turned down Russian aid for Hurricane Katrina before ultimately accepting it.\footnote{`U.S. accepts Russian Katrina aid.' \textit{UPI}. 2 September 2005. Accessed September 2017. \url{https://www.upi.com/US-accepts-Russian-Katrina-aid/39221125680989/}.}

The logic of hard-nosed realism leads us to the following hypothesis as to how the interaction between natural disasters and strategic aid affects humanitarian aid allocations: 

\textit{H1A: Donors are driven by self-interest and in the event of a natural disaster, donors are more likely to send \textbf{humanitarian aid} to their strategic allies than their strategic opponents.}

\subsubsection*{Natural Disasters as (Temporary) Humanizers of Strategic Opponents}

Conversely, social context, rather than material interest, may be decisive in foreign aid allocation decisions. Recent research in behavioral economics also underscores the idea that different social contexts lead to varying behavior in identical situations \citep{kahneman:2003,do:2011}. While there is evidence that non-governmental organizations are driven by the norms of humanitarian discourse when allocating aid \citep{buthe:2012}, evidence for similar behavior in governments has been mixed at best. Natural disasters may reorient the social context of a dyadic relationship to encourage donors to increase aid to their strategic opponents. That is, the loss of human life and destruction of infrastructure, which natural disasters provoke, can temporarily serve to emphasize the human aspect of the bilateral relationship as opposed to the  political, economic, and military aspects that generally define foreign relations between two countries.

Moreover, if natural disasters do have a humanizing effect, than we might expect strategic opponents to be particularly sensitive to it. This is, given that strategic opponents are more likely to ``otherize'' each other, then dyadic opponents must traverse a greater gap to humanize each another compared to dyadic allies  \citep{de:2012}. For example, seeing disaster befall Cubans may humanize them more for Americans compared to Russians. More explicitly, historically hostile relations between the US and Cuba may mean that the baseline extent to which they ``otherize'' each other is much greater than in the Russian-Cuban relationship, increasing the potential for Cubans to be humanized in American eyes compared to Russian ones. \footnote{Note, regardless of the actual motivation of donors when they give aid to their strategic opponents, there is anecdotal evidence to suggest that aid given under such circumstances can also serve to humanize the donors as well. For example, in the wake of US and South Korean aid for the North Korean famine, one refugee summarized his reaction to the US Institute for Peace this way: ``We were taught all these years that the South Koreans and Americans were our enemies. Now we see they are trying to feed us. We are wondering who our real enemies are'' \citet{natsios:1999}. This suggests that to some extent, social context can matter.}

That is not to say that natural disasters can always bridge this divide among strategic opponents. For example, India and Pakistan have had an uneasy history in accepting aid from each other following natural disasters.\footnote{Ravishankar, Siddharth. `Cooperation between India and Pakistan after Natural Disasters.' \textit{Stimson Center}. 9 January 2015. Accessed September 2017: \url{https://www.stimson.org/content/cooperation-between-india-and-pakistan-after-natural-disasters}} In general, we contend only that natural disasters may make it more \textit{likely} that a strategic adversary will contribute aid because the humanitarian disaster temporarily reframes the context of bilateral relations. An understanding of the interaction between natural disasters and strategic interests affects humanitarian aid allocations based on social context thus leads us to the following hypothesis:

\textit{H1B:  Donors who are strategic opponents of the recipient are more likely than strategic allies to be sensitive to the humanizing effect of natural disasters, and, consequently, are likely to send more \textbf{humanitarian aid}.} 

\subsection*{Long-term Responses to Natural Disasters}

Donor countries may dispense aid that not only serves to immediately address the natural disaster at hand, but also through other channels that have longer-term objectives. Here, we make a distinction between civil society aid and development aid. Civil society aid is aimed at supporting non-governmental organizations (NGOs) and their programs. Often, the goal of such aid is to empower grass-roots advocacy and improve governance and government accountability. Meanwhile, development aid is targeted toward promoting long-term economic development in a recipient country often through the building of infrastructure like roads and hospitals as well as the building of human resources via technical training and education. In what follows, we develop hypotheses as to how the interaction between strategic interest and natural disasters can affect the allocation of these two different types of aid.

\subsubsection*{Natural Disasters as Strategic Opportunities}

If, as following the realist logic, foreign aid is used to promote donor interests, then donor governments should be especially inclined to increase the allocation of civil society aid. This is because aiding the development of civil society is an inherently political act.\footnote{\url{http://foreignpolicy.com/2013/05/21/the-prickly-politics-of-aid/}} From supporting the growth of government watch dogs to increasing the domestic capacity for grass roots advocacy, whether it is their intention or not, donors are able to exert influence over a recipients domestic politics by directing funds to civil society.


With respect to natural disasters, countries may be motivated to give more civil society aid to their strategic opponents because the temporary suspension in the normal dynamics of the relationship represents a unique opportunity to increase civil society aid and initiate a shift in the nature of the bilateral relationship. That is, donor countries may either already recognize all to well or come to recognize that the natural disasters offers an opportunity to improve the terms of their relationship with the affected country. Either way, donors can seize on a country's inherent vulnerability following a natural disaster to  decide to \textit{strategically} increase their civil aid so as to increase their chances of exerting domestic influence over the recipient countries. As such, we derive the following hypothesis:

\textit{H2: Natural disasters present an opportune window for donors to exert influence over recipients who are their strategic opponents and as such, donors are more likely to send additional \textbf{civil society aid} to their strategic opponents.}  

\subsubsection*{Natural Disasters as (Permanent) Humanizers of Strategic Opponents}

Whereas humanitarian aid provides stop-gap measures to address the immediate aftermath of a natural disaster, the focus of development aid is to build the conditions for long-term, sustainable economic growth. As such, if social context matters, natural disasters can have a humanizing effect on strategic opponents that is not only short-term, but long-term, then this should be manifested in an increase in development aid. That is, the initial devastation wrought by a natural disaster may not only encourage greater humanitarian aid, but may also encourage greater contact and mutual understanding between donors and recipients and a greater donor commitment to recipient development. As in H1B, strategic opponents may be particularly sensitive to this humanizing effect because relative to strategic allies, they are more likely to have had a dehumanizing view of the recipient country. This results in the final hypothesis:

\textit{H3: Natural disasters have a long-term humanizing effect on their strategic opponents and as such, donors are more likely to send greater \textbf{development aid} to their strategic opponents.}  

\section*{Measuring Strategic Relationships}

One reason for evaluating the \textit{motivations} for aid and not aid \textit{outcomes} is that aid given for strategic reasons may still further development objectives, albeit incidentally, while aid given for humanitarian reasons may also bring unexpected strategic benefits \citep{maizels:1984}. However, evaluating the motivations for aid is not a straightforward process -- any given aid project may work toward providing assistance to a recipient country as well as strategic benefits to a donor country. 

Of critical importance to investigating whether strategic considerations (and by extension, the interaction between strategic considerations and humanitarian need) affects foreign aid considerations then is constructing a reliable measure of strategic interest. Unfortunately, we find that \citet{alesina:2000}'s remark that  ``the measurement of what a `strategic interest' is varies from study to study and is occasionally tautological,'' still holds true.  Indeed, strategic interest has alternately been operationalized as: trade intensity \citep{berthelemy:2004,bermeo:2008,hoeffler:2011}, UN voting scores \citep{alesina:2000, alesina:2002,hoeffler:2011,dreher:2012}, arms transfers \citep{maizels:1984}, colonial legacy \citep{alesina:2000, bermeo:2008, berthelemy:2004,berthelemy:2006}, alliances \citep{bermeo:2008,schraeder.etal:1998}, regional dummies \citep{bermeo:2008,berthelemy:2006, maizels:1984}, bilateral dummies \citep{alesina:2000, berthelemy:2004, berthelemy:2006}\footnote{A US-Egypt or US-Israel dummy seems to be the most common instance of a bilateral dummy} or some combination of the above.\footnote{Meanwhile other papers take a negative approach and argue that any shortfall between what would theoretically be expected from poverty-efficient aid allocation and actual aid allocation \citep{collier:2002,nunnenkamp:2006,thiele:2007}, or similarly between a theoretical allocation based on good governance and actual aid allocation \citep{dollar:2006,neumayer:2005}, is evidence of strategic interest at play.} 

Such inconsistency in the operationalization of strategic interest is not simply a matter of using different variables to measure the same concept but a matter of using different variables to measure different \textit{aspects} of the underlying concept. However, while a dyad's strategic bilateral relationship is quite multifaceted, to date, there has not been a readily available measure of strategic relationships which captures its various aspects the same way that scholars have done for other complex concepts.\footnote{For example, Polity and Freedom House have provided measures or political institutions while the World Bank's World Governance Indicators (WGI) project provides measures for six dimensions of governance} To address this problem, we create a new measure of strategic interest that is able to account for varying aspects of strategic interest. 

\subsection*{A new measure of strategic relationships}

To generate a measure of strategic relationships we adopt a latent variable approach that enables us to estimate a relational measure of interest between countries by taking into account the direct and indirect ways in which states are connected across a variety of dimensions. Specifically, we utilize three dimensions of state relations to construct our strategic interest measure: dyadic alliances, UN voting, and joint membership in intergovernmental organizations (IGOs). We focus on these dimensions because each provides a representation of the political and military relations between countries in the international system. Additionally, these measures are commonly employed in the foreign aid literature to measure strategic interest. Dyadic alliances largely capture the strategic and military aspect of country relationships. Meanwhile, joint membership in IGOs reflects the dyadic relationship across many political issue areas, and UN voting is better able to capture this relationship in a centralized forum. 

To estimate a measure of strategic interest across these dimensions, we take a network based approach that allows us to leverage both the direct and indirect ways in which states are connected to one another. To do this we employ a latent factor model as described in \citet{hoff:2005}. The model is structured as follows:

\begin{align}
\begin{aligned}
	Y = \textbf{u}_{i}^{T} & \textbf{u}_{j} + \epsilon_{ij} \text{, where} \\
	&\textbf{u}_{i} \in \mathbb{R}^{R=2}, \; i \in \{1, \ldots, n \} \\
	% &\Lambda \text{ a } K \times K \text{ diagonal matrix}
\end{aligned}
\end{align}

$Y$ here is a $n \times n$ undirected sociomatrix in which $y_{ij}$ designates whether there exists a link (e.g., an alliance) between $i$ and $j$. The goal of the model is to provide a projection of the systematic variation in $Y$ into a low-dimensional social space.\footnote{The latent factor model we utilize here is based on an eigvenvalue decomposition that seeks to represent relations between countries as the weighted inner-product of country-specific vectors of latent characteristics. In this application, we project our $n \times n$ sociomatrix into a $n \times 2$ matrix of country positions in a latent social space.} More precisely, the types of systematic variation that we are interested in include the concepts of (a) transitivity, (b) balance and (c) clusterability. Formally, a set of three countries $ijk$ is said to be transitive, if for whenever $y_{ij} = 1$ and $y_{jk} = 1$, we also observe that $y_{ik} = 1$. This follows the logic of `` a friend of a friend is a friend''. Meanwhile, the relationships between $ijk$ are said to be balanced if $y_{ij} \times y_{jk} \times y_{ki} >0$. Conceptually, if the relationship between $i$ and $j$ is ``positive'', then both will relate to another unit $k$ identically, either both positive or both negative. Finally, relationships between $ijk$ are said to be clusterable if it is balanced or all the relations are all negative. It is a relaxation of the concept of balance and seeks to capture groups where the measurements are positive within groups and negative between groups.

Thus third order dependencies suggest that ``knowing something about the relationship between $i$ and $j$ as well as between $i$ and $k$ may reveal something about the relationship between $i$ and $k$, even when we do not directly observe it'' \citep{hoff:2004}. Such dependences would seem especially relevant for our purposes, as one cannot understand the strategic relationship between two countries without taking into account their respective relationships with other countries. The importance for accounting for these dynamics have long been acknowledged in the foreign aid literature. \citet{trumbull:1994} for example, note that, ``donors do make their decisions with knowledge of what each other are doing, and may actually act cooperatively. Any study that ignores the interrelationship of donor behavior risks problems with simultaneity bias.'' However, we find that until now, this critique has largely gone unaddressed by the existing literature. 

The main advantage of calculating the latent space of different dyadic variables as opposed to using alternative specifications such as the S Score algorithm\footnote{\citet{leeds:2007}, for example, measure a states ``threat environment'' as the set of all states for which ones is contiguous with or which is a major power and with an S score below the population median.} is that it allows us to better account for indirect ties that states share. Indirect ties are accounted for within this framework because the latent factor model takes patterns such as transitivity into account, as a result, the relation between two actors can be inferred even if no
direct interaction between them is observed.

We employ this latent factor model on every year for each of our three measures.\footnote{The models are estimated via Gibbs sampling from the full conditional distributions of $\textbf{u}_{i}^{T} \textbf{u}_{j}$. For a more detailed discussion of this model, see \citet{hoff:2005}.} In Figure \ref{fig:polLat}, we present a visualization of the resultant latent space we calculated for each variable for the year 2005.

\begin{figure}[h!] 
\caption{Latent Spaces for components of Political Strategic Interest Measure during 2005}
\label{fig:polLat}
\centering
	\begin{minipage}{.33\linewidth}
		\centering
		\label{fig:ally}
		\includegraphics[width = 1.1\textwidth]{\detokenize{ally_2005.jpg}}
		\caption*{(a) Alliances}
		\end{minipage}
		\hspace{0.2in}
		\begin{minipage}{.33\linewidth}
		\centering
		\label{fig:un}
		\includegraphics[width = 1.1\textwidth]{\detokenize{un_2005.jpg}}
		\caption*{(b) UN voting}
		\end{minipage}
		%\hspace{.2in}
		\begin{minipage}{.33\linewidth}
		\centering
		\label{fig:igo}
		\includegraphics[width = 1.1\textwidth]{\detokenize{igo_2005.jpg}}
		\caption*{(c) IGO membership}
	\end{minipage}
\end{figure}

\indent\indent Countries that cluster together in this two-dimensional latent space are more likely to interact with each other. The plots for alliances, UN voting and IGO membership suggest that there is distinct clustering among countries. Moreover, these clusters are different across the three measures, suggesting that each variable is indeed capturing different aspects of strategic interest.

\indent\indent After estimating the latent spaces for these components, we estimate the standardized distance between each pair of each countries for the three component. We then combine them in a principal components analysis (PCA) to reduce the dimensionality of our measure while retaining as much variance as possible. That is, alliances, UN voting and joint membership in IGOs all capture certain aspects of political strategic interest. Instead of choosing only one of them as our measure of strategic interest as other papers have done, we combine them in order to increase our explanatory power. We estimate the PCA of these variables for each year separately\footnote{For each year, we conduct a bootstrap PCA of 1000 subsamples.} and use the first principal component for each year as our measure of strategic interest. On average over all the years, we find that the first component of our PCA of alliances, UN voting and joint membership in IGOs, which we use as our measure of strategic interest, explains about 51\% of its variance. 
For further validation checks of our strategic interest variable, please see Appendix \ref{sec:appendix}.

The end result of this process is a measure of strategic interest that takes into account indirect ties while also accounting for multiple dimensions in which states interact with one another. 

% Our measure of strategic relationships introduces greater coherency to the literaturex by providing a more rigorous measure that captures aspects of strategic interest. We do so by providing a latent space representation of indirect ties that states share on three dimensions: dyadic alliances, UN voting and joint membership in an intergovernmental organizations (IGOs). After generating each latent space representation, we calculate the distances between countries in that space, and then utilize a principal components analysis (PCA) across the distances to generate a single measure.\footnote{As such, our political strategic relationship measure is the first principal component that results from the PCA of the latent distance between each variable.} 
%%%%%%%%%%%%%%%%%%%%%%


%%%%% Theory %%%%%
\section*{Data}
\label{data}

\subsection*{Aid flows}

Our data from foreign aid flows is taken from the AidData project \citep{tierney2011more}. This database includes information on over a million aid activities from the 1940s to the present. We use the country level aggregated version of this database to create a directed-dyadic dataset of total aid dollars committed. In this analysis, we focus specifically on OECD donor countries as they both are the best able and have the best incentive to give foreign aid to advance their strategic interests. In the final tally, our dataset includes the 18 most active senders\footnote{More specifically, the included donor countries are: Australia, Belgium, Canada, Denmark, France, Germany, Greece, Iceland, Ireland, Italy, Luxembourg, the Netherlands, Norway, Portugal, Spain, Sweden, the United Kingdom and the United States. These countries were chosen both to maximize comparability with previous work as well as for reasons of data availability. Research on non-DAC donors suggests that like DAC donors, they seem to be primarily driven by strategic motivations in distributing aid \citep{neumayer:2003,dreher_etal:2011,fuchs:Vadlamannati:2013,dreher:2015,dreher:2018}.  Existing evidence suggests that non-DAC donors do seem more likely to give aid following a natural disaster however \citep{dreher_etal:2011}, though they still only account for at most 12\% of humanitarian aid in any given year \citep{harmer:2005}. This research suggests that our findings might be even stronger among non-DAC donors. Future work investigating this possibility will become increasingly important the more foreign aid non-DAC donors distribute. } and 167 receivers of aid flows from 1975 to 2005. Accounting for all possible senders of aid during this time frame is difficult because of the amount of missing data. That being said, issues with missingness in our dataset still exist and we deal with them by employing a multiple imputation method developed by \citet{hoff:2007} and shown to have good performance by \citet{hollenbach:2014}. %However, even with the limited number of senders in this version of our analysis we still have approximiately 40,000 observations worth of data to work with.

We use the AidData's Sector coding scheme in order to disaggregate bilateral ODA into humanitarian aid, development aid, and civil society aid.\footnote{``AidData's Sector Coding Scheme.'' \url{http://docs.aiddata.org/ad4/files/aiddata_coding_scheme_0.pdf}}  To that end, our measure of humanitarian aid encompasses the sectors of:\\ 

% \begin{quote}
% 	``Emergency Response'', ``Reconstruction Relief'', and ``Disaster Prevention and Preparedness''.
% \end{quote}

\begin{tabular}{lll}
	\hline
    \multirow{2}{*}{``Emergency response''} & \multirow{2}{*}{``Reconstruction Relief''} & ``Disaster Prevention  \\[-.5mm]
	~ & ~ &  $\quad$and Preparedness'' \\
    \hline
\end{tabular}\\

\noindent Meanwhile, civil society aid is measured as aid to the sectors of:\\

% \begin{quote}
% 	``Government and Civil Society',  ``Women'', ``Support to Non-Governmental Organizations and Governmental Organizations''.
% \end{quote}

\begin{tabular}{lll}
	\hline
	\multirow{3}{*}{``Government and Civil Society''} & \multirow{3}{*}{``Women''} & ``Support to Non-Governmental  \\[-.5mm]
    ~ & ~ & $\quad$Organizations and Governmental \\[-.5mm]
    ~ & ~ &  $\quad$Organizations'' \\
    \hline
\end{tabular}\\

\noindent Finally, development aid is defined as aid given to the following sectors:\\ 

\begin{tabular}{lll}
  \hline
  ``Education'' & ``Health'' & ``Water Sanitation'' \\
  ``Other Infrastructure & ``Economic Infrastructure & \multirow{2}{*}{``Environmental Protection''} \\[-.5mm]
  $\quad$ and Services'' & $\quad$ and Services'' & ~ \\
  ``Other Social  & ``Agriculture Forestry & ``Industry, Mining \\[-.5mm]
  $\quad$Infrastructure and Services'' & $\quad$ and Fishing'' &  $\quad$and Construction'' \\
  ``Other Development Aid'' & '``Food Aid'' & ``Debt Relief'' \\
  \hline
\end{tabular}\\

We note that bilateral ODA often represents only one channel through which donors may allocate foreign aid and that an increasing number of papers have argued for accounting for the heterogeneity of aid channels donors may use when estimating drivers of foreign aid \citep{nunnenkamp:2011,buthecheng:2013,dietrich:2013}. Here, we choose to focus solely on bilateral aid in order to maintain greater comparability with previous studies. % but we can look at the differentiated effects as robustness checks?

\subsection*{Strategic Interest}

As previously stated, we created our measure of strategic relationships by conducting a PCA on the latent distances for alliances, UN voting and joint IGO membership. Data for alliances was retrieved from the Correlates of War (COW) Formal Alliance dataset \citep{gibler:2009}. Following \citet{demesquita:1975} and \citet{signorino:1999}, we distinguish between different types of alliances with the following weighting scheme: 0 = no alliance, 1 = entente, 2 = neutrality or nonaggression pact, 3 = mutual defense pact.\footnote{Note, as for alliances, we had attempted to distinguish between different types of membership but found that very few states were listed as Associate Members or Observers of an IGO for the time period that we are conducting our analysis. Thus we used the simpler coding scheme.} 

\indent\indent UN voting data was obtained from the United Nations General Assembly Data set \citep{strezhnev:2012}.  We calculate the proportion of times two states agree out of the total number of votes they both voted on. Agreement means either both vote yes, both vote no, or both abstain. This measure is similar to the `voting similarity index' readily available from the dataset except the voting similarity index does not account for mutual abstentions. 

\indent\indent  Meanwhile IGO voting data was obtained from the Correlates of War International Governmental Organizations Data Set \citep{pevehouse:2004}. A total of 529 IGOs across a broad swath of topics, including trade, communications, and health and security,  are represented in this dataset. Dyads were coded as 1 if they belonged to the same IGO as a full member or an associate member and coded as 0 if one or both of them was an observer, had no membership, was not yet a state or was missing data. \footnote{Information on the IGOs included in the dataset are available from the Correlates of War website: \url{http://www.correlatesofwar.org/data-sets/IGOs}} 

\subsection*{Natural Disasters}
%\footnote{\url{http://www.emdat.be/}}
Almost all the empirical work on natural disasters relies on the publicly available Emergency Events Database (EM-DAT) maintained by the Center for Research on the Epidemiology of Disasters (CRED) at the Catholic University of Louvain, Belgium. EM- DAT defines a disaster as a natural situation or event which overwhelms local capacity and/or necessitates a request for external assistance. For a disaster to be entered into the EM-DAT database, at least one of the following criteria must be met: i) 10 or more people are reported killed; ii) 100 people are reported affected; iii) a state of emergency is declared; or iv) a call for international assistance is issued.  We  use a count of the number of natural disasters a country has experienced a year as our measure of natural disaster severity. 

\subsection*{Additional Covariates}

In addition to our dyadic strategic relationship measures, we include a number of covariates to capture characteristics of aid recipients.

For our measure of political institutions, we use Polity IV data available from the Center for Systemic Peace \citep{gurr:2010}. Polity IV captures differences in regime characteristics on a 21 point scale ranging from -10 (hereditary monarchy) to +10 (consolidated democracy), rescaling it to range from 1 to 21 for greater ease of interpretation.   We also controlled for colonial history using the Colonial History Data Set from the Issue Correlates of War Project \citep{hensel:2009}. This variable is coded as a one when the receiver in a sender-receiver dyad is a former colony of the sender and zero otherwise. 

Meanwhile, for our measures of developmental need, we use (1) Log GDP per capita and (2) life expectancy at birth.  Both of these measures are extracted from the World \citet{wb:2013}. Finally, we control for the incidence of civil war in a recipient country as it informs the ability for a donor country to dispense aid. We do so with data retrieved from the Uppsala Conflict Data Program (UCDP)/International Peace Research Institute (PRIO) Armed Conflict Database. \citep{gleditsch:2002}. We code as civil war any armed conflict which either (a) ``Internal armed conflict occurs between the government of a state and one or more internal opposition group(s) without intervention from other states'' or (b) ``Internationalized internal armed conflict occurs between the government of a state and one or more internal opposition group(s) with intervention from other states (secondary parties) on one or both sides.''
%%%%%%%%%%%%%%%%%%%%%%

%%%%% Empirics %%%%%
\section*{Analysis}
\label{empirics}

%% just taking this out because this model equation doesnt account for the hierarchical structure
% \begin{align*}
%   Log(Aid_{sr,t}) &= \beta_{1}(Pol. \; Strat.  \; Distance_{sr,t-1}) \\
%   & \;+\; \beta_{2}(Colony_{sr,t-1}) \;+\; \beta_{3}(Polity_{r,t-1}) \\
%   & \;+\; \beta_{4}Log(GDP \; per \; capita_{r,t-1}) \;+\; \beta_{5}(Life \; Expect_{r,t-1}) \\  
%   & \;+\; \beta_{6}(No. \; Disasters_{r,t-1}) \;+\; \beta_{7}(Civil \; War_{r,t-1}) 
% \end{align*}

\subsection*{Estimation Method}

To model aid flows using our directed-dyadic panel dataset, we utilize a hierarchical model. To implement this, we nest aid receivers within aid senders and aid senders within years. We include random intercepts in our model for every sender and year. More concretely, we fit the following model: 

\begin{align*}
  Log(Aid)_{sr,t} &= \beta_{1}(Pol. \; Strat.  \; Distance_{sr,t-1})  \\
  & \;+\; \beta_{2}(Colony_{sr,t-1}) \;+\; \beta_{3}(Polity_{r,t-1}) \\
  & \;+\; \beta_{4}Log(GDP \; per \; capita_{r,t-1}) \;+\; \beta_{5}(Life \;Expect_{r,t-1}) \\  
  & \;+\; \beta_{6}(No. \; Disasters_{r,t-1}) \;+\; \beta_{7}(Civil \; War_{r,t-1}) \\
  & \;+\; \beta_{8}(Pol. \; Strat.  \; Interest_{sr,t-1} \times No. \; Disasters_{r,t-1}) \\
   & \;+\; \delta_s  \;+\; \rho_r \\
\end{align*}

\noindent Where $\delta_s$ and $\rho_r$ are the sender and receiver random effects respectively. 

The results of this analysis are shown below in a coefficient plot in Figure \ref{fig:intCoef}.\footnote{Note, to examine the model results without the interaction effects, please see Figure \ref{fig:nointCoef} in Appendix \ref{app:rawModels}} We test Hypothesis 1A and 1B using the model with `Humanitarian Aid' as the dependent variable. The results show a positive and statistically significant relationship between the interaction of \textit{Strategic Distance} and the \textit{No. Disasters}. To interpret these results, we turn to Figure \ref{fig:simEffects} (`Humanitarian Aid' panel) where we plot the substantive effect of this interaction term on humanitarian aid over the range of \textit{Strategic Distance} for different levels of natural disaster severity. 

\begin{figure}
\centering
\includegraphics[height = 5in]{intCoef.pdf}
\caption{Coefficient plots for the main analyses with interaction terms across each dependent variable, humanitarian aid, civil society aid and development aid.  Coefficients that are significant at the 5\% level are shaded in blue if the coefficient is positive and red if the coefficient is negative. Coefficients that are not significant at the 5\% level are shaded in gray. }
\label{fig:intCoef}
\end{figure}

\begin{figure}
\centering
\includegraphics[height = 5.5in]{simComboPlot.pdf}
\caption{Simulated substantive effect plots for each dependent variable (humanitarian aid, civil society aid, and development aid) for different levels of natural disaster severity across the range of the strategic distance measure.}
\label{fig:simEffects}
\end{figure}

These results suggest that the greater the number of natural disasters a country experiences, the more likely it is to receive humanitarian aid from a strategic adversary. This is apparent in the rising slope of the relationship between strategic interest and humanitarian aid as the number of natural disasters increases. As such, these results are consistent with H1B, which suggests that donors may be more likely to dispense humanitarian aid to their strategic adversaries because such disasters humanize them. Conversely, these results do not support H1A, which hypothesizes that donors are more likely to give to their strategic allies in the wake of a natural disaster to further their own self-interest. This result suggests that donors do not always act in their self interest when dispensing foreign aid under some circumstances. Notably when natural disasters are particularly severe, they may have, at least in the short term, a humanizing effect on strategic adversaries. 

Meanwhile, we test H2 by examining the effect of the interaction between strategic interest and natural disasters on civil society aid. In Figure \ref{fig:intCoef} we similarly find a positive and significant relationship between this interaction and civil society aid. The substantive effects plot (in the `Civil Society Aid' panel in Figure \ref{fig:simEffects}) meanwhile also suggests that donors are more likely to target aid to civil society in their strategic adversaries the more natural disasters that country experiences, supporting H2. These results suggest that donors may not be completely abandoning their strategic interest but are also acting to take advantage of vulnerable recipients to mold the relationship to their interests. 

Finally, we test H3 by analyzing how the interaction between strategic interest and natural disasters affects development aid allocation. From, Figure  \ref{fig:intCoef}, we can see that this coefficient is not statistically significant. Examining the substantive significance in Figure \ref{fig:simEffects} (`Development Aid' panel) we can see that while the level of development aid does increase as the number of natural disasters increases, the slope between strategic interest and development aid is only minimally affected, suggesting little support for H3. These results suggest that whatever humanitarian impulse donors may have felt toward their strategic adversaries (which H1B suggests exists), this effect only applies to short-term humanitarian aid and not long-term development aid. 

\subsection*{Persistence of foreign aid allocation over time}

How persistent are these changes that we observe? To answer this question, we re-estimate the original models for different lag lengths for the main interaction terms\footnote{The covariates are measured using a one-year lag throughout}. These models are estimated separately for each lag length (lags of 1 through 6 years). The coefficient plots for the interaction term of interest and its constituent terms, number of natural disasters and strategic distance are shown in Figures \ref{fig:humanIntCoef}, \ref{fig:civIntCoef}, and \ref{fig:devIntCoef} for the outcome variables humanitarian aid, civil society aid and development aid respectively across each lag value.

From Figure \ref{fig:humanIntCoef}, we can see that the interaction between strategic interest and natural disasters is rather persistent. This suggests that donors are more likely to allocate humanitarian aid to their strategic adversaries for some time following a natural disaster, suggesting some staying power for the ability of natural disasters to shift donors from acting in their own strategic interests to acting in the humanitarian interest of aid recipients (supporting H1B).

\begin{figure}[h!]
\centering
\includegraphics[height = 4in]{simComboPlot_lag_hAid.pdf}
\caption{ }
\label{fig:simLag}
\end{figure}

\begin{figure}[h!]
\centering
\includegraphics[height = 5in]{humanitarianTotal_int_lagEffect.pdf}
\caption{ }
\label{fig:humanIntCoef}
\end{figure}

Meanwhile, Figure \ref{fig:civIntCoef} shows that while the interaction between strategic interest and natural disasters positively affects the allocation of civil society aid, this effect is only consistently significant for the first three years following a natural disaster (supporting H2). One way to interpret these results is that donors recognize the difficulty of trying to influence domestic politics through civil society aid relatively quickly and waste relatively little time in abandoning such attempts. Another interpretation is that civil society aid is actually rather effective and as such, recipients governments are likely to push back against allowing it in fairly short order. Teasing out the exact mechanism would be a fruitful area for future research.

Finally, Figure \ref{fig:devIntCoef} extends the earlier finding that the interaction between strategic interest and natural disasters has little effect on development aid across a variety of different lags. This result further suggests that there is little support for H3, that is natural disasters do not seem to prompt donors to care concern themselves with recipient's long-term interests.

\begin{figure}[h!]
\centering
\includegraphics[height = 5in]{civSocietyTotal_int_lagEffect.pdf}
\caption{ }
\label{fig:civIntCoef}
\end{figure}

\begin{figure}[h!]
\centering
\includegraphics[height = 5in]{developTotal_int_lagEffect.pdf}
\caption{ }
\label{fig:devIntCoef}
\end{figure}
%%%%%%%%%%%%%%%%%%%%%%

%%%%% Conclusion %%%%%
\section*{Conclusion}
\label{conclusion}

Our analysis suggests that a more nuanced understanding of the drivers of foreign aid is in order. While recent work has shown that accounting for the channel of aid delivery can go a long way toward understanding aid allocation decisions \citep{dietrich:2013,dietrich:2016}, we argue that the social context may also be an important consideration. In particular, we show that while donors are generally driven by strategic interests, this does not always manifest itself by the allocation of greater aid to strategic allies. Indeed, we find that donors are be driven to allocate humanitarian aid to strategic adversaries struck by natural disasters. We argue that one explanation for this finding is that donors see natural disasters as an opportunity improve relations with their strategic opponents. As shown in our lag models, these findings are surprisingly persistent. 

Moreover, natural disasters may not only increase short-term humanitarian aid. We find that strategic considerations also reign large when one considers the effect on the distribution of aid with longer-term targets.  We find that strategic adversaries are more likely to distribute civil society aid in the more natural disasters a country experiences, they are not more likely to distribute development aid. Because civil society aid inherently involves engagement and intervention in the domestic politics of a recipient country, an increase in civil society aid is indicative of a greater desire to increase donor influence over a recipient country, at least relative to development aid. Our analysis suggests however, that these results are rather short-term.

These results should be of particular interest as climate change continues to increase the incidence and the intensity of natural disasters. They suggest that while countries that experience natural disasters can expect humanitarian aid even from their strategic adversaries, such help can also open the doors to efforts to influence domestic politics in line with the interests of donors who have historically been antagonistic.

%During this hurricane season alone, residents in the United States have faced the wrath of Hurricane Harvey, Hurricane Irma . Meanwhile, wildfires continue to rage in Northern California. Neither is the rest of the world untouched, as the Mexico City earthquake, flooding in South Asia. \footnote{\url{https://www.nytimes.com/2017/08/29/world/asia/floods-south-asia-india-bangladesh-nepal-houston.html?_r=0}}

%Meanwhile, to revisit the original illustrative example presented in the introduction, note that though the US' offer and Iran's acceptance of humanitarian aid was a head-turning deviation from the status quo, the swiftness with which both countries reverted back to it was equally remarkable.  Indeed,  following the immediate fallout of the earthquake a couple of weeks later, Iran declined US offers of further humanitarian aid \footnote{``Iran to prosecute over building law breaches in Bam.'' \textit{China Daily.} 3 January 2004. Accessed October 2017: \url{https://web.archive.org/web/20090619204216/http://www.chinadaily.com.cn/en/doc/2004-01/03/content_295446.htm}}. Meanwhile, President Bush denied attempts to interpret US aid as evidence of thaw in US-Iran relations.\footnote{\url{http://news.bbc.co.uk/2/hi/middle_east/3362443.stm}}

%In this particular case then, the exchange of aid led to only a temporary reprieve from the generally contentious bilateral relations. However, whether aid given exchanged between historically contentious dyads can lead to a more permanent softening of relations remains an open question. We plan to explore this question more fully in future work. 



%%%%%%%%%%%%%%%%%%%%%%

\newpage
\bibliographystyle{chicago}
\bibliography{Paper/fAidRefs.bib}

%%%%% Appendix %%%%%
\newpage

\appendix
\section{Appendix}
\label{sec:appendix}

\subsection*{Validating our measure of strategic interest}

\indent\indent We further conduct a series of post-estimation validation tests for our resulting strategic variable. In particular, we (1) evaluate the relationship between our political strategic interest variable  against S scores and Kendall's $\tau_b$ for alliances and (2) investigate how our measure of strategic interest describe well-known dyadic relationships. 

First, we perform a simple bivariate OLS with and with year fixed effects to evaluate how our measures compare to S scores and Kendall's $\tau_b$.\footnote{Note for comparison that the bivariate relationship of S scores on Kendall's $\tau_b$ is statistically significant with a coefficient of 0.62 while the bivariate relationship of Kendall's $\tau_b$ on S Scores is statistically significant with a coefficient of 0.31.} Note in order to make our strategic measures somewhat interpretable, for the validation we scale our strategic measures to be between 0 and 1 just as S scores and Kendall $\tau_b$ is scaled. The results are shown in Table \ref{table:polval}. % for political strategic interest and Table \ref{table:milval} for military strategic interest. \\

\begin{table}[h!]
\small
\caption{Validation of Political Strategic Interest Variable against S scores and Kendall's $\tau_b$}
\begin{center}
\begin{tabular}{l c c c c c c }
\hline
                    & Unweighted   & Unweighted & Weighted  & Weighted  & Tau-B & Tau-B \\
                   &   S Scores &   S Scores &  S Scores &  S Scores &  &   \\
\hline
(Intercept)         & $0.97^{***}$  & $1.03^{***}$  & $1.01^{***}$  & $1.02^{***}$  & $0.29^{***}$  & $0.25^{***}$  \\
                    & $(0.00)$      & $(0.00)$      & $(0.00)$      & $(0.00)$      & $(0.00)$      & $(0.00)$      \\
Strategic Interest             & $-0.80^{***}$ & $-0.84^{***}$ & $-1.22^{***}$ & $-1.26^{***}$ & $-0.89^{***}$ & $-0.87^{***}$ \\
                    & $(0.00)$      & $(0.00)$      & $(0.00)$      & $(0.00)$      & $(0.00)$      & $(0.00)$      \\
Year FE? 	   & No 		& Yes 		& No		& Yes	& No		& Yes\\
% \hline
% R$^2$               & 0.28          & 0.32          & 0.32          & 0.34          & 0.17          & 0.17          \\
% Adj. R$^2$          & 0.28          & 0.32          & 0.32          & 0.34          & 0.17          & 0.17          \\
% Num. obs.           & 824426        & 824426        & 824426        & 824426        & 824148        & 824148        \\
\hline
\multicolumn{7}{l}{\scriptsize{$^{***}p<0.001$, $^{**}p<0.01$, $^*p<0.05$}}
\end{tabular}
\label{table:polval}
\end{center}
\end{table}

\indent\indent  In brief, we find that our political strategic measure performs well against S scores and Kendall's $\tau_b$ for alliances  with and without fixed effects. Note that because the PCA is of latent distances between any two dyads, dyads that are closer in space will have smaller values and therefore represent a stronger strategic relationship. Therefore the negative relationship we find between the political strategic measure and S scores and $\tau_b$ are interpreted to mean the greater the foreign policy similarity as measured by the S score or Kendal's $\tau_b$ , the smaller the latent distance or the greater the political strategic relationship between a dyad.


Next, we assess the relative model fit of our strategic interest variable compared to the raw components of our strategic interest variable (``Raw UN Votes'', ``Alliances'', ``IGO Membership'')  as well as alternative measures of strategic interest (``UN Ideal Point'', ``S-Score, Unweighted'', ``S-Score, Weighted''). We assess the model fit by conducting 10-Fold Cross validations of each imputed dataset and calculating the subsequent root mean squared error (RMSE) for each model. We plot the RMSES for each possible partition of the data, donor country (Figure \ref{rmse:donor}), recipient country (Figure \ref{rmse:recipient}), and year (Figure \ref{rmse:year}) and find that the model fit when using our strategic interest variable is not significantly different to the model fit when using other possible measures of strategic interest.
	


\begin{figure}
\centering
\caption{RMSES of 10-Fold Cross validation, partitioned by Donor Country}
\label{rmse:donor}
\includegraphics[height = 4in]{rmse_10FoldCrossVal_ccodeS.pdf}
\end{figure}

\begin{figure}
\centering
\caption{RMSES of 10-Fold Cross validation, partitioned by Recipient Country}
\label{rmse:recipient}
\includegraphics[height = 4in]{rmse_10FoldCrossVal_ccodeR.pdf}
\end{figure}

\begin{figure}
\centering
\caption{RMSES of 10-Fold Cross validation, partitioned by Year}
\label{rmse:year}
\includegraphics[height = 4in]{rmse_10FoldCrossVal_year.pdf}
\end{figure}



\subsection{Analyses for models without interaction terms}
\label{app:rawModels}

\begin{figure}[h!]
\centering
\includegraphics[height = 5in]{noIntCoef.pdf}
\caption{Coefficient plots for the analyses  without interaction terms for each dependent variable, humanitarian aid, civil society aid and development aid.  Coefficients that are significant at the 5\% level are shaded in blue if the coefficient is positive and red if the coefficient is negative. Coefficients that are not significant at the 5\% level are shaded in gray. }
\label{fig:nointCoef}
\end{figure}




%%%%%%%%%%%%%%%%%%%%%%

\end{document} 